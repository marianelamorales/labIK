\documentclass[sigplan,10pt]{acmart}\settopmatter{}


%% Conference information
%% Supplied to authors by publisher for camera-ready submission;
%% use defaults for review submission.
\acmConference[WiL'18]{the Second Women in Logic Workshop}
\acmYear{2018}
\acmISBN{} % \acmISBN{978-x-xxxx-xxxx-x/YY/MM}
\acmDOI{} % \acmDOI{10.1145/nnnnnnn.nnnnnnn}
\startPage{1}

%% Copyright information
%% Supplied to authors (based on authors' rights management selection;
%% see authors.acm.org) by publisher for camera-ready submission;
%% use 'none' for review submission.
\setcopyright{none}
%\setcopyright{acmcopyright}
%\setcopyright{acmlicensed}
%\setcopyright{rightsretained}
%\copyrightyear{2017}           %% If different from \acmYear

%% Bibliography style
\bibliographystyle{ACM-Reference-Format}

%% Citation style
%\citestyle{acmauthoryear}  %% For author/year citations
%\citestyle{acmnumeric}     %% For numeric citations
%\setcitestyle{nosort}      %% With 'acmnumeric', to disable automatic
                            %% sorting of references within a single citation;
                            %% e.g., \cite{Smith99,Carpenter05,Baker12}
                            %% rendered as [14,5,2] rather than [2,5,14].
%\setcitesyle{nocompress}   %% With 'acmnumeric', to disable automatic
                            %% compression of sequential references within a
                            %% single citation;
                            %% e.g., \cite{Baker12,Baker14,Baker16}
                            %% rendered as [2,3,4] rather than [2-4].


%%%%%%%%%%%%%%%%%%%%%%%%%%%%%%%%%%%%%%%%%%%%%%%%%%%%%%%%%%%%%%%%%%%%%%
%% Note: Authors migrating a paper from traditional SIGPLAN
%% proceedings format to PACMPL format must update the
%% '\documentclass' and topmatter commands above; see
%% 'acmart-pacmpl-template.tex'.
%%%%%%%%%%%%%%%%%%%%%%%%%%%%%%%%%%%%%%%%%%%%%%%%%%%%%%%%%%%%%%%%%%%%%%


%% Some recommended packages.
\usepackage{booktabs}   %% For formal tables:
                        %% http://ctan.org/pkg/booktabs
\usepackage{subcaption} %% For complex figures with subfigures/subcaptions
                        %% http://ctan.org/pkg/subcaption

%% Added Packages
\usepackage{virginialake}\vlnosmallleftlabels
\usepackage{graphicx}
\usepackage{amsmath}
\usepackage{amssymb}
\usepackage{float}
\usepackage{color}
\floatstyle{boxed} 
\restylefloat{figure}
%\usepackage{colonequals}
%\newcommand{\vlhtr}[2]{\vlpd{#1}{}{#2}}
\usepackage{bm}

%% Marianela's macros
\renewcommand{\G}{\mathcal{G}}
\newcommand{\Left}{\mathcal{L}}
\newcommand{\Right}{\mathcal{R}}

%Symbols for System labK
\newcommand{\id}{id^{lab}}
\newcommand{\tolab}{\top^{lab}}
\newcommand{\vlab}{\wedge^{lab}}
\newcommand{\olab}{\vlor^{lab}}
\newcommand{\blab}{\square^{lab}}
\newcommand{\dlab}{\lozenge^{lab}}

%Labelled proof system
\renewcommand{\toprule}{\G \Rightarrow \Right, x  \colon   \top}
\newcommand{\vlabr}{\G \Rightarrow \Right, x  \colon   A}
\newcommand{\vlabu}{\G \Rightarrow \Right, x  \colon   B}
\newcommand{\olabr}{\G \Rightarrow \Right, x  \colon   A, x  \colon   B}
\newcommand{\blabr}{\G \Rightarrow \Right, x  \colon   \square A}
\newcommand{\blabu}{\G, x$R$y \Rightarrow \Right, y  \colon   A}
\newcommand{\dlabr}{\G, x$R$y \Rightarrow \Right, x  \colon   \lozenge A}
\newcommand{\dlabu}{\G, x$R$y \Rightarrow \Right, x  \colon   \lozenge A, y  \colon} 

%Symbols for system labIK
\newcommand{\botlab}{\bot_{L}^{lab}}
\newcommand{\toplab}{\top_{R}^{lab}}
\newcommand{\andleflab}{\wedge_{L}^{lab}}
\newcommand{\andriglab}{\wedge_{R}^{lab}}
\newcommand{\orleflab}{\vlor_{L}^{lab}}
\newcommand{\orriglabo}{\vlor_{R1}^{lab}}
\newcommand{\orriglabt}{\vlor_{R2}^{lab}}
\newcommand{\irlab}{\vljm_{R}^{lab}}
\newcommand{\illab}{\vljm_{L}^{lab}}
\newcommand{\dllab}{\lozenge_{L}^{lab}}
\newcommand{\drlab}{\lozenge_{R}^{lab}}
\newcommand{\bllab}{\square_{L}^{lab}}
\newcommand{\brlab}{\square_{R}^{lab}}

%Sumbols for System labheartIK
\newcommand{\gklmn}{\boxtimes_{gklmn}}
\newcommand{\ids}{id}
\newcommand{\idg}{id_{g}}
\newcommand{\refl}{refl}
\newcommand{\trans}{trans}
\newcommand{\cut}{cut}
\newcommand{\fone}{F1}
\newcommand{\ftwo}{F2}
\newcommand{\sbot}{\bot_{L}}
\newcommand{\Stop}{\top_{R}}
\newcommand{\svlef}{\wedge_{L}}
\newcommand{\svrig}{\wedge_{R}}
\newcommand{\solef}{\vlor_{L}}
\newcommand{\sorig}{\vlor_{R}}
\newcommand{\sorone}{\vlor_{R1}}
\newcommand{\sotwo}{\vlor_{R2}}
\newcommand{\sir}{\vljm_{R}}
\newcommand{\sil}{\vljm_{L}}
\newcommand{\sdl}{\lozenge_{L}}
\newcommand{\sdr}{\lozenge_{R}}
\newcommand{\sbl}{\square_{L}}
\newcommand{\sbr}{\square_{R}}
\newcommand{\smon}{mon_{L}}
\newcommand{\M}{\mathcal{M}}
\newcommand{\F}{\mathcal{F}}
\newcommand{\Gone}{\mathcal{G}_{1}}
\newcommand{\Gtwo}{\mathcal{G}_{2}}
\newcommand{\Dw}{\mathcal{D}^{w}}
\newcommand{\Dwone}{\mathcal{D}_{1}^{w}}
\newcommand{\Dwtwo}{\mathcal{D}_{2}^{w}}
\newcommand{\D}{\mathcal{D}}
\newcommand{\Done}{\mathcal{D}_{1}}
\newcommand{\Dtwo}{\mathcal{D}_{2}}

%System LABIK
\newcommand{\conjrig}{\G, \Left \Rightarrow \Right, x \colon A}
\newcommand{\conjrigh}{\G, \Left \Rightarrow \Right, x  \colon B}
\newcommand{\conjlef}{\G, \Left, x  \colon  A, x \colon B \Rightarrow \Right}



%% Sonia's macros
\newcommand{\marianela}[1]{{\color{purple}[Marianela: #1]}}
\newcommand{\sonia}[1]{{\color{blue}[Sonia: #1]}}
\newcommand{\lutz}[1]{{\color{green}[Lutz: #1]}}
\newcommand{\todo}[1]{{\color{red}[TODO: #1]}}

%%% Systems
\newcommand*{\lab}{\mathsf{lab}}
\newcommand*{\IK}{\mathsf{IK}}


%%% Connectives
\newcommand*{\NOT}{\neg}
\newcommand*{\AND}{\mathbin{\wedge}}
\newcommand*{\TOP}{\mathord{\top}}
\newcommand*{\OR}{\mathbin{\vee}}
\newcommand*{\BOT}{\mathord{\bot}}
\newcommand*{\IMP}{\mathbin{\supset}}

\newcommand*{\BOX}{\mathord{\Box}}
\newcommand*{\DIA}{\mathord{\Diamond}}

%%% Labelled sequents
\newcommand{\B}{\mathcal{B}}
\newcommand*{\rel}{R}
\newcommand*{\labels}[2]{{#1}\:\colon{#2}}
\newcommand{\SEQ}{\Rightarrow}
%\newcommand*{\DD}{\mathcal{D}}
\newcommand*{\rn}[1]  {\ensuremath{\mathsf{#1}}}

%%% Labelled rules
\newcommand*{\labrn}[2][]  {\rn{#2}_{#1}}%^{\lab}}}
\newcommand*{\rlabrn}[2][]  {\rn{#2}_{R#1}}%^\lab}}
\newcommand*{\llabrn}[2][]  {\rn{#2}_{L#1}}%^\lab}}
%\newcommand*{\brsym}{\mathord{\scalebox{.8}{$\blacksquare$}}}
%\newcommand*{\boxbrn}[1]{\rn{\brsym_\rn{#1}}}%^{\lab}}}
%
%%% Extracting symbols from MnSymbol 
\DeclareFontFamily{U} {MnSymbolC}{}

\DeclareFontShape{U}{MnSymbolC}{m}{n}{
	<-6>  MnSymbolC5
	<6-7>  MnSymbolC6
	<7-8>  MnSymbolC7
	<8-9>  MnSymbolC8
	<9-10> MnSymbolC9
	<10-12> MnSymbolC10
	<12->   MnSymbolC12}{}
\DeclareFontShape{U}{MnSymbolC}{b}{n}{
	<-6>  MnSymbolC-Bold5
	<6-7>  MnSymbolC-Bold6
	<7-8>  MnSymbolC-Bold7
	<8-9>  MnSymbolC-Bold8
	<9-10> MnSymbolC-Bold9
	<10-12> MnSymbolC-Bold10
	<12->   MnSymbolC-Bold12}{}

\DeclareSymbolFont{MnSyC}         {U}  {MnSymbolC}{m}{n}

%\DeclareMathSymbol{\triangleright}{\mathbin}{MnSyC}{80}
\DeclareMathSymbol{\diamondplus}{\mathbin}{MnSyC}{124}
%\DeclareMathSymbol{\boxtimes}{\mathbin}{MnSyC}{117}
%\DeclareMathSymbol{\meddiamond}{\mathbin}{MnSyC}{110}
%\DeclareMathSymbol{\medsquare}{\mathbin}{MnSyC}{106}
%\DeclareMathSymbol{\vee}{\mathbin}{MnSyC}{45}
%\DeclareMathSymbol{\wedge}{\mathbin}{MnSyC}{44}
%\DeclareMathSymbol{\bot}{\mathbin}{MnSyC}{150}
%\DeclareMathSymbol{\top}{\mathbin}{MnSyC}{151}
%\DeclareMathSymbol{\forall}{\mathbin}{MnSyC}{166}
%\DeclareMathSymbol{\exists}{\mathbin}{MnSyC}{167}
%\DeclareMathSymbol{\smalldiamond}{\mathbin}{MnSyC}{108}
%\DeclareMathSymbol{\filleddiamond}{\mathbin}{MnSyC}{109}


\begin{document}

%% Title information
\title{Decomposing labelled proof theory for~intuitionistic~modal~logic}         %% [Short Title] is optional;
                                        %% when present, will be used in
                                        %% header instead of Full Title.
%\titlenote{with title note}             %% \titlenote is optional;
                                        %% can be repeated if necessary;
                                        %% contents suppressed with 'anonymous'
%\subtitle{Subtitle}                     %% \subtitle is optional
%\subtitlenote{with subtitle note}       %% \subtitlenote is optional;
                                        %% can be repeated if necessary;
                                        %% contents suppressed with 'anonymous'


%% Author information
%% Contents and number of authors suppressed with 'anonymous'.
%% Each author should be introduced by \author, followed by
%% \authornote (optional), \orcid (optional), \affiliation, and
%% \email.
%% An author may have multiple affiliations and/or emails; repeat the
%% appropriate command.
%% Many elements are not rendered, but should be provided for metadata
%% extraction tools.

%% Author with single affiliation.
\author{Sonia Marin}
\authornote{This publication was made possible in part by NPRP Grant \#7-988-1-178 from the Qatar National
	Research Fund (a member of Qatar Foundation). The statements made herein are
	solely the responsibility of the authors.}          %% \authornote is optional;
                                        %% can be repeated if necessary
%\orcid{nnnn-nnnn-nnnn-nnnn}             %% \orcid is optional
\affiliation{
%  \position{Position1}
%  \department{Department1}              %% \department is recommended
  \institution{IT-Universitetet i K{\o}benhavn}            %% \institution is required
%  \streetaddress{Street1 Address1}
%  \city{City1}
%  \state{State1}
%  \postcode{Post-Code1}
  \country{Denmark}                    %% \country is recommended
}
%\email{first1.last1@inst1.edu}          %% \email is recommended

\author{Marianela Morales}
%\authornote{with author2 note}          %% \authornote is optional;
%% can be repeated if necessary
%\orcid{nnnn-nnnn-nnnn-nnnn}             %% \orcid is optional
\affiliation{
	%  \position{Position2a}
	%  \department{Department2a}             %% \department is recommended
	\institution{Universidad Nacional de C\'ordoba}           %% \institution is required
	%  \streetaddress{Street2a Address2a}
	%  \city{City2a}
	%  \state{State2a}
	%  \postcode{Post-Code2a}
	\country{Argentina}                   %% \country is recommended
}
%\email{first1.last1@inst1.edu}          %% \email is recommended

\author{Lutz Stra{\ss}burger}
%\authornote{with author2 note}          %% \authornote is optional;
                                        %% can be repeated if necessary
%\orcid{nnnn-nnnn-nnnn-nnnn}             %% \orcid is optional
\affiliation{
%  \position{Position2a}
%  \department{Department2a}             %% \department is recommended
  \institution{Inria Saclay \& LIX} %, \'Ecole Polytechnique}           %% \institution is required
%  \streetaddress{Street2a Address2a}
%  \city{City2a}
%  \state{State2a}
%  \postcode{Post-Code2a}
  \country{France}                   %% \country is recommended
}
%\email{first2.last2@inst2a.com}         %% \email is recommended


%% Abstract
%% Note: \begin{abstract}...\end{abstract} environment must come
%% before \maketitle command

%% 2012 ACM Computing Classification System (CSS) concepts
%% Generate at 'http://dl.acm.org/ccs/ccs.cfm'.
%\begin{CCSXML}
%<ccs2012>
%<concept>
%<concept_id>10011007.10011006.10011008</concept_id>
%<concept_desc>Software and its engineering~General programming languages</concept_desc>
%<concept_significance>500</concept_significance>
%</concept>
%<concept>
%<concept_id>10003456.10003457.10003521.10003525</concept_id>
%<concept_desc>Social and professional topics~History of programming languages</concept_desc>
%<concept_significance>300</concept_significance>
%</concept>
%</ccs2012>
%\end{CCSXML}
%
%\ccsdesc[500]{Software and its engineering~General programming languages}
%\ccsdesc[300]{Social and professional topics~History of programming languages}
%% End of generated code


%% Keywords
%% comma separated list
%\keywords{Proof theory, Intuitionistic modal logic, Labelled sequents.}  %% \keywords are mandatory in final camera-ready submission


%% \maketitle
%% Note: \maketitle command must come after title commands, author
%% commands, abstract environment, Computing Classification System
%% environment and commands, and keywords command.
\maketitle

Structural proof theoretic accounts of intuitionistic modal logic can adopt the paradigm of \emph{labelled deduction} in the form of labelled natural deduction and sequent systems~\cite{Simpson}, or the one of \emph{unlabelled deduction} in the form of sequent~\cite{Bierman} or nested sequent systems~\cite{Strassburger} (for a survey see~\cite[Chap.~3]{Marin}).

Simpson's labelled sequents make use only of relational atoms referring to the accessibility relation of a Kripke model. 
%
In this short note we propose a system that represents both the \emph{accessibility relation} (for modal logics) and the \emph{preorder relation} (for intuitionistic logic), 
using the full power of the bi-relational semantics for intuitionistic modal logics~\cite{Fischer, Plotkin}, and developing fully the idea of~\cite{Maffezioli}. 


A \emph{bi-relational frame}~\cite{Fischer, Plotkin} $\B$ is a triple $\langle W, R, \le \rangle$ of a non-empty set of worlds $W$ equipped with an {accessibility relation} $R$ and a preorder $\le$, satisfying:
\begin{itemize}
	\item[($F_1$)] For all worlds $x$, $y$, $z$, if $xRy$ and $y \le z$, there exists a $u$ such that $x \le u$ and $uRz$.
	
	\item[($F_2$)] For all worlds $x$, $y$, $z$, if $xRy$ and $x \le z$, there exists a $u$ such that $y \le u$ and $zRu$.
\end{itemize}


Reflecting this definition, we define our two-sided intuitionistic labelled sequents, similarly to~\cite{Maffezioli}, to be of the form $\mathcal{B}, \Left \Rightarrow \Right$ with $\B$ a set of relational atoms $xRy$ and preorder atoms $x \le y$, and $\Left,\Right$ multi-sets of labelled formulas $x \colon A$ (for $x, y$ labels and $A$ an intuitionistic modal formula).
%

Furthermore, our system has to incorporate the two semantic conditions into deductive rules as follows:
%
$$\vlinf{\rn{F_1}}{\text{\footnotesize $u$ fresh}}{\B, xRy, y \le z, \Left \SEQ \Right}{\B, xRy, y \le z, x \le u, uRz, \Left \SEQ \Right}$$
%
$$\vlinf{\rn{F_2}}{\text{\footnotesize $u$ fresh}}{\B, xRy,x \le z, \Left \SEQ \Right}{\B, xRy, x \le z, y \le u, zRu, \Left \SEQ \Right }$$

In the intuitionistic setting, the validity of a modal formula has to be defined using both the $R$ and the $\le$ relation as:
%\begin{center}
$x \Vdash \BOX A$ iff for all $y$ and $z$ s.t.~$x \le y$ and $yRz$, $z \Vdash A$.
%\end{center}

Again, our system reflects exactly this definition in the rules introducing the $\BOX$-operator:
%
$$\vlinf{\BOX_\rn{L}}{}{\B, \Left, x \le y, yRz, x \colon \BOX A \SEQ \Right}{\B, x \le y, yRz, \Left, x \colon \BOX A, z \colon A \SEQ \Right}$$
%
$$\vlinf{\BOX_\rn{R}}{\text{\footnotesize $y$, $z$ fresh}}{\B, \Left \Rightarrow \Right, x \colon \BOX A}{\B, x \le y, yRz, \Left \Rightarrow \Right, z \colon A}$$

By complementing these rules with the standard labelled rules for intuitionistic modal logic of~\cite{Simpson}, we get a system that is sound and complete wrt.~the birelational semantics.

In~\cite{Plotkin}, Plotkin and Stirling give a correspondence result for intuitionistic modal logic extended with a family of axioms wrt.~some classes of bi-relational frames.
%
For example, the frames that validate the axiom $\rn{4}_\rn\DIA \colon \DIA\DIA A \IMP \DIA A$ are exactly the ones satisfying the condition:
\begin{itemize}
	\item[($\diamondplus_\rn{4}$)] if $w \rel v$ and $v \rel u$, there exists a $u'$ s.t.~$u \le u'$ and $wRu'$.
\end{itemize}

Incorporating the preorder symbol into the syntax of our sequents allows us to also obtain a sound and complete proof system for the intuitionistic modal logic extended with axiom $\rn{4}_\rn\DIA$, by designing the following rule:
$$\vlinf{\diamondplus_\rn{4}}{\text{\footnotesize $u'$ fresh}}{\B, w \rel v, v \rel u, \Left \SEQ \Right}{\B, w \rel v, v \rel u, u \le u', w \rel u' , \Left \SEQ \Right}$$

Therefore, we decompose further the formalism of labelled sequents and extend the reach of labelled deduction to the logics studied in~\cite{Plotkin}.
%
These systems enjoy cut-elimination via usual arguments, the generality of the result is subject of ongoing study.


\begin{thebibliography}{4}
	%	\bibitem{Brunnler}
	%	Kai Br{\"u}nnler. \emph{Deep Sequent Systems for Modal Logic}. Archive for Mathematical Logic, 48(6):551-577, 2009.

	\bibitem{Bierman}
	G.~M.~Bierman and V.~de Paiva. \emph{On an Intuitionistic Modal Logic}
	Studia Logica, 2000.		%, 65(3):383-416
	
	\bibitem{Maffezioli}
	P.~Maffezioli, A.~Naibo, and S.~Negri. \emph{The Church-Fitch knowability paradox in the light of structural proof theory}. 
	Synthese, 2013. %190(14):2677-2716, 
	
	%	\bibitem{Fitch}
	%	Frederic B Fitch. \emph{Tree proofs in modal logic}. Journal of Symbolic Logic, 31(1):152, 1966.
	
%	\bibitem{Negri}
%	Sara Negri. \emph{Proof analysis in modal logics}. Journal of Philosophical Logic, 34:507-544, 2005. 
	
	\bibitem{Simpson}
	A.~Simpson. \emph{The Proof Theory and Semantics of Intuitionistic Modal Logic}. PhD thesis, University of Edinburgh, 1994. 
	
%	\bibitem{Vigano}
%	Luca Vigan\`o. \emph{Labelled Non-Classical Logic}. Kluwer Academic Publisher, 2000. 
	
	\bibitem{Marin}
	S.~Marin. \emph{Modal proof theory through a focused telescope}. PhD thesis, Universit\'e Paris-Saclay, 2018.
	
	\bibitem{Fischer}
	G.~Fischer-Servi.\emph{ Axiomatizations for some intuitionistic modal logics}. R.~del Seminario Matematico della Univ.~Politecnica di Torino, 1984. %, 42(3):179-194
	
	\bibitem{Plotkin}
	G.~D.~Plotkin and C.~P.~Stirling. \emph{A framework for intuitionistic modal logic}. In J.~Y.~Halpern, editor, 1st Conference on Theoretical Aspects of Reasoning About Knowledge.
	Morgan Kaufmann, 1986.
	
	%\bibitem{Gabbay}
	%Dov M. Gabbay. \emph{Labelled Deductive Systems}. Clarendon Press, 1996.

	%	\bibitem{kuz:str}
	%	Roman Kuznets and Lutz Stra{\ss}burger. \emph{Maehara-style Modal Nested Calculi}. Research report, Inria RR-9123, 2017.
	
	%	\bibitem{mar:str}
	%	Sonia Marin and Lutz Stra{\ss}burger. \emph{Label-free Modular Systems for Classical and Intuitionistic Modal Logics}. In Advances in Modal Logics, 2014.
	
	\bibitem{Strassburger}
	L.~Stra{\ss}burger. \emph{Cut Elimination in Nested Sequents for Intuitionistic Modal Logics}. In F.~Pfenning, editor, 16th Conference on Foundations of Software Science and Computation Structures.
	Springer, 2013. % , LNCS, 7794:209-224
	
	
\end{thebibliography}

%% Acknowledgments
%\begin{acks}                            %% acks environment is optional
%                                        %% contents suppressed with 'anonymous'
%  %% Commands \grantsponsor{<sponsorID>}{<name>}{<url>} and
%  %% \grantnum[<url>]{<sponsorID>}{<number>} should be used to
%  %% acknowledge financial support and will be used by metadata
%  %% extraction tools.
%  This material is based upon work supported by the
%  \grantsponsor{GS100000001}{National Science
%    Foundation}{http://dx.doi.org/10.13039/100000001} under Grant
%  No.~\grantnum{GS100000001}{nnnnnnn} and Grant
%  No.~\grantnum{GS100000001}{mmmmmmm}.  Any opinions, findings, and
%  conclusions or recommendations expressed in this material are those
%  of the author and do not necessarily reflect the views of the
%  National Science Foundation.
%\end{acks}


%% Bibliography
%\bibliography{bibfile}

%% Appendix
%\appendix
%\section{Appendix}
%
%Text of appendix \ldots

\end{document}
