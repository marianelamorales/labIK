\documentclass[twoside]{aiml18}

\usepackage{aiml18macro}
\usepackage{virginialake}
\usepackage{graphicx}
\usepackage{amsmath}
\usepackage{amssymb}
\usepackage{float}
\floatstyle{boxed} 
\restylefloat{figure}


%%%%%%%%%%%%%%%%%%%%%%%%%%%%%%%%%%%%%%%%%%%%%%%%%%%%%%%%%
% Setting the correct page numbers                      
% Ignore the next two commented lines                   
% but please don't delete                               
%%%%%%%%%%%%%%%%%%%%%%%%%%%%%%%%%%%%%%%%%%%%%%%%%%%%%%%%%
%\input{../procnum.tex}
%\numbering{../aiml18db}{paper}
\newcommand{\G}{\mathcal{G}}
\newcommand{\Left}{\mathcal{L}}
\newcommand{\Right}{\mathcal{R}}

%Symbols for System labK
\newcommand{\id}{id^{lab}}
\newcommand{\tolab}{\top^{lab}}
\newcommand{\vlab}{\wedge^{lab}}
\newcommand{\olab}{\vlor^{lab}}
\newcommand{\blab}{\square^{lab}}
\newcommand{\dlab}{\lozenge^{lab}}

%Labelled proof system
\newcommand{\toprule}{\B \Rightarrow \Right, x  \colon   \top}
\newcommand{\vlabr}{\B \Rightarrow \Right, x  \colon   A}
\newcommand{\vlabu}{\B \Rightarrow \Right, x  \colon   B}
\newcommand{\olabr}{\B \Rightarrow \Right, x  \colon   A, x  \colon   B}
\newcommand{\blabr}{\B \Rightarrow \Right, x  \colon   \square A}
\newcommand{\blabu}{\B, x$R$y \Rightarrow \Right, y  \colon   A}
\newcommand{\dlabr}{\B, x$R$y \Rightarrow \Right, x  \colon   \lozenge A}
\newcommand{\dlabu}{\B, x$R$y \Rightarrow \Right, x  \colon   \lozenge A, y  \colon} 


%Symbols for system labIK
\newcommand{\botlab}{\bot_{L}^{lab}}
\newcommand{\toplab}{\top_{R}^{lab}}
\newcommand{\andleflab}{\wedge_{L}^{lab}}
\newcommand{\andriglab}{\wedge_{R}^{lab}}
\newcommand{\orleflab}{\vlor_{L}^{lab}}
\newcommand{\orriglabo}{\vlor_{R1}^{lab}}
\newcommand{\orriglabt}{\vlor_{R2}^{lab}}
\newcommand{\irlab}{\vljm_{R}^{lab}}
\newcommand{\illab}{\vljm_{L}^{lab}}
\newcommand{\dllab}{\lozenge_{L}^{lab}}
\newcommand{\drlab}{\lozenge_{R}^{lab}}
\newcommand{\bllab}{\square_{L}^{lab}}
\newcommand{\brlab}{\square_{R}^{lab}}

%Sumbols for System labheartIK
\newcommand{\gklmn}{\boxtimes_{gklmn}}
\newcommand{\ids}{id}
\newcommand{\idg}{id_{g}}
\newcommand{\refl}{refl}
\newcommand{\trans}{trans}
\newcommand{\cut}{cut}
\newcommand{\fone}{F1}
\newcommand{\ftwo}{F2}
\newcommand{\sbot}{\bot_{L}}
\newcommand{\Stop}{\top_{R}}
\newcommand{\svlef}{\wedge_{L}}
\newcommand{\svrig}{\wedge_{R}}
\newcommand{\solef}{\vlor_{L}}
\newcommand{\sorig}{\vlor_{R}}
\newcommand{\sorone}{\vlor_{R1}}
\newcommand{\sotwo}{\vlor_{R2}}
\newcommand{\sir}{\vljm_{R}}
\newcommand{\sil}{\vljm_{L}}
\newcommand{\sdl}{\lozenge_{L}}
\newcommand{\sdr}{\lozenge_{R}}
\newcommand{\sbl}{\square_{L}}
\newcommand{\sbr}{\square_{R}}
\newcommand{\smon}{mon_{L}}
\newcommand{\M}{\mathcal{M}}
\newcommand{\F}{\mathcal{F}}
\newcommand{\Gone}{\mathcal{G}_{1}}
\newcommand{\Gtwo}{\mathcal{G}_{2}}
\newcommand{\Dw}{\mathcal{D}^{w}}
\newcommand{\Dwone}{\mathcal{D}_{1}^{w}}
\newcommand{\Dwtwo}{\mathcal{D}_{2}^{w}}
\newcommand{\D}{\mathcal{D}}
\newcommand{\Done}{\mathcal{D}_{1}}
\newcommand{\Dtwo}{\mathcal{D}_{2}}


%System LABIK
\newcommand{\conjrig}{\G, \Left \Rightarrow \Right, x \colon A}
\newcommand{\conjrigh}{\G, \Left \Rightarrow \Right, x  \colon B}
\newcommand{\conjlef}{\G, \Left, x  \colon  A, x \colon B \Rightarrow \Right}


% definitions specific to your article
\newcommand{\ob}{[}
\newcommand{\cb}{]}


%%%%%%%%%%%%%%%%%%%%%%%%%%%%%%%%%%%%%%%%%%%%%%%%%%%%%%%%%

%The following line defines the page header consisting of the surnames of the authors.
% Please include only the last names! 
% Separate by commas except the last two surnames which are separated by an "and".
\def\lastname{Surname1, Surname2, Surname3, ..., Surname(n-1) and Surname(n)}

\begin{document}

\begin{frontmatter}
  \title{Descomposing labelled proof theory for intuitionistic modal logic}
  \author{Author}
  \address{Affiliation \\ Address }
 \author{Author}
 \address{Affiliation \\ Address \\ Address}
  
  \begin{abstract}
  We study labelled deduction for intuitionistic modal logic. We show the possibility of extend labelled sequents with a preorder relation symbol in order to capture intuitionistic modal logic. For this, we obtain a proof system which is complete with respect to Hilbert system.  Also, we present the proof for completeness using Simpson system.
  \end{abstract}

  \begin{keyword}
  Modal logic, Intuitionistic modal logic, labelled sequents, proof theory.
  \end{keyword}
 \end{frontmatter}


\section{Introduction}
One possible-world semantics was established as a solid base to define modal logics, the idea of incorporating these notions into the proof theory of modal logics emerged. Fitch seems to have been the first one to formalise it, directly including symbols representing worlds into the language of his proofs in natural deduction \cite{Fitch}.

Labelled deduction has been more generally proposed by Gabbay in the 80’s as a unifying framework throughout proof theory in order to provide proof systems for a wide range of logics. For modal logics it can also take the form of labelled natural deduction and labelled sequent systems as used, for example, by Simpson \cite{Simpson}, Vigano \cite{Vigano} and Negri\cite{Negri}. These formalisms make explicit use not only of labels, but also of relational atoms. 

Labelled sequents are formed from by labelled formulas of the form $x \colon A$  and relational atoms of the form $x$R$y$, where $x$, $y$ range over a set of variables and $A$ is a modal formula. A one-sided labelled sequent is then of the form $\G \Rightarrow \Right$ where $\G$ denotes a set of relational atoms and $\Right$ a multiset of labelled formulas. A simple proof system for classical modal logic K can be obtained in this formalism (\textbf{Fig. 1}). 

\begin{figure}[h]
\begin{center}

$\vlderivation {\vlinf{\id}{}{\G \Rightarrow \Right, x: a, x: \vls-a}{}}$
\hspace{7mm}$\vlderivation {\vlinf{\tolab}{}{\toprule}{}}$


$\vliinf{\vlab}{}{\G \Rightarrow \Right, x :\vls(A.B)}{\vlabr}{\vlabu}$
\hspace{7mm}$\vlinf{\olab}{}{\G \Rightarrow \Right, x \colon \vls[A.B]}{\olabr}$


$\vlinf{\blab}{y$ fresh$}{\blabr}{\blabu}$
\hspace{7mm}$\vlinf{\dlab}{}{\dlabr}{\dlabu}$


\end{center}
\caption{System labK}
\end{figure}
These rule schemes can occur in different contexts and different calculi. The context that interests us is when it is applied to modal logic.

Echoing to the definition of bi-relational structures, the straightforward extension of labelled deduction to the intuitionistic setting would be to use two sorts of relational atoms, one for the modal relation R and another one for the intuitionistic relation $\leq$. This is the approach developed by Maffezioli, Naibo and Negri in \cite{Maffezioli}. Therefore, the idea is to extend labelled sequents with a preorder relation symbol in order to capture intuitionistic modal logics, that is to define intuitionistic labelled sequents from labelled formulas $x \colon A$, relational atoms $x$R$y$, and preorder atoms of the form $x \leq y$, where x, y range over a set of labels and A is an intuitionistic modal formula.

A two-sided intuitionistic labelled sequent would be of the form $\G$, $\Left $ $\Rightarrow \Right$ where $\G$ denotes a set of relational and preorder atoms, and $\Left$ and $\Right$ are multiset of labelled formulas. We then obtain a proof system lab$\heartsuit $IK (\textbf{Fig. 2}) for intuitionistic modal logic in this formalism.

\section{Capturing intuitionistic modal logics with labels}
As we mentioned, we obtain a proof system lab$\heartsuit$IK which allows us to give an extension of labelled deduction to the intuitionistic world and then we prove the next theorem:

\begin{theorem}
A formula $A$ is provable in the calculus lab$\heartsuit$IK if and only if $A$ is valid in every bi-relational frame.
\end{theorem}

We show that each rule from our system is sound as we also present a syntactic completeness proof with respect the Hilbert system: we prove all Hilbert axioms using the rules from our system (i.e. proof of all propositional intuitionistic axioms, the five variants of k axiom from the intuitionistic syntax, simulate the necessitation rule and simulate modus ponens). 

Furthermore, we present a completeness proof for our system lab$\heartsuit$IK using the Simpson system. The idea comes from knowing that the Simpson system is a Cut-free system, so this proof lets us know that our system is complete without the cut rule. We show the proof by case analysis. Most of the rules from Simpson system are the same as the rules in the system lab$\heartsuit$IK, then we prove for the rules that are different.

\begin{figure}[h]
\begin{center}

$\vlinf{\sbot}{}{\G,\Left, x \colon \bot \Rightarrow \Right}{}$
\hspace{7mm}$\vlinf{\ids}{}{\G, \Left,x \le y, x \colon a \Rightarrow \Right, y \colon a}{}$ \hspace{7mm}$\vlinf{\Stop}{}{\G, \Left \Rightarrow \Right, x \colon \top}{}$


\vspace{4mm}

$\vlinf{\svlef}{}{\G,\Left, x \colon \vls(A.B) \Rightarrow \Right}{\conjlef}$
\hspace{7mm}$\vliinf{\svrig}{}{\G,\Left \Rightarrow \Right, x \colon \vls(A.B)}{\conjrig}{\conjrigh}$

\vspace{4mm}

$\vliinf{\solef}{}{\G, \Left, x \colon \vls[A.B] \Rightarrow \Right}{\G, \Left, x   \colon   A \Rightarrow \Right}{\G, \Left, x   \colon   B \Rightarrow \Right}$
\hspace{7mm}$\vlinf{\sorig}{}{\G, \Left \Rightarrow \Right, x \colon \vls[A.B]}{\G, \Left \Rightarrow \Right, x   \colon   A, x   \colon   B}$

\vspace{4mm}

$\vlinf{\sir}{$ $y$ fresh$}{\G, \Left \Rightarrow \Right, x \colon A \vljm B}{\G, \Left, x \le y, y \colon A \Rightarrow \Right, y \colon B}$

\vspace{4mm}

$\vliinf{\sil}{}{\G, \Left, x \le y, x \colon A \vljm B \Rightarrow \Right}{\G, \Left, x \le y, x \colon A \vljm B \Rightarrow \Right, y \colon A}{\G, \Left, x \le y, x \colon A \vljm B, y \colon B \Rightarrow \Right}$

\vspace{4mm}


\small $\vlderivation {\vlinf{\sbl}{}{\G, \Left, x \le y, y$R$z, x \colon \square A \Rightarrow \Right}{\G,\Left, x \le y, y$R$z, x \colon \square A, z \colon A \Rightarrow \Right}}$
\hspace{5mm} \small $\vlinf{\sbr}{$ $y, z$ fresh$}{\G, \Left \Rightarrow \Right, x \colon \square A}{\G, \Left, x \le y, y$R$z \Rightarrow \Right, z \colon A}$


\vspace{4mm}

$\vlinf{\sdl}{$ $y$ fresh $}{\G, \Left, x \colon \lozenge A \Rightarrow \Right}{\G, \Left, x$R$y, y \colon A \Rightarrow \Right}$
\hspace{5mm}$\vlinf{\sdr}{}{\G, \Left, x$R$y \Rightarrow \Right, x \colon \lozenge A}{\G, \Left, x$R$y \Rightarrow \Right, x \colon \lozenge A, y \colon A}$


\vspace{2mm}


\vspace{2mm}

$\vlinf{\refl}{}{\G, \Left \Rightarrow \Right}{\G, x\le x, \Left, \Right}$
\hspace{7mm} $\vlinf{\trans}{}{\G, x \le y, y \le z, \Left \Rightarrow \Right}{\G, x \le y, y \le z, x \le z, \Left \Rightarrow \Right}$


\vspace{2mm}


\footnotesize $\vlinf{\fone}{$ $u$ fresh$}{\G, \Left, x$R$y, y \le z \Rightarrow \Right}{\G, \Left, x$R$y, y \le z, x \le u, u$R$z \Rightarrow \Right}$
\hspace{3mm} $\vlinf{\ftwo}{u$ fresh$}{\G, \Left, x$R$y,x \le z \Rightarrow \Right}{\G, \Left, x$R$y, x \le z, y \le u, z$R$u \Rightarrow \Right }$

\end{center}

\caption{System lab$\heartsuit$IK}

\end{figure}



\newpage

\begin{thebibliography}{4}
\bibitem{Maffezioli}
Paolo Maffezioli, Alberto Naibo, and Sara Negri. \emph{The Church–Fitch knowability paradox in the light of structural proof theory}. Synthese, 190(14):2677–2716, 2013. 

\bibitem{Fitch}
Frederic B Fitch. \emph{Tree proofs in modal logic}. Journal of Symbolic Logic, 31(1):152, 1966.

\bibitem{Negri}
Sara Negri. \emph{Proof analysis in modal logics}. Journal of Philosophical Logic, 34:507–544, 2005. 

\bibitem{Simpson}
Alex Simpson. \emph{The Proof Theory and Semantics of Intuitionistic Modal Logic}. PhD thesis, University of Edinburgh, 1994. 

\bibitem{Vigano}
Luca Viganò. \emph{Labelled Non-Classical Logic}. Kluwer Academic Publisher, 2000. 

\bibitem{Marin}
Sonia Marin. \emph{Modal proof theory through a focused telescope}. PhD thesis, Université Paris-Saclay, 2017.

\end{thebibliography}

\end{document}
