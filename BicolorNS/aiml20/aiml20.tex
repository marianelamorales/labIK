\documentclass[twoside]{aiml20}

% Please include these macros
\usepackage{aiml20macro}

% Here you can include the standard packages you use.
% Try to avoid using non-standard packages.
% If you use a non-standard package you will have
% to submit it when you submit the final version of
% your paper.

\usepackage{graphicx}
\usepackage{amsmath}
\usepackage{amssymb,latexsym}
\usepackage{stmaryrd}
\usepackage{colonequals}

%%%%%%%%%%%%%%%%%%%%%%%%%%%%%%%%%%%%%%%%%%%%%%%%%%%%%%%%%
% Setting the correct page numbers                      
% Ignore the next two commented lines                   
% but please don't delete                               
%%%%%%%%%%%%%%%%%%%%%%%%%%%%%%%%%%%%%%%%%%%%%%%%%%%%%%%%%
%\input{../procnum.tex}
%\numbering{../aiml20db}{paper}

%%%%%%%%%%%%%%%%%%%%%%%%%%%%%%%%%%%%%%%%%%%%%%%%%%%%%%%%%%%%%%%
%%%%%%%%%%%%%%%%%%%%%%%%%%%%%%%%%%%%%%%%%%%%%%%%%%%%%%%%%%%%%%%
%%%% Logic
\newcommand*{\ax}[1]{\mathsf{#1}}
\newcommand*{\kax}[1][]		{\ax{k_{#1}}}
%\newcommand*{\A}{\mathcal{A}}
%
%\newcommand{\force}[2]{#1\Vdash#2}
%\newcommand{\nforce}[2]{#1\nVdash#2}
%%
%\newcommand{\height}[1]{|#1|}
\newcommand*{\rel}{R}
%
%%%%%%%%%%%%%%%%%%%%%%%%%%%%%%%%%%%%%%%%%%%%%%%%%%%%%%%%%%%%%%%
%%%% Connectives
\newcommand*{\NOT}{\neg}
\newcommand*{\AND}{\mathbin{\wedge}}
\newcommand*{\TOP}{\mathord{\top}}
\newcommand*{\OR}{\mathbin{\vee}}
\newcommand*{\BOT}{\mathord{\bot}}
\newcommand*{\IMP}{\mathbin{\scalebox{.9}{\raise.2ex\hbox{$\supset$}}}}

\newcommand*{\BOX}{\mathord{\Box}}
\newcommand*{\DIA}{\mathord{\Diamond}}

%%%%%%%%%%%%%%%%%%%%%%%%%%%%%%%%%%%%%%%%%%%%%%%%%%%%%%%%%%%%%%%
%%%%%%%%%%%%%%%%%%%%%%%%%%%%%%%%%%%%%%%%%%%%%%%%%%%%%%%%%%%%%%%
%%%% Systems
%\newcommand*{\K}{\mathsf{K}}
\newcommand*{\IK}{\mathsf{IK}}
%
\newcommand*{\lab}{\mathsf{lab}}
\newcommand*{\n}{\mathsf{n}}
%
\newcommand*{\labIKp}{\lab\IK_{\le}}
\newcommand*{\nIKp}{\n\IK_{\le}}
%
%%%%%%%%%%%%%%%%%%%%%%%%%%%%%%%%%%%%%%%%%%%%%%%%%%%%%%%%%%%%%%%
%%%%%%%%%%%%%%%%%%%%%%%%%%%%%%%%%%%%%%%%%%%%%%%%%%%%%%%%%%%%%%%
%%%% Nested sequents
%%%%%%%%%%%%%%%%%
%% intuitionistic annotations
\newcommand*{\rt}[1]{#1^\circ}
\newcommand*{\lf}[1]{#1^\bullet}
%
\newcommand*\mdelim[3]{%
	\mathopen{}\left#1%
	#3%
	\right#2\mathclose{}%
}
%
\makeatletter
%\newcommand*\BR[2][]{\mdelim{\lbrack\strut^{#1}}{\rbrack}{#2}}
\newcommand*{\BR}{%
	\@ifnextchar\i{\br@two}{%
		\@ifnextchar\bgroup{\br@one}{% 
}}}
\newcommand*{\br@one}[1]{%
	\def\br@{#1}%
	\mdelim{\lbrack}{\rbrack}{\ifx\br@\empty\mkern 3mu\else #1\fi}%
}
\newcommand*{\br@two}[3]{%
	\def\br@{#3}%
	\mdelim{\lbrack\strut^{#2}}{\rbrack}{\ifx\br@\empty\mkern 3mu\else #3\fi}%
}
%
\newcommand*{\bBR}{%
	\@ifnextchar\i{\bbr@two}{%
		\@ifnextchar\bgroup{\bbr@one}{% 
}}}
\newcommand*{\bbr@one}[1]{%
	\def\br@{#1}%
	\mdelim{\llbracket}{\rrbracket}{\ifx\br@\empty\mkern 3mu\else #1\fi}%
}
\newcommand*{\bbr@two}[3]{%
	\def\br@{#3}%
	\mdelim{\llbracket\strut^{#2}}{\rrbracket}{\ifx\br@\empty\mkern 3mu\else #3\fi}%
}
%
%%%%%%%%%%%%%%%%%%%%%%%%%%%%%%%%%%%%%%%%%%%%%%%%%%%%%%%%%%%%%%%
%% contexts
\makeatletter
\newcommand*{\cxs}{%
	\@ifnextchar\i{\cxs@two}{%
		\@ifnextchar\bgroup{\cxs@one}{% % \bgroup is the same as {
}}}
\newcommand*{\cxs@one}[1]{%
	\def\cxs@{#1}%
	\mdelim{\lbrace}{\rbrace}{\ifx\cxs@\empty\mkern 3mu\else #1\fi}%
	%\cxs@one@decor%
}
\newcommand*{\cxs@two}[3]{%
	\def\cxs@{#3}%
	\mdelim{\lbrace\strut^{#2}}{\rbrace}{\ifx\cxs@\empty\mkern 3mu\else #3\fi}%
	%\cxs@one@decor%
}
%\def\cxs@one@decor{%
%\@ifnextchar_{\cxs@one@sub}{%
%\@ifnextchar^{\cxs@one@sup}{%
%\@ifnextchar\dots{\@firstoftwo{\dotsm\cxs@one@decor}}{%
%	\@ifnextchar[{\cxs@one@arg}%]
%	\cxs}}}%
%}
%\def\cxs@one@sub_#1{_{#1}\cxs@one@decor}
%\def\cxs@one@sup^#1{^{#1}\cxs@one@decor}
%\def\cxs@one@arg[#1]{{#1}\cxs@one@decor}

%% Change these as needed, but keep the \cx@continuation at the end
%\def\cx@delete@always#1{{#1}^{\ast}\cx@continuation}
%\def\cx@delete@right#1*{{#1}^{\star}\cx@continuation}
%\def\cx@delete@focus#1\f{{#1}^{\scriptscriptstyle\not{\langle}{\rangle}}\cx@continuation}
%
%\def\cx@delete@star#1*{%
%\@ifnextchar*{\cx@delete@right{#1}}{\cx@delete@always{#1}}%
%}

\newcommand*{\@makecontextual}[2]{
	\newcommand*{#1}{%
		\@ifnextchar*{\cx@delete@star{#2}}{%
			\@ifnextchar\f{\cx@delete@focus{#2}}{%
				#2\cx@continuation}}%
	}%
}
\newcommand*{\cx@continuation}[1][]{_{#1}\cxs}

\@makecontextual{\Cx}{\Gamma}
\makeatother
%%%%%%%%%%%%%%%%%%%%%%%%%%%%%%%%%%%%%%%%%%%%%%%%%%%%%%%%%%%%%%%
%%%% Rules
\newcommand*{\rn}[1]  {\ensuremath{\mathsf{#1}}}
%\newcommand*{\invr}[1]{#1^\mathsf{inv}}
%
\newcommand*{\rrn}[2][]  {\rn{#2}_\rn{R#1}}
\newcommand*{\lrn}[2][]  {\rn{#2}_\rn{L#1}}
%
%%%%%%%%%%%%%%%%%%%%%%%%%%%%%%%%%%%%%%%%%%%%%%%%%%%%%%%%%%%%%%%
%%%%%%%%%%%%%%%%%%%%%%%%%%%%%%%%%%%%%%%%%%%%%%%%%%%%%%%%%%%%%%%
%%%% Labelled sequents

\newcommand*{\lb}[1]{#1}%{{\color{blue}{#1}}}
\newcommand*{\fm}[1]{#1}%{{\color{notgreen}{#1}}}
%
\newcommand{\SEQ}{\Rightarrow}
\newcommand{\B}{\mathcal{B}}
\newcommand{\Left}{\mathcal{L}}
\newcommand{\Right}{\mathcal{R}}
%
\newcommand*{\labels}[2]{\lb{#1}\mathord{:}\fm{#2}}
\newcommand*{\accs}[2]{\lb{#1}R\lb{#2}}
\newcommand*{\futs}[2]{\lb{#1}\le{\lb{#2}}}
%
%%%%%%%%%%%%%%%%%%%%%%%%%%%%%%%%%%%%%%%%%%%%%%%%%%%%%%%%%%%%%%%
%%%% Labelled rules
\newcommand*{\labrn}[2][]  {\rn{#2}_{#1}}%^{\lab}}}
\newcommand*{\rlabrn}[2][]  {\rn{#2}_\rn{R#1}}%^\lab}}
\newcommand*{\llabrn}[2][]  {\rn{#2}_\rn{L#1}}%^\lab}}
%%
%%%%%%%%%%%%%%%%%%%%%%%%%%%%%%%%%%%%%%%%%%%%%%%%%%%%%%%%%

%The following line defines the page header consisting of the surnames of the authors.
% Please include only the last names! 
% Separate by commas except the last two surnames which are separated by an "and".
\def\lastname{Marin and Morales}

\begin{document}

\begin{frontmatter}
  \title{Fully structured proof theory for intuitionistic modal logics}
  \author{Sonia Marin}
  %\footnote{You can put your email address or grant acknowledgement as a footnote, if you wish.}
  \address{University College London, UK}
  \author{Marianela Morales}
  \address{LIX, \'Ecole Polytechnique  \&  Inria Saclay, France}
  
  \begin{abstract}
  We present a labelled sequent system and a nested sequent system for intuitionistic modal logics equipped with two relation symbols, one for the accessibility relation associated with the Kripke semantics for modal logics and one for the preorder relation associated with the Kripke semantics for intuitionistic logic. Both systems are in close correspondence with the birelational Kripke semantics for intuitionistic modal logic.
  \end{abstract}

  \begin{keyword}
  Nested sequents, Labelled sequents, Intuitionistic modal logic, Proof theory.
  \end{keyword}
 \end{frontmatter}

%
%\section{Instructions for authors}
%Important information pertaining to the preparation of your paper:
%\begin{itemize}
%  \item Please prepare your paper by editing this file (\verb|aiml20.tex|) as well as the bibliography file, \verb|aiml20.bib|.
%
%  \item Please do not use includegraphics on any PDF files not having all fonts embedded. If in doubt, please convert to JPG before including.
%
%  \item It is strictly prohibited to tamper with the text dimensions, including adding to the text height or width. The dimensions of the style files are already taken to the limit of what the printer of the final proceedings is able to handle. Also, please refrain from using any kind of negative white space. 
%
%  \item Please make sure too split lines that are too wide in the final output (avoid overfull \verb|\hbox|es).
%\end{itemize}


\section{Introduction}
Structural proof theoretic accounts of modal logic can adopt the paradigm of \emph{labelled deduction}, in the form of e.g.~labelled sequent systems~\cite{vigano2000,negri2005}, or the one of \emph{unlabelled deduction}, in the form of e.g.~nested sequent systems~\cite{brunnler2009,poggiolesi2009}.

These generalisations of the sequent framework, inspired by relational semantics, are needed to treat modalities uniformly. 
%
By extending the ordinary sequent structure with \emph{one} extra element, either relational atoms between labels or nested bracketing, they encode respectively graphs or trees in the sequents, giving them enough power to represent modalities. 

Similarly, proof systems have been designed for \emph{intuitionistic} modal logic both as labelled~\cite{simpson1994} and as nested~\cite{strassburger2013,kuznets:strassburger:maehara,galmiche2018} sequent systems.
%
Surprisingly, in nested and labelled sequents, extending the sequent structure with the same \emph{one} extra element is enough to obtain sound and complete systems.
%

This no longer matches the relational semantics of these logics, which requires to combine the relation for intuitionistic propositional logic and the one for modal logic. 
%
More importantly, it leads to deductive systems that are not entirely satisfactory; they cannot as modularly capture axiomatic extensions (or equivalently, restricted semantical conditions) and, in particular, can only provide decision procedures for a handful of them~\cite{simpson1994}.

This lead us to develop a \emph{fully structured} approach to intuitionistic modal proof theory capturing \emph{both} the modal accesibility relation and the intuitionistic preorder relation. 
%
A fully labelled framework, described succintly in Section~\ref{sec:labelled}, has already allowed us to obtain modular systems for all intuitionistic Scott-Lemmon logics~\cite{marin:morales:strassburger:hal}. 
%
In an attempt to make this system amenable for proof-search and decision procedures, we have started investigated a fully nested framework, presented in Section~\ref{sec:nested}.
%
We would be particularly interested in a suitable system for logic $\mathsf{IS4}$, whose decidability is not known; we discuss this direction in Section~\ref{sec:extensions}.

%We present a labelled sequent system and a nested sequent system for intuitionistic modal logics capturing both the modal accesibility relation and the intuitionistic preorder relation. 
%
%In the labelled calculus we made explicity both relations with $\rel$ and $\le$ into the system. 
%
%In the nested sequent system, we represent this relations with one bracket in a sequent (for the accesibility relation) and a with two brackets in a sequent (preorder relation).  
%


\section{Intuitionistic modal logic}
The language of {intuitionisitic modal logic} is the one of intuitionistic propositional logic with the modal operators $\BOX$ and $\DIA$. %standing most generally for \emph{necessity} and \emph{possibility}.
%
Starting with a set $\mathcal{A}$ of atomic propositions, denoted $\fm a$, modal formulas are constructed from the grammar:\vspace*{-.1cm}
%
$$
\fm A \coloncolonequals
\fm a \mid \fm{\BOT} \mid \fm{(A \AND A)} \mid \fm{(A \OR A)} \mid \fm{(A \IMP A)} \mid \fm{\BOX A} \mid \fm{\DIA A}
$$
%
The axiomatisation of intuitionistic modal logic $\IK$ \cite{plotkin1986,fischer1984}
%, and by Ewald~\cite{Ewald} in the case of intuitionistic tense logic. 
%
is obtained from intuitionistic propositional logic by adding:
\begin{itemize}
	\item the \emph{necessitation rule}: $\fm{\BOX A}$ is a theorem if $\fm{A}$ is a theorem; and
	\item the following five variants of the \emph{distributivity axiom}:\vspace*{-.1cm}
	\begin{equation*}
	\label{eq:ik}%\hskip-2em
	\begin{array}[t]{r@{\;}l@{\quad}r@{\;}l@{\quad}r@{\;}l}
	\kax[1]\colon&\fm{\BOX(A\IMP B)\IMP(\BOX A\IMP\BOX B)}
	&
	\kax[3]\colon&\fm{\DIA(A\OR B)\IMP(\DIA A\OR\DIA B)}
	&
	\kax[5]\colon&\fm{\DIA\BOT\IMP\BOT}
	\\
	\kax[2]\colon&\fm{\BOX(A\IMP B)\IMP(\DIA A\IMP\DIA B)}
	&
	\kax[4]\colon&\fm{(\DIA A\IMP \BOX B)\IMP\BOX(A\IMP B)}\\%x[1ex]
	\end{array}
	\end{equation*}
\end{itemize}

%The relational semantics for intuitionistic modal logic combines the Kripke semantics for intuitionistic propositional logic and the one for classical modal logic. % using two distinct relations on the set of worlds.


\begin{definition}
	A \emph{bi-relational frame} consists %is a triple $\langle W, R, \le \rangle$ 
	%	of a non-empty set of worlds $W$ equipped with two binary relations $R$ and $\le$, where $R$ being the modal \emph{accessibility relation} and $\le$ a preorder (\emph{i.e.} a reflexive and transitive relation), satisfying the following conditions:
	of a set of worlds $W$ equipped with an {accessibility relation} $\rel$ and a preorder $\le$ satisfying:
	\begin{enumerate}
		\item[($F_1$)] For $\lb x, \lb y, \lb z \in W$, if $\accs xy$ and $\futs yz$, there exists $\lb u$ s.t.~$\futs xu$ and $\accs uz$.
		
		\item[($F_2$)] For $\lb x, \lb y, \lb z \in W$, if $\futs xy$ and $\accs xz$, there exists $\lb u$ s.t.~$\accs yu$ and $\futs zu$.
	\end{enumerate}
	%	
\end{definition}

\begin{definition}
	A \emph{bi-relational model} is %a quadruple $\langle W, R,\le,V \rangle$ with $\langle W, R, \le \rangle$ 
	a bi-relational frame with a monotone valuation function $V\colon W \to 2^\mathcal{A}$.%, that is, a function mapping each world $\lb w$ to the subset of propositional atoms true at $\lb w$, additionally subject to: if $\futs{w}{w'}$ then $V(\lb w)\subseteq V(\lb{w'})$.
\end{definition}

We write $\lb x \Vdash \fm a$ if $\fm a \in V(\lb x)$ and, by definition, it is never the case that $\lb x \Vdash \fm{\BOT}$. %and always have $\lb w \Vdash \fm{\TOP}$
%
The relation $\Vdash$ is extended to all formulas by induction, following the rules for both intuitionistic and modal Kripke models:\vspace*{-.1cm}
\begin{align}
	&\lb x \Vdash \fm{A \AND B} && \text{iff}\qquad 
	\lb x \Vdash \fm A \text{ and } \lb x \Vdash \fm B
	\nonumber\\
	&\lb x \Vdash \fm{A \OR B} && \text{iff}\qquad 
	\lb x \Vdash \fm A \text{ or } \lb x \Vdash \fm B
	\nonumber\\
	&\lb x \Vdash \fm{A \IMP B} && \text{iff}\qquad 
	\text{for all } \lb y \text{ with } \futs xy\text{, if }\lb y \Vdash \fm A\text{ then }\lb y \Vdash \fm B
	\nonumber\\
	&\lb x \Vdash \fm{\BOX A} && \text{iff}\qquad
	\text{for all }\lb y\text{ and }\lb z\text{ with }\futs xy\text{ and }\accs yz, \lb z \Vdash \fm A %\hfill $(\ast)$
	\label{eq:box}\\
	&\lb x \Vdash \fm{\DIA A} && \text{iff}\qquad
	\text{there exists a }\lb y\text{ such that }\accs xy\text{ and }\lb y \Vdash \fm A
	\nonumber
\end{align}

\begin{definition}
%	A formula $\fm A$ is \emph{satisfied} in a model $\langle W, R, \le, V \rangle$, if for all $\lb w \in W$ we have $\lb w \Vdash \fm A$.
	%
	A formula $\fm A$ is \emph{valid} in a frame $\langle W, R, \le \rangle$, if for all monotone valuations $V$ and for all $\lb w \in W$, we have $\lb w \Vdash \fm A$
\end{definition}

%Similarly to the classical case, in the case of $\IK$, the correspondence between syntax and semantics is recovered.

\begin{theorem}[\cite{fischer1984,plotkin1986}]\label{thm:plotkin}
	A formula $\fm A$ is a theorem of $\IK$ if and only if $\fm A$ is valid in every bi-relational frame.
\end{theorem}

\section{Fully labelled sequent calculus}\label{sec:labelled}

\begin{figure}
	\centering
	\fbox{
		\includegraphics[scale=.9]{fig-labIK}
	}		
	\caption{System $\labIKp$}
	\label{fig:labIKp}
\end{figure}

%Simpson~\cite{Simpson} followed the lines of Gentzen in a labelled context, namely, he developed a labelled natural deduction framework for modal logics and then converted it into sequent systems with the consequent restriction to one formula on the right-hand side of each sequent.
%%
%This worked as well in the labelled setting as in the ordinary sequent case: we followed Simpson's sequent system where intuitionistic labelled sequents are written $\B, \Left \SEQ \labels{z}{C}$ for some multiset of labelled formulas $\Left$, some formula $C$, some label $z$ and a set of relational atoms $\B$. 

%\emph{Labelled deduction} has been proposed by Gabbay~\cite{gabbay1996} in the 80s as a unifying framework throughout proof theory in order to provide proof systems for a wide range of logics. 
%
%For modal logics it can take the form of labelled natural deduction and labelled sequent systems as used, for example, by Simpson~\cite{simpson1994}, Vigan\`o~\cite{vigano2000}, and Negri~\cite{negri2005}. 
%
%Simpson's~\cite{simpson1994} labelled sequent system for intuitionistic modal logic is representing explicitly only the accessibility relation $\rel$ in the syntax.
%
Echoing the definition of bi-relational structures, we consider an extension of labelled deduction to the intuitionistic setting
%
that uses two sorts of relational atoms, one for the modal relation $\rel$ and another one for the intuitionistic relation $\leq$ (similarly to~\cite{maffezioli2013} for epistemic logic). 
%


\begin{definition}
	A two-sided intuitionistic \emph{fully labelled sequent} is of the form $\B, \Left \SEQ \Right$ where $\B$ denotes a set of relational atoms $\accs xy$ and preorder atoms $\futs xy$, and $\Left$ and $\Right$ are multi-sets of labelled formulas $\labels{x}{A}$ (for $\lb x$ and $\lb y$ taken from a countable set of labels and $A$ an intuitionistic modal formula).
\end{definition}




%Labelled sequents are formed from by labelled formulas of the form $\labels{x}{A}$ and relational or equality atoms of the form $x \rel y$ or $x = y $ respectively, where $x$,$y$ range over  a set of variables (called labels) and $A$ is a modal formula. A (onde-sided) \emph{labelled sequent} is then of the form $\B \SEQ \Right$ where $\B$ denotes a set of relational or equality atoms, and $\Right$ a multiset of labelled formulas.


%
We obtain a proof system $\labIKp$, displayed on Figure~\ref{fig:labIKp}, for intuitionistic modal logic in this formalism. 
%
Most rules are similar to the ones of Simpson~\cite{simpson1994}, but some are more explicitly in correspondence with the semantics by using the preorder atoms. 
%
For instance, the rules introducing the $\BOX$-connective correspond to~\eqref{eq:box}.
%
Furthermore, our system must incorporate the conditions ($F_1$) and ($F_2$) into the deductive rules $\rn{F_1}$ and $\rn{F_2}$, and rules $\rn{refl_\le}$ and $\rn{trans_\le}$ are necessary to ensure that the preorder atoms behave as a preorder on labels.
%
%As we mentioned, we obtain a proof system $\labIKp$ which allows us to give an extension of labelled deduction to the intuitionistic world and then we prove the next theorem:

%\begin{theorem}
%	\label{thm:sound-compl}
%	A formula $A$ is provable in the calculus $\labIKp$ if and only if $A$ is valid in every bi-relational frame.
%\end{theorem}
%
%On the one hand, we prove directly that each rule from our system is sound wrt.~bi-relational structures.
%
%On the other hand, we show that $\labIKp$ is complete wrt.~Simpson's $\lab\IK$, and the theorem then follows from Theorem~\ref{thm:simpson-sound-compl}. 

%Finally we can prove the following theorem ensuring soundness and (cut-free) completeness of $\labIKp$.

\begin{theorem}\label{thm:cutfree-compl}
	%	Let $\CC$ be a set of geometric frame properties as in~\eqref{eq:cla-geometric} and $\labbrn{\CC}$ be the corresponding set of rules following schema~\eqref{eq:modal-grs}.
	%
	For any formula $A$, the following are equivalent.
	%
	\begin{enumerate}
		\item\label{i} $A$ is a theorem of $\IK$ 
		%
		\item\label{ii} $A$ is provable in $\labIKp +\labrn{cut}$ with \quad
		%		
		\smash{\parbox[c]{5cm}{\includegraphics[scale=0.8]{cut}}}
		%
		\item\label{iii} $A$ is provable in $\labIKp$
		%
		%\item\label{iv} $A$ is valid in every birelational frames %satisfying the properties in $\CC$.
	\end{enumerate}
\end{theorem}

The proof is a careful adaptation of standard techniques (see~\cite{marin:morales:strassburger:hal} for details).


%%%%%%%%%%%%%%%%%%%%%%%%%%%%%%%%%%%%%%%%%%%%%%%%%%%%%%%%%
%%%%%%%%%%%%%%%%%%%%%%%%%%%%%%%%%%%%%%%%%%%%%%%%%%%%%%%%%
%%%%%%%%%%%%%%%%%%%%%%%%%%%%%%%%%%%%%%%%%%%%%%%%%%%%%%%%%


\section{Fully nested sequent calculus}\label{sec:nested}

\begin{figure}
	\centering
	\fbox{
		\includegraphics[scale=.9]{fig-nIK}
	}		
\caption{System $\nIKp$}
\label{fig:nIK}
\end{figure}

%Nested sequents were introduced by Kashima \cite{kashima1994cut} and then independently rediscovered by Br\"{u}nnler \cite{brunnler2009deep} and Poggiolesi \cite{poggiolesi2009method}.
%
%Nested sequents are a generalisation of sequents from a multiset of formulas to a tree of multisets of formulas.
%
In standard nested sequent notation, brackets $\BR{\cdot}$ are used to indicate the parent-child relation in the modal accessibility tree.
%, and can be interpreted as the modal $\BOX$ (similarly to how the comma is interpreted as $\OR$). 
%
$\lf{(\cdot)}$ and $\rt{(\cdot)}$ annotations are used to indicate that the formulas would occur on the left-hand-side or right-hand-side of a sequent respectively in the absence of the sequent arrow.  

To make it fully structured again, we enhance the structure with a second type of bracketting $\bBR{\cdot}$ to encode the preorder relation.
%For intuitionistic logic we need two-sided sequents, which are formally generated by:
%$\Cx \coloncolonequals \lf{A_{1}}, ..., \lf{A_{n}}, \rt{B_{1}},..., \rt{B_{k}}, \BR{\Cx[1]},..., \BR{\Cx[m]}$
%where $\fm{A_{1}}, . . . , \fm{A_{n}}$ are the formulas that would occur on the left of the turnstile if there was a turnstile, $\fm{B_{1}}, . . . , \fm{B_{k}}$ are the formulas that would occur on the right of the turnstile if there was a turnstile, and $\Cx[1], . . . , \Cx[m]$ are nested sequents. We use the $\bullet$ and $\circ$ as polarities indication for formulas.
%

\begin{definition}
	A two-sided intuitionistic \emph{fully nested sequent} is constructed from the grammar:
	$\Cx \coloncolonequals \varnothing \mid \lf A, \Cx \mid \rt A, \Cx \mid \BR{\Cx} \mid \bBR{\Cx}$
\end{definition}

The obtained nested sequent calculus $\nIKp$ is displayed in Figure~\ref{fig:nIK}.
%
The idea is similar to the fully labelled calculus but the shift of paradigm forced us to make different design choices.
%
In particular, the underlying tree-structure prevents us to express the rule $\rn{F_2}$, but its absence is compensated by the \emph{monotonicity rules} $\lrn{mon}$ and $\rrn{mon}$ which were admissible in $\labIKp$.
%
Another beneficial result of this addition is that rules $\rn{refl_\le}$ and $\rn{trans_\le}$ do not need any equivalent here.

%%%%%%%%%%%%%%%%%%%%%%%%%%%%%%%%%%%%%%%%%%%%%%%%%%%%%%%%%
%%%%%%%%%%%%%%%%%%%%%%%%%%%%%%%%%%%%%%%%%%%%%%%%%%%%%%%%%

\section{Extensions: example of transitivity}\label{sec:extensions}

As mentionned in the introduction, one of our motivation is to investigate decision procedure for axiomatic extensions of $\IK$, for instance for $\mathsf{IS4}$, intuitionistic logic of reflexive transitive frames.
%
We will therefore illustrate the benefit of our approach taking transitivity as a test-case.

The frame condition of transitivity ($\forall \lb {xyz}.\:\accs xy\AND \accs yz\IMP\accs xz$) can be axiomatised by the conjunction of the two versions of the $\ax 4$-axiom:\vspace*{-.2cm}
$$\small
\ax{4_{\BOX}}\colon\fm{\BOX A\IMP \BOX\BOX A}
\qquad\qquad\qquad
\ax{4_{\DIA}}\colon\fm{\DIA\DIA A\IMP\DIA A}
$$
%
which are equivalent in classical modal logic. 
%
However, in intuitionistic modal logic they are not and they can be added independently to the logic $\IK$. %(an intuitionistic version of $\K$). 
%
From~\cite{plotkin1986} we know they are in correspondance with the following frame conditions:\vspace*{-.2cm}
{\small
\begin{equation}
\label{eq:4frame}
\forall\lb{xyz}.\:\accs xy\AND \accs yz\IMP(\exists\lb{u}.\:\futs x{u}\AND\accs {u}z)
\quad
\forall\lb{xyz}.\:\accs xy\AND \accs yz\IMP(\exists\lb{u}.\:\futs z{u}\AND\accs {x}{u})
\end{equation}
}

Following Simpson~\cite{simpson1994} we could extend our basic sequent system for $\IK$ to $\mathsf{IK4} = \IK+(\ax{4_{\BOX}}\AND\ax{4_{\DIA}})$ with the rule%\vspace*{-.2cm}

\begin{center}
\includegraphics[scale=.9]{fig-trans}
\end{center}

Incorporating the preorder symbol into the syntax, however,  allowed us~\cite{marin:morales:strassburger:hal} to 
translate the conditions in~\eqref{eq:4frame} into separate inference rules for $\ax{4_{\BOX}}$ and $\ax{4_{\DIA}}$:%\vspace*{-.1cm}

\begin{center}
	\includegraphics[scale=.9]{fig-trans2}
\end{center}

A great interest of a nested system is that the rules for axiomatic extensions are usually more suitable for proof-search procedures that in the labelled setting.
%
Previous nested systems for intuitionistic modal logics~\cite{strassburger2013,kuznets:strassburger:maehara} can be extended from $\IK$ to $\mathsf{IK4}$ by simply adding the following rules:%\vspace*{-.2cm}

\begin{center}
	\includegraphics[scale = 0.8]{fig-trans3}
\end{center}

%The generalisation of these results to our fully structured framework is subject of ongoing study.
%
These extensions for $\labIKp$ are already proven~\cite{marin:morales:strassburger:hal} but these results for our fully nested sequent system are subject of ongoing study. 

%% Appendix.
%% Remove the \Appendix command if an 
%% appendix is not required.
%\Appendix


%% Bibliography
%% Make sure to use the bibliographystyle aiml20.
%\bibliographystyle{aiml20}
%\bibliography{aiml20}

\begin{thebibliography}{}
	\bibitem{brunnler2009}Br{\"u}nnler, K., \emph{Deep sequent systems for modal logic}, Archive for Mathematical Logic~\textbf{48} (2009), pp. 551--577.
	\bibitem{fischer1984}Fischer Servi, G., \emph{Axiomatizations for some intuitionistic modal logics}, Rend. Sem. Mat. Univers. Politecn. Torino~\textbf{42} (1984), pp. 179--194.
	\bibitem{galmiche2018}Galmiche, D. and Salhi, Y., \emph{Tree-sequent calculi and decision procedures for intuitionistic modal logics}, Journal of Logic and Computation~\textbf{28} (2018), pp. 967--989.
	\bibitem{kuznets:strassburger:maehara} Kuznets, R. and Stra{\ss}burger, L., \emph{Maehara-style modal nested calculi}, Archive for Mathematical Logic~\textbf{58} (2018), pp. 359--385.
	\bibitem{maffezioli2013} Maffezioli, P. and Naibo, A. and Negri, S.,\emph{The {C}hurch--{F}itch knowability paradox in the light of structural proof theory}, Synthese~\textbf{190} (2013), pp. 2677--2716.
	\bibitem{marin:morales:strassburger:hal} Marin, S., Morales, M. and Stra{\ss}burger, L., \emph{A fully labelled proof system for intuitionistic modal logics} (2019), working paper or preprint.
	\bibitem{negri2005} Negri, S., \emph{Proof analysis in modal logic}, Journal of Philosophical Logic~\textbf{34} (2005), pp. 107--128.
	\bibitem{plotkin1986} Plotkin, G. and Stirling, C., \emph{A framework for intuitionistic modal logics}, Proceedings of the 1st Conference on Theoretical Aspects of Reasoning about Knowledge (TARK), (1986), pp. 399--406.
	\bibitem{poggiolesi2009} Poggiolesi, F., \emph{The method of tree-hypersequents for modal propositional logic}, Towards mathematical philosophy (2009), pp. 31--51.
	\bibitem{simpson1994} Simpson, A. K., \emph{The proof theory and semantics of intuitionistic modal logic}, Ph.D. thesis, University of Edinburgh. College of Science and Engineering (1994).
	\bibitem{strassburger2013} Stra{\ss}burger, L., \emph{Cut Elimination in Nested Sequents for Intuitionistic Modal Logics}, FoSSaCS'13,LNCS~\textbf{7794} (2013), pp. 209--224.
	\bibitem{vigano2000}Vigan\`o, L., \emph{Labelled non-classical logics}, Kluwer Academic Publisher (2000).
\end{thebibliography}

\end{document}
