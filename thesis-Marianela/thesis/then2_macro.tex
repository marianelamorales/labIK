%%%%%%%%THEN-2 MACRO%%%%%%%%%%%%
\newcommand{\thentwo}{\begin{center}	
		$\vlderivation {
			\vlin{\sir}
			{}
			{\Rightarrow x \colon (A \vljm (B \vljm C)) \vljm ((A \vljm B)\vljm (A \vljm C))}
			{\vlin {\sir}
				{}
				{x \le y, y \colon A \vljm (B \vljm C)\Rightarrow y \colon (A \vljm B) \vljm (A \vljm C)}
				{\vlin {\sir}
					{}
					{x \le y, y \le z, y \colon A \vljm (B \vljm C), z \colon A \vljm B \Rightarrow z \colon A \vljm C}
					{\vlin {\trans}
						{}
						{x \le y, y \le z, z \le w, y \colon A \vljm (B \vljm C), z \colon A \vljm B, w \colon A \Rightarrow w \colon C}
						{\vliin {\sil}
							{}
							{x \le y, y \le z, z \le w, y \le w, y \colon A \vljm (B \vljm C), z \colon A \vljm B, w \colon A \Rightarrow w \colon C}
							{\vlhy{\circledast \hspace{75mm}}}
							{\vlhy{\star}}}}}
			}}$
			
		\end{center}
		
		\bigskip
		
		La derivación, luego de aplicar la regla izquierda de la implicación $\sil$, genera dos premisas. La premisa que continua a partir $\circledast$ es de la siguiente forma:
		
		\begin{center}
			
			$\vlderivation{
				\vlin{\sil}{}{\circledast}
				{\vlin {\refl}
					{}
					{x \le y, y \le z, z \le w, y \le w, y \colon A \vljm (B \vljm C), z \colon A \vljm B, w \colon A \Rightarrow w \colon C, w \colon A}
					{\vlin {\ids}
						{}
						{x \le y, y \le z, z \le w, y \le w, w \le w, y \colon A \vljm (B \vljm C), z \colon A \vljm B, w \colon A \Rightarrow w \colon C, w \colon A}
						{\vlhy {}}}}	
			}$
			
		\end{center}
		
		Mientras que la premisa que continua en $\star$ es de esta forma:
		
		\begin{center}
			\scalebox{0.96}{
		$\vlderivation{
			\vlin{\sil}{}{\star}
			{\vlin {\trans}
				{}
				{x \le y, y \le z, z \le w, y \le w, y \colon A \vljm (B \vljm C), z \colon A \vljm B, w \colon A, w \colon B \vljm C \Rightarrow w \colon C}
				{\vliin {\sil}
					{}
					{x \le y, y \le z, z \le w, y \le w, w \le w, y \colon A \vljm (B \vljm C), z \colon A \vljm B, w \colon A, w \colon B \vljm C \Rightarrow w \colon C}
					{\vlhy{\heartsuit \hspace{75mm}}}
					{\vlhy{\clubsuit}}}}
		}$}
		
	\end{center}
	
		Vemos que en $\star$ al volver aplicar la regla de implicación izquierda $\sil$, genera nuevamente dos premisas que las representamos con $\heartsuit$ y $\clubsuit$ para continuar con nuestra derivación.
		
		Por un lado, la derivación correspondiente a $\clubsuit$ está dada por:
		
		\begin{center}
			\scalebox{0.94}{
		$\vlderivation{
			\vlin{\sil}{}{\clubsuit}{
				\vlin {\ids}
				{}
				{x \le y, y \le z, z \le w, y \le w, w \le w, y \colon A \vljm (B \vljm C), z \colon A \vljm B, w \colon A, w \colon B \vljm C, w \colon C \Rightarrow w \colon C}
				{\vlhy {}}}
		}$}
	\end{center}
	
		Por otro lado, de $\heartsuit$ obtenemos la siguiente derivación que de nuevo utiliza $\sil$. Es por ello que volvimos a representar las premisas que se obtienen como resultado de esta regla con $\spadesuit$ (para la premisa que parte del lado izquierdo) y $\dagger$ (para la premisa que parte del lado derecho):
		
		\begin{center}
			\scalebox{0.94}{
		$\vlderivation{
			\vlin{\sil}{}{\heartsuit}
			{\vliin {\sil}
				{}
				{x \le y, y \le z, z \le w, y \le w, w \le w, y \colon A \vljm (B \vljm C), z \colon A \vljm B, w \colon A, w \colon B \vljm C \Rightarrow w \colon C, w \colon B}
				{\vlhy{\spadesuit \hspace{105mm}}}
				{\vlhy{\dagger}}}
		}$}
	\end{center}
		
	
		
		Finalmente por $\spadesuit$ concluye la derivación de esta rama de la siguiente forma:
		
		\begin{center}
		\scalebox{0.90}{
			$\vlderivation{
				\vlin{\sil}{}{\spadesuit}
				{\vlin {\ids}
					{}
					{x \le y, y \le z, z \le w, y \le w, w \le w, y \colon A \vljm (B \vljm C), z \colon A \vljm B, w \colon A, w \colon B \vljm C \Rightarrow w \colon C, w \colon B, w \colon A}
					{\vlhy {}}}}$
		}
	\end{center}
	
		Y por $\dagger$, la derivación concluye de esta otra manera:
		
		\begin{center}
		
		\scalebox{0.90}{
			$\vlderivation{
				\vlin{\sil}{}{\dagger}
				{\vlin {\ids}
					{}
					{x \le y, y \le z, z \le w, y \le w, w \le w, y \colon A \vljm (B \vljm C), z \colon A \vljm B, w \colon A, w \colon B \vljm C, w \colon B \Rightarrow w \colon C, w \colon B}
					{\vlhy {}}}
			}$
		}
	\end{center}
		}