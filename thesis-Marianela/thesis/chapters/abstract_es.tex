
\paragraph{Resumen}

El estudio de la lógica modal \cite{blackburn01}, desde Aristóteles, proviene del deseo de analizar ciertos argumentos filosóficos y así poder establecer la verdad de una proposición: por ejemplo, una proposición puede ser falsa ahora pero verdadera más tarde, o por el contrario necesariamente verdadera, y así sucesivamente. Lo que llamamos lógica modal describe el comportamiento abstracto de las modalidades $\square$ y $\Diamond$, pero abarca una amplia cantidad de modalidades 'reales' en las expresiones ling\"{u}ísticas: tiempo, necesidad, posibilidad, obligación, conocimiento, creencia, etc. Semánticamente, los operadores modales y más precisamente el $\Diamond$, nos permite describir propiedades de los diferentes estados que podemos alcanzar a partir de un punto de evalación.

Desde que la teoría de modelos modal comenzó a desarrollarse, existe una tendencia a utilizar métodos basados en la teoría de modelos en lugar de aquellos basados en la teoría de prueba ya que los sistemas de prueba clásicos eran generalmente insuficientes. Sin embargo, en los últimos años, nuevas técnicas fueron desarrolladas y los sistemas de prueba comenzaron a ser mayormente utilizados ya que proveen ciertas ventajas cuando se trabaja con el análisis y estandarización de pruebas. Para tener más detalles sobre esta dicotomía vease \cite{negri2005}. La deducción etiquetada propuesta por Gabbay \cite{gabbay1996} en los 80's se presentó como un marco unificador de la teoría de prueba para proporcionar sistemas de prueba para una amplia gama de lógicas. Para las lógicas modales, también puede tomar la forma de deducción natural etiquetada y sistemas de secuentes etiquetados como los de Simpson \cite{simpson1994}, Vigano \cite{vigano2013} y Negri \cite{negri2005}. Estos formalismos hacen un uso explícito no sólo de las etiquetas sino también de los átomos relacionales que hacen referencia a la relación de accesibilidad con un modelo de Kripke \cite{kripke1959}.

Continuaremos con la elección de una presentación de secuentes. Más precisamente, en este trabajo, el objetivo principal consiste en desarrollar un sistema de secuentes etiquetados para la lógica modal intuicionista, que involucra dos símbolos de relación: uno para la relación de accesibilidad asociada con la semántica de Kripke para lógicas modales clásicas, y otra relación de pre-orden asociada a la semántica de Kripke para la lógica intuicionista. Para obtener este resultado, utilizamos un cálculo de secuentes etiquetado propuesto por Negri \cite{negri2005} para lógicas modales clásicas y lo extendimos con una relación de pre-orden. Esto permite tener un sistema etiquetado en estrecha correspondencia con los modelos bi-relacionales de Kripke.