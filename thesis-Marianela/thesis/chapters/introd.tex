\renewcommand{\chaptername}{}
\chapter{Introducción}
%Las lógicas modales clásicas son extensiones de la lógica clásica con nuevos operadores llamados $modalidades$.
Existen al menos dos formas de caracterizar todas las verdades o teoremas de una lógica: por un lado, el formato conocido como \emph{axiomatización a la Hilbert} y por otra parte, el formato que sigue el \emph{estilo de Gentzen}, el cual utiliza reglas de secuentes. 
En este trabajo final, nuestra contribución se basa en un cálculo de secuentes para lógicas modales intuicionistas. El resultado se obtiene a partir de extender un cálculo para lógicas modales clásicas con una relación de pre-orden ($\le$) (además de la relación de accesibilidad que nos provee la lógica modal básica) con el objetivo de poder capturar el comportamiento intuicionista. Desde un punto de vista de la lógica modal, esencialmente la lógica modal intuicionista es simplemente una lógica con dos modalidades, donde una de ellas es interpretada en una relación de pre-orden. En esta introducción, haremos un breve recorrido por la historia de la \emph{teoría de prueba} y su relación con las lógicas modales, así como también veremos qué aporta nuestro nuevo cálculo sobre otros.

\section{Teoría de prueba}
En el razonamiento matemático, existen los teoremas y sus demostraciones. Frecuentemente, existen distintas demostraciones para un mismo teorema y nos interesa estudiar y entender sus similitudes y sus diferencias. Por otro lado, es posible que las demostraciones de distintos teoremas compartan ciertos patrones, como por ejemplo, inducción o reducción por el absurdo. Parte de nuestro trabajo es entender cuáles son los contextos de aplicación de estas formas particulares de razonamiento y cómo hacerlo. Sin embargo, con el objetivo de comunicar demostraciones (es decir, presentar demostraciones a otras personas y que ellas las comprendan) los matemáticos usualmente utilizan lenguaje natural con algunos símbolos especiales. Con el objeto de describir de manera precisa las demostraciones, es necesario contar con un lenguaje puramente matemático. De esta manera nace una disciplina dedicada a estudiar las demostraciones como un objeto formal: la \textit{teoría de prueba} (en inglés \textit{proof theory}). Frege lideró esta disciplina sugiriendo que las pruebas deben ser consideradas como objetos de estudio matemático en 1879 en su libro \textit{Begriffschrift} \cite{frege1879}. Luego, Hilbert siguió las ideas de Frege proponiendo una definición de un sistema deductivo para formalizar el razonamiento \cite{hilbert1967}.

\textit{La Teoría de Prueba} puede ser considerada como uno de los cuatro pilares de la lógica matemática junto con la teoría de modelos, la teoría de conjuntos y la teoría de la recursión. Al diseñar diversos formalismos para ver las pruebas como objetos matemáticos, debemos entender sus propiedades a través de métodos matemáticos formales. Desde el punto de vista de las ciencias de la computación, una aplicación directa podría ser el desarrollo de algoritmos para demostrar teoremas automáticamente (demostradores automáticos), o verificar pruebas matemáticas (verificación/comprobación automática de teoremas). Otra aplicación podría ser extraer de una prueba un algoritmo; por ejemplo, si el teorema establece la existencia de un objeto, este objeto también podría construirse efectivamente; otro ejemplo podría ser el uso de las demostraciones fallidas para la construcción de contraejemplos. Por otro lado, desde un punto de vista más abstracto, podríamos tratar de entender qué axiomas se requieren para demostrar ciertos teoremas (matemática inversa), como así también comparar diferentes métodos de prueba y, en particular, los tamaños de las pruebas que producen (complejidad de la prueba).

Así, como mencionamos al comienzo, podríamos decir que las matemáticas están hechas de muchos lemas, algunos teoremas y de sus pruebas. Se podría decir que los matemáticos no demuestran un teorema desde cero, si no que construyen la prueba a partir de demostrar otras afirmaciones intermedias que son llamadas \emph{lemas}.
En un sistema deductivo \emph{a la Hilbert y Frege}, para probar un teorema T, es posible utilizar un lema dado L si, por un lado, podemos probar L y si, por otra parte, podemos inferir T de L. Este paso de razonamiento es llamado \emph{Modus Ponens}. El desafío radica en encontrar el lema apropiado.

Sin embargo, para las aplicaciones que mencionamos, cambiar a una teoría diferente del teorema que queremos demostrar puede no resultar tan sencillo, por lo que muchas veces preferimos una demostración sin lemas. Es importante aclarar que no todas las pruebas matemáticas pueden realizarse libres de lemas.

Fue así que Gentzen en su \emph{Untersuchungen {\"u}ber das logische Schlie{\ss}en. I} en 1934 \cite{gentzen1934} introdujo métodos teóricos de prueba para demostrar resultados en la lógica matemática. En particular, Gentzen fue quién desarrolló el \emph{cálculo de secuentes}, una representación alternativa de las pruebas que promueven las reglas de inferencia sobre los axiomas. El teorema de \emph{cut-elimination} o también conocido como \emph{Hauptsatz} declara que cualquier prueba en el cálculo de secuentes puramente lógica puede transformarse en una forma analítica normal. Esto quiere decir que en cálculo de secuentes, cualquier prueba puede ser realizada sin lemas. Para poder probar el \emph{Hauptsatz}, uno necesita mostrar que una regla llamada \emph{cut}, la cual reformula Modus Ponens en el cálculo de secuentes, es redundante en el sistema. Es decir, establece que cualquier prueba que tenga una demostración en el cálculo de secuentes utilizando \emph{cut}, también tiene una demostración sin esta regla.

Otra de las grandes contribuciones de Gentzen fue el estudio paralelo de las lógicas clásicas e intuicionistas. Las lógicas intuicionisticas surgen con Brouwer \cite{brouwer1920} como una formalización del razonamiento constructivista (la existencia de un objeto es equivalente a la posibilidad de su construcción), rechazando el \emph{Principio del Tercero Excluído} el cual establece que una proposición tiene que ser verdadera o falsa. Mientras que en la lógica intuicionista puede haber incertidumbre sobre si una proposición se mantiene o no, en la lógica clásica, el principio es aceptado.

En este trabajo, vale la pena destacar que por \emph{lógica intuicionista} nos referimos a la lógica intuicionista proposicional básica. A la hora de definir cálculos de prueba para esta lógica, una caracterísitica importante es que es posible obtener tales cálculos, simplemente restringiendo sintácticamente el cálculo de prueba para la lógica clásica correspondiente \cite{vandalen2004} (en nuestro caso, lógica proposicional).

El enfoque intuicionista fue utilizado sobre diferentes lógicas. En este trabajo, nos centraremos en el desarrollo de un cálculo de secuentes para una lógica modal intuicionista. Como se discutirá en las siguientes secciones, el enfoque modal se acopla adecuadamente, ya que una de las características principales es la definición y combinación de nuevos operadores, dando origen a nuevas familias de lógicas modales. Desde esta perspectiva, la componente intuicionista es simplemente una modalidad en el lenguaje.


\section{Lógicas modales y su teoría de prueba}

En sus comienzos, las lógicas modales surgen del deseo de analizar ciertos conceptos filosóficos como \emph{necesidad, creencia, conocimiento, obligación}, entre otros. Lo que llamamos \emph{lógicas modales} se obtienen como extensiones de lógicas clásicas (por ejemplo: lógica proposicional o lógica de predicados) describiendo el comportamiento de modalidades como $\square$ y $\Diamond$ que representan los conceptos a investigar. Más precisamente estos operadores permiten describir propiedades de los estados accesibles desde el punto de evaluación. Las lógicas modales más estudiadas son aquellas basadas en el razonamiento clásico. El interés en las versiones intuicionistas de las lógicas modales se produjo mucho más tarde y de dos fuentes diferentes: por un lado, los lógicos se interesaron por obtener versiones intuicionistas (es decir, utilizando razonamiento constructivista), y por otra parte, aplicaciones de las mismas.

Desde Aristóteles y con un gran desarrollo en la Edad Media se fundan las bases de la lógica modal. Su consolidación se produjo a fines de la década de 1950 y principios de la década de 1960 con el desarrollo de una semántica basada en los \emph{``mundos posibles"} introducida por Kripke \cite{kripke1959} (de allí el nombre de la semántica). Nos permite ver a la lógica modal como un lenguaje para grafos o bien como un lenguaje para describir procesos, es decir, ver a los elementos del grafo como un conjunto de estados computacionales y ver a las relaciones como acciones que transforman un estado a otro.

Como consecuencia de lo propuesto por Kripke, en términos de \emph{teoría de prueba}, algunas extensiones para secuentes fueron incorporadas beneficiando el manejo de las modalidades. Dos enfoques se destacaron dentro de estas extensiones: por un lado, sistemas que incorporan semántica relacional explícita en su formalismo, llamados \emph{sistemas etiquetados deductivos} como los \emph{cálculos etiquetados} \cite{negri2005} y el \emph{método de tablas semánticas} (mejor conocido en inglés como \emph{semantic tableaux} \cite{fitting1983}), y por otra parte, sistemas que utilizan dispositivos sintácticos para recuperar el lenguaje modal, llamados \emph{sistemas deductivos sin etiquetas} (por ejemplo: \emph{hypersequents} \cite{avron1996}, \emph{nested sequents} \cite{brunnler2009}). Así, los teóricos de prueba interesados en la lógica modal tienen a su disposición una amplia gama de diferentes formalismos de prueba para elegir, que nos permiten tener herramientas complementarias para poder explorar la teoría de prueba de la lógica modal. El trabajo de esta tesis nos muestra uno de estos formalismos nombrados, el de \emph{sistemas de pruebas etiquetados}, para analizar pruebas modales.

\section{Motivación}

%Como se nombró anteriormente, los teoremas de una lógica pueden caracterizarse a través de dos formatos: por un lado, \emph{axiomatización a la Hilbert} y por otro lado, a través de reglas de secuentes conocido como \emph{estilo Gentzen}. 
Nuestro trabajo se centra en sistemas de prueba etiquetados para lógicas modales intuicionistas usando dos tipos de relaciones de accesibilidad. Esto significa que hay un símbolo relacional para la relación de accesibilidad modal y una para la relación futura (o conocida como relación de pre-orden). Esta es una novedad ya que nos permite poner el sistema de prueba en estrecha relación con la semántica birelacional de Kripke.

Extendemos el sistema etiquetado propuesto por Negri para la lógica modal básica \cite{negri2005} con la relación de pre-orden que se nombró anteriormente (la relación de accesibilidad ya esta presente en el cálculo para la lógica clásica) para capturar lógicas modales intuicionistas. Para nuestro nuevo sistema, probamos tanto correctitud como completitud, así como también utilizamos el sistema etiquetado propuesto por Simpson \cite{simpson1994} para demostrar que nuestro sistema es un sistema libre de $\mathsf{cut}$. Por último y como trabajo a futuro, extendemos nuestro sistema agregando el axioma de Scott-Lemmon volviendo a demostrar correctitud y completitud para nuestro nuevo sistema (con el axioma agregado).


\section{Organización del trabajo}

Esta tesis se estructura de la siguiente manera:

En el \textbf{Capítulo \ref{cap:logclasica}} presentamos a la lógica modal básica con su sintaxis, semántica y sus distintos formatos de caracterización: presentamos una axiomatización a la Hilbert y una axiomatización a la Gentzen, en donde vemos en detalle el cálculo de secuentes etiquetado propuesto por Negri \cite{negri2005}.

En el \textbf{Capítulo \ref{cap:logintuicionista}} se presentan los conceptos de sintaxis y semántica para la lógica modal intuicionista como así también una axiomatización a la Hilbert.

En el \textbf{Capítulo \ref{cap:labIKh}} se plantea en detalle el problema principal que abordamos y se presenta en detalle el sistema de secuentes etiquetado propuesto para la lógica modal intuicionista (llamado $\labIKh$). Se ven en detalle cada una de las reglas de secuentes que conforman al sistema. En este capítulo también se desarrolla la demostración de que nuestro sistema es correcto.

En el \textbf{Capítulo \ref{cap:completeness}} se desarrolla una prueba de completitud sintáctica por medio del sistema a la Hilbert para el sistema $\labIKh$. Se realiza una demostración en la que se demuestran todos los axiomas de la lógica modal proposicional, se simula modus ponens y necesitación, y se prueban las variantes $\kaxiom_{1}$, ... $\kaxiom_{5}$ del axioma de distributividad $\kaxiom$ (presentados en detalle en el Capítulo 3). De esta prueba se obtiene un sistema completo con la regla de $\mathsf{cut}$ (utilizada para simular modus ponens). 

En el \textbf{Capítulo \ref{cap:future}} se presentan algunas líneas de trabajo futuro. Entre ellas, se continuará trabajando en una prueba de completitud utilizando el sistema de secuentes propuesto por Simpson \cite{simpson1994}. Y por otra parte, se presentan algunas extensiones del sistema $\labIKh$ para generar lógicas más fuertes. En particular, en el Capítulo 6 extendemos nuestro sistema con el axioma de Scott-Lemmon o también conocido como $\agklmn$.

Por último, en el \textbf{Capítulo \ref{cap:conclusion}} presenta un breve resumen de lo realizado a lo largo de este trabajo final de Licenciatura como así también se incluye una pequeña sección de la experiencia que tuve a lo largo de este trabajo.