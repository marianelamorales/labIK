\documentclass[a4paper,12pt,oneside,spanish]{book}

\usepackage[left=2.3cm,top=3cm,right=2.3cm,bottom=3cm]{geometry} 
\usepackage[noxy]{virginialake}
\usepackage{lmodern}
\usepackage{amssymb,amsmath,amsthm} 
\usepackage{mathtools}
\usepackage[spanish]{babel}
\usepackage[utf8]{inputenc} 
\usepackage{enumerate}
\usepackage{graphicx}
\usepackage{pgf}
\usepackage[all]{xy}
\usepackage{tikz}
\usepackage[hidelinks]{hyperref}
\usepackage[textwidth=0.89in,textsize=scriptsize]{todonotes}
\usetikzlibrary{babel} 
\usepackage{pgffor}
\usepackage{named}
\usetikzlibrary{arrows,automata}
\usetikzlibrary{quotes,arrows.meta}

\newcommand*{\ax}[1]{\mathsf{#1}}
\newcommand*{\kax}[1][]		{\ax{k_{#1}}}
\newcommand*{\pax}[2]		{\ax{p_{#1#2}}}
\newcommand*{\gax}[4]		{\ax{g_{#1#2#3#4}}}

\tikzset{
	annotated cuboid/.pic={
		\tikzset{%
			every edge quotes/.append style={midway, auto},
			/cuboid/.cd,
			#1
		}
		\draw [every edge/.append style={pic actions, opacity=.5}, pic actions]
		(0,0,0) coordinate (o) -- ++(-\cubescale*\cubex,0,0) coordinate (a) -- ++(0,-\cubescale*\cubey,0) coordinate (b) edge coordinate [pos=1] (g) ++(0,0,-\cubescale*\cubez)  -- ++(\cubescale*\cubex,0,0) coordinate (c) -- cycle
		(o) -- ++(0,0,-\cubescale*\cubez) coordinate (d) -- ++(0,-\cubescale*\cubey,0) coordinate (e) edge (g) -- (c) -- cycle
		(o) -- (a) -- ++(0,0,-\cubescale*\cubez) coordinate (f) edge (g) -- (d) -- cycle;
		
		;
	},
	/cuboid/.search also={/tikz},
	/cuboid/.cd,
	width/.store in=\cubex,
	height/.store in=\cubey,
	depth/.store in=\cubez,
	units/.store in=\cubeunits,
	scale/.store in=\cubescale,
	width=10,
	height=10,
	depth=10,
	units=cm,
	scale=.1,
}
%tikz parameters
\usepgflibrary{arrows}
\tikzstyle{point}=[circle,draw]
\usetikzlibrary{arrows,automata,shapes,decorations.markings,
decorations.pathmorphing,backgrounds,fit,snakes,calc}
%\usepackage[nottoc]{tocbibind}
\newlength{\crestheight}
\setlength{\crestheight}{0.15\paperheight} 
\newcommand{\todo}[1]{\textcolor{red}{TODO: #1}}
\newcommand{\red}[1]{{\color{red}{#1}}}
\newcommand{\blue}[1]{{\color{blue}{#1}}}
%%%%%%%%%%%%%%%%%%%%%%%%%%%%%%%%%%%%%%%%%%%%%%%%%%%%%%%%%%%%%%
%%%%%%%%%%%%%%%%%%%%%%%%%%%%%%%%%%%%%%%%%%%%%%%%%%%%%%%%%%%%%%
%%% Packages
\usepackage{geometry}
\geometry{vmargin=3.2cm,hmargin=3.5cm}%
\usepackage{rotating}

\usepackage{amsmath} % for improving the structure and printed output of documents containing mathematical formulas

\usepackage{amssymb} % for mathematical symbols and fonts
\usepackage{bm} % command \bm makes its ar­gu­ment bold, better than \boldsymbol{}
\usepackage{colonequals} % symbols := and ::=
\usepackage{cmll}  % Linear Logic symbols
\usepackage{wasysym} % for smiley 

\usepackage[matrix,arrow]{xy}

\usepackage{amsthm}
\newtheorem{theorem}{Theorem}%[section]
\newtheorem{proposition}[theorem]{Proposition}
\newtheorem{lemma}[theorem]{Lemma}
\newtheorem{question}[theorem]{Question}
\newtheorem{conjecture}[theorem]{Conjecture}
\renewcommand\qedsymbol{$\smiley$}

\usepackage{graphicx} % for graphics, and commands like \scalebox{h-scale}{text} in particular

%%%%%%%%%%%%%%%%%%%%%%%%%%%%%%%%%%%%%%%%%%%%%%%%%%%%%%%%%%%%%
%%%%%%%%%%%%%%%%%%%%%%%%%%%%%%%%%%%%%%%%%%%%%%%%%%%%%%%%%%%%%%
%%% Proof package
\usepackage[noxy]{virginialake}
\vlnosmallleftlabels
\vlnostructuressyntax
%
\newcommand{\marianela}[1]{{\color{purple}[Marianela: #1]}}
\newcommand{\sonia}[1]{{\color{blue}[Sonia: #1]}}

\newcommand{\vlhtr}[2]{\vlpd{#1}{}{#2}}
%
\newcommand{\vlderivationauxnc}[1]{#1{\box\derboxone}\vlderivationterm}
\newcommand{\vlderivationnc}{\vlderivationinit\vlderivationauxnc}
%
%
\makeatletter
\newbox\@conclbox
\newdimen\@conclheight
%
\newcommand\vlderibase[4]{{%
\setbox\@conclbox=\hbox{$#3$}\relax%
\@conclheight=\ht\@conclbox%
\setbox\@conclbox=\hbox{$%
\vlderivationnc{%
\vlin{#1}{#2}{\box\@conclbox}{#4}%
}$}%
\lower\@conclheight\box\@conclbox%
}}
%
\newcommand\vlderidbase[4]{{%
\setbox\@conclbox=\hbox{$#3$}\relax%
\@conclheight=\ht\@conclbox%
\setbox\@conclbox=\hbox{$%
\vlderivationnc{%
\vlid{#1}{#2}{\box\@conclbox}{#4}%
}$}%
\lower\@conclheight\box\@conclbox%
}}
%
\newcommand\vlderiibase[5]{{%
\setbox\@conclbox=\hbox{$#3$}\relax%
\@conclheight=\ht\@conclbox%
\setbox\@conclbox=\hbox{$%
\vlderivationnc{%
\vliin{#1}{#2}{\box\@conclbox}{#4}{#5}%
}$}%
\lower\@conclheight\box\@conclbox%
}}
\makeatother
%

%%%%%%%%%%%%%%%%%%%%%%%%%%%%%%%%%%%%%%%%%%%%%%%%%%%%%%%%%%%%%%
%%%%%%%%%%%%%%%%%%%%%%%%%%%%%%%%%%%%%%%%%%%%%%%%%%%%%%%%%%%%%%
%%% Equation environments
\newdimen\mydisplayskip
\mydisplayskip=.4\abovedisplayskip
\newenvironment{smallequation}
{\par\nobreak\vskip\mydisplayskip\noindent\bgroup\small\csname equation\endcsname}{\csname endequation\endcsname\egroup}
\newenvironment{smallequation*}
{\par\nobreak\vskip\mydisplayskip\noindent\bgroup\small\csname equation*\endcsname}{\csname endequation*\endcsname\egroup}
\newenvironment{smallalign}
{\par\nobreak\noindent\bgroup\small\csname align\endcsname}{\csname endalign\endcsname\egroup}
\newenvironment{smallalign*}
{\par\nobreak\noindent\bgroup\small\csname align*\endcsname}{\csname endalign*\endcsname\egroup}
%\newenvironment{smallmultline}
%{\par\nobreak\vskipmydisplayskip\noindent\bgroup\small\csname multline*\endcsname}{\csname endmultline*\endcsname\egroup}

%%%%%%%%%%%%%%%%%%%%%%%%%%%%%%%%%%%%%%%%%%%%%%%%%%%%%%%%%%%%%%
%%%%%%%%%%%%%%%%%%%%%%%%%%%%%%%%%%%%%%%%%%%%%%%%%%%%%%%%%%%%%%
%%% Extracting symbols from MnSymbol 
\DeclareFontFamily{U} {MnSymbolC}{}

\DeclareFontShape{U}{MnSymbolC}{m}{n}{
<-6>  MnSymbolC5
<6-7>  MnSymbolC6
<7-8>  MnSymbolC7
<8-9>  MnSymbolC8
<9-10> MnSymbolC9
<10-12> MnSymbolC10
<12->   MnSymbolC12}{}
\DeclareFontShape{U}{MnSymbolC}{b}{n}{
<-6>  MnSymbolC-Bold5
<6-7>  MnSymbolC-Bold6
<7-8>  MnSymbolC-Bold7
<8-9>  MnSymbolC-Bold8
<9-10> MnSymbolC-Bold9
<10-12> MnSymbolC-Bold10
<12->   MnSymbolC-Bold12}{}

\DeclareSymbolFont{MnSyC}         {U}  {MnSymbolC}{m}{n}

%\DeclareMathSymbol{\triangleright}{\mathbin}{MnSyC}{80}
\DeclareMathSymbol{\diamondplus}{\mathbin}{MnSyC}{124}
\DeclareMathSymbol{\boxtimes}{\mathbin}{MnSyC}{117}
\DeclareMathSymbol{\meddiamond}{\mathbin}{MnSyC}{110}
\DeclareMathSymbol{\medsquare}{\mathbin}{MnSyC}{106}
\DeclareMathSymbol{\vee}{\mathbin}{MnSyC}{45}
\DeclareMathSymbol{\wedge}{\mathbin}{MnSyC}{44}
\DeclareMathSymbol{\bot}{\mathbin}{MnSyC}{150}
\DeclareMathSymbol{\top}{\mathbin}{MnSyC}{151}
%\DeclareMathSymbol{\forall}{\mathbin}{MnSyC}{166}
%\DeclareMathSymbol{\exists}{\mathbin}{MnSyC}{167}
%\DeclareMathSymbol{\smalldiamond}{\mathbin}{MnSyC}{108}
%\DeclareMathSymbol{\filleddiamond}{\mathbin}{MnSyC}{109}	 

%%%%%%%%%%%%%%%%%%%%%%%%%%%%%%%%%%%%%%%%%%%%%%%%%%%%%%%%%%%%%%
%%%%%%%%%%%%%%%%%%%%%%%%%%%%%%%%%%%%%%%%%%%%%%%%%%%%%%%%%%%%%%
%%% General maths
\newcommand*\mdelim[3]{%
\mathopen{}\left#1%
#3%
\right#2\mathclose{}%
}

\newcommand*{\tuple}{\mdelim{\langle}{\rangle}}

\newcommand*{\DD}{\mathcal{D}}

\newcommand*{\reducesto}{\quad{\leadsto}\quad}
%%%%%%%%%%%%%%%%%%%%%%%%%%%%%%%%%%%%%%%%%%%%%%%%%%%%%%%%%%%%%%
%%%%%%%%%%%%%%%%%%%%%%%%%%%%%%%%%%%%%%%%%%%%%%%%%%%%%%%%%%%%%%
%%% Connectives
\newcommand*{\NEG}[1]{\bar{#1}}
\newcommand*{\NOT}{\neg}
\newcommand*{\AND}{\mathbin{\scalebox{.85}{\raise.1ex\hbox{\large$\wedge$}}}}
\newcommand*{\TOP}{\mathord{\top}}
\newcommand*{\OR}{\mathbin{\scalebox{.85}{\raise.1ex\hbox{\large$\vee$}}}}
\newcommand*{\BOT}{\mathord{\bot}}
\newcommand*{\IMP}{\mathbin{\scalebox{.6}{\raise.4ex\hbox{\large$\bm\supset$}}}}%\supset}}%

\newcommand*{\BOX}{\mathord{\medsquare}}
\newcommand*{\DIA}{\mathord{\scalebox{.9}{\raise.1ex\hbox{$\meddiamond$}}}}

%%%%%%%%%%%%%%%%%%%%%%%%%%%%%%%%%%%%%%%%%%%%%%%%%%%%%%%%%%%%%%
%%%%%%%%%%%%%%%%%%%%%%%%%%%%%%%%%%%%%%%%%%%%%%%%%%%%%%%%%%%%%%
%%% Logic
\newcommand*{\ax}[1]{\mathsf{#1}}
\newcommand*{\kax}[1][]		{\ax{k_{#1}}}
\newcommand*{\accs}[2]{{#1}R{#2}}
\newcommand*{\futs}[2]{{#1}\le{#2}}
\newcommand*{\A}{\mathcal{A}}
\newcommand*{\F}{\mathcal{F}}
\newcommand*{\M}{\mathfrak{M}}

\newcommand{\force}[2]{#1\Vdash#2}
\newcommand{\nforce}[2]{#1\nVdash#2}
%
\newcommand{\height}[1]{|#1|}
%%%%%%%%%%%%%%%%%%%%%%%%%%%%%%%%%%%%%%%%%%%%%%%%%%%%%%%%%%%%%%
%%%%%%%%%%%%%%%%%%%%%%%%%%%%%%%%%%%%%%%%%%%%%%%%%%%%%%%%%%%%%%
%%% Systems
\newcommand*{\sys}[1]{\ensuremath{\mathsf{#1}}}%\xspace}
\newcommand*{\K}{\sys{K}}
\newcommand*{\IK}{\sys{IK}}
\newcommand*{\CK}{\sys{CK}}

\newcommand*{\lab}{\mathsf{lab}}
\newcommand*{\labK}{\lab\K}

%%%% System NKK
\newcommand{\IMPW}{\IMP^{\circ}}
\newcommand{\IMPB}{\IMP^{\bullet}}
\newcommand{\ANDW}{\AND^{\circ}}
\newcommand{\ANDB}{\AND^{\bullet}}
\newcommand{\ORW}{\OR^{\circ}}
\newcommand{\ORB}{\OR^{\bullet}}
\newcommand{\BOXW}{\BOX^{\circ}}
\newcommand{\BOXB}{\BOX^{\bullet}}
\newcommand{\DIAW}{\DIA^{\circ}}
\newcommand{\DIAB}{\DIA^{\bullet}}

%%%%%%%%%%%%%%%%%%%%%%%%%%%%%%%%%%%%%%%%%%%%%%%%%%%%%%%%%%%%%%
%%%%%%%%%%%%%%%%%%%%%%%%%%%%%%%%%%%%%%%%%%%%%%%%%%%%%%%%%%%%%%
%%% Labelled sequents
\newcommand{\SEQ}{\Rightarrow}

\newcommand*{\Labx}{\mathcal{L}}
\newcommand*{\Rabx}{\mathcal{R}}
\newcommand*{\Gx}{\mathcal{G}}
\newcommand*{\Bx}{\mathcal{B}}
\newcommand*{\labels}[2]{{\color{blue}{#1}\:\colon}{#2}}
\newcommand*{\rel}{R}

\newcommand*{\BBot}{\Perp} % \usepackage{cmll} 

%%%%%%%%%%%%%%%%%%%%%%%%%%%%%%%%%%%%%%%%%%%%%%%%%%%%%%%%%%%%%%
%%%%%%%%%%%%%%%%%%%%%%%%%%%%%%%%%%%%%%%%%%%%%%%%%%%%%%%%%%%%%%
%%% Rules
\newcommand*{\rn}[1]  {\ensuremath{\mathsf{#1}}}
\newcommand*{\invr}[1]{#1^\mathsf{inv}}
%
\newcommand*{\orn}[2][]  {\rn{#2}_{#1}}%^{\lab}}}
\newcommand*{\rrn}[2][]  {\rn{#2}_\rn{R#1}}%^\lab}}
\newcommand*{\lrn}[2][]  {\rn{#2}_\rn{L#1}}%^\lab}}
%
\newcommand*{\labrn}[2][]  {\rn{#2}_{#1}}%^{\lab}}}
%\newcommand*{\rlabrn}[2][]  {\rn{#2}_\rn{R#1}}%^\lab}}
%\newcommand*{\llabrn}[2][]  {\rn{#2}_\rn{L#1}}%^\lab}}

\newcommand*{\brsym}{\mathord{\scalebox{.9}{$\boxtimes$}}}
\newcommand*{\boxbrn}[1]{\rn{\brsym_\rn{#1}}}%^{\lab}}}
\newcommand*{\diasym}{\mathord{\diamondplus}}
\newcommand*{\diabrn}[1][]{\rn{\diasym_\rn{#1}}}
%%%%%%%%%%%%%%%%%%%%%%%%%%%%%%%%%%%%%%%%%%%%%%%%%%%%%%%%%%%%%%
%%%%%%%%%%%%%%%%%%%%%%%%%%%%%%%%%%%%%%%%%%%%%%%%%%%%%%%%%%%%%%
%%% Nested sequents
\makeatletter
%\newcommand*\BR[2][]{\mdelim{\lbrack\strut^{#1}}{\rbrack}{#2}}
\newcommand*{\BR}{%
\@ifnextchar\i{\br@two}{%
\@ifnextchar\bgroup{\br@one}{% 
}}}
\newcommand*{\br@one}[1]{%
\def\br@{#1}%
\mdelim{\lbrack}{\rbrack}{\ifx\br@\empty\mkern 3mu\else #1\fi}%
}
\newcommand*{\br@two}[3]{%
\def\br@{#3}%
\mdelim{\lbrack\strut^{#2}}{\rbrack}{\ifx\br@\empty\mkern 3mu\else #3\fi}%
}
%
\newcommand*{\bBR}{%
\@ifnextchar\i{\bbr@two}{%
\@ifnextchar\bgroup{\bbr@one}{% 
}}}
\newcommand*{\bbr@one}[1]{%
\def\br@{#1}%
\mdelim{\llbracket}{\rrbracket}{\ifx\br@\empty\mkern 3mu\else #1\fi}%
}
\newcommand*{\bbr@two}[3]{%
\def\br@{#3}%
\mdelim{\llbracket\strut^{#2}}{\rrbracket}{\ifx\br@\empty\mkern 3mu\else #3\fi}%
}
%
%\newcommand*{\@makeoperator}[2]{
%	\newcommand*{#1}{%
%		\mathrm{#2}\mdelim{\lparen}{\rparen}
%	}
%}
%%
%\@makeoperator{\fm}{fm^{\n}}
%\@makeoperator{\ofm}{fm^{\o}}
%\@makeoperator{\hfm}{fm^{\h}}
%%
%\@makeoperator{\depth}{dp}
%\@makeoperator{\rank}{rk}
%\@makeoperator{\height}{ht}
%\@makeoperator{\tree}{tree}
%\@makeoperator{\graph}{graph}
\makeatother
%%%%%%%%%%%%%%%%%%%%%%%%%%%%%%%%%%%%%%%%%%%%%%%%%%%%%%%%%%%%%%
% contexts
\makeatletter
\newcommand*{\cxs}{%
\@ifnextchar\i{\cxs@two}{%
\@ifnextchar\bgroup{\cxs@one}{% % \bgroup is the same as {
}}}
\newcommand*{\cxs@one}[1]{%
\def\cxs@{#1}%
\mdelim{\lbrace}{\rbrace}{\ifx\cxs@\empty\mkern 3mu\else #1\fi}%
\cxs@one@decor%
}
\newcommand*{\cxs@two}[3]{%
\def\cxs@{#3}%
\mdelim{\lbrace\strut^{#2}}{\rbrace}{\ifx\cxs@\empty\mkern 3mu\else #3\fi}%
\cxs@one@decor%
}
\def\cxs@one@decor{%
\@ifnextchar_{\cxs@one@sub}{%
\@ifnextchar^{\cxs@one@sup}{%
\@ifnextchar\dots{\@firstoftwo{\dotsm\cxs@one@decor}}{%
	\@ifnextchar[{\cxs@one@arg}%]
	\cxs}}}%
}
\def\cxs@one@sub_#1{_{#1}\cxs@one@decor}
\def\cxs@one@sup^#1{^{#1}\cxs@one@decor}
\def\cxs@one@arg[#1]{{#1}\cxs@one@decor}

%% Change these as needed, but keep the \cx@continuation at the end
\def\cx@delete@always#1{{#1}^{\ast}\cx@continuation}
\def\cx@delete@right#1*{{#1}^{\star}\cx@continuation}
\def\cx@delete@focus#1\f{{#1}^{\scriptscriptstyle\not{\langle}{\rangle}}\cx@continuation}

\def\cx@delete@star#1*{%
\@ifnextchar*{\cx@delete@right{#1}}{\cx@delete@always{#1}}%
}

\newcommand*{\@makecontextual}[2]{
\newcommand*{#1}{%
\@ifnextchar*{\cx@delete@star{#2}}{%
\@ifnextchar\f{\cx@delete@focus{#2}}{%
	#2\cx@continuation}}%
}%
}
\newcommand*{\cx@continuation}[1][]{_{#1}\cxs}

\@makecontextual{\Ex}{}

\@makecontextual{\Cx}{\Gamma}
\@makecontextual{\Dx}{\Delta}
\@makecontextual{\Lx}{\Lambda}
\@makecontextual{\Rx}{\Pi}

\@makecontextual{\LLx}{\Omega}
\@makecontextual{\RRx}{\Xi}
\@makecontextual{\CCx}{\Sigma}
\@makecontextual{\DDx}{\Theta}
\makeatother
%%%%%%%%%%%%%%%%%%%%%%%%%%%%%%%%%%%%%%%%%%%%%%%%%%%%%%%%%%%%%%
% intuitionistic
\newcommand*{\rt}[1]{#1^\circ}
\newcommand*{\lf}[1]{#1^\bullet}

\newcommand{\kfour}{	
	\begin{center}
		$\vlderivation {
			\vlin{\sir}
			{y$ fresh$}
			{\Rightarrow x \colon (\lozenge A \vljm \square B) \vljm \square (A \vljm B)}
			{\vlin {\sbr}
				{z, w$ fresh$}
				{x \le y, y\colon \lozenge A \vljm \square B \Rightarrow y \colon \square (A \vljm B) }
				{\vlin {\sir}
					{u$ fresh$}
					{x \le y, y\le z, zRw, y \colon \lozenge A \vljm \square B \Rightarrow w \colon A \vljm B }
					{\vlin {\fone}
						{}
						{x \le y, y \le z, w \le u, zRw, y \colon \lozenge A \vljm \square B, u \colon A \Rightarrow u \colon B}
						{\vlin {\trans}
							{}
							{x \le y, y \le z, w \le u, z \le t, zRw, tRu, y \colon \lozenge A \vljm \square B, u \colon A \Rightarrow u \colon B}
							{\vliin {\sil}
								{}
								{x \le y, y \le z, w \le u, z \le t, y \le t, zRw, tRu, y \colon \lozenge A \vljm \square B, u \colon A \Rightarrow u \colon B}
								{\vlhy{\circledast  \hspace{75mm}}}
								{\vlhy{\star}}}}}}}
		}$
		
	\end{center}
	
	
	La derivación continúa como resultado de aplicar la regla izquierda de la implicación $\sil$. De esta regla obtenemos dos árboles:
	
	Por un lado, la derivación que se sigue de $\circledast$:
	
	\begin{center}
		
		$
		\vlderivation{
			\vlin{\sil}{}{\circledast}{\vlin {\sdr}
				{}
				{x \le y, y \le z, w \le u, z \le t, y \le t, zRw, tRu, y \colon \lozenge A \vljm \square B, u \colon A \Rightarrow u \colon B, t \colon \lozenge A}
				{\vlin {\refl}
					{}
					{x \le y, y \le z, w \le u, z \le t, y \le t, zRw, tRu, y \colon \lozenge A \vljm \square B, u \colon A \Rightarrow u \colon B, t \colon \lozenge A, u \colon A}
					{\vlin {\ids}
						{}
						{x \le y, y \le z, w \le u, z \le t, y \le t, u \le u, zRw, tRu, y \colon \lozenge A \vljm \square B, u \colon A \Rightarrow u \colon B, t \colon \lozenge A, u \colon A}
						{\vlhy {}}}}}
		}$
	\end{center}
	
	Y por otro lado, la derivación que se sigue de $\star$:
	
	\begin{center}
		
		\scalebox{0.92}{
		$\vlderivation{
			\vlin{\sil}{}{\star}{\vlin {\refl}
				{}
				{x \le y, y \le z, w \le u, z \le t, y \le t, zRw, tRu, y \colon \lozenge A \vljm \square B, u \colon A, t \colon \square B \Rightarrow u \colon B}
				{\vlin {\sbl}
					{}
					{x \le y, y \le z, w \le u, z \le t, y \le t, t \le t, zRw, tRu, y \colon \lozenge A \vljm \square B, u \colon A, t \colon \square B \Rightarrow u \colon B}
					{\vlin {\refl}
						{}
						{x \le y, y \le z, w \le u, z \le t, y \le t, t \le t, zRw, tRu, y \colon \lozenge A \vljm \square B, u \colon A, t \colon \square B, u \colon B \Rightarrow u \colon B}
						{\vlin {\ids}
							{}
							{x \le y, y \le z, w \le u, z \le t, y \le t, t \le t, u \le u, zRw, tRu, y \colon \lozenge A \vljm \square B, u \colon A, t \colon \square B, u \colon B \Rightarrow u \colon B}
							{\vlhy {}}}}}}
		}$}
	\end{center}
}

%%%%%%%%THEN-2 MACRO%%%%%%%%%%%%
\newcommand{\thentwo}{\begin{center}	
		$\vlderivation {
			\vlin{\sir}
			{}
			{\Rightarrow x \colon (A \vljm (B \vljm C)) \vljm ((A \vljm B)\vljm (A \vljm C))}
			{\vlin {\sir}
				{}
				{x \le y, y \colon A \vljm (B \vljm C)\Rightarrow y \colon (A \vljm B) \vljm (A \vljm C)}
				{\vlin {\sir}
					{}
					{x \le y, y \le z, y \colon A \vljm (B \vljm C), z \colon A \vljm B \Rightarrow z \colon A \vljm C}
					{\vlin {\trans}
						{}
						{x \le y, y \le z, z \le w, y \colon A \vljm (B \vljm C), z \colon A \vljm B, w \colon A \Rightarrow w \colon C}
						{\vliin {\sil}
							{}
							{x \le y, y \le z, z \le w, y \le w, y \colon A \vljm (B \vljm C), z \colon A \vljm B, w \colon A \Rightarrow w \colon C}
							{\vlhy{\circledast \hspace{75mm}}}
							{\vlhy{\star}}}}}
			}}$
			
		\end{center}
		
		\bigskip
		
		La derivación, luego de aplicar la regla izquierda de la implicación $\sil$, genera dos premisas. La premisa que continua a partir $\circledast$ es de la siguiente forma:
		
		\begin{center}
			
			$\vlderivation{
				\vlin{\sil}{}{\circledast}
				{\vlin {\refl}
					{}
					{x \le y, y \le z, z \le w, y \le w, y \colon A \vljm (B \vljm C), z \colon A \vljm B, w \colon A \Rightarrow w \colon C, w \colon A}
					{\vlin {\ids}
						{}
						{x \le y, y \le z, z \le w, y \le w, w \le w, y \colon A \vljm (B \vljm C), z \colon A \vljm B, w \colon A \Rightarrow w \colon C, w \colon A}
						{\vlhy {}}}}	
			}$
			
		\end{center}
		
		Mientras que la premisa que continua en $\star$ es de esta forma:
		
		\begin{center}
			\scalebox{0.96}{
		$\vlderivation{
			\vlin{\sil}{}{\star}
			{\vlin {\trans}
				{}
				{x \le y, y \le z, z \le w, y \le w, y \colon A \vljm (B \vljm C), z \colon A \vljm B, w \colon A, w \colon B \vljm C \Rightarrow w \colon C}
				{\vliin {\sil}
					{}
					{x \le y, y \le z, z \le w, y \le w, w \le w, y \colon A \vljm (B \vljm C), z \colon A \vljm B, w \colon A, w \colon B \vljm C \Rightarrow w \colon C}
					{\vlhy{\heartsuit \hspace{75mm}}}
					{\vlhy{\clubsuit}}}}
		}$}
		
	\end{center}
	
		Vemos que en $\star$ al volver aplicar la regla de implicación izquierda $\sil$, genera nuevamente dos premisas que las representamos con $\heartsuit$ y $\clubsuit$ para continuar con nuestra derivación.
		
		Por un lado, la derivación correspondiente a $\clubsuit$ está dada por:
		
		\begin{center}
			\scalebox{0.94}{
		$\vlderivation{
			\vlin{\sil}{}{\clubsuit}{
				\vlin {\ids}
				{}
				{x \le y, y \le z, z \le w, y \le w, w \le w, y \colon A \vljm (B \vljm C), z \colon A \vljm B, w \colon A, w \colon B \vljm C, w \colon C \Rightarrow w \colon C}
				{\vlhy {}}}
		}$}
	\end{center}
	
		Por otro lado, de $\heartsuit$ obtenemos la siguiente derivación que de nuevo utiliza $\sil$. Es por ello que volvimos a representar las premisas que se obtienen como resultado de esta regla con $\spadesuit$ (para la premisa que parte del lado izquierdo) y $\dagger$ (para la premisa que parte del lado derecho):
		
		\begin{center}
			\scalebox{0.94}{
		$\vlderivation{
			\vlin{\sil}{}{\heartsuit}
			{\vliin {\sil}
				{}
				{x \le y, y \le z, z \le w, y \le w, w \le w, y \colon A \vljm (B \vljm C), z \colon A \vljm B, w \colon A, w \colon B \vljm C \Rightarrow w \colon C, w \colon B}
				{\vlhy{\spadesuit \hspace{105mm}}}
				{\vlhy{\dagger}}}
		}$}
	\end{center}
		
	
		
		Finalmente por $\spadesuit$ concluye la derivación de esta rama de la siguiente forma:
		
		\begin{center}
		\scalebox{0.90}{
			$\vlderivation{
				\vlin{\sil}{}{\spadesuit}
				{\vlin {\ids}
					{}
					{x \le y, y \le z, z \le w, y \le w, w \le w, y \colon A \vljm (B \vljm C), z \colon A \vljm B, w \colon A, w \colon B \vljm C \Rightarrow w \colon C, w \colon B, w \colon A}
					{\vlhy {}}}}$
		}
	\end{center}
	
		Y por $\dagger$, la derivación concluye de esta otra manera:
		
		\begin{center}
		
		\scalebox{0.90}{
			$\vlderivation{
				\vlin{\sil}{}{\dagger}
				{\vlin {\ids}
					{}
					{x \le y, y \le z, z \le w, y \le w, w \le w, y \colon A \vljm (B \vljm C), z \colon A \vljm B, w \colon A, w \colon B \vljm C, w \colon B \Rightarrow w \colon C, w \colon B}
					{\vlhy {}}}
			}$
		}
	\end{center}
		}


\begin{document}

%\renewcommand{\proofname}{Demostración}
%\providecommand{\innp}[1]{\langle#1\rangle} 
%\newcommand{\seno}{\mathrm{sen}}
%\newcommand{\diff}{\mathrm{d}}

\newtheorem{teo}{Teorema} 
\newtheorem{lemma}{Lema}
%\newtheorem{cor}[teo]{Corolario}
%\newtheorem{lem}[teo]{Lema}

\theoremstyle{definition}
\newtheorem{dfn}{Definición}

%\theoremstyle{remark}
%\newtheorem{obs}[teo]{Observación}



\begin{titlepage}
\begin{center}

\textsc{\Large UNIVERSIDAD NACIONAL DE CÓRDOBA}\\[1em]
\textsc{FACULTAD DE MATEMÁTICA, ASTRONOMÍA, FÍSICA Y COMPUTACIÓN}
\vspace{4mm}
%Figura
\begin{figure}[h]
\begin{center}
\includegraphics[height=\crestheight]{unc.png}
\end{center}
\end{figure}

\vspace{4mm}

\textsc{\huge \textbf{Teoría de prueba con etiquetas para lógicas modales intuicionistas}}\\

\vspace{15mm}

\textsc{\Large Marianela Morales}\\[4em]

\textsc{\large Directores:\\\vspace{2mm} Dr. Raul Fervari \\ Dr. Lutz Stra$\ss$burger}

\vspace{15mm}
\textsc{\Large Licenciatura en Ciencias de la Computación}\\[1em]
\bigskip
\bigskip

\includegraphics[width=0.1\textwidth]{icono.png}


Teoría de prueba con etiquetas para lógicas modales intuicionistas por Marianela Morales se distribuye bajo una \href{https://creativecommons.org/licenses/by-nc-sa/4.0/}{Licencia Creative Commons Atribución-NoComercial-CompartirIgual 4.0 Internacional}

\end{center}

\vspace*{\fill}
\textsc{Córdoba, Argentina. \hspace*{\fill} 28 de Marzo del 2019}

\end{titlepage}




\paragraph{Resumen}

El estudio de la lógica modal \cite{blackburn01}, desde Aristóteles, proviene del deseo de analizar ciertos argumentos filosóficos y así poder establecer la verdad de una proposición: por ejemplo, una proposición puede ser falsa ahora pero verdadera más tarde, o por el contrario necesariamente verdadera, y así sucesivamente. Lo que llamamos lógica modal describe el comportamiento abstracto de las modalidades $\square$ y $\Diamond$, pero abarca una amplia cantidad de modalidades 'reales' en las expresiones ling\"{u}ísticas: tiempo, necesidad, posibilidad, obligación, conocimiento, creencia, etc. Semánticamente, los operadores modales y más precisamente el $\Diamond$, nos permite describir propiedades de los diferentes estados que podemos alcanzar a partir de un punto de evalación.

Desde que la teoría de modelos modal comenzó a desarrollarse, existe una tendencia a utilizar métodos basados en la teoría de modelos en lugar de aquellos basados en la teoría de prueba ya que los sistemas de prueba clásicos eran generalmente insuficientes. Sin embargo, en los últimos años, nuevas técnicas fueron desarrolladas y los sistemas de prueba comenzaron a ser mayormente utilizados ya que proveen ciertas ventajas cuando se trabaja con el análisis y estandarización de pruebas. Para tener más detalles sobre esta dicotomía vease \cite{negri2005}. La deducción etiquetada propuesta por Gabbay \cite{gabbay1996} en los 80's se presentó como un marco unificador de la teoría de prueba para proporcionar sistemas de prueba para una amplia gama de lógicas. Para las lógicas modales, también puede tomar la forma de deducción natural etiquetada y sistemas de secuentes etiquetados como los de Simpson \cite{simpson1994}, Vigano \cite{vigano2013} y Negri \cite{negri2005}. Estos formalismos hacen un uso explícito no sólo de las etiquetas sino también de los átomos relacionales que hacen referencia a la relación de accesibilidad con un modelo de Kripke \cite{kripke1959}.

Continuaremos con la elección de una presentación de secuentes. Más precisamente, en este trabajo, el objetivo principal consiste en desarrollar un sistema de secuentes etiquetados para la lógica modal intuicionista, que involucra dos símbolos de relación: uno para la relación de accesibilidad asociada con la semántica de Kripke para lógicas modales clásicas, y otra relación de pre-orden asociada a la semántica de Kripke para la lógica intuicionista. Para obtener este resultado, utilizamos un cálculo de secuentes etiquetado propuesto por Negri \cite{negri2005} para lógicas modales clásicas y lo extendimos con una relación de pre-orden. Esto permite tener un sistema etiquetado en estrecha correspondencia con los modelos bi-relacionales de Kripke.
\paragraph{Abstract}


The study of modal logic \cite{blackburn01}, going back to Aristotle, comes from the desire to analyse certain philosophical arguments, and thus qualify finely the truth of a proposition: for example a proposition may be false now but true later, or on the contrary true and necessarily so, and so on. What is now called modal logic describes the behaviour of the abstract modalities $\square$ and $\Diamond$, but covers a wide range of ‘real’ modalities in linguistic expressions: time, necessity, possibility, obligation, knowledge, belief, etc. Semantically, the modal operators and more precisely the $\Diamond$, allows to describe properties of the different states that you can reach from the evaluation point. 

Since the modal model theory began to develop, there is a tendency to use methods based on model theory instead of proof theory, because the classical proof systems were generally insufficient. However, in the last years, new techniques were developed, and proof systems began to be more familiar, that’s why they have certain advantages when it comes to analysis or standardization of proofs. Look at \cite{negri2005} to have more details of this dichotomy. Labelled deduction has been more generally proposed by Gabbay \cite{gabbay1996} in the 80’s as a unifying framework throughout proof theory in order to provide proof systems for a wide range of logics. For modal logics it can also take the form of labelled natural deduction and labelled sequent systems as used, for example, by Simpson \cite{simpson1994}, Vigano \cite{vigano2013}, and Negri \cite{negri2005}. These formalisms make explicit use not only of labels, but also of relational atoms that reference the accessibility relation with a Kripke model \cite{kripke1959}. 

We will continue with a choice of a sequent presentation. More precisely, in this work, the main goal consists of developing a labelled sequent system for intuitionistic modal logic, that comes with two relation symbols: one for the accessible world relation associated with the Kripke semantics for classical modal logics, and one for the preorder relation associated with the Kripke semantics for intuitionistic logic. To obtain this result, we used a labelled sequent calculus proposed by Negri \cite{negri2005} for classical modal logics and we extended it with a preorder relation. That allows to have a labelled system in close correspondence to the birelational Kripke models. 

\bigskip
\bigskip
\bigskip


\textbf{Palabras clave:} lógica modal, lógica intuicionista, cálculo de secuentes etiquetados, teoría de prueba, Gentzen, modelos bi-relacionales.


\textbf{Key words:} modal logic, intuitionistic logic, labelled sequent calculus, proof theory, Gentzen, bi-relational models.



%\clearpage\mbox{}\clearpage

\newpage
\bigskip

\paragraph{Menciones especiales}

\begin{quote}
	Antes de comenzar, quiero destacar en este pequeño párrafo que gran parte de este trabajo fue resultado de un trabajo en conjunto con Sonia Marin. Ella realizó sus estudios de doctorado con uno de mis directores, el Dr. Lutz Stra$\ss$burger, y el desarrollo de su tesis doctoral (Ver \cite{marin2018}) fue el punto de partida de esta tesis de Licenciatura que se fue conformando con ideas provenientes de Sonia. El labor y el acompañamiento por parte de ella fue fundamental a lo largo de este trabajo final de Licenciatura.s
\end{quote}




\tableofcontents

\clearpage\mbox{}\clearpage

%\chapter*{}
%\begin{flushright}
%\textit{DEDICATORIA}
%\end{flushright}



%\pagestyle{headings}


\renewcommand{\chaptername}{}
\chapter{Introducción}
%Las lógicas modales clásicas son extensiones de la lógica clásica con nuevos operadores llamados $modalidades$.
Existen al menos dos formas de caracterizar todas las verdades o teoremas de una lógica: por un lado, el formato conocido como \emph{axiomatización a la Hilbert} y por otra parte, el formato que sigue el \emph{estilo de Gentzen}, el cual utiliza reglas de secuentes. 
En este trabajo final, nuestra contribución se basa en un cálculo de secuentes para lógicas modales intuicionistas. El resultado se obtiene a partir de extender un cálculo para lógicas modales clásicas con una relación de pre-orden ($\le$) (además de la relación de accesibilidad que nos provee la lógica modal básica) con el objetivo de poder capturar el comportamiento intuicionista. Desde un punto de vista de la lógica modal, esencialmente la lógica modal intuicionista es simplemente una lógica con dos modalidades, donde una de ellas es interpretada en una relación de pre-orden. En esta introducción, haremos un breve recorrido por la historia de la \emph{teoría de prueba} y su relación con las lógicas modales, así como también veremos qué aporta nuestro nuevo cálculo sobre otros.

\section{Teoría de prueba}
En el razonamiento matemático, existen los teoremas y sus demostraciones. Frecuentemente, existen distintas demostraciones para un mismo teorema y nos interesa estudiar y entender sus similitudes y sus diferencias. Por otro lado, es posible que las demostraciones de distintos teoremas compartan ciertos patrones, como por ejemplo, inducción o reducción por el absurdo. Parte de nuestro trabajo es entender cuáles son los contextos de aplicación de estas formas particulares de razonamiento y cómo hacerlo. Sin embargo, con el objetivo de comunicar demostraciones (es decir, presentar demostraciones a otras personas y que ellas las comprendan) los matemáticos usualmente utilizan lenguaje natural con algunos símbolos especiales. Con el objeto de describir de manera precisa las demostraciones, es necesario contar con un lenguaje puramente matemático. De esta manera nace una disciplina dedicada a estudiar las demostraciones como un objeto formal: la \textit{teoría de prueba} (en inglés \textit{proof theory}). Frege lideró esta disciplina sugiriendo que las pruebas deben ser consideradas como objetos de estudio matemático en 1879 en su libro \textit{Begriffschrift} \cite{frege1879}. Luego, Hilbert siguió las ideas de Frege proponiendo una definición de un sistema deductivo para formalizar el razonamiento \cite{hilbert1967}.

\textit{La Teoría de Prueba} puede ser considerada como uno de los cuatro pilares de la lógica matemática junto con la teoría de modelos, la teoría de conjuntos y la teoría de la recursión. Al diseñar diversos formalismos para ver las pruebas como objetos matemáticos, debemos entender sus propiedades a través de métodos matemáticos formales. Desde el punto de vista de las ciencias de la computación, una aplicación directa podría ser el desarrollo de algoritmos para demostrar teoremas automáticamente (demostradores automáticos), o verificar pruebas matemáticas (verificación/comprobación automática de teoremas). Otra aplicación podría ser extraer de una prueba un algoritmo; por ejemplo, si el teorema establece la existencia de un objeto, este objeto también podría construirse efectivamente; otro ejemplo podría ser el uso de las demostraciones fallidas para la construcción de contraejemplos. Por otro lado, desde un punto de vista más abstracto, podríamos tratar de entender qué axiomas se requieren para demostrar ciertos teoremas (matemática inversa), como así también comparar diferentes métodos de prueba y, en particular, los tamaños de las pruebas que producen (complejidad de la prueba).

Así, como mencionamos al comienzo, podríamos decir que las matemáticas están hechas de muchos lemas, algunos teoremas y de sus pruebas. Se podría decir que los matemáticos no demuestran un teorema desde cero, si no que construyen la prueba a partir de demostrar otras afirmaciones intermedias que son llamadas \emph{lemas}.
En un sistema deductivo \emph{a la Hilbert y Frege}, para probar un teorema T, es posible utilizar un lema dado L si, por un lado, podemos probar L y si, por otra parte, podemos inferir T de L. Este paso de razonamiento es llamado \emph{Modus Ponens}. El desafío radica en encontrar el lema apropiado.

Sin embargo, para las aplicaciones que mencionamos, cambiar a una teoría diferente del teorema que queremos demostrar puede no resultar tan sencillo, por lo que muchas veces preferimos una demostración sin lemas. Es importante aclarar que no todas las pruebas matemáticas pueden realizarse libres de lemas.

Fue así que Gentzen en su \emph{Untersuchungen {\"u}ber das logische Schlie{\ss}en. I} en 1934 \cite{gentzen1934} introdujo métodos teóricos de prueba para demostrar resultados en la lógica matemática. En particular, Gentzen fue quién desarrolló el \emph{cálculo de secuentes}, una representación alternativa de las pruebas que promueven las reglas de inferencia sobre los axiomas. El teorema de \emph{cut-elimination} o también conocido como \emph{Hauptsatz} declara que cualquier prueba en el cálculo de secuentes puramente lógica puede transformarse en una forma analítica normal. Esto quiere decir que en cálculo de secuentes, cualquier prueba puede ser realizada sin lemas. Para poder probar el \emph{Hauptsatz}, uno necesita mostrar que una regla llamada \emph{cut}, la cual reformula Modus Ponens en el cálculo de secuentes, es redundante en el sistema. Es decir, establece que cualquier prueba que tenga una demostración en el cálculo de secuentes utilizando \emph{cut}, también tiene una demostración sin esta regla.

Otra de las grandes contribuciones de Gentzen fue el estudio paralelo de las lógicas clásicas e intuicionistas. Las lógicas intuicionisticas surgen con Brouwer \cite{brouwer1920} como una formalización del razonamiento constructivista (la existencia de un objeto es equivalente a la posibilidad de su construcción), rechazando el \emph{Principio del Tercero Excluído} el cual establece que una proposición tiene que ser verdadera o falsa. Mientras que en la lógica intuicionista puede haber incertidumbre sobre si una proposición se mantiene o no, en la lógica clásica, el principio es aceptado.

En este trabajo, vale la pena destacar que por \emph{lógica intuicionista} nos referimos a la lógica intuicionista proposicional básica. A la hora de definir cálculos de prueba para esta lógica, una caracterísitica importante es que es posible obtener tales cálculos, simplemente restringiendo sintácticamente el cálculo de prueba para la lógica clásica correspondiente \cite{vandalen2004} (en nuestro caso, lógica proposicional).

El enfoque intuicionista fue utilizado sobre diferentes lógicas. En este trabajo, nos centraremos en el desarrollo de un cálculo de secuentes para una lógica modal intuicionista. Como se discutirá en las siguientes secciones, el enfoque modal se acopla adecuadamente, ya que una de las características principales es la definición y combinación de nuevos operadores, dando origen a nuevas familias de lógicas modales. Desde esta perspectiva, la componente intuicionista es simplemente una modalidad en el lenguaje.


\section{Lógicas modales y su teoría de prueba}

En sus comienzos, las lógicas modales surgen del deseo de analizar ciertos conceptos filosóficos como \emph{necesidad, creencia, conocimiento, obligación}, entre otros. Lo que llamamos \emph{lógicas modales} se obtienen como extensiones de lógicas clásicas (por ejemplo: lógica proposicional o lógica de predicados) describiendo el comportamiento de modalidades como $\square$ y $\Diamond$ que representan los conceptos a investigar. Más precisamente estos operadores permiten describir propiedades de los estados accesibles desde el punto de evaluación. Las lógicas modales más estudiadas son aquellas basadas en el razonamiento clásico. El interés en las versiones intuicionistas de las lógicas modales se produjo mucho más tarde y de dos fuentes diferentes: por un lado, los lógicos se interesaron por obtener versiones intuicionistas (es decir, utilizando razonamiento constructivista), y por otra parte, aplicaciones de las mismas.

Desde Aristóteles y con un gran desarrollo en la Edad Media se fundan las bases de la lógica modal. Su consolidación se produjo a fines de la década de 1950 y principios de la década de 1960 con el desarrollo de una semántica basada en los \emph{``mundos posibles"} introducida por Kripke \cite{kripke1959} (de allí el nombre de la semántica). Nos permite ver a la lógica modal como un lenguaje para grafos o bien como un lenguaje para describir procesos, es decir, ver a los elementos del grafo como un conjunto de estados computacionales y ver a las relaciones como acciones que transforman un estado a otro.

Como consecuencia de lo propuesto por Kripke, en términos de \emph{teoría de prueba}, algunas extensiones para secuentes fueron incorporadas beneficiando el manejo de las modalidades. Dos enfoques se destacaron dentro de estas extensiones: por un lado, sistemas que incorporan semántica relacional explícita en su formalismo, llamados \emph{sistemas etiquetados deductivos} como los \emph{cálculos etiquetados} \cite{negri2005} y el \emph{método de tablas semánticas} (mejor conocido en inglés como \emph{semantic tableaux} \cite{fitting1983}), y por otra parte, sistemas que utilizan dispositivos sintácticos para recuperar el lenguaje modal, llamados \emph{sistemas deductivos sin etiquetas} (por ejemplo: \emph{hypersequents} \cite{avron1996}, \emph{nested sequents} \cite{brunnler2009}). Así, los teóricos de prueba interesados en la lógica modal tienen a su disposición una amplia gama de diferentes formalismos de prueba para elegir, que nos permiten tener herramientas complementarias para poder explorar la teoría de prueba de la lógica modal. El trabajo de esta tesis nos muestra uno de estos formalismos nombrados, el de \emph{sistemas de pruebas etiquetados}, para analizar pruebas modales.

\section{Motivación}

%Como se nombró anteriormente, los teoremas de una lógica pueden caracterizarse a través de dos formatos: por un lado, \emph{axiomatización a la Hilbert} y por otro lado, a través de reglas de secuentes conocido como \emph{estilo Gentzen}. 
Nuestro trabajo se centra en sistemas de prueba etiquetados para lógicas modales intuicionistas usando dos tipos de relaciones de accesibilidad. Esto significa que hay un símbolo relacional para la relación de accesibilidad modal y una para la relación futura (o conocida como relación de pre-orden). Esta es una novedad ya que nos permite poner el sistema de prueba en estrecha relación con la semántica birelacional de Kripke.

Extendemos el sistema etiquetado propuesto por Negri para la lógica modal básica \cite{negri2005} con la relación de pre-orden que se nombró anteriormente (la relación de accesibilidad ya esta presente en el cálculo para la lógica clásica) para capturar lógicas modales intuicionistas. Para nuestro nuevo sistema, probamos tanto correctitud como completitud, así como también utilizamos el sistema etiquetado propuesto por Simpson \cite{simpson1994} para demostrar que nuestro sistema es un sistema libre de $\mathsf{cut}$. Por último y como trabajo a futuro, extendemos nuestro sistema agregando el axioma de Scott-Lemmon volviendo a demostrar correctitud y completitud para nuestro nuevo sistema (con el axioma agregado).


\section{Organización del trabajo}

Esta tesis se estructura de la siguiente manera:

En el \textbf{Capítulo \ref{cap:logclasica}} presentamos a la lógica modal básica con su sintaxis, semántica y sus distintos formatos de caracterización: presentamos una axiomatización a la Hilbert y una axiomatización a la Gentzen, en donde vemos en detalle el cálculo de secuentes etiquetado propuesto por Negri \cite{negri2005}.

En el \textbf{Capítulo \ref{cap:logintuicionista}} se presentan los conceptos de sintaxis y semántica para la lógica modal intuicionista como así también una axiomatización a la Hilbert.

En el \textbf{Capítulo \ref{cap:labIKh}} se plantea en detalle el problema principal que abordamos y se presenta en detalle el sistema de secuentes etiquetado propuesto para la lógica modal intuicionista (llamado $\labIKh$). Se ven en detalle cada una de las reglas de secuentes que conforman al sistema. En este capítulo también se desarrolla la demostración de que nuestro sistema es correcto.

En el \textbf{Capítulo \ref{cap:completeness}} se desarrolla una prueba de completitud sintáctica por medio del sistema a la Hilbert para el sistema $\labIKh$. Se realiza una demostración en la que se demuestran todos los axiomas de la lógica modal proposicional, se simula modus ponens y necesitación, y se prueban las variantes $\kaxiom_{1}$, ... $\kaxiom_{5}$ del axioma de distributividad $\kaxiom$ (presentados en detalle en el Capítulo 3). De esta prueba se obtiene un sistema completo con la regla de $\mathsf{cut}$ (utilizada para simular modus ponens). 

En el \textbf{Capítulo \ref{cap:future}} se presentan algunas líneas de trabajo futuro. Entre ellas, se continuará trabajando en una prueba de completitud utilizando el sistema de secuentes propuesto por Simpson \cite{simpson1994}. Y por otra parte, se presentan algunas extensiones del sistema $\labIKh$ para generar lógicas más fuertes. En particular, en el Capítulo 6 extendemos nuestro sistema con el axioma de Scott-Lemmon o también conocido como $\agklmn$.

Por último, en el \textbf{Capítulo \ref{cap:conclusion}} presenta un breve resumen de lo realizado a lo largo de este trabajo final de Licenciatura como así también se incluye una pequeña sección de la experiencia que tuve a lo largo de este trabajo.
\chapter{Lógica modal clásica}
\label{cap:logclasica}

Usualmente, existen situaciones donde necesitamos como herramienta de modelado una lógica que tenga buenas propiedades computacionales. En estos casos, la lógica de primer orden ($\mathsf{LPO}$)~\cite{enderton72} no es la mejor alternativa. En particular, las tareas de inferencia para $\mathsf{LPO}$ no son tratables computacionalmente. Por ejemplo, el problema de determinar si una fórmula $\varphi$ de $\mathsf{LPO}$ es satisfacible, es indecidible \cite{chur:note36,turi:comp37,berg:unde66}, mientras que \emph{model checking} (es decir, dado un modelo $\M$ y una fórmula $\varphi$, determinar si $\M$ satisface $\varphi$) es $\sf{PSPACE}$-completo \cite{Stockmeyer74,chan:opti77,Vardi82}. Por este motivo, es natural buscar nuevos lenguajes con mejores propiedades computacionales. Las lógicas modales~\cite{blackburn01,blackmodal06} se encuentran dentro de esta familia de lógicas: son lógicas diseñadas para hablar de estructuras relacionales lo cual permite su utilización en diversos campos: ling\"uística, verificación de software, teoría de juegos, etc. Es por este motivo que las lógicas modales resultan una elección tan recurrente en ciencias de la computación: en general, poseen un buen balance entre expresividad y comportamiento computacional.

En este trabajo nos dedicaremos al estudio de lógicas modales intuicionistas. En este capítulo, comenzaremos por introducir las nociones básicas de la lógica modal clásica: su sintaxis, su semántica, y una discusión sobre su teoría de prueba, para luego trabajar sobre el caso intuicionista en los capítulos siguientes.

\section{Sintaxis y semántica}
El lenguaje de la lógica modal básica se obtiene extendiendo lógica proposicional clásica con los conectivos modales $\square$ y $\Diamond$, los cuales históricamente representan \textit{necesidad} y \textit{posibilidad} respectivamente. 

\dfn Sea $\mathsf{PROP}$ un conjunto infinito contable de proposiciones atómicas, las fórmulas modales se construyen a partir de la siguiente gramática:
\begin{center}
	 $A ::=  a$ $| $ $\vls-a$ $ | $  $\top$ $|$ $\bot $ $|$ $\vls(A.A)$ $|$ $\vls[A.A]$ $|$  $\square A$ $|$ $\Diamond A$,
\end{center} 
\noindent donde $a \in A$ y $\vls-a$ es su negación.


Siempre asumimos que las fórmulas están en \textit{negation normal form}, esto es, la negación está restringida a los átomos. Cuando escribimos $\neg A$, nos referimos al resultado de computar la \emph{negation normal form}, utilizando las siguientes equivalencias: $\neg a \equiv \vls-a$, $\neg \vls-a \equiv a$, $\neg \neg A $ $\equiv$ $A$, $\neg (\vls(A.B))\equiv \vls[\neg A. \neg B]$ y $\neg \square A \equiv \Diamond \neg A$, donde $\equiv$ denota equivalencia. La implicación material puede ser definida a partir de este conjunto de conectivos como: $A \vljm B \coloneqq \vls[\neg A. B]$. $\top$ and $\bot$ son las constantes usuales denotando verdadero y falso respectivamente.

Como dijimos anteriormente, las fórmulas de lógica modal se interpretan sobre estructuras relacionales. Por razones históricas, dichas estructuras se conocen como modelos de Kripke \cite{kripke1959}. La semántica en términos de estos modelos se conoce como \emph{semántica de Kripke} o por la interpretación original de los operadores modales, como \emph{semántica de mundos posibles}. Intuitivamente, un modelo es un grafo dirigido con etiquetas en los nodos, y las fórmulas modales permiten expresar propiedades de estos grafos. A la estructura relacional, sin considerar las etiquetas en los nodos, se la conoce como \emph{frame}. A continuación introducimos formalmente la definición de frame y de modelo de Kripke.


\dfn(Frame) Un \textit{frame} es un par $\langle W, R\rangle$ donde $W$ es un conjunto no vacío de mundos posibles y $R$ es una relación binaria (mejor conocida como \textit{relación de accesibilidad}) tal que $R \subseteq W\times W$.

Las lógicas modales permiten describir axiomáticamente propiedades estructurales de un modelo, es decir propiedades del frame. Ejemplos de esto pueden ser reflexividad, transitividad, o simetría de la relación de accesibilidad. 

Agregando valuaciones a los mundos, obtenemos los modelos de Kripke:

\dfn(Modelos de Kripke) Un \textit{modelo} $\M$ es una tupla $\langle W, R, V \rangle$, donde $\langle W, R \rangle$ es un frame, y $V : W \rightarrow 2^{\mathsf{PROP}}$ es una función de valuación, que asigna a cada mundo $w$ un conjunto de variables proposicionales que se hacen verdaderas en $w$. Sea $w \in W$, llamamos a $\M, w$ un \emph{pointed model}.

La Figura~\ref{fig:kmodel}  muestra un ejemplo de un modelo de Kripke. Podemos observar que el modelo $\M$ es un grafo con 3 elementos, $\cset{w,v,u}$. 
$w$ está etiquetado con $p$, $u$ con $p$ y $q$, y $v$ no posee etiqueta.
Formalmente, $\M{=}\tup{W,R,V}$, donde $W=\cset{w,v,u}$, $R=\cset{(w,v),(w,u),(v,v),\linebreak (v,u),(u,v)}$, y $V(w){=}\cset{p},V(u)=\cset{p, q}$. 

\begin{figure}[h]
	\begin{center}
		\begin{tikzpicture}[>=latex]
		\node (a0) at (-1,0)   [shape=circle,draw,fill, inner sep=1pt,label=above left:$w$,label=below left:$\cset{p}$] {} ;
		\node (a1) at (1,1)  [shape=circle,draw,fill, inner sep=1pt, label=right:$v$] {} ;
		\node (a2) at (1,-1)  [shape=circle,draw,fill, inner sep=1pt, label=right:$u$,label=below:$\cset{p,q}$] {} ;
		\draw [ar,->] (a0) edge [bend left] (a1);
		\draw [ar,->] (a0) edge [bend right] (a2);
		\path [ar,->] (a1) edge [loop] (b0);
		\draw [ar,->] (a1) edge [bend left] (a2);
		\draw [ar,->] (a2) edge [bend left] (a1);
		\node (a3) at (-1.5,-.9)  [] {$\M$} ;
		\end{tikzpicture}
	\end{center}
	\caption{Ejemplo de un modelo de Kripke.}
	\label{fig:kmodel}
\end{figure}


Una proposición expresada por una fórmula que involucra solo los conectivos lógicos de la lógica proposicional se determina localmente sobre un mundo en particular y es independiente del estado de otros \emph{mundos}. Por otra parte, una proposición expresada por una fórmula que involucra las modalidades depende fundamentalmente del estado de otros \emph{mundos}. En un mundo \emph{w}, la fórmula $\Diamond A$ expresa la proposición de que $A$ es cierta en algún mundo $v$ considerado posible desde el punto de vista de $w$ (técnicamente, la definición de que $v$ es posible de acuerdo con $w$ será modelada por la relación de accesibilidad del modelo). Dualmente, la fórmula $\square A$ expresa la proposición (en $w$) de que $A$ es verdadera en todos los mundos $v$ considerados posibles por $w$. De este modo, el significado de las modalidades $\square$ y $\Diamond$ recibe una lectura basada en la noción primitiva de verdad relativa, es decir, verdad en un mundo.


Los operadores de la lógica modal básica describen propiedades \emph{locales} de los modelos, lo que significa que las fórmulas son evaluadas en algún punto específico. Para eso usamos \emph{pointed models}.

\dfn Sea $\M = \langle W, R, V \rangle $ un modelo, $w \in W$ y $A$ una fórmula, decimos que $\M, w $ satisface $A$ (denotado $\M, w \Vdash A$) si: %\male{se ve feo así! no te parece?}
	$$
	\begin{array}{rcl}
	
 \M, w \Vdash a & \mbox{sii} & a \in V(w)\\
 
	\M, w \Vdash \vls-a& \mbox{sii} & a \not \in V(w)\\
	
	\M, w \Vdash A \vlan B & \mbox{sii} & \M, w \Vdash A\mbox{ y }\M, w \Vdash B \\
	
	\M, w \Vdash A \vlor B & \mbox{sii} & \M, w \Vdash A\mbox{ o }\M, w \Vdash B \\
	
	\M, w \Vdash \square A & \mbox{sii} & \mbox{para todo } v \in W \mbox{ tal que } wRv \mbox{ y } \M, v \Vdash A \\
	
	\M, w \Vdash \Diamond A & \mbox{sii} & \mbox{existe } v \in W \mbox{ tal que } wRv \mbox{ y } \M, v \Vdash A \\
\end{array}$$
%\hspace{9mm}	$\M, w \Vdash a$ si y sólo si $a \in V(w)$

%\hspace{9mm}	$\M, w \Vdash \vls-a$ si y sólo si $a \not \in V(w)$

%\hspace{9mm}	$\M, w \Vdash \vls(A.B)$ si y sólo si $\M, w \Vdash A$ y $\M, w \Vdash B$

%\hspace{9mm} $\M, w \Vdash \vls[A.B]$ si y sólo si $\M, w \Vdash A$ o $\M, w \Vdash B$

%\hspace{9mm} $\M, w \Vdash \square A$ si y sólo si para todo $v \in W$ tal que $(w, v) \in $ R y $\M, v \Vdash A$

%\hspace{9mm} $\M, w \Vdash \Diamond A$ si y sólo si existe un $v \in W$ tal que $(w, v) \in R$ y $\M, v \Vdash A$

\vspace{1mm}

Una fórmula $A$ es satisfacible si existe un pointed model $\M, w$ tal que $\M,w \Vdash A$. Diremos que una fórmula $A$ es válida en un modelo (notación $\M \vDash A$) $\M$ si y sólo si en todos los estados $w \in W$ vale $\M, w \Vdash A$.



Notemos la correspondencia entre $\square$ y el cuantificador $\forall$, así como también la correspondencia entre el operador $\Diamond$ y el cuantificador $\exists$. En particular, podemos identificar a la lógica modal clásica como un fragmento de la lógica de primer orden clásica simplemente internalizando su semántica en $\mathsf{LPO}$.

Como se mencionó anteriormente, en este trabajo nos interesa estudiar los teoremas y sus demostraciones, en particular, de la lógica modal intuicionista. Para ello veremos las demostraciones como objetos formales. Existen diferentes tipos de sistemas de pruebas para un lenguaje lógico, cada uno con ciertas características particulares. Por ejemplo, los sistemas \emph{á la Hilbert} en general consisten de un conjunto de axiomas, y unas pocas reglas que permiten inferir teoremas a partir de axiomas. En contrapartida, en los sistemas \emph{á la Gentzen} predominan las reglas de inferencia por sobre los axiomas. A continuación presentaremos ambos tipos de sistemas para la lógica modal clásica.
% Podemos definir la satisfabilidad de una fórmula de la siguiente manera:

% \dfn(Satisfabilidad y Validez de una fórmula) Una formula $A$ se \emph{satisface} en un modelo $M$=$\langle W, R, V \rangle$ si existe  $w \in W$ tal que $\M, w \Vdash A$. Lo denotaremos con $\M, w\vDash A$. Diremos que una fórmula $A$ es \emph{válida} en un modelo $\M$ si para todo $w \in W$, $\M, w \vDash A$. Lo denotamos con $\M \vDash A$


%%%%%%%%%%%%%%%%%%%%%%%%%
\section{Axiomatizaciones \emph{á la Hilbert}}

Presentaremos a la lógica modal básica y su teoría como la llamada lógica modal $\K$. La misma se obtiene a partir de una axiomatización arbitraria de la lógica proposicional clásica, agregando los siguientes componentes:


\begin{itemize}
	\item la regla de \textit{necesitación:} si $A$ es un teorema de $\K$, entonces $\square A$ también lo es.
	\item el axioma de \textit{distributividad}, normalmente escrito de la siguiente forma:
		\begin{center}
		$\kaxiom$ $\colon$ $\square(A \vljm B) \vljm (\square A \vljm \square B)$.
		\end{center}
\end{itemize}

Es importante mencionar que lo que llamamos un axioma aquí es de hecho, como es estándar en las axiomatizaciones al estilo de Hilbert, un esquema de axiomas, para evitar una mención explícita de la regla de sustitución. Más concretamente, una derivación es una lista de fórmulas que son instancias de algunos esquemas de axiomas dados o deducciones de fórmulas anteriores combinadas usando una regla de inferencia. Además, como estamos aceptando el esquema del axioma $k$ en nuestro sistema base, todas las lógicas que consideramos pertenecen a la familia de las \emph{lógicas modales normales}. Se pueden obtener lógicas modales más fuertes agregando a $\K$ otros axiomas que contienen los conectivos modales. Una clase de axiomas modales que serán de nuestro interés han sido propuestos por Lemmon y Scott en \cite{lemmon1977}. Estos axiomas denominados \emph{de Scott-Lemmon} están definidos esquemáticamente por una 4-upla de números naturales $\langle h, i, j, k \rangle$ como:
\begin{center}
	$\mathsf{\mbox{g}_{hijk}} := \Diamond^{h} \square^{i} A \vljm \square^{j}  \Diamond^{k} A$,
\end{center}
donde $\square^{n}$ denota $n$ ocurrencias de $\square$ ($ \Diamond^{n}$ denota $n$ ocurrencias de $\Diamond$). Notemos que en el caso clásico $\mathsf{\mbox{g}_{hijk}}$ y $\mathsf{\mbox{g}_{jkhi}}$ son equivalentes por De Morgan.

Consideraremos también una subclase particular de estos axiomas llamados \emph{path Scott-Lemmon axioms} que corresponden al caso donde $i + k$ es exactamente igual a 1. Por ejemplo, obtenemos el siguiente axioma:

\begin{center}
		$\mathsf{\mbox{g}_{h1j0}} := \Diamond^{h} \square A \vljm \square^{j} A$
\end{center}

Algunas ocurrencias específicas del axioma de Scott-Lemmon se han estudiando con distintos nombres, como por ejemplo:
  
  $$
\begin{array}{rcl}
	\mathsf{\mbox{g}_{0001}} & \mbox{corresponde al axioma} & \taxiom := A \vljm \Diamond A \\
	
	\mathsf{\mbox{g}_{0011}} & \mbox{corresponde al axioma} & \baxiom := A \vljm \square \Diamond A \\
	
	\mathsf{\mbox{g}_{0101}} & \mbox{corresponde al axioma} & \daxiom := \square A \vljm \Diamond A \\
	
	\mathsf{\mbox{g}_{1002}} & \mbox{corresponde al axioma} & \fouraxiom := \Diamond A \vljm \Diamond \Diamond A \\
	
	\mathsf{\mbox{g}_{1011}} & \mbox{corresponde al axioma} & \fiveaxiom := \Diamond A \vljm \square \Diamond A \\
	
	\mathsf{\mbox{g}_{1111}} & \mbox{corresponde al axioma} & \twoaxiom := \Diamond \square A \vljm \square \Diamond A
\end{array}
$$

Podemos notar que $\taxiom$, $\baxiom$, $\fouraxiom$ y $\fiveaxiom$ son \emph{path Scott-Lemmon axioms}, pero $\daxiom$ y $\twoaxiom$ no lo son.

Seleccionando subconjuntos de estos axiomas nos permite \emph{a priori} definir treinta y dos lógicas modales. Sin embargo, algunas de ellas coinciden. Por ejemplo, los conjuntos $\{ \baxiom, \fouraxiom \}$ y $\{ \taxiom, \fiveaxiom\}$ ambos dan como resultado la lógica modal conocida como $\sfive$. Se obtienen quince lógicas modales distintas que se extienden entre $\K$ y $\sfive$ y se pueden representar en un cubo, como se muestra en la Figura \ref{fig:cubesfive}. Este cubo es conocido como el \emph{cubo $\sfive$}.

\begin{figure}[h]
	\begin{center}
	\begin{tikzpicture}
	[every node/.style={inner sep=1pt,outer sep=0},
	logic/.style={shape=circle,draw}]
	\def\xx{4.5}\def\hxx{2}\def\hhxx{2}
	\def\yy{4}\def\hyy{2}\def\hhyy{1}
	\def\zz{-4}\def\hzz{-2}\def\hhzz{-1}
	\node[logic,label=225:K]   (ik)   at (0, 0, 0)            {} ;
	\node[logic,label=-45:KB]  (ikb)  at (\xx, 0, 0)          {} ;
	\node[logic,label=-45:KB5] (ikb5) at (\xx, 0, \zz)        {} ;
	\node[logic,label=170:K4]  (ik4)  at (0, 0, \zz)          {} ;
	\node[logic,label=180:D]   (id)   at (0, \hyy, 0)         {} ;
	\node[logic,label=135:T]   (it)   at (0, \yy, 0)          {} ;
	\node[logic,label=135:S4]  (is4)  at (0, \yy, \zz)        {} ;
	\node[logic,label=45:S5]   (is5)  at (\xx, \yy, \zz)      {} ;
	\node[logic,label=-45:TB]  (itb)  at (\xx, \yy, 0)        {} ;
	\node[logic,label=0:DB]    (idb)  at (\xx, \hyy, 0)       {} ;
	\node[logic,label=135:D4]  (id4)  at (0, \hyy, \zz)       {} ;
	\node[logic,label=-45:D45] (id45) at (\hxx, \hyy, \zz)    {} ;
	\node[logic,label=-45:K45] (ik45) at (\hxx, 0, \zz)       {} ;
	\node[logic,label=-10:D5]  (id5)  at (\hhxx, \hyy, \hhzz) {} ;
	\node[logic,label=-10:K5]  (ik5)  at (\hhxx, 0, \hhzz)    {} ;
	\draw[line width=.6pt,color=black!40]
	(ik) -- (ik4) -- (id4) (ik4) -- (ik45) ;
	\draw[line width=.7pt,color=black!60]
	(ik45) -- (ikb5)
	(id) -- (id4) -- (is4) (id4) -- (id45) -- (id5) -- (id)
	(ik) -- (ik5) -- (ik45) -- (id45) (ik5) -- (id5)
	(id45) to[bend left] (is5) ;
	\draw[line width=.8pt]
	(ik) -- (ikb) -- (ikb5) -- (is5) -- (is4) -- (it) -- (id) -- (ik)
	(it) -- (itb) -- (is5)
	(itb) -- (idb) -- (ikb)
	(id) -- (idb) ;
	\end{tikzpicture}
	\end{center}
	\caption{Cubo $\sfive$}
	\label{fig:cubesfive}
\end{figure}


%\male{Definir el problema de SAT y VAL?}
%Como se mencionó al comienzo de la sección, la lógica modal tiene mejores propiedades computacionales que la lógica de primer orden. En efecto, una vez introducida la definición de satisfabilidad, queremos destacar que este problema de determinar si una fórmula es satisfacible o no en la lógica modal, es decidible y su complejidad es {\sc PSPACE}-complete.

%\dfn (Validez de una fórmula) Una fórmula $A$ es \emph{válida} en un frame $\F = \langle W, R \rangle$ si para cada función de valuación $V$ se tiene que $\langle W, R, V \rangle \vDash A$. Lo denotamos con $\F \vDash A$.

%Con esta definición presentada anteriormente para \emph{validez}, surgen varias preguntas naturales con respecto a las relaciones entre las lógicas modales y las clases de frames.

 La sintáxis y la semántica descripta en la Sección 2.1 se vinculan mediante el hecho de que la lógica $\K$ es correcta y completa con respecto a la clase de todos los \emph{frames}.

\begin{teo}(\cite{kripke1963}). 
	Una fórmula $A$ es un teorema de $\K$ si y sólo si $A$ es válida en cada frame. 
\end{teo}

Además, el poder de esta construcción es que este enlace no está restringido a $\K$: algunas clases de fórmulas modales corresponden a propiedades específicas de frames. Una forma alternativa, entonces, para obtener lógicas modales más fuertes que $\K$ es restringiendo la clase de frames que queremos considerar, imponiendo algunas restricciones en la relación de accesibilidad.

\section{Axiomatizaciones \emph{á la Gentzen}}
A pesar de que la noción de pruebas dada para un sistema axiomático \emph{á la Hilbert} es claro y amigable, encontrar una prueba para un teorema específico puede resultar muy complicado. Uno de los objetivos del formalismo de cálculo de secuentes propuesto por Gentzen es brindar una forma más intuitiva de estudiar las pruebas. 

En el caso clásico, definimos un \emph{secuente} $\Right = A_{1}, ..., A_{n}$ como un conjunto de fórmulas donde la coma representa la unión. Una \emph{regla de secuentes} es una expresión de la forma:

\begin{center}
$\vlderivation{\vlinf{}{}{S}{S_{1}$ ... $S_{n}}}$,
\end{center}

\noindent tal que $n \geq 0$, y el secuente $S$, la \emph{conclusión}, puede deducirse de los secuentes $S_{1}$ ... $S_{n}$ llamados \emph{premisas}.

Una \emph{derivación}, denotada con $\D$, es construida de acuerdo a estas reglas; tendrá una estructura de un árbol, donde cada arista es un secuente y cada nodo interno es una regla. Una derivación es una \emph{prueba} de un secuente en la raíz, si cada hoja es una regla sin premisas. La altura de una derivación $\D$, denotada por $ht(\D)$, es la altura de $\D$ vista como un árbol, es decir, la longitud del camino más largo en el árbol desde su raíz hasta una de sus hojas.

Diseñar un sistema de secuentes \emph{completo} y \emph{correcto} para una lógica dada significa definir el conjunto correcto de reglas de secuentes, de manera que cada fórmula válida de la lógica en cuestión tenga una prueba en el sistema de secuentes (\emph{completitud}), y que cualquier prueba que pueda construirse en el sistema de secuentes deriva una fórmula válida de la lógica (\emph{correctitud}).

Decimos que una regla es \emph{admisible} por un sistema $\system$ si, siempre que sus premisas son demostrables en $\system$, existe una prueba de su conclusión en $\system$. Decimos que es \emph{derivable} si existe una derivación en $\system$ de sus premisas a su conclusión, posiblemente utilizando premisas varias veces.

\subsection{Sistemas de secuentes etiquetados}

Una vez que la semántica de los \emph{mundos posibles} se estableció como una base sólida para definir lógicas modales, surgió la idea de incorporar estas nociones en la teoría de la prueba de las mismas. Una de las primeras formalizaciones de este tipo fue introducida por Fitch \cite{fitch1948} incluyendo símbolos que representan mundos posibles, dentro del lenguaje de sus pruebas en deducción natural.

Los sistemas de prueba etiquetados han sido propuestos por Gabbay \cite{gabbay1996} en los años 80, como un marco unificador para proporcionar sistemas de prueba para una amplia gama de lógicas. Para las lógicas modales, un ejemplo de ello  son los sistemas de deducción natural etiquetados y sistemas de secuentes etiquetados, tales como los introducidos por Simpson \cite{simpson1994}, Vigano \cite{vigano2013} y Negri \cite{negri2005}. Estos formalismos hacen uso explícito no sólo de etiquetas, sino también de átomos relacionales que se refieren a la relación de accesibilidad de un modelo de Kripke. En esta sección presentaremos el cálculo de secuentes de Negri para la lógica modal clásica.

Los \emph{secuentes etiquetados} están formados por fórmulas etiquetadas de la forma $x \colon A$ y por átomos relacionales de la forma $xRy$, donde $x, y$ pertenecen a un conjunto de variables (llamados etiquetas) y $A$ es una fórmula modal. Un \emph{secuente etiquetado unilateral} es de la forma $\G \Rightarrow \Right$ donde $\G$ denota un conjunto de átomos relacionales y $\Right$ un conjunto de fórmulas etiquetadas. \emph{Labels} será el conjunto de variables llamadas etiquetas.

Un sistema de pruebas simple para la lógica modal clásica $\K$ puede ser obtenido a partir del formalismo presentado en la Figura \ref{fig:labK}.


\begin{figure}[!h]
	\begin{center}
		$\vlinf{\id}{}{\G \Rightarrow \Right, x \colon a, x \colon \vls-a}{}$ \hspace{25mm}	
		$\vlinf{\tolab}{}{\toprule}{}$
		
		\vspace{8mm}
		$\vliinf{\vlab}{}{\G \Rightarrow \Right, x :\vls(A.B)}{\vlabr}{\vlabu}$\hspace{9mm}
		$\vlinf{\olab}{}{\G \Rightarrow \Right, x \colon \vls[A.B]}{\olabr}$
		
		\vspace{8mm}
		\hspace{8mm}$\vlinf{\blab}{y$ fresh$}{\blabr}{\blabu}$ \hspace{9mm} $\vlinf{\dlab}{}{\dlabr}{\dlabu}$
		
		
	\end{center}
	\caption{Sistema $\labK$}
	\label{fig:labK}
\end{figure}

Las reglas de la Figura \ref{fig:labK} hacen uso de etiquetas para expresar explícitamente la semántica de una fórmula en un estado en particular. Expliquemos más en detalle la regla para la conjunción, la regla para el box y la regla para el diamante:

\begin{itemize}
	\item Regla $\vlab$.
	
	La regla propuesta en el sistema $\labK$ para la conjunción es la que se observa a continuación:
	\begin{center}
		$\vliinf{\vlab}{}{\G \Rightarrow \Right, x :\vls(A.B)}{\vlabr}{\vlabu}$
	\end{center}
	Dado $\G$ un conjunto de átomos relacionales y dado $\Right$ un conjunto de fórmulas etiquetadas, esta regla (leyendo desde la conclusión hacia las premisas) semánticamente nos dice que si $A \vlan B$ se satisface en $x$, entonces en $x$ debe valer $A$ y en $x$ también debe ser verdadera $B$.
	
	\item Regla $\blab$.
	
	El caso del $\square$ queda representado en el sistema $\labK$ con la regla que vemos a continuación:
	\begin{center}
		$\vlinf{\blab}{y$ fresh$}{\blabr}{\blabu}$
	\end{center}
	Nuevamente dado $\G$ un conjunto de átomos relacionales y dado $\Right$ un conjunto de fórmulas etiquetadas, la regla para el $\square$ expresa semánticamente que si en $x$ es verdadera la fórmula $\square A$ (conclusión de la regla) entonces si existe una etiqueta (o mundo) $y$ tal que $xRy$, la fórmula $A$ es verdadera en $y$. Cabe destacar que cuando decimos \emph{existe una etiqueta (o mundo) $y$}, no utilizamos el existe como cuantificador, si no que hacemos referencia a $y fresh$. Es decir, $y$ es una nueva etiqueta que no estaba presente en la conclusión de la regla.
	
	\item Regla $\dlab$:
	
	Para representar el operador $\Diamond$, el sistema $\labK$ presenta la siguiente regla:
	
	\begin{center}
		$\vlinf{\dlab}{}{\dlabr}{\dlabu}$
	\end{center}
	
	Al igual que en las reglas anteriores, dado $\G$ un conjunto de átomos relacionales y dado $\Right$ un conjunto de fórmulas etiquetadas, la regla para el $\Diamond$ expresa semánticamente (leyendo desde la conclusión hacia las premisas) que dado mundos $x, y$ tal que $xRy$ entonces en $x$ se satisface $\Diamond A$, tenemos que la fórmula $A$ se hace verdadera en $y$.
\end{itemize}

De esta manera queda reflejado cómo el uso de etiquetas nos provee una correspondencia casi directa entre expresiones sintácticas y su semántica, haciendo la teoría de prueba mucho más intuitiva. Por ejemplo, en \cite{Brauner07} se discuten los beneficios de las etiquetas en la teoría de prueba modal.

Concluímos este capítulo enunciando el siguiente resultado:

\begin{teo}(\cite{negri2005}) 
	Una fórmula $A$ es demostrable en el cálculo $\labK$ si y sólo si $A$ es válida en cada frame.
\end{teo}



\chapter{Lógica modal intuicionista}
\label{cap:logintuicionista}

El intuicionismo surgió como una escuela de matemáticas fundada por el matemático holandés Brouwer \cite{brouwer1920}. Él rechazaba los métodos matemáticos cuya justificación requería apelar a un concepto abstracto de ``verdad". Más precisamente, Brouwer creía que el significado matemático se originaba en el acto humano de ``hacer" \vspace{0.02mm} las matemáticas. Por lo tanto, para Brouwer, un objeto matemático debía ser dado por una construcción, y no hay un sentido abstracto en el que una declaración pueda ser verdadera a menos que tengamos una prueba de ella (o los medios para encontrar una). Además, los pasos dados en cualquier prueba deben ser legítimos de acuerdo con esta rígida interpretación de las matemáticas. Como se ha mencionado anteriormente, tales consideraciones llevaron a Brouwer a rechazar varios principios clásicos como, por ejemplo, el conocido \emph{Tercero excluído}: $\vls[A.\neg A]$ es válida para cualquier proposición $A$.

En la década de 1930, Heyting desarrolló la lógica intuicionista, una lógica que incorpora los principios subyacentes del razonamiento intuicionista. La lógica intuicionista ha sido enormemente exitosa. En primer lugar, se acepta ampliamente que ha logrado su objetivo original de aislar los métodos de prueba intuicionistas aceptables. En segundo lugar, ha logrado proporcionar una base para la investigación meta-matemática de las matemáticas revelando que las matemáticas intuicionistas son un campo de notable coherencia y de belleza matemática, ya sea que uno acepte o no sus principios filosóficos subyacentes. En tercer lugar, hay conexiones profundas con algunos conceptos de reciente crecimiento dentro de las ciencias de la computación (como ejemplo de dos aplicaciones diferentes ver Martin-L\"of \cite{martin1982} y Scott \cite{scott1980}).  La teoría de prueba de la lógica intuicionista también ha encontrado aplicación filosófica reciente. Dummett ha argumentado que la teoría de prueba justifica la lógica intuicionista como la lógica subyacente de una filosofía anti-realista \cite{dummett1991}. Su argumento da lugar a un intuicionismo que es sustancialmente diferente al de Brouwer y que se aplica tanto al razonamiento no matemático como al razonamiento matemático. Para una introducción general a la filosofía y las matemáticas del intuicionismo y la lógica intuicionista, ver Dummett \cite{dummett1977}.

Dada esta breve introducción a la lógica intuicionista, en este trabajo final, como se dijo en la intruducción, nos centraremos en las lógicas modales intuicionistas. Es decir, nos aprovecharemos de la semántica de Kripke subyacente, para combinar operadores modales con el razonamiento intuicionista. Como veremos más adelante, esta combinación resultará natural a la hora de estudiar su teoría de prueba. En esta sección presentaremos los conceptos y las notaciones de la lógica modal intuicionista que serán necesarios para el desarrollo de nuestro trabajo (para mayor detalle, ver \cite{simpson1994}).

\section{Sintaxis y semántica}

En el caso intuicionista, trabajamos con un conjunto  diferente de conectivos. Empezando con un conjunto de proposiciones atómicas que seguimos denotando con $a$, las fórmulas se construyen a partir de la siguiente gramática:


\begin{center}
	$A ::=  a$ $| $$\top$ $|$ $\bot $ $|$ $\vls(A.A)$ $|$ $\vls[A.A]$ $|$ $A \vljm A$ $|$ $\square A$ $|$ $\Diamond A$ 
\end{center}

Cuando escribimos $\neg A$, queremos representar $A \vljm \bot$.

La semántica de Kripke para la lógica modal intuicionista combina la semántica de Kripke para la lógica proposicional intuicionista con la de la lógica modal clásica, utilizando dos relaciones distintas en el conjunto de mundos. Para introducir la semántica que es de nuestro interés introducimos las siguientes definiciones:



\dfn(Frame bi-relacional) Un \emph{frame bi-relacional} $\F$ es una tupla $\langle W, R, \le \rangle$ donde $W$ es un conjunto no vacío y $R$, $\le$ son dos relaciones binarias $R$ $\subseteq$ $W \times W$ y $\le$ $\subseteq$ $W \times W$. Insistimos además que $\le$ sea un pre-orden (es decir, reflexividad y transitividad) que satisface las siguientes condiciones:\\

(F1) Para todo $u, v, v'$ $\in W$, si $uRv$ y $v \le v'$ entonces existe $u'$ tal que $u \le u'$ y $u'Rv'$.\\


\begin{figure}[h]
	\begin{center}
		\begin{tikzpicture}[>=latex]
		\node (a0) at (-1,0)   [shape=circle,draw,fill, inner sep=1pt,label=left:$u$] {} ;
		\node (a1) at (-1,2)  [shape=circle,draw,fill, inner sep=1pt, label=left:$u'$] {} ;
		\node (a2) at (1,0)  [shape=circle,draw,fill, inner sep=1pt, label=right:$v$] {} ;
		\node (a3) at (1,2)  [shape=circle,draw,fill, inner sep=1pt, label=right:$v'$] {} ;
		\draw [dotted,->] node[left] at (-1,1){$\le$}(a0) edge [ left] (a1);
		\draw [ar,->] node[above] at (0,2){$R$}(a0) edge [ right] (a2);
		\draw [dotted, ->] node[right] at (1,1){$\le$}(a1) edge [ left] (a3);
		\draw [ar,->] node[above] {$R$} (a2) edge [ left] (a3);
 ;
		\end{tikzpicture}
	\end{center}
\end{figure}

(F2) Para todo $u', u, v$ $\in W$, si $u \le u'$ y $u R v$ entonces existe $v'$ tal que $u'Rv'$ y $v\le v'$.\\


\begin{figure}[h]
	\begin{center}
		\begin{tikzpicture}[>=latex]
		\node (a0) at (-1,0)   [shape=circle,draw,fill, inner sep=1pt,label=left:$u$] {} ;
		\node (a1) at (-1,2)  [shape=circle,draw,fill, inner sep=1pt, label=left:$u'$] {} ;
		\node (a2) at (1,0)  [shape=circle,draw,fill, inner sep=1pt, label=right:$v$] {} ;
		\node (a3) at (1,2)  [shape=circle,draw,fill, inner sep=1pt, label=right:$v'$] {} ;
		\draw [ar,->] node[left] at (-1,1){$\le$}(a0) edge [ left] (a1);
		\draw [ar,->] node[above] at (0,2){$R$}(a0) edge [ right] (a2);
		\draw [dotted, ->] node[right] at (1,1){$\le$}(a1) edge [ left] (a3);
		\draw [dotted,->] node[above] {$R$} (a2) edge [ left] (a3);
		;
		\end{tikzpicture}
	\end{center}
\end{figure}

De acuerdo al paradigma de Kripke para la lógica intuicionista, los hechos atómicos se acumulan a medida que ascendemos el orden parcial. Es decir, podría razonablemente sostenerse que el hecho de que un mundo $w$ accedía a otro mundo $v$, es un tipo razonable de hecho atómico que debería persistir. Así, cualquier mundo $ w \le w'$ debería también, en efecto, ver $v$; pero es razonable esperar que también hayamos acumulado más datos sobre $v$ que, por lo tanto, pueden haber 'evolucionado' en algún mundo $v \le v'$. Formalizando estas consideraciones, llegamos a obtener la condición (F1). Un argumento dual partiendo de que el mundo $v$ está viendo al mundo $w$ justifica la condición (F2).

\dfn(Modelo bi-relacional) Un \emph{modelo bi-relacional} $\M$ es una cuatro-upla $\langle W, R, \break\le, V \rangle$ donde $\langle W, R, \le \rangle$ es un frame bi-relacional y $V$ una función de valuación monótona $V: W$$\rightarrow$ $2^{\mathsf{PROP}}$ tal que es una función que asigna a cada mundo $w$ el subconjunto de átomos proposicionales que son verdaderos en $w$, sujeto a:

\begin{center}
	$w \le w'$ $\Rightarrow$ $V(w)$ $\subseteq$ $V(w')$
\end{center}

De esta manera, obtenemos el \emph{lema de monotonía}:

\begin{lemma} (Lema de monotonía) 
	Si $w \le w'$ y $\M, w \Vdash A$ entonces $\M, w' \Vdash A$.
\end{lemma}

Como vimos en el caso de la lógica clásica, escribimos $\M, w \Vdash a$ si y sólo si $a \in V(w)$ y lo extendemos con inducción a todas las fórmulas siguiendo las reglas para los modelos de Kripke tanto intuicionistas como clásicos:

\dfn Sea $\M = \langle W, R, \le, V \rangle $ un modelo bi-relacional, $w \in W$ y $A$ una fórmula, decimos que $\M,w $ satisface $A$ (denotado $\M, w \Vdash A$) si:$$
\begin{array}{rcl}

%\M, w \Vdash a & \mbox{si y sólo si} & a \in V(w)\\

%\M, w \Vdash \vls-a& \mbox{si y sólo si} & a \not \in V(w)\\

\M, w \Vdash A \vlan B & \mbox{sii} & \M, w \Vdash A\mbox{ y }\M, w \Vdash B \\

\M, w \Vdash A \vlor B & \mbox{sii} & \M, w \Vdash A\mbox{ o }\M, w \Vdash B \\

\M, w \Vdash A \vljm B & \mbox{sii} & \mbox{para todo } w' \mbox{ con } w \le w', \mbox{ si } \M, w' \Vdash A \mbox{ entonces } \M, w' \Vdash B \\

\M, w \Vdash \square A & \mbox{sii} & \mbox{para todo } w' \mbox{ y } u \mbox{ con } w \le w' \mbox{ y } w'Ru \mbox{ entonces } \M, u \Vdash A \\

\M, w \Vdash \Diamond A & \mbox{sii} & \mbox{existe } u \in W \mbox{ tal que } wRu \mbox{ y } \M, u \Vdash A \\

\end{array}$$

 %\male{de nuevo, queda feo, no te parece?}
 %\hspace{9mm} $\M, w \Vdash \vls(A.B)$ si y sólo si $\M, w \Vdash A$ y $\M, w \Vdash B$

%\hspace{9mm} $\M, w \Vdash \vls[A.B]$ si y sólo si $\M, w \Vdash A$ o $\M, w \Vdash B$


%\hspace{9mm} $\M, w \Vdash A \vljm B$  si y sólo si para todo $w'$ con $w \le w'$, si $\M, w' \Vdash A$ entonces $\M, w' \Vdash B$.


%\hspace{9mm} $\M, w \Vdash \square A$ si y sólo si para todo $w'$ y $u$ con $w \le w'$ y $w'Ru$ entonces 
%se tiene que $\M, u \Vdash A$

%\hspace{9mm} $\M, w \Vdash \Diamond A$ si y sólo si existe un mundo $u \in W$ tal que $wRu$ y $\M, u \Vdash A$.

Escribimos $\M, w \not \Vdash A$ si no se da el caso de $\M, w\Vdash A$, en particular $\M, w ,\not \Vdash \bot $.\\


\dfn (Satisfabilidad y Validez de una fórmula) Una fórmula $A$ es \textit{satisfacible} en un modelo $\M = \langle W, R, \le, V \rangle$, si existe $w \in W$ tal que $\M, w \Vdash A$. Diremos que una fórmula $A$ es \emph{válida} en un modelo $\M$ si para todo $w \in W$, $\M, w \Vdash A$. Lo denotamos con $\M \vDash A$.

%%%%%%%

\section{Axiomatizaciones para la lógica modal intuicionista}
La lógica modal intuicionista $\IK$ (una variante intuicionista de la lógica modal $\K$) es una extensión de la lógica proposicional intuicionista \textbf{IPL} \cite{vandalen2004} compuesta por los siguientes axiomas:

\begin{itemize}
	
	\item{THEN-1}: $A  \vljm (B \vljm A)$
	
	\item{THEN-2}: $(A \vljm (B \vljm C)) \vljm ((A \vljm B) \vljm (A \vljm C))$
	
	\item{AND-1}: $\vls(A.B)\vljm A$
	
	\item{AND-2}: $\vls(A.B) \vljm B$
	
	\item{AND-3}: $A \vljm (B \vljm (\vls(A.B)))$
	
	\item{OR-1}: $A \vljm \vls[A.B]$
	
	\item{OR-2}: $B \vljm \vls[A.B]$
	
	\item{OR-3}: $(A \vljm C) \vljm ((B \vljm C) \vljm (\vls[A.B] \vljm C))$
	
	\item{FALSE}: $\bot \vljm A$

\end{itemize}

Una vez introducida la lógica \textbf{IPL}, obtenemos $\IK$ añadiendo los siguientes axiomas y reglas:

\begin{itemize}
	\item{la \emph{regla de necesitación} ($\mathsf{nec}$)}: si $A$ es un teorema, entonces también $\square A$ es un teorema.
	\item{las siguientes \emph{cinco variantes del axioma de distributividad $\kaxiom$}}:
	
	$\kaxiom_{1}$: $\square(A \vljm B) \vljm (\square A \vljm \square B)$
	
	$\kaxiom_{2}$: $\square (A \vljm B) \vljm (\Diamond A \vljm \Diamond B)$
	
	$\kaxiom_{3}$: $\Diamond (\vls[A.B]) \vljm (( \vls [\Diamond A. \Diamond B]))$
	
	$\kaxiom_{4}$: $(\Diamond A \vljm \square B) \vljm \square(A \vljm B)$
	
	$\kaxiom_{5}$: $\Diamond \bot \vljm \bot$ 
	
\end{itemize} 

%El sistema modal intuicionista más básico que uno puede pensar sería considerar únicamente la modalidad $\square$ según lo establecido por el axioma $k$, mejor conocido en esta sección como $k_{1}$, obteniendo así el sistema \textbf{IPL+nec+$k_{1}$}. Aunque de aquí no podríamos tener ningún tipo de información acerca de la modalidad $\Diamond$.
 El sistema modal intuicionista más básico que podría ser considerado, consiste únicamente de la modalidad $\square$, axiomatizada por el axioma $\kaxiom_{1}$. Sin embargo, dicho sistema no contempla la modalidad $\Diamond$.
Fitch \cite{fitch1948} fue el primero en proponer un sistema intuicionista para trabajar con $\Diamond$ obteniéndolo a partir de \textbf{IPL+nec+$\kaxiom_{1}$+$\kaxiom_{2}$}, el cual muchas veces es conocido como \textbf{CK} por \emph{lógicas modales constructivas} (en inglés \emph{constructive modal logics}). En \cite{wijesekera1990} se introdujo un sistema intuicionista que no contempla la distribuividad de $\Diamond$ sobre $\vlor$; el mismo está compuesto por \textbf{IPL+nec+$\kaxiom_{1}$+$\kaxiom_{2}$+$\kaxiom_{5}$}.
%: luego obtenemos \textbf{CK} de constructive y de la anteriormente mencioada lógica modal K). Wijekesera \cite{wijesekera1990} propuso un sistema intuicionista que agregaba al sistema de Fitch el axioma $k_{5}$, que establece que $\Diamond$ distribuye sobre disyunciones 0-aria, pero no asumió que siempre distribuiría sobre disyunciones binarias; el sistema que él propuso entonces fue \textbf{IPL+nec+$k_{1}$+$k_{2}$+$k_{5}$}. 
Estos sistemas fueron diseñados para algunas aplicaciones concretas, tal como analizar algunos sistemas de tipos \cite{benton1998} o para razonar sobre los estados de una máquina bajo información parcial  \cite{wijesekera2005}. Sin embargo, desde un punto de vista estrictamente lógico, no resulta satisfactorio, ya que la adición del principio del Tercero Excluído al sistema no produce la lógica modal clásica $\K$, es decir, en este caso no es posible recuperar la dualidad de De Morgan entre $\square$ y $\Diamond$.

La axiomatización que es ahora generalmente aceptada como \emph{lógica modal intuicionista} denotada como $\IK$ fue introducida por Plotkin y Stirling \cite{plotkin1986} y es equivalente a la propuesta por Fischer-Servi \cite{servi1984} y por Ewald \cite{ewald1986}. Como se mencionó anteriormente, consiste de \textbf{IPL}, la regla de necesitación, y los axiomas $\kaxiom_{1}$, $\kaxiom_{2}$, $\kaxiom_{3}$, $\kaxiom_{4}$, $\kaxiom_{5}$. 

Es posible demostrar que el sistema $\IK$ es correcto y completo con respecto a la clase de modelos indicada.


%\dfn (Validez de una fórmula) Una fórmula $A$ es \textit{válida} en un frame $\F = \langle W, R, \le \rangle$, si para toda valuación $V$ se tiene que $A$ es satisfacible en $\langle W, R, \le, V \rangle$.

\begin{teo}
	 (\cite{servi1984,plotkin1986}) Una fórmula $A$ es un teorema de $\IK$ si y sólo si $A$ es válida en cada frame bi-relacional.
\end{teo}

Para culminar, notar el paralelismo establecido entre las diferentes axiomatizaciones para la lógica $\K$ (Capítulo 2). En este capítulo, presentamos un sistema \emph{á la Hilbert} para la lógica modal intuicionista. Nuestra contribución consiste en la definición de un sistema \emph{a la Gentzen} para lógica modal intuicionista, y al estudio de las relaciones con otros sistemas existentes. Los siguientes capítulos estarán dedicados a la presentación de dichas contribuciones.
%Siguiendo Getnzen, un primer intento de cálculo de secuentes para la lógica modal intuicionista \textbf{IK} sería considerar una versión bilateral del cálculo de secuentes clásico para la lógica modal \textbf{K} con la restricción de que sólo una fórmula puede aparecer en el lado derecho. Un cálculo de secuentes intuicionista es un conjunto múltiple de fórmulas 
%\chapter{Sistemas de pruebas etiquetados}

Una vez que la semántica de los \emph{posibles mundos} se estableció como una base sólida para definir lógicas modales, surgió la idea de incorporar estas nociones en la teoría de la prueba de las lógicas modales. Fitch parece haber sido el primero en formalizarlo, directamente, incluyendo símbolos que representan mundos en el lenguaje de sus pruebas en deducción natural.

Los sistemas de prueba etiquetados han sido propuestos por Gabbay \cite{gabbay1996} en los años 80 como un marco unificador a través de la teoría de prueba con el fin de proporcionar sistemas de prueba para una amplia gama de lógicas. Para las lógicas modales, un ejemplo de ello  son, los sistemas de deducción natural etiquetados y sistemas de secuentes etiquetados, tales como los introducidos por Simpson \cite{simpson1994}, Vigano \cite{vigano2013} y Negri \cite{negri2005}. Estos formalismos hacen uso explícito no sólo de las etiquetas, sino también de los átomos relacionales que se refieren a la relación de accesibilidad de un modelo de Kripke. En esta sección presentaremos el cálculo de secuentes de Negri para la lógica modal clásica.

Los \emph{secuentes etiquetados} están formados por fórmulas etiquetadas de la forma $x \colon A$ y por átomos relacionales de la forma $x$R$y$, donde $x, y$ pertenecen a un conjunto de variables (llamados etiquetas) y $A$ es una fórmula modal. Un \emph{secuente etiquetado unilateral} es de la forma $\G \Rightarrow \Right$ donde $\G$ denota un conjunto de átomos relacionales y $\Right$ un conjunto de fórmulas etiquetadas.

Un sistema de pruebas simple para la lógica modal clásica K puede ser obtenido a partir del siguiente formalismo como se ve en la siguiente figura:

\vspace{3mm}

\begin{figure}[h]
	\begin{center}
			$\vlinf{\id}{}{\G \Rightarrow \Right, x \colon a, x \colon \vls-a}{}$ \hspace{25mm}	
			$\vlinf{\tolab}{}{\toprule}{}$
		
			
			$\vliinf{\vlab}{}{\G \Rightarrow \Right, x :\vls(A.B)}{\vlabr}{\vlabu}$\hspace{9mm}
			$\vlinf{\olab}{}{\G \Rightarrow \Right, x \colon \vls[A.B]}{\olabr}$
			
		\hspace{8mm}$\vlinf{\blab}{y$ fresh$}{\blabr}{\blabu}$ \hspace{9mm} $\vlinf{\dlab}{}{\dlabr}{\dlabu}$
			

	\end{center}
	\caption{Sistema labK}
\end{figure}

\vspace{5mm}

\begin{teo}(\cite{negri2005}) 
	Una fórmula $A$ es demostrable en el cálculo labK si y sólo si $A$ es válida en cada frame.
\end{teo}

\newpage
% \section{El problema principal}

Como se dijo anteriormente, nuestro trabajo consiste en la definición de un sistema de prueba etiquetado para lógicas modales intuicionistas. La extensión directa de la deducción etiquetada al entorno intuicionista consiste en la utilización de dos tipos de átomos relaciones: uno para la relación de accesibilidad modal representada con $R$, y otro para la relación intuicionista o mejor conocida como relación de preorden $\le$. Esto resulta interesante ya que nos permite poner el sistema de prueba en estrecha relación con la semántica birelacional de Kripke que se detalló en el Capítulo~\ref{cap:logintuicionista}.

En \cite{negri2005} se introdujo un cálculo de secuentes etiquetado para la lógica modal básica. En este trabajo extenderemos dicho sistema con el objetivo de capturar lógicas modales intuicionistas. Además de las fórmulas etiquetadas de la forma $x \colon A$, y de las expresiones $xRy$, capturando la relación de accesibilidad (donde $x, y$ pertenecen a un conjunto de etiquetas, y $A$ es un fórmula), utilizaremos expresiones del tipo $x \le y$ para capturar la relación de preorden intuicionista. De una manera más formal introducimos la siguiente definición:

%La idea es extender un cálculo de secuentes etiquetado con un símbolo de relación de preorden con el objetivo de capturar lógicas modales intuicionistas; esto quiere decir, buscamos definir secuentes etiquetados intuicionistas a partir de fórmulas etiquetadas $x \colon A$, de la relación de accesibilidad modal $x$R$y$, y la nueva relación de preorden de la forma $x \le y$, donde $x, y$ pertenecen a un conjunto de etiquetas y $A$ es una fórmula modal intuicionista. El cálculo de secuentes etiquetado en el cual nos basamos fue el propuesto por Negri para la lógica modal básica \cite{negri2005}.



\dfn{Un \emph{secuente etiquetado intuicionista} es de la forma $\G$, $\Left \Rightarrow \Right$ donde $\G$ denota un conjunto de átomos relacionales y de preorden, y  $\Left$ y $\Right$ son conjuntos de fórmulas etiquetadas. }

\vspace{2mm}

De esta manera obtenemos el sistema de prueba $\labIKh$ para lógicas modales intuicionistas en este formalismo, y podemos demostrar el siguiente teorema:

\begin{quote}
 \emph{Una fórmula $A$ es demostrable en el cálculo $\labIKh$ si y sólo si $A$ es válida en cada frame bi-relacional.}
\end{quote}

La dirección de izquierda a derecha de la sentencia anterior indica que el cálculo es \emph{correcto}: las reglas me permiten inferir solo fórmulas válidas. Por otro lado, la implicación de derecha a izquierda se conoce como \emph{completitud}: todos los teoremas son demostrables en el cálculo.

En la siguiente sección, se presenta el nuevo sistema de cálculo de secuentes etiquetados para las lógicas modales intuicionistas. 


% \section{El sistema $\labIKh$}
En esta sección buscamos resolver el problema recientemente propuesto a partir del sistema de prueba etiquetado que se observa en la Figura \ref{fig:labIKheart}. Como se dijo anteriormente, para obtener este nuevo sistema, se extendió el cálculo de secuentes etiquetado propuesto por Negri para la lógica modal básica agregando un símbolo de relación de pre-orden. En las siguientes secciones desarrollamos en detalle las reglas que forman parte de este sistema de secuentes etiquetado para la lógica modal intuicionista llamado $\labIKh$.

\subsection{Condiciones para frames}

Para poder capturar la lógica modal intuicionista, en primer lugar, buscamos responder a los requisitos de las definiciones presentadas en el Capítulo 3. La semántica de la lógica modal intuicionista habla de un modelo bi-relacional el cual esta compuesto por dos relaciones binarias $R$ y $\le$, donde exigimos que $\le$ sea una relación de pre-orden, es decir, que sea reflexiva y transitiva. Por lo tanto, el sistema $\labIKh$ debe tener reglas que caractericen tales condiciones (estas reglas se representan con \emph{refl} y \emph{trans}). También debe poseer reglas que caractericen las condiciones F1 y F2 (en el sistema $\labIKh$ estas reglas son $\fone$ y $\ftwo$). Por otra parte, recordemos que la definición de la función de valuación monótona $V$ nos introduce el \emph{lema de monotonía} que será capturado por la regla $\ids$ del sistema. Estas reglas son las siguientes:

\begin{center}
	$\vlinf{\refl}{}{\G, \Left \Rightarrow \Right}{\G, x\le x, \Left \Rightarrow \Right}$\hspace{8mm}
	$\vlinf{\trans}{}{\G, x \le y, y \le z, \Left \Rightarrow \Right}{\G, x \le y, y \le z, x \le z, \Left \Rightarrow \Right}$
	
	\vspace{6mm}
	
	$\vlinf{\ids}{}{\G, \Left,x \le y, x \colon a \Rightarrow \Right, y \colon a}{}$
	
	\vspace{6mm}
	
	$\vlinf{\fone}{$ $u$ fresh$}{\G, \Left, xRy, y \le z \Rightarrow \Right}{\G, \Left, xRy, y \le z, x \le u, uRz \Rightarrow \Right}$\hspace{4mm}$\vlinf{\ftwo}{u$ fresh$}{\G, \Left, xRy,x \le z \Rightarrow \Right}{\G, \Left, xRy, x \le z, y \le u, zRu \Rightarrow \Right }$
		
\end{center}

\bigskip

A modo de ejemplo, explicaremos en detalle las reglas $\fone$ y $\ftwo$:

\begin{itemize}
		\item Regla $\fone$.
		
		La regla presentada para el cumplimiento de una de las condiciones ($\fone$) de la relación de preorden $\le$ es la siguiente:
		
		\begin{center}
			$\vlinf{\fone}{$ $u$ fresh$}{\G, \Left, xRy, y \le z \Rightarrow \Right}{\G, \Left, xRy, y \le z, x \le u, uRz \Rightarrow \Right}$
		\end{center}
		
		Dado $\G$ un conjunto de átomos relacionales y de pre-orden, y dados $\Left$ y $\Right$ conjuntos de fórmulas etiquetadas, la regla en cuestión expresa la semántica de la condición $\fone$. Leyendo desde la conclusión hacia la premisa tenemos etiquetas $x, y$ tal que $xRy$ y $y \le z$, entonces existe un mundo $u$ tal que $x \le u$ y $uRz$.
		
		\item Regla $\ftwo$:
		
		Como vimos anteriormente, la regla para el cumplimiento de F2 se presenta de la siguiente manera en $\labIKh$:
		
		\begin{center}
			$\vlinf{\ftwo}{u$ fresh$}{\G, \Left, xRy,x \le z \Rightarrow \Right}{\G, \Left, xRy, x \le z, y \le u, zRu \Rightarrow \Right }$
		\end{center}
		
		Al igual que con F1, sea $\G$ un conjunto de átomos relacionales y de pre-orden, y dados $\Left$ y $\Right$ conjuntos de fórmulas etiquetadas, esta regla expresa semánticamente desde su conclusión hace su premisa, que dadas las etiquetas $x, y$ y $z$, si $xRy$ y $x \le z$, entonces hay una etiqueta (o mundo) $u$ (fresh) tal que $y \le u$ y $zRu$.
		
\end{itemize}

\subsection{Conectivos lógicos de la lógica proposicional intuicionista}

Para seguir construyendo un sistema completo y correcto etiquetado para la lógica modal intuicionista, en esta sección presentamos las reglas que expresan la semántica de los conectivos lógicos usuales. Es decir, presentamos reglas para la conjunción y la disjunción que forman parte del sistema $\labIKh$. Las mismas se presentan a continuación:

\begin{center}
		\hspace{4mm}$\vlinf{\svlef}{}{\G,\Left, x \colon \vls(A.B) \Rightarrow \Right}{\conjlef}$\hspace{10mm}
		$\vliinf{\svrig}{}{\G,\Left \Rightarrow \Right, x \colon \vls(A.B)}{\conjrig}{\conjrigh}$
		
		
		\vspace{5mm}
		
		$\vliinf{\solef}{}{\G, \Left, x \colon \vls[A.B] \Rightarrow \Right}{\G, \Left, x   \colon   A \Rightarrow \Right}{\G, \Left, x   \colon   B \Rightarrow \Right}$\hspace{10mm}
		$\vlinf{\sorig}{}{\G, \Left \Rightarrow \Right, x \colon \vls[A.B]}{\G, \Left \Rightarrow \Right, x   \colon   A, x   \colon   B}$
\end{center}

También el sistema $\labIKh$ debe capturar las constantes proposicionales $\top$ y $\bot$. Para ello, incorpora las siguientes dos reglas al sistema:

\begin{center}
	$\vlinf{\sbot}{}{\G,\Left, x \colon \bot \Rightarrow \Right}{}$\hspace{18mm}
	$\vlinf{\Stop}{}{\G, \Left \Rightarrow \Right, x \colon \top}{}$
\end{center}

A continuación explicaremos en detalle la regla de conjunción izquierda $\svlef$ y la regla de disjunción derecha $\sorig$:

\begin{itemize}
	\item Regla $\svlef$:
	
	El sistema $\labIKh$ presenta la siguiente regla para la conjunción en el lado izquierdo:
	
	\begin{center}
		$\vlinf{\svlef}{}{\G,\Left, x \colon \vls(A.B) \Rightarrow \Right}{\conjlef}$
	\end{center}
	
	Al igual que las reglas presentadas en la sección anterior $\G$ es un conjunto de átomos relacionales y de pre-orden, y $\Left$ y $\Right$ son conjuntos de fórmulas etiquetadas. Desde la conclusión hacia la premisa de esta regla, semánticamente nos dice que si en $x$ se satisface la fórmula $A \vlan B$, entonces en $x$ se satisface la fórmula $A$ y también se satisface la fórmula $B$.
	
	\item Regla $\sorig$:
	
	Sea la regla presentada en $\labIKh$ para la disjunción en el lado derecho:
	
	\begin{center}
		$\vlinf{\sorig}{}{\G, \Left \Rightarrow \Right, x \colon \vls[A.B]}{\G, \Left \Rightarrow \Right, x   \colon   A, x   \colon   B}$
	\end{center}
	
	Tenemos que, leyendo desde la conclusión hacia la premisa, si en $x$ se satisface $A \vlor B$ (del lado derecho), entonces en $x$ es verdadera la fórmula $A$ y también la fórmula $B$.
	
\end{itemize}

\subsection{Capturando los operadores modales}

En esta sección, introduciremos las reglas que expresan la semántica de los operadores modales $\square$ y $\Diamond$. El sistema $\labIKh$ presenta las siguientes cuatro reglas (dos para el $\square$ y dos para el $\Diamond$): 

\begin{center}
	\hspace{5mm}$\vlinf{\sbl}{}{\G, \Left, x \le y, yRz, x \colon \square A \Rightarrow \Right}{\G,\Left, x \le y, yRz, x \colon \square A, z \colon A \Rightarrow \Right}$\hspace{10mm}$\vlinf{\sbr}{$ $y, z$ fresh$}{\G, \Left \Rightarrow \Right, x \colon \square A}{\G, \Left, x \le y, yRz \Rightarrow \Right, z \colon A}$
	
	
	\vspace{5mm}
	
	$\vlinf{\sdl}{$ $y$ fresh $}{\G, \Left, x \colon \Diamond A \Rightarrow \Right}{\G, \Left, xRy, y \colon A \Rightarrow \Right}$\hspace{10mm}
	$\vlinf{\sdr}{}{\G, \Left, xRy \Rightarrow \Right, x \colon \Diamond A}{\G, \Left, xRy \Rightarrow \Right, x \colon \Diamond A, y \colon A}$
	
\end{center}

Expliquemos más en detalle algunas de ellas:

\begin{itemize}
	\item Regla $\sbl$.
	
	Para capturar el $\square$ del lado izquierdo, en nuestro sistema introducimos la siguiente regla:
	\begin{center}
		$\vlinf{\sbl}{}{\G, \Left, x \le y, yRz, x \colon \square A \Rightarrow \Right}{\G,\Left, x \le y, yRz, x \colon \square A, z \colon A \Rightarrow \Right}$
	\end{center}
	Nuevamente dado $\G$ un conjunto de átomos relacionales y de pre-orden, y sean $\Left$ y $\Right$ conjuntos de fórmulas etiquetadas. Esta regla semánticamente expresa, desde su conclusión hacia su premisa, que dado $x, y$ y $z$ tal que $x \le y$ y $yRz$ donde en $x$ se hace verdadera la fórmula $\square A$, entonces en $z$ se hace válida la fórmula $A$.
	
	\item Regla $\sdr$.
	
	Sea la regla presentada en $\labIKh$ para $\Diamond$ en el lado derecho:	
	\begin{center}
		$\vlinf{\sdr}{}{\G, \Left, xRy \Rightarrow \Right, x \colon \Diamond A}{\G, \Left, xRy \Rightarrow \Right, x \colon \Diamond A, y \colon A}$
	\end{center}
	
	Semánticamente, el diamante establece que en un modelo $\M$ y en un mundo $v$ vale $\Diamond A$ si y sólo si existe un mundo $u$ tal que $vRu$ y en $u$ la fórmula $A$ se satisface. Para capturar esta definición semántica, la regla introducida (leyendo desde la conclusión hacia la premisa) nos dice que, si $xRy$ y en $x$ se satisface $\Diamond A$, luego en $y$ la fórmula $A$ tiene que ser verdadera.
	
\end{itemize}

\subsection{Implicación intuicionista}

Por último, nos resta ver que el sistema $\labIKh$ captura la semántica del operador de implicación. Es por ello que este sistema de secuentes etiquetado para que sea completo necesita de las siguientes dos reglas:

\begin{center}
		$\vlinf{\sir}{$ $y$ fresh$}{\G, \Left \Rightarrow \Right, x \colon A \vljm B}{\G, \Left, x \le y, y \colon A \Rightarrow \Right, y \colon B}$
		
		
		\vspace{5mm}
		
		
		$\vliinf{\sil}{}{\G, \Left, x \le y, x \colon A \vljm B \Rightarrow \Right}{\G, \Left, x \le y, x \colon A \vljm B \Rightarrow \Right, y \colon A}{\G, \Left, x \le y, x \colon A \vljm B, y \colon B \Rightarrow \Right}$
\end{center}

Veamos en detalle el significado de la regla de implicación del lado derecho:

\begin{itemize}
	\item Regla $\sir$.
	
	El sistema $\labIKh$ presenta la siguiente regla para la implicación en el lado derecho:
	
	\begin{center}
		$\vlinf{\sir}{$ $y$ fresh$}{\G, \Left \Rightarrow \Right, x \colon A \vljm B}{\G, \Left, x \le y, y \colon A \Rightarrow \Right, y \colon B}$
	\end{center}
	
	Dado $\G$ un conjunto de átomos relacionales y de preorden, y dados $\Left$ y $\Right$ conjuntos de fórmulas etiquetadas, esta regla (leyendo desde la conclusión hacia la premisa) semánticamente nos dice que si $A \vljm B$ se satisface en $x$, entonces existe un nuevo (mundo) $y$ tal que $x \le y$ y en $y$ vale $A$, luego en $y$ también $B$ es válida.
	
	
\end{itemize}

%\subsection{\textbf{¿Por qué necesitamos dos reglas para cada operador?}}

\subsection{Uniendo todas las reglas: sistema $\labIKh$}
Como pudimos ver a lo largo de las subsecciones anteriores, el sistema $\labIKh$ tiene dos reglas para cada operador. Esto se debe a que la sintaxis para la lógica modal intuicionista no presenta la negación $\neg$, por lo que los operadores $\Diamond$ y $\square$ dejan de ser operadores duales como lo eran en el caso clásico.

En la Figura \ref{fig:labIKheart} se puede ver el sistema de secuentes etiquetado $\labIKh$ completo con cada una de las reglas mencionadas anteriormente.

\begin{figure}[!h]
	\small
	\begin{center}
			
			$\vlinf{\sbot}{}{\G,\Left, x \colon \bot \Rightarrow \Right}{}$\hspace{6mm}
			$\vlinf{\ids}{}{\G, \Left,x \le y, x \colon a \Rightarrow \Right, y \colon a}{}$\hspace{6mm}
			$\vlinf{\Stop}{}{\G, \Left \Rightarrow \Right, x \colon \top}{}$
		
		\vspace{5mm}
			
			$\vlinf{\svlef}{}{\G,\Left, x \colon \vls(A.B) \Rightarrow \Right}{\conjlef}$\hspace{10mm}
			$\vliinf{\svrig}{}{\G,\Left \Rightarrow \Right, x \colon \vls(A.B)}{\conjrig}{\conjrigh}$

		
		\vspace{5mm}
			
			$\vliinf{\solef}{}{\G, \Left, x \colon \vls[A.B] \Rightarrow \Right}{\G, \Left, x   \colon   A \Rightarrow \Right}{\G, \Left, x   \colon   B \Rightarrow \Right}$\hspace{10mm}
			$\vlinf{\sorig}{}{\G, \Left \Rightarrow \Right, x \colon \vls[A.B]}{\G, \Left \Rightarrow \Right, x   \colon   A, x   \colon   B}$

		
		\vspace{5mm}
			
			$\vlinf{\sir}{$ $y$ fresh$}{\G, \Left \Rightarrow \Right, x \colon A \vljm B}{\G, \Left, x \le y, y \colon A \Rightarrow \Right, y \colon B}$
		
		
		\vspace{5mm}
		
		
		$\vliinf{\sil}{}{\G, \Left, x \le y, x \colon A \vljm B \Rightarrow \Right}{\G, \Left, x \le y, x \colon A \vljm B \Rightarrow \Right, y \colon A}{\G, \Left, x \le y, x \colon A \vljm B, y \colon B \Rightarrow \Right}$
		
		\vspace{5mm}
		
			
			$\vlinf{\sbl}{}{\G, \Left, x \le y, yRz, x \colon \square A \Rightarrow \Right}{\G,\Left, x \le y, yRz, x \colon \square A, z \colon A \Rightarrow \Right}$\hspace{10mm}$\vlinf{\sbr}{$ $y, z$ fresh$}{\G, \Left \Rightarrow \Right, x \colon \square A}{\G, \Left, x \le y, yRz \Rightarrow \Right, z \colon A}$
			

		\vspace{5mm}
		
			$\vlinf{\sdl}{$ $y$ fresh $}{\G, \Left, x \colon \Diamond A \Rightarrow \Right}{\G, \Left, xRy, y \colon A \Rightarrow \Right}$\hspace{10mm}
			$\vlinf{\sdr}{}{\G, \Left, xRy \Rightarrow \Right, x \colon \Diamond A}{\G, \Left, xRy \Rightarrow \Right, x \colon \Diamond A, y \colon A}$
			
		\vspace{5mm}
		
				
		\vspace{2mm}
			$\vlinf{\refl}{}{\G, \Left \Rightarrow \Right}{\G, x\le x, \Left \Rightarrow \Right}$\hspace{10mm}
			$\vlinf{\trans}{}{\G, x \le y, y \le z, \Left \Rightarrow \Right}{\G, x \le y, y \le z, x \le z, \Left \Rightarrow \Right}$
			
		
		\vspace{5mm}
		
			
			$\vlinf{\fone}{$ $u$ fresh$}{\G, \Left, xRy, y \le z \Rightarrow \Right}{\G, \Left, xRy, y \le z, x \le u, uRz \Rightarrow \Right}$\hspace{4mm}
			$\vlinf{\ftwo}{u$ fresh$}{\G, \Left, xRy,x \le z \Rightarrow \Right}{\G, \Left, xRy, x \le z, y \le u, zRu \Rightarrow \Right }$

	\end{center}
	\caption{Sistema $\labIKh$}
	\label{fig:labIKheart}
\end{figure}





%Veamos un ejemplo en el cual la ausencia de una regla no nos permitiría

%\raul{Completar esta idea, o sacar la subseccion y dejar solo el comentario sin ejemplo.}
% \section{Corrección de $\labIKh$}

Como se mencionó anteriormente, debemos mostrar que el sistema introducido tiene dos propiedades. Por un lado, es necesario demostrar que es un sistema completo (como se verá en el Capítulo~\ref{cap:completeness}, y por otro lado, debemos asegurar que es correcto. Esto último quiere decir que cada una de las reglas de secuentes que posee nuestro sistema $\labIKh$ es correcta. Formalmente:

\begin{center}
	\emph{Para todo modelo $\M$, si $\M$ satisface la premisa entonces $\M$ satisface la conclusión.}
\end{center}

Para poder demostrar la sentencia anterior, primero necesitamos introducir algunas definiciones y notación:

\dfn{Sea $\M = \langle W, R, \le, V \rangle$ un modelo, una \emph{función de asignación} es una $\f : Labels \rightarrow W$, que a cada etiqueta ( en inglés label) le asigna un mundo en el modelo $\M$.
%\raul{nunca definiste Labels. Introducir ese nombre cuando se introducen sistemas etiquetados por primera vez.}
\dfn Sea $\M = \langle W, R, \le, V \rangle$ un modelo y sea $\G, \Left \Rightarrow \Right$ un secuente. Decimos que $\M \Vdash \G, \Left \Rightarrow \Right$ si existe una función de asignación $\f$ tal que:
\medskip
 Si se cumplen:
\begin{enumerate}
	\item Para todo $x \colon A \in \Left$ tenemos que $\M, f(x) \Vdash A$ (Notación: $\M \Vdash \Left$)
	\item Para todo $xRy \in \G$ tenemos que $f(x)Rf(y)$ (Notación: $\M \Vdash \G$)
	\item Para todo $x \le y \in \G$ tenemos que $f(x) \le f(y)$ (Notación: $\M \Vdash \G$)
\end{enumerate}
Entonces para todo $z \colon B \in \Right$ tenemos que $\M, f(z) \Vdash B$ (Notación: $\M \Vdash \Right$).
Diremos que $\M \not \Vdash \G, \Left \Rightarrow \Right$ si no es cierto que $\M \Vdash \G, \Left \Rightarrow \Right$.}

\medskip
Siguiendo estas definiciones, en esta sección demostramos que todas las reglas presentadas en la Figura \ref{fig:labIKheart} son correctas. Como la prueba es similar para cada una de las reglas, demostramos la correctitud de algunas reglas particulares:


\begin{itemize}
\item {Regla $\svlef$}:

Sea $\svlef$ la regla izquierda para la conjunción definida en la Figura~\ref{fig:labIKheart}:

\begin{center}
		$\vlinf{\svlef}{}{\G,\Left, x \colon \vls(A.B) \Rightarrow \Right}{\conjlef}$
\end{center}

Queremos ver que es correcta. Aplicando la definición, queremos ver que:

\begin{center}
\emph{Para todo modelo $\M$, si $\M \Vdash \conjlef$, entonces $\M \Vdash \G, \Left, x \colon A \vlan B \Rightarrow \Right$.}
	
\end{center}

Para demostrar este enunciado lo hacemos a partir del uso de la contrarrecíproca, es decir queremos ver que si un existe un modelo $\M_{0}$ tal que $\M_{0} \not \Vdash \G, \Left, x \colon \vls(A.B) \Rightarrow \Right$, entonces $\M_{0} \not \Vdash \conjlef$. Asumimos la primera parte de la implicación y tenemos que $\M_{0} \not \Vdash \G, \Left, x \colon \vls(A.B) \Rightarrow \Right$, es decir, por un lado tenemos que $\M_{0} \Vdash \G$, $\M_{0} \Vdash \Left$ y $\M_{0} \Vdash x \colon \vls(A.B)$, y por otro lado, $\M_{0} \not \Vdash \Right$. Como dijimos, en particular tenemos que $\M_{0} \Vdash x \colon \vls (A.B)$.  Por definición, obtenemos que $\M_{0} \Vdash x \colon A$ y $\M_{0} \Vdash x \colon B$. Finalmente, concluimos que, por un lado: $\M_{0} \Vdash \G$, $\M_{0} \Vdash \Left$, $\M_{0} \Vdash x \colon A$ y $\M_{0} \Vdash x \colon B$ y por otro lado, $\M_{0} \not \Vdash \Right$, que es lo mismo que $\M_{0} \not \Vdash \conjlef $. Por lo tanto, la regla $\svlef$ del sistema $\labIKh$ es correcta.


\item {Regla $\sorig$}:

Sea la regla $\sorig$ propuesta para el sistema $\labIKh$ la siguiente:

\begin{center}
	$\vlinf{\sorig}{}{\G, \Left \Rightarrow \Right, x \colon \vls[A.B]}{\G, \Left \Rightarrow \Right, x   \colon   A, x   \colon   B}$
\end{center}

Queremos ver la correctitud de esta regla. Más precisamente, queremos ver que:

\begin{center}
	\emph{Para todo modelo $\M$, si $\M \Vdash \G, \Left \Rightarrow \Right, x   \colon   A, x   \colon   B$, entonces $\M \Vdash \G, \Left \Rightarrow \Right, x \colon A \vlor B$.}
\end{center}

Para esta prueba utilizamos la contrarrecíproca, es decir, deseamos ver que, si existe un modelo $\M_{0}$ tal que $\M_{0} \not \Vdash \G, \Left \Rightarrow \Right, x \colon \vls[A.B]$, entonces $\M_{0} \not \Vdash \G, \Left \Rightarrow \Right, x   \colon   A, x   \colon   B$. Asumimos entonces que $\M_{0} \not \Vdash \G, \Left \Rightarrow \Right, x \colon \vls[A.B]$, es decir, tenemos que por un lado $\M_{0} \Vdash \G$ y $\M_{0} \Vdash \Left$, y por otro, $\M_{0} \not \Vdash \Right$ y $\M_{0} \not \Vdash x \colon \vls[A.B]$. Por definición de $\Vdash$ para $\M_{0} \not \Vdash x \colon \vls[A.B]$ tenemos que $\M_{0} \not \Vdash A$ y $\M_{0} \not \Vdash B$. Por lo tanto, vimos por un lado que $\M_{0} \Vdash \G$ y $\M_{0} \Vdash \Left$, y por otro lado, vimos que $\M_{0} \not \Vdash \Right$, $\M_{0} \not \Vdash x \colon A$ y $\M_{0} \not \Vdash x \colon B$. Estas observaciones pueden reescribirse de la siguiente forma: $\M_{0} \not \Vdash \G, \Left \Rightarrow \Right, x \colon A, x \colon B$. Luego, queda demostrado que la regla $\sorig$ es correcta.


\item{Regla $\sbr$}:

La regla presentada en la Figura \ref{fig:labIKheart} para $\sbr$ es la siguiente:

\begin{center}
$\vlinf{\sbr}{}{\G, \Left \Rightarrow \Right, x \colon \square A}{\G, \Left, x \le y, yRz \Rightarrow \Right, x \colon \square A, z \colon A}$
\end{center}

Queremos saber si esta regla es correcta. Utilizando la definición de corrección que vimos anteriormente queremos ver que:
\begin{center}
	\emph{Para todo modelo $\M$, si $\M \Vdash \G, \Left, x \le y, yRz \Rightarrow \Right, x \colon \square A, z \colon A$, entonces $\M \Vdash \G, \Left \Rightarrow \Right, x \colon \square A.$}
\end{center}
 
Asumimos que existe un modelo $\M_{0}$ tal que $\M_{0} \not \Vdash \G, \Left \Rightarrow \Right, x \colon \square A$ y queremos ver que $\M_{0} \not \Vdash \G, \Left, x \le y, y$R$z \Rightarrow \Right, x \colon \square A, z \colon A$. De nuestra hipótesis obtenemos que $\M_{0} \Vdash \G$ y $\M_{0} \Vdash \Left$, y también que $\M_{0} \not \Vdash \Right$ y $\M_{0} \not \Vdash x: \square A$. Por definición de $\M_{0},x \not \Vdash \square A $ tenemos que existen mundos $y, z$ pertenecientes a $\M_{0}$, tal que $x \le y$, $yRz$ donde $\M_{0}\not \Vdash z \colon A$. Por lo tanto, $\M_{0} \not \Vdash \G, \Left, x \le y, yRz \Rightarrow \Right, x \colon \square A, z \colon A$. Finalmente, utilizando la contrarrecíproca queda demostrado que la regla $\sbr$ es correcta. 


\item {Regla $\sdr$}:

La regla presentada en la Figura \ref{fig:labIKheart} para $\sdr$ es la siguiente:

\begin{center}
	$\vlinf{\sdr}{}{\G, \Left, xRy \Rightarrow \Right, x \colon \Diamond A}{\G, \Left, xRy \Rightarrow \Right, x \colon \Diamond A, y \colon A}$
\end{center}

Al igual que como se demostró para $\sbr$, queremos saber si la regla $\sdr$ es correcta. Para ello, volvemos a utilizar la definición de corrección para esta regla en cuestión. Es decir, correctitud para la regla derecha del diamante $\sdr$ es:

\begin{center}
	\emph{Para todo modelo $\M$, si $\M \Vdash \G, \Left, xRy \Rightarrow \Right, x \colon \Diamond A, y \colon A$, entonces $\M \Vdash \G, \Left, xRy \Rightarrow \Right, x \colon \Diamond A$.}
\end{center}

Utilizando la contrarrecíproca a la definición planteada, asumimos que existe un modelo $\M_{0}$ tal que $\M_{0} \not \Vdash \G, \Left, xRy \Rightarrow \Right, x \colon \Diamond A$ y queremos ver que $\M_{0} \not \Vdash \G, \Left, xRy \Rightarrow \Right, x \colon \Diamond A, y \colon A$. Desglosando nuestra hipótesis tenemos que $\M_{0} \Vdash \G$, $\M_{0} \Vdash \Left$, $\M_{0} \Vdash xRy$, $\M_{0} \not \Vdash \Right$ y $ \M_{0} \not \Vdash x: \Diamond A$. Por definición de $\M_{0} \not \Vdash x: \Diamond A$ tenemos que para todo mundo $y$ en $\M_{0}$ donde $xRy$, $\M_{0} \not \Vdash y \colon A$. Por lo tanto, podemos concluir que $\M_{0} \not \Vdash \G, \Left, xRy \Rightarrow \Right, x \colon \Diamond A, y \colon A$.

\item {Regla $\sdl$}:

La regla presentada en la Figura \ref{fig:labIKheart} para la regla del diamante izquierda (denotada con $\sdl$) es la siguiente:

 \begin{center}  
$\vlinf{\sdl}{}{\G, \Left, x: \Diamond A \Rightarrow \Right}{\G, \Left, x$R$y,  y \colon A \Rightarrow \Right}$
\end{center}


Siguiendo el mismo criterio que venimos utilizando para las reglas anteriores para la prueba de corrección, definimos corrección para $\sdl$:

\begin{center}
	\emph{Para todo modelo $\M$, si $\M \Vdash \G, \Left, xRy,  y \colon A \Rightarrow \Right$, entonces $\M \Vdash \G, \Left, x: \Diamond A \Rightarrow \Right$.}
\end{center}

Para probar este enunciado, utilizamos la contrarrecíproca: asumimos que existe un modelo $\M_{0}$ tal que $\M_{0} \not \Vdash \G, \Left, x: \Diamond A \Rightarrow \Right$, y queremos ver que $\M_{0} \not \Vdash \G, \Left, xRy,  y \colon A$. De $\M_{0} \not \Vdash \G, \Left, x: \Diamond A \Rightarrow \Right$ tenemos que $\M_{0} \Vdash \G$, $\M_{0} \Vdash \Left$, $\M_{0} \Vdash x \colon \Diamond A$ y $\M_{0} \not \Vdash \Right$. Por $\M_{0}, x \Vdash \Diamond A$ sabemos que existe un mundo $y$ en $\M_{0}$ tal que $xRy$ y $\M_{0}, y \Vdash A$. Por lo tanto, $\M_{0} \not \Vdash \G, \Left, xRy,  y \colon A \Rightarrow \Right$.

\item {Regla $\sbl$}:

La regla para $\sbl$ presentada en nuestro sistema $\labIKh$ es la siguiente:

\begin{center}
	$\vlinf{\sbl}{}{\G, \Left, x \le y, yRz, x \colon \square A \Rightarrow \Right}{\G,\Left, x \le y, yRz, x \colon \square A, z \colon A \Rightarrow \Right}$
\end{center}

Queremos ver que esta regla es correcta. Para ello, así como hicimos con las reglas anteriores, lo que queremos demostrar es:

\begin{center}
	\emph{Para todo modelo $\M$, si $\M \Vdash \G,\Left, x \le y, yRz, x \colon \square A, z \colon A \Rightarrow \Right$, entonces $\M \Vdash \G, \Left, x \le y, yRz, x \colon \square A \Rightarrow \Right$.}
\end{center}

La prueba de este enunciado la hacemos a partir del uso de la contrarrecíproca, es decir, lo que queremos ver es que si existe un modelo $\M_{0}$ tal que $\M_{0} \not \Vdash conclusion$, entonces $\M_{0} \not \Vdash premisa$. Más particularmente para nuestra regla: si existe un modelo $\M_{0}$ tal que $\M_{0} \not \Vdash \G, \Left, x \le y, yRz, x \colon \square A \Rightarrow \Right$, entonces $\M_{0} \not \Vdash \G,\Left, x \le y, yRz, x \colon \square A, z \colon A \Rightarrow \Right$.

Asumimos entonces que existe un modelo $\M_{0}$ tal que $\M_{0} \not \Vdash \G, \Left, x \le y, yRz, x \colon \square A \Rightarrow \Right$, es decir $\M_{0} \Vdash \G$, $\M_{0} \Vdash \Left$, $\M_{0} \Vdash x \le y$, $ \M_{0} \Vdash yRz$, $\M_{0} \Vdash x\colon \square A$ y  $\M_{0} \not \Vdash \Right$. Por lo tanto, en particular de $\M_{0} \Vdash \square A$ tenemos que para todo mundo $y, z$ tal que $x \le y$ y $yRz$, $\M_{0}, f(z) \Vdash A$, o lo que es lo mismo $\M_{0} \Vdash z \colon A$. Finalmente obtenemos que $\M_{0} \not \Vdash \G, \Left, x \le y, yRz, x \colon \square A, z \colon A \Rightarrow \Right$.

\end{itemize}

La corrección del resto de las reglas puede demostrarse de manera similar. De esta forma podemos concluir:

\begin{teo} El sistema $\labIKh$ es correcto.\end{teo}

Las siguientes secciones serán dedicadas a demostrar la otra propiedad que buscamos, es decir, que el sistema $\labIKh$ es completo.
\chapter{Un c\'alculo de secuentes para la l\'ogica modal intuicionista}
\label{cap:labIKh}

En los capítulos anteriores se introdujeron todos los conceptos fundamentales que resultan necesarios para comprender nuestra contribución, que presentaremos en detalle en el resto de la tesis. A lo largo de este capítulo, en diferentes secciones, presentamos un nuevo sistema de prueba etiquetado para la lógica modal intuicionista, como así también la prueba de correctitud para cada uno de las reglas que conforman el sistema en cuestión.

\section{El problema principal}

Como se dijo anteriormente, nuestro trabajo consiste en la definición de un sistema de prueba etiquetado para lógicas modales intuicionistas. La extensión directa de la deducción etiquetada al entorno intuicionista consiste en la utilización de dos tipos de átomos relaciones: uno para la relación de accesibilidad modal representada con $R$, y otro para la relación intuicionista o mejor conocida como relación de preorden $\le$. Esto resulta interesante ya que nos permite poner el sistema de prueba en estrecha relación con la semántica birelacional de Kripke que se detalló en el Capítulo~\ref{cap:logintuicionista}.

En \cite{negri2005} se introdujo un cálculo de secuentes etiquetado para la lógica modal básica. En este trabajo extenderemos dicho sistema con el objetivo de capturar lógicas modales intuicionistas. Además de las fórmulas etiquetadas de la forma $x \colon A$, y de las expresiones $xRy$, capturando la relación de accesibilidad (donde $x, y$ pertenecen a un conjunto de etiquetas, y $A$ es un fórmula), utilizaremos expresiones del tipo $x \le y$ para capturar la relación de preorden intuicionista. De una manera más formal introducimos la siguiente definición:

%La idea es extender un cálculo de secuentes etiquetado con un símbolo de relación de preorden con el objetivo de capturar lógicas modales intuicionistas; esto quiere decir, buscamos definir secuentes etiquetados intuicionistas a partir de fórmulas etiquetadas $x \colon A$, de la relación de accesibilidad modal $x$R$y$, y la nueva relación de preorden de la forma $x \le y$, donde $x, y$ pertenecen a un conjunto de etiquetas y $A$ es una fórmula modal intuicionista. El cálculo de secuentes etiquetado en el cual nos basamos fue el propuesto por Negri para la lógica modal básica \cite{negri2005}.



\dfn{Un \emph{secuente etiquetado intuicionista} es de la forma $\G$, $\Left \Rightarrow \Right$ donde $\G$ denota un conjunto de átomos relacionales y de preorden, y  $\Left$ y $\Right$ son conjuntos de fórmulas etiquetadas. }

\vspace{2mm}

De esta manera obtenemos el sistema de prueba $\labIKh$ para lógicas modales intuicionistas en este formalismo, y podemos demostrar el siguiente teorema:

\begin{quote}
 \emph{Una fórmula $A$ es demostrable en el cálculo $\labIKh$ si y sólo si $A$ es válida en cada frame bi-relacional.}
\end{quote}

La dirección de izquierda a derecha de la sentencia anterior indica que el cálculo es \emph{correcto}: las reglas me permiten inferir solo fórmulas válidas. Por otro lado, la implicación de derecha a izquierda se conoce como \emph{completitud}: todos los teoremas son demostrables en el cálculo.

En la siguiente sección, se presenta el nuevo sistema de cálculo de secuentes etiquetados para las lógicas modales intuicionistas. 


\section{El sistema $\labIKh$}
En esta sección buscamos resolver el problema recientemente propuesto a partir del sistema de prueba etiquetado que se observa en la Figura \ref{fig:labIKheart}. Como se dijo anteriormente, para obtener este nuevo sistema, se extendió el cálculo de secuentes etiquetado propuesto por Negri para la lógica modal básica agregando un símbolo de relación de pre-orden. En las siguientes secciones desarrollamos en detalle las reglas que forman parte de este sistema de secuentes etiquetado para la lógica modal intuicionista llamado $\labIKh$.

\subsection{Condiciones para frames}

Para poder capturar la lógica modal intuicionista, en primer lugar, buscamos responder a los requisitos de las definiciones presentadas en el Capítulo 3. La semántica de la lógica modal intuicionista habla de un modelo bi-relacional el cual esta compuesto por dos relaciones binarias $R$ y $\le$, donde exigimos que $\le$ sea una relación de pre-orden, es decir, que sea reflexiva y transitiva. Por lo tanto, el sistema $\labIKh$ debe tener reglas que caractericen tales condiciones (estas reglas se representan con \emph{refl} y \emph{trans}). También debe poseer reglas que caractericen las condiciones F1 y F2 (en el sistema $\labIKh$ estas reglas son $\fone$ y $\ftwo$). Por otra parte, recordemos que la definición de la función de valuación monótona $V$ nos introduce el \emph{lema de monotonía} que será capturado por la regla $\ids$ del sistema. Estas reglas son las siguientes:

\begin{center}
	$\vlinf{\refl}{}{\G, \Left \Rightarrow \Right}{\G, x\le x, \Left \Rightarrow \Right}$\hspace{8mm}
	$\vlinf{\trans}{}{\G, x \le y, y \le z, \Left \Rightarrow \Right}{\G, x \le y, y \le z, x \le z, \Left \Rightarrow \Right}$
	
	\vspace{6mm}
	
	$\vlinf{\ids}{}{\G, \Left,x \le y, x \colon a \Rightarrow \Right, y \colon a}{}$
	
	\vspace{6mm}
	
	$\vlinf{\fone}{$ $u$ fresh$}{\G, \Left, xRy, y \le z \Rightarrow \Right}{\G, \Left, xRy, y \le z, x \le u, uRz \Rightarrow \Right}$\hspace{4mm}$\vlinf{\ftwo}{u$ fresh$}{\G, \Left, xRy,x \le z \Rightarrow \Right}{\G, \Left, xRy, x \le z, y \le u, zRu \Rightarrow \Right }$
		
\end{center}

\bigskip

A modo de ejemplo, explicaremos en detalle las reglas $\fone$ y $\ftwo$:

\begin{itemize}
		\item Regla $\fone$.
		
		La regla presentada para el cumplimiento de una de las condiciones ($\fone$) de la relación de preorden $\le$ es la siguiente:
		
		\begin{center}
			$\vlinf{\fone}{$ $u$ fresh$}{\G, \Left, xRy, y \le z \Rightarrow \Right}{\G, \Left, xRy, y \le z, x \le u, uRz \Rightarrow \Right}$
		\end{center}
		
		Dado $\G$ un conjunto de átomos relacionales y de pre-orden, y dados $\Left$ y $\Right$ conjuntos de fórmulas etiquetadas, la regla en cuestión expresa la semántica de la condición $\fone$. Leyendo desde la conclusión hacia la premisa tenemos etiquetas $x, y$ tal que $xRy$ y $y \le z$, entonces existe un mundo $u$ tal que $x \le u$ y $uRz$.
		
		\item Regla $\ftwo$:
		
		Como vimos anteriormente, la regla para el cumplimiento de F2 se presenta de la siguiente manera en $\labIKh$:
		
		\begin{center}
			$\vlinf{\ftwo}{u$ fresh$}{\G, \Left, xRy,x \le z \Rightarrow \Right}{\G, \Left, xRy, x \le z, y \le u, zRu \Rightarrow \Right }$
		\end{center}
		
		Al igual que con F1, sea $\G$ un conjunto de átomos relacionales y de pre-orden, y dados $\Left$ y $\Right$ conjuntos de fórmulas etiquetadas, esta regla expresa semánticamente desde su conclusión hace su premisa, que dadas las etiquetas $x, y$ y $z$, si $xRy$ y $x \le z$, entonces hay una etiqueta (o mundo) $u$ (fresh) tal que $y \le u$ y $zRu$.
		
\end{itemize}

\subsection{Conectivos lógicos de la lógica proposicional intuicionista}

Para seguir construyendo un sistema completo y correcto etiquetado para la lógica modal intuicionista, en esta sección presentamos las reglas que expresan la semántica de los conectivos lógicos usuales. Es decir, presentamos reglas para la conjunción y la disjunción que forman parte del sistema $\labIKh$. Las mismas se presentan a continuación:

\begin{center}
		\hspace{4mm}$\vlinf{\svlef}{}{\G,\Left, x \colon \vls(A.B) \Rightarrow \Right}{\conjlef}$\hspace{10mm}
		$\vliinf{\svrig}{}{\G,\Left \Rightarrow \Right, x \colon \vls(A.B)}{\conjrig}{\conjrigh}$
		
		
		\vspace{5mm}
		
		$\vliinf{\solef}{}{\G, \Left, x \colon \vls[A.B] \Rightarrow \Right}{\G, \Left, x   \colon   A \Rightarrow \Right}{\G, \Left, x   \colon   B \Rightarrow \Right}$\hspace{10mm}
		$\vlinf{\sorig}{}{\G, \Left \Rightarrow \Right, x \colon \vls[A.B]}{\G, \Left \Rightarrow \Right, x   \colon   A, x   \colon   B}$
\end{center}

También el sistema $\labIKh$ debe capturar las constantes proposicionales $\top$ y $\bot$. Para ello, incorpora las siguientes dos reglas al sistema:

\begin{center}
	$\vlinf{\sbot}{}{\G,\Left, x \colon \bot \Rightarrow \Right}{}$\hspace{18mm}
	$\vlinf{\Stop}{}{\G, \Left \Rightarrow \Right, x \colon \top}{}$
\end{center}

A continuación explicaremos en detalle la regla de conjunción izquierda $\svlef$ y la regla de disjunción derecha $\sorig$:

\begin{itemize}
	\item Regla $\svlef$:
	
	El sistema $\labIKh$ presenta la siguiente regla para la conjunción en el lado izquierdo:
	
	\begin{center}
		$\vlinf{\svlef}{}{\G,\Left, x \colon \vls(A.B) \Rightarrow \Right}{\conjlef}$
	\end{center}
	
	Al igual que las reglas presentadas en la sección anterior $\G$ es un conjunto de átomos relacionales y de pre-orden, y $\Left$ y $\Right$ son conjuntos de fórmulas etiquetadas. Desde la conclusión hacia la premisa de esta regla, semánticamente nos dice que si en $x$ se satisface la fórmula $A \vlan B$, entonces en $x$ se satisface la fórmula $A$ y también se satisface la fórmula $B$.
	
	\item Regla $\sorig$:
	
	Sea la regla presentada en $\labIKh$ para la disjunción en el lado derecho:
	
	\begin{center}
		$\vlinf{\sorig}{}{\G, \Left \Rightarrow \Right, x \colon \vls[A.B]}{\G, \Left \Rightarrow \Right, x   \colon   A, x   \colon   B}$
	\end{center}
	
	Tenemos que, leyendo desde la conclusión hacia la premisa, si en $x$ se satisface $A \vlor B$ (del lado derecho), entonces en $x$ es verdadera la fórmula $A$ y también la fórmula $B$.
	
\end{itemize}

\subsection{Capturando los operadores modales}

En esta sección, introduciremos las reglas que expresan la semántica de los operadores modales $\square$ y $\Diamond$. El sistema $\labIKh$ presenta las siguientes cuatro reglas (dos para el $\square$ y dos para el $\Diamond$): 

\begin{center}
	\hspace{5mm}$\vlinf{\sbl}{}{\G, \Left, x \le y, yRz, x \colon \square A \Rightarrow \Right}{\G,\Left, x \le y, yRz, x \colon \square A, z \colon A \Rightarrow \Right}$\hspace{10mm}$\vlinf{\sbr}{$ $y, z$ fresh$}{\G, \Left \Rightarrow \Right, x \colon \square A}{\G, \Left, x \le y, yRz \Rightarrow \Right, z \colon A}$
	
	
	\vspace{5mm}
	
	$\vlinf{\sdl}{$ $y$ fresh $}{\G, \Left, x \colon \Diamond A \Rightarrow \Right}{\G, \Left, xRy, y \colon A \Rightarrow \Right}$\hspace{10mm}
	$\vlinf{\sdr}{}{\G, \Left, xRy \Rightarrow \Right, x \colon \Diamond A}{\G, \Left, xRy \Rightarrow \Right, x \colon \Diamond A, y \colon A}$
	
\end{center}

Expliquemos más en detalle algunas de ellas:

\begin{itemize}
	\item Regla $\sbl$.
	
	Para capturar el $\square$ del lado izquierdo, en nuestro sistema introducimos la siguiente regla:
	\begin{center}
		$\vlinf{\sbl}{}{\G, \Left, x \le y, yRz, x \colon \square A \Rightarrow \Right}{\G,\Left, x \le y, yRz, x \colon \square A, z \colon A \Rightarrow \Right}$
	\end{center}
	Nuevamente dado $\G$ un conjunto de átomos relacionales y de pre-orden, y sean $\Left$ y $\Right$ conjuntos de fórmulas etiquetadas. Esta regla semánticamente expresa, desde su conclusión hacia su premisa, que dado $x, y$ y $z$ tal que $x \le y$ y $yRz$ donde en $x$ se hace verdadera la fórmula $\square A$, entonces en $z$ se hace válida la fórmula $A$.
	
	\item Regla $\sdr$.
	
	Sea la regla presentada en $\labIKh$ para $\Diamond$ en el lado derecho:	
	\begin{center}
		$\vlinf{\sdr}{}{\G, \Left, xRy \Rightarrow \Right, x \colon \Diamond A}{\G, \Left, xRy \Rightarrow \Right, x \colon \Diamond A, y \colon A}$
	\end{center}
	
	Semánticamente, el diamante establece que en un modelo $\M$ y en un mundo $v$ vale $\Diamond A$ si y sólo si existe un mundo $u$ tal que $vRu$ y en $u$ la fórmula $A$ se satisface. Para capturar esta definición semántica, la regla introducida (leyendo desde la conclusión hacia la premisa) nos dice que, si $xRy$ y en $x$ se satisface $\Diamond A$, luego en $y$ la fórmula $A$ tiene que ser verdadera.
	
\end{itemize}

\subsection{Implicación intuicionista}

Por último, nos resta ver que el sistema $\labIKh$ captura la semántica del operador de implicación. Es por ello que este sistema de secuentes etiquetado para que sea completo necesita de las siguientes dos reglas:

\begin{center}
		$\vlinf{\sir}{$ $y$ fresh$}{\G, \Left \Rightarrow \Right, x \colon A \vljm B}{\G, \Left, x \le y, y \colon A \Rightarrow \Right, y \colon B}$
		
		
		\vspace{5mm}
		
		
		$\vliinf{\sil}{}{\G, \Left, x \le y, x \colon A \vljm B \Rightarrow \Right}{\G, \Left, x \le y, x \colon A \vljm B \Rightarrow \Right, y \colon A}{\G, \Left, x \le y, x \colon A \vljm B, y \colon B \Rightarrow \Right}$
\end{center}

Veamos en detalle el significado de la regla de implicación del lado derecho:

\begin{itemize}
	\item Regla $\sir$.
	
	El sistema $\labIKh$ presenta la siguiente regla para la implicación en el lado derecho:
	
	\begin{center}
		$\vlinf{\sir}{$ $y$ fresh$}{\G, \Left \Rightarrow \Right, x \colon A \vljm B}{\G, \Left, x \le y, y \colon A \Rightarrow \Right, y \colon B}$
	\end{center}
	
	Dado $\G$ un conjunto de átomos relacionales y de preorden, y dados $\Left$ y $\Right$ conjuntos de fórmulas etiquetadas, esta regla (leyendo desde la conclusión hacia la premisa) semánticamente nos dice que si $A \vljm B$ se satisface en $x$, entonces existe un nuevo (mundo) $y$ tal que $x \le y$ y en $y$ vale $A$, luego en $y$ también $B$ es válida.
	
	
\end{itemize}

%\subsection{\textbf{¿Por qué necesitamos dos reglas para cada operador?}}

\subsection{Uniendo todas las reglas: sistema $\labIKh$}
Como pudimos ver a lo largo de las subsecciones anteriores, el sistema $\labIKh$ tiene dos reglas para cada operador. Esto se debe a que la sintaxis para la lógica modal intuicionista no presenta la negación $\neg$, por lo que los operadores $\Diamond$ y $\square$ dejan de ser operadores duales como lo eran en el caso clásico.

En la Figura \ref{fig:labIKheart} se puede ver el sistema de secuentes etiquetado $\labIKh$ completo con cada una de las reglas mencionadas anteriormente.

\begin{figure}[!h]
	\small
	\begin{center}
			
			$\vlinf{\sbot}{}{\G,\Left, x \colon \bot \Rightarrow \Right}{}$\hspace{6mm}
			$\vlinf{\ids}{}{\G, \Left,x \le y, x \colon a \Rightarrow \Right, y \colon a}{}$\hspace{6mm}
			$\vlinf{\Stop}{}{\G, \Left \Rightarrow \Right, x \colon \top}{}$
		
		\vspace{5mm}
			
			$\vlinf{\svlef}{}{\G,\Left, x \colon \vls(A.B) \Rightarrow \Right}{\conjlef}$\hspace{10mm}
			$\vliinf{\svrig}{}{\G,\Left \Rightarrow \Right, x \colon \vls(A.B)}{\conjrig}{\conjrigh}$

		
		\vspace{5mm}
			
			$\vliinf{\solef}{}{\G, \Left, x \colon \vls[A.B] \Rightarrow \Right}{\G, \Left, x   \colon   A \Rightarrow \Right}{\G, \Left, x   \colon   B \Rightarrow \Right}$\hspace{10mm}
			$\vlinf{\sorig}{}{\G, \Left \Rightarrow \Right, x \colon \vls[A.B]}{\G, \Left \Rightarrow \Right, x   \colon   A, x   \colon   B}$

		
		\vspace{5mm}
			
			$\vlinf{\sir}{$ $y$ fresh$}{\G, \Left \Rightarrow \Right, x \colon A \vljm B}{\G, \Left, x \le y, y \colon A \Rightarrow \Right, y \colon B}$
		
		
		\vspace{5mm}
		
		
		$\vliinf{\sil}{}{\G, \Left, x \le y, x \colon A \vljm B \Rightarrow \Right}{\G, \Left, x \le y, x \colon A \vljm B \Rightarrow \Right, y \colon A}{\G, \Left, x \le y, x \colon A \vljm B, y \colon B \Rightarrow \Right}$
		
		\vspace{5mm}
		
			
			$\vlinf{\sbl}{}{\G, \Left, x \le y, yRz, x \colon \square A \Rightarrow \Right}{\G,\Left, x \le y, yRz, x \colon \square A, z \colon A \Rightarrow \Right}$\hspace{10mm}$\vlinf{\sbr}{$ $y, z$ fresh$}{\G, \Left \Rightarrow \Right, x \colon \square A}{\G, \Left, x \le y, yRz \Rightarrow \Right, z \colon A}$
			

		\vspace{5mm}
		
			$\vlinf{\sdl}{$ $y$ fresh $}{\G, \Left, x \colon \Diamond A \Rightarrow \Right}{\G, \Left, xRy, y \colon A \Rightarrow \Right}$\hspace{10mm}
			$\vlinf{\sdr}{}{\G, \Left, xRy \Rightarrow \Right, x \colon \Diamond A}{\G, \Left, xRy \Rightarrow \Right, x \colon \Diamond A, y \colon A}$
			
		\vspace{5mm}
		
				
		\vspace{2mm}
			$\vlinf{\refl}{}{\G, \Left \Rightarrow \Right}{\G, x\le x, \Left \Rightarrow \Right}$\hspace{10mm}
			$\vlinf{\trans}{}{\G, x \le y, y \le z, \Left \Rightarrow \Right}{\G, x \le y, y \le z, x \le z, \Left \Rightarrow \Right}$
			
		
		\vspace{5mm}
		
			
			$\vlinf{\fone}{$ $u$ fresh$}{\G, \Left, xRy, y \le z \Rightarrow \Right}{\G, \Left, xRy, y \le z, x \le u, uRz \Rightarrow \Right}$\hspace{4mm}
			$\vlinf{\ftwo}{u$ fresh$}{\G, \Left, xRy,x \le z \Rightarrow \Right}{\G, \Left, xRy, x \le z, y \le u, zRu \Rightarrow \Right }$

	\end{center}
	\caption{Sistema $\labIKh$}
	\label{fig:labIKheart}
\end{figure}





%Veamos un ejemplo en el cual la ausencia de una regla no nos permitiría

%\raul{Completar esta idea, o sacar la subseccion y dejar solo el comentario sin ejemplo.}
\section{Corrección de $\labIKh$}

Como se mencionó anteriormente, debemos mostrar que el sistema introducido tiene dos propiedades. Por un lado, es necesario demostrar que es un sistema completo (como se verá en el Capítulo~\ref{cap:completeness}, y por otro lado, debemos asegurar que es correcto. Esto último quiere decir que cada una de las reglas de secuentes que posee nuestro sistema $\labIKh$ es correcta. Formalmente:

\begin{center}
	\emph{Para todo modelo $\M$, si $\M$ satisface la premisa entonces $\M$ satisface la conclusión.}
\end{center}

Para poder demostrar la sentencia anterior, primero necesitamos introducir algunas definiciones y notación:

\dfn{Sea $\M = \langle W, R, \le, V \rangle$ un modelo, una \emph{función de asignación} es una $\f : Labels \rightarrow W$, que a cada etiqueta ( en inglés label) le asigna un mundo en el modelo $\M$.
%\raul{nunca definiste Labels. Introducir ese nombre cuando se introducen sistemas etiquetados por primera vez.}
\dfn Sea $\M = \langle W, R, \le, V \rangle$ un modelo y sea $\G, \Left \Rightarrow \Right$ un secuente. Decimos que $\M \Vdash \G, \Left \Rightarrow \Right$ si existe una función de asignación $\f$ tal que:
\medskip
 Si se cumplen:
\begin{enumerate}
	\item Para todo $x \colon A \in \Left$ tenemos que $\M, f(x) \Vdash A$ (Notación: $\M \Vdash \Left$)
	\item Para todo $xRy \in \G$ tenemos que $f(x)Rf(y)$ (Notación: $\M \Vdash \G$)
	\item Para todo $x \le y \in \G$ tenemos que $f(x) \le f(y)$ (Notación: $\M \Vdash \G$)
\end{enumerate}
Entonces para todo $z \colon B \in \Right$ tenemos que $\M, f(z) \Vdash B$ (Notación: $\M \Vdash \Right$).
Diremos que $\M \not \Vdash \G, \Left \Rightarrow \Right$ si no es cierto que $\M \Vdash \G, \Left \Rightarrow \Right$.}

\medskip
Siguiendo estas definiciones, en esta sección demostramos que todas las reglas presentadas en la Figura \ref{fig:labIKheart} son correctas. Como la prueba es similar para cada una de las reglas, demostramos la correctitud de algunas reglas particulares:


\begin{itemize}
\item {Regla $\svlef$}:

Sea $\svlef$ la regla izquierda para la conjunción definida en la Figura~\ref{fig:labIKheart}:

\begin{center}
		$\vlinf{\svlef}{}{\G,\Left, x \colon \vls(A.B) \Rightarrow \Right}{\conjlef}$
\end{center}

Queremos ver que es correcta. Aplicando la definición, queremos ver que:

\begin{center}
\emph{Para todo modelo $\M$, si $\M \Vdash \conjlef$, entonces $\M \Vdash \G, \Left, x \colon A \vlan B \Rightarrow \Right$.}
	
\end{center}

Para demostrar este enunciado lo hacemos a partir del uso de la contrarrecíproca, es decir queremos ver que si un existe un modelo $\M_{0}$ tal que $\M_{0} \not \Vdash \G, \Left, x \colon \vls(A.B) \Rightarrow \Right$, entonces $\M_{0} \not \Vdash \conjlef$. Asumimos la primera parte de la implicación y tenemos que $\M_{0} \not \Vdash \G, \Left, x \colon \vls(A.B) \Rightarrow \Right$, es decir, por un lado tenemos que $\M_{0} \Vdash \G$, $\M_{0} \Vdash \Left$ y $\M_{0} \Vdash x \colon \vls(A.B)$, y por otro lado, $\M_{0} \not \Vdash \Right$. Como dijimos, en particular tenemos que $\M_{0} \Vdash x \colon \vls (A.B)$.  Por definición, obtenemos que $\M_{0} \Vdash x \colon A$ y $\M_{0} \Vdash x \colon B$. Finalmente, concluimos que, por un lado: $\M_{0} \Vdash \G$, $\M_{0} \Vdash \Left$, $\M_{0} \Vdash x \colon A$ y $\M_{0} \Vdash x \colon B$ y por otro lado, $\M_{0} \not \Vdash \Right$, que es lo mismo que $\M_{0} \not \Vdash \conjlef $. Por lo tanto, la regla $\svlef$ del sistema $\labIKh$ es correcta.


\item {Regla $\sorig$}:

Sea la regla $\sorig$ propuesta para el sistema $\labIKh$ la siguiente:

\begin{center}
	$\vlinf{\sorig}{}{\G, \Left \Rightarrow \Right, x \colon \vls[A.B]}{\G, \Left \Rightarrow \Right, x   \colon   A, x   \colon   B}$
\end{center}

Queremos ver la correctitud de esta regla. Más precisamente, queremos ver que:

\begin{center}
	\emph{Para todo modelo $\M$, si $\M \Vdash \G, \Left \Rightarrow \Right, x   \colon   A, x   \colon   B$, entonces $\M \Vdash \G, \Left \Rightarrow \Right, x \colon A \vlor B$.}
\end{center}

Para esta prueba utilizamos la contrarrecíproca, es decir, deseamos ver que, si existe un modelo $\M_{0}$ tal que $\M_{0} \not \Vdash \G, \Left \Rightarrow \Right, x \colon \vls[A.B]$, entonces $\M_{0} \not \Vdash \G, \Left \Rightarrow \Right, x   \colon   A, x   \colon   B$. Asumimos entonces que $\M_{0} \not \Vdash \G, \Left \Rightarrow \Right, x \colon \vls[A.B]$, es decir, tenemos que por un lado $\M_{0} \Vdash \G$ y $\M_{0} \Vdash \Left$, y por otro, $\M_{0} \not \Vdash \Right$ y $\M_{0} \not \Vdash x \colon \vls[A.B]$. Por definición de $\Vdash$ para $\M_{0} \not \Vdash x \colon \vls[A.B]$ tenemos que $\M_{0} \not \Vdash A$ y $\M_{0} \not \Vdash B$. Por lo tanto, vimos por un lado que $\M_{0} \Vdash \G$ y $\M_{0} \Vdash \Left$, y por otro lado, vimos que $\M_{0} \not \Vdash \Right$, $\M_{0} \not \Vdash x \colon A$ y $\M_{0} \not \Vdash x \colon B$. Estas observaciones pueden reescribirse de la siguiente forma: $\M_{0} \not \Vdash \G, \Left \Rightarrow \Right, x \colon A, x \colon B$. Luego, queda demostrado que la regla $\sorig$ es correcta.


\item{Regla $\sbr$}:

La regla presentada en la Figura \ref{fig:labIKheart} para $\sbr$ es la siguiente:

\begin{center}
$\vlinf{\sbr}{}{\G, \Left \Rightarrow \Right, x \colon \square A}{\G, \Left, x \le y, yRz \Rightarrow \Right, x \colon \square A, z \colon A}$
\end{center}

Queremos saber si esta regla es correcta. Utilizando la definición de corrección que vimos anteriormente queremos ver que:
\begin{center}
	\emph{Para todo modelo $\M$, si $\M \Vdash \G, \Left, x \le y, yRz \Rightarrow \Right, x \colon \square A, z \colon A$, entonces $\M \Vdash \G, \Left \Rightarrow \Right, x \colon \square A.$}
\end{center}
 
Asumimos que existe un modelo $\M_{0}$ tal que $\M_{0} \not \Vdash \G, \Left \Rightarrow \Right, x \colon \square A$ y queremos ver que $\M_{0} \not \Vdash \G, \Left, x \le y, y$R$z \Rightarrow \Right, x \colon \square A, z \colon A$. De nuestra hipótesis obtenemos que $\M_{0} \Vdash \G$ y $\M_{0} \Vdash \Left$, y también que $\M_{0} \not \Vdash \Right$ y $\M_{0} \not \Vdash x: \square A$. Por definición de $\M_{0},x \not \Vdash \square A $ tenemos que existen mundos $y, z$ pertenecientes a $\M_{0}$, tal que $x \le y$, $yRz$ donde $\M_{0}\not \Vdash z \colon A$. Por lo tanto, $\M_{0} \not \Vdash \G, \Left, x \le y, yRz \Rightarrow \Right, x \colon \square A, z \colon A$. Finalmente, utilizando la contrarrecíproca queda demostrado que la regla $\sbr$ es correcta. 


\item {Regla $\sdr$}:

La regla presentada en la Figura \ref{fig:labIKheart} para $\sdr$ es la siguiente:

\begin{center}
	$\vlinf{\sdr}{}{\G, \Left, xRy \Rightarrow \Right, x \colon \Diamond A}{\G, \Left, xRy \Rightarrow \Right, x \colon \Diamond A, y \colon A}$
\end{center}

Al igual que como se demostró para $\sbr$, queremos saber si la regla $\sdr$ es correcta. Para ello, volvemos a utilizar la definición de corrección para esta regla en cuestión. Es decir, correctitud para la regla derecha del diamante $\sdr$ es:

\begin{center}
	\emph{Para todo modelo $\M$, si $\M \Vdash \G, \Left, xRy \Rightarrow \Right, x \colon \Diamond A, y \colon A$, entonces $\M \Vdash \G, \Left, xRy \Rightarrow \Right, x \colon \Diamond A$.}
\end{center}

Utilizando la contrarrecíproca a la definición planteada, asumimos que existe un modelo $\M_{0}$ tal que $\M_{0} \not \Vdash \G, \Left, xRy \Rightarrow \Right, x \colon \Diamond A$ y queremos ver que $\M_{0} \not \Vdash \G, \Left, xRy \Rightarrow \Right, x \colon \Diamond A, y \colon A$. Desglosando nuestra hipótesis tenemos que $\M_{0} \Vdash \G$, $\M_{0} \Vdash \Left$, $\M_{0} \Vdash xRy$, $\M_{0} \not \Vdash \Right$ y $ \M_{0} \not \Vdash x: \Diamond A$. Por definición de $\M_{0} \not \Vdash x: \Diamond A$ tenemos que para todo mundo $y$ en $\M_{0}$ donde $xRy$, $\M_{0} \not \Vdash y \colon A$. Por lo tanto, podemos concluir que $\M_{0} \not \Vdash \G, \Left, xRy \Rightarrow \Right, x \colon \Diamond A, y \colon A$.

\item {Regla $\sdl$}:

La regla presentada en la Figura \ref{fig:labIKheart} para la regla del diamante izquierda (denotada con $\sdl$) es la siguiente:

 \begin{center}  
$\vlinf{\sdl}{}{\G, \Left, x: \Diamond A \Rightarrow \Right}{\G, \Left, x$R$y,  y \colon A \Rightarrow \Right}$
\end{center}


Siguiendo el mismo criterio que venimos utilizando para las reglas anteriores para la prueba de corrección, definimos corrección para $\sdl$:

\begin{center}
	\emph{Para todo modelo $\M$, si $\M \Vdash \G, \Left, xRy,  y \colon A \Rightarrow \Right$, entonces $\M \Vdash \G, \Left, x: \Diamond A \Rightarrow \Right$.}
\end{center}

Para probar este enunciado, utilizamos la contrarrecíproca: asumimos que existe un modelo $\M_{0}$ tal que $\M_{0} \not \Vdash \G, \Left, x: \Diamond A \Rightarrow \Right$, y queremos ver que $\M_{0} \not \Vdash \G, \Left, xRy,  y \colon A$. De $\M_{0} \not \Vdash \G, \Left, x: \Diamond A \Rightarrow \Right$ tenemos que $\M_{0} \Vdash \G$, $\M_{0} \Vdash \Left$, $\M_{0} \Vdash x \colon \Diamond A$ y $\M_{0} \not \Vdash \Right$. Por $\M_{0}, x \Vdash \Diamond A$ sabemos que existe un mundo $y$ en $\M_{0}$ tal que $xRy$ y $\M_{0}, y \Vdash A$. Por lo tanto, $\M_{0} \not \Vdash \G, \Left, xRy,  y \colon A \Rightarrow \Right$.

\item {Regla $\sbl$}:

La regla para $\sbl$ presentada en nuestro sistema $\labIKh$ es la siguiente:

\begin{center}
	$\vlinf{\sbl}{}{\G, \Left, x \le y, yRz, x \colon \square A \Rightarrow \Right}{\G,\Left, x \le y, yRz, x \colon \square A, z \colon A \Rightarrow \Right}$
\end{center}

Queremos ver que esta regla es correcta. Para ello, así como hicimos con las reglas anteriores, lo que queremos demostrar es:

\begin{center}
	\emph{Para todo modelo $\M$, si $\M \Vdash \G,\Left, x \le y, yRz, x \colon \square A, z \colon A \Rightarrow \Right$, entonces $\M \Vdash \G, \Left, x \le y, yRz, x \colon \square A \Rightarrow \Right$.}
\end{center}

La prueba de este enunciado la hacemos a partir del uso de la contrarrecíproca, es decir, lo que queremos ver es que si existe un modelo $\M_{0}$ tal que $\M_{0} \not \Vdash conclusion$, entonces $\M_{0} \not \Vdash premisa$. Más particularmente para nuestra regla: si existe un modelo $\M_{0}$ tal que $\M_{0} \not \Vdash \G, \Left, x \le y, yRz, x \colon \square A \Rightarrow \Right$, entonces $\M_{0} \not \Vdash \G,\Left, x \le y, yRz, x \colon \square A, z \colon A \Rightarrow \Right$.

Asumimos entonces que existe un modelo $\M_{0}$ tal que $\M_{0} \not \Vdash \G, \Left, x \le y, yRz, x \colon \square A \Rightarrow \Right$, es decir $\M_{0} \Vdash \G$, $\M_{0} \Vdash \Left$, $\M_{0} \Vdash x \le y$, $ \M_{0} \Vdash yRz$, $\M_{0} \Vdash x\colon \square A$ y  $\M_{0} \not \Vdash \Right$. Por lo tanto, en particular de $\M_{0} \Vdash \square A$ tenemos que para todo mundo $y, z$ tal que $x \le y$ y $yRz$, $\M_{0}, f(z) \Vdash A$, o lo que es lo mismo $\M_{0} \Vdash z \colon A$. Finalmente obtenemos que $\M_{0} \not \Vdash \G, \Left, x \le y, yRz, x \colon \square A, z \colon A \Rightarrow \Right$.

\end{itemize}

La corrección del resto de las reglas puede demostrarse de manera similar. De esta forma podemos concluir:

\begin{teo} El sistema $\labIKh$ es correcto.\end{teo}

Las siguientes secciones serán dedicadas a demostrar la otra propiedad que buscamos, es decir, que el sistema $\labIKh$ es completo.
%%%%%%%%%%%%%%%%%%%%%%%%%%%%%%%%%%%%%%%%%%%%%%%%%%%%%%%%%
%%%%%%%%%%%%%%%%%%%%%%%%%%%%%%%%%%%%%%%%%%%%%%%%%%%%%%%%%
%%%%%%%%%%%%%%%%%%%%%%%%%%%%%%%%%%%%%%%%%%%%%%%%%%%%%%%%%

\section{Completeness}\label{sec:completeness}

In this section we show our system at work, as most of the section
consists of derivations of axioms of $\IK$ in $\labIKp$. More precisely, we prove completeness of $\labIKp$, i.e., the implication \ref{i}$\implies$\ref{ii} of Theorem~\ref{thm:cutfree-compl}, which is stated again below:

\begin{theorem}\label{thm:completeness}
	For any formula $\fm A$. If $\fm A$ is a theorem of $\IK$ then $\fm A$ is provable in $\labIKp +\labrn{cut}$.
\end{theorem}

\begin{remark}
  We have seem already in the proof of Proposition~\ref{prop:id} the
  need of the rule $\rn{F_2}$. In the following proof of
  Theorem~\ref{thm:completeness} we also see the need of the rules
  $\rn{F_1}$, $\rn{refl}$, and $\rn{trans}$.
\end{remark}

\begin{proof}[Proof of Theorem~\ref{thm:completeness}]
  We begin by showning how the axioms $\kax[1]$--$\kax[5]$ are proved in system $\labIKp$.
  \begin{itemize}
  \item $\kax[1]$:
    $$
    \vlderivation{
				\vlin{\rlabrn{\IMP}}
				{\lb y \mbox{ fresh}}
				{\SEQ \labels{x}{\BOX (A \IMP B) \IMP (\BOX A \IMP \BOX B)}}
				{\vlin {\rlabrn{\IMP}}
					{\lb z \mbox{ fresh}}
					{\lseq{\futs xy}{\labels{y}{\BOX(A \IMP B)}}{\labels{y}{\BOX A \IMP \BOX B}}}
					{\vlin {\rlabrn{\BOX}}
						{\lb w, \lb u \mbox{ fresh}}
						{\lseq{\futs xy, \futs yz}{\labels{y}{\BOX(A \IMP B)}, \labels{z}{\BOX A}}{\labels{z}{\BOX B}}}
						{\vlin {\llabrn{\BOX}}
							{}
							{\lseq{\futs xy, \futs yz, \futs zw, \accs wu}{\labels{y}{\BOX(A \IMP B)}, \labels{z}{\BOX A}}{\labels{u}{B}}}
							{\vlin {\color{red}\rn{trans}}
								{}
								{\lseq{\futs xy, \futs yz, \futs zw, \accs wu}{\labels{y}{\BOX(A \IMP B)}, \labels{z}{\BOX A}, \labels{u}{A}}{\labels{u}{B}}}
								{\vlin {\llabrn{\BOX}}
									{}
									{\lseq{\futs xy, \futs yz, \futs zw, \futs yw, \accs wu}{\labels{y}{\BOX(A \IMP B)}, \labels{z}{\BOX A}, \labels{u}{A}}{\labels{u}{B}}}
									{\vlin {\color{red}\rn{refl}}
										{}
										{\lseq{\futs xy, \futs yz, \futs zw, \futs yw, \accs wu}{\labels{y}{\BOX(A \IMP B)}, \labels{z}{\BOX A}, \labels{u}{A}, \labels{u}{A \IMP B}}{\labels{u}{B}}}
										{\vliin{\llabrn{\IMP}}
											{}
											{\lseq{\futs xy, \futs yz, \futs zw, \futs yw, \futs uu, \accs wu}{\labels{y}{\BOX(A \IMP B)}, \labels{z}{\BOX A}, \labels{u}{A}, \labels{u}{A \IMP B}}{\labels{u}{B}}}
											{\vlin {\rn{id}}
												{}
												{\lseq{\B}{\labels{y}{\BOX(A \IMP B)}, \labels{z}{\BOX A}, \labels{u}{A}, \labels{u}{A \IMP B}}{\labels{u}{B}, \labels{u}{A}}}
												{\vlhy {}}}
											{\vlin {\rn{id}}
												{}
												{\lseq{\B}{\labels{y}{\BOX(A \IMP B)}, \labels{z}{\BOX A}, \labels{u}{A}, \labels{u}{A \IMP B}, \labels{u}{B}}{\labels{u}{B}}}
												{\vlhy {}}}}}}}}}}
    }
    $$
    where $\B$ is equal to: $\futs xy, \futs yz, \futs zw, \futs yw, \futs uu, \accs wu$.
  \item $\kax[2]$:
    $$
    \vlderivation {
		\vlin{\rlabrn{\IMP}}
		{\lb y \mbox{ fresh}}
		{ \SEQ \labels{x}{\BOX (A \IMP B) \IMP (\DIA A \IMP \DIA B)}}
		{\vlin {\rlabrn{\IMP}}
			{\lb z \mbox{ fresh}}
			{\lseq{\futs xy}{\labels{y}{\BOX (A \IMP B)}}{\labels{y}{(\DIA A \IMP \DIA B)}}}
			{\vlin {\llabrn{\DIA}}
				{\lb u \mbox{ fresh}}
				{\lseq{\futs xy, \futs yz}{\labels{y}{\BOX (A \IMP B)}, \labels{z}{\DIA A}}{\labels{z}{\DIA B}}}
				{\vlin{\rlabrn{\DIA}}
					{}
					{\lseq{\futs xy, \futs yz, \accs zu}{\labels{y}{\BOX (A \IMP B)}, \labels{u}{A}}{\labels{z}{\DIA B}}}
					{\vlin {\llabrn{\BOX}}
						{}
						{\lseq{\futs xy, \futs yz, \accs zu}{\labels{y}{\BOX (A \IMP B)}, \labels{u}{A}}{\labels{z}{\DIA B}, \labels{u}{B}}}
						{\vlin {\color{red}\rn{refl}}
							{}
							{\lseq{\futs xy, \futs yz, \accs zu}{\labels{y}{\BOX (A \IMP B)}, \labels{u}{A}, \labels{u}{A \IMP B}}{\labels{z}{\DIA B}, \labels{u}{B}}}
							{\vliin{\llabrn{\IMP}}
								{}
								{\lseq{\futs xy, \futs yz, \accs zu, \futs uu}{\labels{y}{\BOX (A \IMP B)}, \labels{u}{A}, \labels{u}{A \IMP B}}{\labels{z}{\DIA B}, \labels{u}{B}}}
								{\vlin {\rn{id}}
									{}
									{\lseq{\B}{\labels{y}{\BOX (A \IMP B)}, \labels{u}{A}, \labels{u}{A \IMP B}}{\labels{z}{\DIA B}, \labels{u}{B}, \labels{u}{A}}}
									{\vlhy {}}}
								{\vlin {\rn{id}}
									{}
									{\lseq{\B}{\labels{y}{\BOX (A \IMP B)}, \labels{u}{A}, \labels{u}{A \IMP B}, \labels{u}{B}}{\labels{z}{\DIA B}, \labels{u}{B}}}
									{\vlhy {}}}								}}}}}}
    }
    $$
    where $\B$ is equal to $\futs xy, \futs yz, \accs zu, \futs uu$.
  \item $\kax[3]$:
    $$
    \vlderivation {
			\vlin{\rlabrn{\IMP}}
			{}
			{\SEQ \labels{x}{\DIA (A \OR B) \IMP (\DIA A \OR \DIA B)}}
			{\vlin {\llabrn{\DIA}}
				{}
				{\lseq{\futs xy}{\labels{y}{\DIA (A \OR B)}}{\labels{y}{\DIA A \OR \DIA B}}}
				{\vliin{\llabrn{\OR}}{}{\lseq{\futs xy, \accs yz}{\labels{z}{A \OR B}}{\labels{y}{\DIA A \OR \DIA B}}}{\vlin {\rlabrn{\OR}}
						{}
						{\lseq{\futs xy, \accs yz}{\labels{z}{A}}{\labels{y}{\DIA A \OR \DIA B}}}
						{\vlin {\rlabrn{\DIA}}
							{}
							{\lseq{\futs xy, \accs yz}{\labels{z}{A}}{\labels{y}{\DIA A}, \labels{y}{\DIA B}}}
							{\vlin {\color{red}\rn{refl}}
								{}
								{\lseq{\futs xy, \accs yz}{\labels{z}{A}}{\labels{y}{\DIA A}, \labels{z}{A}, \labels{y}{\DIA B}}}
								{\vlin {\rn{id}}
									{}
									{\lseq{\futs xy, \futs zz, \accs yz}{\labels{z}{A}}{\labels{y}{\DIA A}, \labels{z}{A}, \labels{y}{\DIA B}}}
									{\vlhy {}}}}}}{\vlin {\rlabrn{\OR}}
						{}
						{\lseq{\futs xy, \accs yz}{\labels{z}{B}}{\labels{y}{\DIA A \OR \DIA B}}}
						{\vlin {\rlabrn{\DIA}}
							{}
							{\lseq{\futs xy, \accs yz}{\labels{z}{B}}{\labels{y}{\DIA A}, \labels{y}{\DIA B}}}
							{\vlin {\color{red}\rn{refl}}
								{}
								{\lseq{\futs xy, \accs yz}{\labels{z}{B}}{\labels{y}{\DIA A}, \labels{y}{\DIA B}, \labels{z}{B}}}
								{\vlin {\rn{id}}
									{}
									{\lseq{\futs xy, \futs zz, \accs yz}{\labels{z}{B}}{\labels{y}{\DIA A}, \labels{y}{\DIA B}, \labels{z}{B}}}
									{\vlhy {}}}}}}}}
		}
    $$
  \item $\kax[4]$:
    $$
\vlderivation {
		\vlin{\rlabrn{\IMP}}
		{\lb y \mbox{ fresh}}
		{\SEQ \labels{x}{(\DIA A \IMP \BOX B) \IMP \BOX (A \IMP B)}}
		{\vlin {\rlabrn{\BOX}}
			{\lb z, \lb w \mbox{ fresh}}
			{\lseq{\futs xy}{\labels{y}{\DIA A \IMP \BOX B}}{\labels{y}{\BOX (A \IMP B)}}}
			{\vlin {\rlabrn{\IMP}}
				{\lb u \mbox{ fresh}}
				{\lseq{\futs xy, \futs yz, \accs zw}{\labels{y}{\DIA A \IMP \BOX B}}{\labels{w}{A \IMP B}}}
				{\vlin {\color{red}{\rn{F_1}}}
					{}
					{\lseq{\futs xy, \futs yz, \futs wu, \accs zw}{\labels{y}{\DIA A \IMP \BOX B}, \labels{u}{A}}{\labels{u}{B}}}
					{\vlin {\color{red}\rn{trans}}
						{}
						{\lseq{\futs xy, \futs yz, \futs wu, \futs zt, \accs zw, \accs tu}{\labels{y}{\DIA A \IMP \BOX B}, \labels{u}{A}}{\labels{u}{B}}}
						{\vliin {\llabrn{\IMP}}
							{}
							{\lseq{\futs xy, \futs yz, \futs wu, \futs zt, \futs yt, \accs zw, \accs tu}{\labels{y}{\DIA A \IMP \BOX B}, \labels{u}{A}}{\labels{u}{B}}}
							{\vlin {\rlabrn{\DIA}}
									{}
									{\lseq{\B}{\labels{y}{\DIA A \IMP \BOX B}, \labels{u}{A}}{\labels{u}{B}, \labels{t}{\DIA A}}}
									{\vlin {\color{red}\rn{refl}}
										{}
										{\lseq{\B}{\labels{y}{\DIA A \IMP \BOX B}, \labels{u}{A}}{\labels{u}{B}, \labels{t}{\DIA A}, \labels{u}{A}}}
										{\vlin {\rn{id}}
											{}
											{\lseq{\B, \futs uu}{\labels{y}{\DIA A \IMP \BOX B}, \labels{u}{A}}{\labels{u}{B}, \labels{t}{\DIA A}, \labels{u}{A}}}
											{\vlhy {}}}}}
							{\vlin {\color{red}\rn{refl}}
								{}
								{\lseq{\B}{\labels{y}{\DIA A \IMP \BOX B}, \labels{u}{A}, \labels{t}{\BOX B}}{\labels{u}{B}}}
								{\vlin {\llabrn{\BOX}}
									{}
									{\lseq{\B, \futs tt}{\labels{y}{\DIA A \IMP \BOX B}, \labels{u}{A}, \labels{t}{\BOX B}}{\labels{u}{B}}}
									{\vlin {\color{red}\rn{refl}}
										{}
										{\lseq{\B, \futs tt}{\labels{y}{\DIA A \IMP \BOX B}, \labels{u}{A}, \labels{t}{\BOX B}, \labels{u}{B}}{\labels{u}{B}}}
										{\vlin {\rn{id}}
											{}
											{\lseq{\B, \futs tt, \futs uu}{\labels{y}{\DIA A \IMP \BOX B}, \labels{u}{A}, \labels{t}{\BOX B}, \labels{u}{B}}{\labels{u}{B}}}
											{\vlhy {}}}}}}}}}}}
}
$$
where $\B$ is equal to $\futs xy, \futs yz, \futs wu, \futs zt, \futs yt, \accs zw, \accs tu$.
\item $\kax[5]$:
  $$
  \vlderivation {
		\vlin{\rlabrn{\IMP}}
		{}
		{\SEQ \labels{x}{\DIA \BOT \IMP \BOT}}
		{\vlin {\llabrn{\DIA}}
			{}
			{\lseq{\futs xy}{\labels{y}{\DIA \BOT}}{\labels{y}{\BOT}}}
			{\vlin {\llabrn{\BOT}}
				{}
				{\lseq{\futs xy, \accs yz}{\labels{z}{\BOT}}{\labels{y}{\BOT}}}
				{\vlhy {}}}}
  }
  $$
  \end{itemize}
  Next, we have to prove all axioms of intuitionistic propositional logic can be shown in $\labIKp$. We do this only for $\fm{A \AND B \IMP B}$ and leave the rest to the reader:
  $$
  \vlderivation {
		\vlin{\rlabrn{\IMP}}
		{}
		{\SEQ \labels{x}{A \AND B \IMP B}}
		{\vlin {\llabrn{\AND}}
			{}
			{\lseq{\futs xy}{\labels{y}{A \AND B}}{\labels{y}{B}}}
			{\vlin {\color{red}\rn{refl}}
				{}
				{\lseq{\futs xy}{\labels{y}{A}, \labels{y}{B}}{\labels{y}{B}}}
				{\vlin {\rn{id}}
					{}
					{\lseq{\futs xy, \futs yy}{\labels{y}{A}, \labels{y}{B}}{\labels{y}{B}}}
					{\vlhy {}}}}}
  }
  $$
  
  
  Finally, we have to show how the rules of modus ponens and
  necessitation can be simulated in our system. For modus ponens, this
  is standard using the cut rule and for necessitation, we can
  transform a proof of $\fm A$ into a proof of $\fm{\BOX A}$ as
  follows:
  
\begin{lemma}
	If there exists a proof of $\vlderivation {\vlpd{\Done}{}{\SEQ \labels{z}{A}}}$ then there exists a proof of $\vlderivation { \vlpd{\Dtwo}{}{\SEQ \labels{x}{\BOX A}}}$
\end{lemma}

\begin{proof}
	
	We assume that there exists a proof of $\vlderivation {\vlpd {\Done}{}{\SEQ \labels{z}{A}}}$ and we want to obtain a proof of $\vlderivation { \vlpd{\Dtwo}{}{\SEQ \labels{x}{\BOX A}}}$.
	
	Using the rule $\rlabrn{\BOX}$ introduced in the system $\labIKp$ and the proof of $\labels{z}{A}$ from our hypothesis, we can build the following proof:
	$\vlderivation{\vlin{\rlabrn{\BOX}}{}{\SEQ \labels{x}{\BOX A}}{\vlin{\rn{w}}{}{\futs xy, \accs yz \SEQ \labels{z}{A}}{\vlhy{ \SEQ \labels{z}{A}}}}}$ \hspace{3mm} or what it is the same $\vlderivation{\vlin{\rlabrn{\BOX}}{}{\SEQ \labels{x}{\BOX A}}{\vlpd {\Dwone}{}{\futs xy, \accs yz \SEQ \labels{z}{A}}}}$.
	
	Let $\Dtwo$ be equal to: $\Dtwo = \vlderivation {\vlpd {\Dwone}{}{\futs xy, \accs yz \SEQ \labels{z}{A}}}$. Therefore, we have the proof $\vlderivation { \vlpd{\Dtwo}{}{\SEQ \labels{x}{\BOX A}}}$.
	
%	If we have a proof of $\vlderivation {\vlpd {\Done}{}{\SEQ \labels{z}{A}}}$ then we can obtain a proof of $\labels{x}{\BOX A}$ using the rule $\rlabrn{\BOX}$ introduced in the system $\labIKp$ and the proof of $\labels{z}{A}$ that we assumed  $\vlderivation{\vlin{\rlabrn{\BOX}}{}{\SEQ \labels{x}{\BOX A}}{\vlpd {\Dwone}{}{\futs xy, \accs yz \SEQ \labels{z}{A}}}}$
%	
%	Let $\Dtwo$ be equal to: $\Dtwo = \vlderivation {\vlpd {\Dwone}{}{\futs xy, \accs yz \SEQ \labels{z}{A}}}$.
%	Then using the \emph{weakening} rule, we have the following proof:
%	
%	\bigskip
%	\begin{center}
%		$\vlderivation{\vlin{\rlabrn{\BOX}}{}{\SEQ \labels{x}{\BOX A}}{\vlin{\rn{w}}{}{\futs xy, \accs yz \SEQ \labels{z}{A}}{\vlhy{ \SEQ \labels{z}{A}}}}}$
%	\end{center}
%	
	
	
\end{proof}

  
  This completes the proof of Proof of Theorem~\ref{thm:completeness}.
\end{proof}


%\sonia{I think the following needs to appear somewhere as it is rather illustrative of the key points. }
%
%This a a reflection on the proof-theoretic side of the model theoretic reasons to enforce $(F1)$ and $(F2)$ conditions on birelational models.
%
%In particular, in order to prove (i) $\rightarrow$ (ii), we present a syntactic completeness proof with respect the Hilbert system which means that we prove all Hilbert axioms using the rules from our system $\labIKp$: we give a proof for all the axioms of propositional intuionistic logic, for the five variants of $\mathsf{k}$ axiom from the intuitionistic syntax and finally, we simulate the necesssitation rule and modus ponens. As we mentioned, in the course of the proof of (i) $\rightarrow$ (ii), we have to derive the five $\kax$ axioms. 
%%
%As an example, we display the derivation of $\kax[4]$ which also illustrates the need of having the rule corresponding to $\rn{F_1}$ in the system.
%\bigskip
%
%\vspace*{-.9cm}
%$$
%\hspace*{-.5cm}
%\scalebox{.9}{
%	$
%	\vlderivation{
%		\vlin{\rlabrn\IMP}{\text{\scriptsize $y$ fresh}} {\SEQ \labels{x}{(\DIA A \IMP \BOX B) \SEQ \BOX (A \IMP B)}}{
%			\vlin{\rlabrn\BOX}{\text{\scriptsize $z, w$ fresh}}{x \le y, \labels{y}{\DIA A \IMP \BOX B} \SEQ \labels{y}{\BOX (A \IMP B)}}{
%				\vlin {\rlabrn\IMP}{\text{\scriptsize $u$ fresh}}{x \le y, y\le z, z \rel w, \labels{y}{\DIA A \IMP \BOX B} \SEQ \labels{w}{A \IMP B}}{
%					\vlin {\color{red}{\rn{F_1}}}{}{x \le y, y \le z, w \le u, z \rel w, \labels{y}{\DIA A \SEQ \BOX B}, \labels{u}{A} \SEQ \labels{u}{B}}{
%						\vlin {\rn{trans}}{}{x \le y, y \le z, w \le u, z \le t, z \rel w, t \rel u, \labels{y}{\DIA A \IMP \BOX B}, \labels{u}{A} \SEQ \labels{u}{B}}{
%							\vliin {\llabrn\IMP}{}{x \le y, y \le z, w \le u, z \le t, y \le t, z \rel w, t \rel u, \labels{y}{\DIA A \IMP \BOX B}, \labels{u}{A} \SEQ \labels{u}{B}}{
%								\vlin {\rlabrn\DIA}{}{x \le y, y \le z, w \le u, z \le t, y \le t, z \rel w, t \rel u, \labels{u}{A} \SEQ \labels{u}{B}, \labels{t}{\DIA A}}{
%									\vlin {\rn{refl}}{}{x \le y, y \le z, w \le u, z \le t, y \le t, z \rel w, t \rel u, \labels{u}{A} \SEQ \labels{u}{B}, \labels{t}{\DIA A}, \labels{u}{A}}{
%										\vlin {\labrn{id_g}}{}{x \le y, y \le z, w \le u, z \le t, y \le t, u \le u, z \rel w, t \rel u, \labels{u}{A} \SEQ \labels{u}{B}, \labels{t}{\DIA A}, \labels{u}{A}}{
%											\vlhy {}
%											}
%											}
%											}
%											}{
%											%						\vlin {\rn{refl}}{}{x \le y, y \le z, w \le u, z \le t, y \le t, z \rel w, t \rel u, \labels{y}{\DIA A \IMP \BOX B}, \labels{u}{A}, \labels{t}{\BOX B} \SEQ \labels{u}{B}}{
%											%							\vlin {\llabrn\BOX}{}{x \le y, y \le z, w \le u, z \le t, y \le t, t \le t, z \rel w, t \rel u, \labels{y}{\DIA A \IMP \BOX B}, \labels{u}{A}, \labels{t}{\BOX B} \SEQ \labels{u}{B}}{
%											%								\vlin {\rn{refl}}{}{x \le y, y \le z, w \le u, z \le t, y \le t, t \le t, z \rel w, t \rel u, \labels{y}{\DIA A \IMP \BOX B}, \labels{u}{A}, \labels{t}{\BOX B}, \labels{u}{B} \SEQ \labels{u}{B}}{
%											%									\vlin {\labrn{id_g}}{}{x \le y, y \le z, w \le u, z \le t, y \le t, t \le t, u \le u, z \rel w, t \rel u, \labels{y}{\DIA A \IMP \BOX B}, \labels{u}{A}, \labels{t}{\BOX B}, \labels{u}{B} \SEQ \labels{u}{B}}{
%											\vlhy {\qquad\vdots\qquad}
%											%										}
%											%									}
%											%								}
%											%							}
%											}
%											}
%											}
%											}
%											}
%											}
%											}$
%											}$$
%											%\end{example}
%											
%											%\vspace*{-.5cm}
%											
%Note that our system offers only an atomic version of the identity rule, though the above derivation uses a general version of the identity rule $\rn{id_g}$ that applies to generic formulas. 
%%
%We therefore have to show that such a rule is admissible in our system.
%%
%As an example, we display one step of this admissibility proof that also illustrates the need for the rule $\rn{F_2}$. The other cases are standard.
%
%\vspace*{-.5cm}
%%\begin{example}
%$$
%\scalebox{.9}{
%	$
%	\vlderivation{
%		\vlin{\llabrn\DIA}{}{\B, x \le y, \Left, \labels{x}{\DIA A} \SEQ \Right, \labels{y}{\DIA A}}{
%			\vlin{\color{red}{\rn{F_2}}}{}{\B, x \le y, x \rel z, \Left, \labels{z}{A} \SEQ \Right, \labels{y}{\DIA A}}{
%				\vlin{\rlabrn\DIA}{}{\B, x \le y, x \rel z, z \le u, y \rel u, \Left, \labels{z}{A} \SEQ \Right, \labels{y}{\DIA A}}{
%					\vlin{\labrn{id_g}}{}{\B, x \le y, x \rel z, z \le u, y \rel u, \Left, \labels{z}{A} \SEQ \Right, \labels{y}{\DIA A}, \labels{u}{A}}{
%						\vlhy{}
%						}
%						}
%						}
%						}
%						}
%						$
%						}
%						$$
%						%\end{example}

\chapter{Trabajo futuro}
\label{cap:future}
%En este capítulo se describirá un resultado que constituye la continuación de este trabajo final. Presentamos algunas extensiones de nuestro sistema para lógicas más fuertes, en particular, extendemos el sistema $\labIKh$ añadiendo el axioma de Scott-Lemmon o también conocido como $\agklmn$
 En este capítulo se describirán brevemente dos resultados que constituyen la continuación de este trabajo final. 
 En primer lugar discutiremos la demostración de completitud por medio del sistema de Simpson~\cite{simpson1994} mencionada anteriormente, y luego presentaremos algunas extensiones de nuestro sistema para lógicas más fuertes. En particular, para esto último, extendemos el sistema $\labIKh$ añadiendo el axioma de Scott-Lemmon o también conocido como $\agklmn$.

%\chapter{Completitud utilizando el sistema de Simpson}
\section{Completitud utilizando el sistema de Simpson}
El objetivo de esta sección, es discutir una demostración alternativa del resultado de completitud. Para ello utilizaremos el sistema de Simpson \cite{simpson1994} que se puede observar en la Figura \ref{fig:simpson}. En la sección anterior se realizó una prueba de completitud sintáctica para el nuevo sistema utilizando la regla de $\mathsf{cut}$ para simular modus ponens. Esto quiere decir que pudimos demostrar \emph{completitud con $\mathsf{cut}$} para nuestro sistema. Lo que deseamos lograr ahora, es tener un sistema completo sin $\mathsf{cut}$. Si bien se puede realizar una prueba conocida como \emph{cut elimination}, decidimos resolver este problema a partir de saber que el sistema propuesto por Simpson ya es un sistema libre de $\mathsf{cut}$. Como podemos observar en la Figura \ref{fig:simpson}, la mayoría de las reglas de secuentes que se encuentran en el sistema de Simpson (lo llamamos $\labIKs$) también están presentes en el sistema $\labIKh$, por lo que nos resta ver que las reglas que son distintas o bien no se encuentran en nuestro sistema, pueden ser deducidas a partir de la aplicación de reglas de secuentes del sistema $\labIKh$. Realizando este análisis a cada una de las reglas con éxito, podemos concluir que nuestro sistema $\labIKh$ es completo sin $\mathsf{cut}$.

\begin{teo}
	El sistema $\labIKs$ es completo sin la regla de $\mathsf{cut}$.
\end{teo}

\begin{figure}[h]
	\begin{center}
		$\vlderivation { \vlin {\idsim}{}{\G, \Left, x \colon a \Rightarrow x\colon a}{\vlhy {}}}$ \hspace{9mm} $\vlderivation { \vlin {\sbot}{}{\G, \Left, x \colon \bot \Rightarrow z\colon A}{\vlhy {}}}$
		
		\vspace{4mm}
		$\vlderivation {\vlin {\svlefs}{}{\G, \Left, x \colon \vls(A.B) \Rightarrow z \colon C}{\vlhy {\G, \Left, x \colon A, x \colon B \Rightarrow z \colon C}}}$
		\hspace{9mm}$\vlderivation { \vliin {\svrigs}{}{\G, \Left, \Rightarrow x \colon \vls(A.B)}{\vlhy {\G, \Left \Rightarrow x \colon A }}{\vlhy {\G, \Left \Rightarrow x \colon B}}}$
		
		\vspace{4mm}
		$\vlderivation {\vliin {\solefs}{}{\G, \Left, x \colon \vls[A.B] \Rightarrow z \colon C}{\vlhy {\G, \Left, x \colon A \Rightarrow z \colon C}}{\vlhy {\G, \Left, x \colon B \Rightarrow z \colon C}}}$
		\hspace{5mm}$\vlderivation { \vlin{\sorones}{}{\G, \Left \Rightarrow x \colon \vls[A.B]}{\vlhy {\G, \Left \Rightarrow x \colon A}}}$
		\hspace{5mm}$\vlderivation { \vlin {\sotwos}{}{\G, \Left \Rightarrow x \colon \vls[A.B]}{\vlhy {\G, \Left \Rightarrow x \colon B}}}$
		
		\vspace{4mm}
		$\vlderivation {\vliin{\sils}{}{\G, \Left, x \colon A \vljm B \Rightarrow z \colon C}{\vlhy {\G, \Left \Rightarrow x \colon A}}{\vlhy {\G, \Left, x \colon B \Rightarrow z \colon C}}}$
		\hspace{7mm}$\vlderivation {\vlin{\sirs}{}{\G,  \Left, x \colon A \Rightarrow x \colon B}{\vlhy {\G, \Left, x \colon A \Rightarrow x \colon B}}}$
		
		\vspace{4mm}
		$\vlderivation { \vlin {\sbls}{}{\G, xRy, \Left, x \colon \square A \Rightarrow z\colon B}{\vlhy {\G, xRy, \Left, x \colon \square A, y \colon A \Rightarrow z\colon B}}}$
		\hspace{23mm}$\vlderivation { \vlin {\sbrs}{y$ fresh$}{\G, \Left \Rightarrow x \colon \square A}{\vlhy {\G, xRy, \Left \Rightarrow y \colon A}}}$
		
		\vspace{4mm}
		$\vlderivation { \vlin{\sdls}{y$ fresh$}{\G, \Left, x \colon \Diamond A \Rightarrow z \colon B}{\vlhy {\G, xRy, \Left, y \colon A \Rightarrow z \colon B}}}$
		\hspace{7mm}$\vlderivation {\vlin {\sdrs}{}{\G,xRy, \Left \Rightarrow x \colon \Diamond A}{\vlhy {\G, xRy, \Left \Rightarrow y \colon A }}}$
	\end{center}
	\caption{System $\labIKs$}
	\label{fig:simpson}
\end{figure}

%\raul{Aca mencionar que es solo un sketch, y decir en una frase qué falta}
Lo siguiente representa un boceto de la prueba que se continuará como trabajo a futuro. Faltan incluir las demostraciones para la regla de $\sirs$ y para $\sbrs$.

\newpage
\begin{proof}
	La demostración de completitud utilizando el sistema propuesto por Simpson, como se detalló rápidamente al comienzo de esta sección, será a través de
	 análisis de casos. La mayoría de las reglas en $\labIKs$ son las mismas reglas que en el sistema $\labIKh$ excepto por las siguientes:
	\begin{center}
		\begin{itemize}
		
		\item Regla $\idsim$:
		
		\begin{center}
		$\vlderivation { \vlin {\idsim}{}{\G, \Left, x \colon A \Rightarrow x \colon a}{\vlhy {}}}$ \hspace{4mm}  \begin{huge}$  \rightarrow$ \end{huge} \hspace{4mm} $\vlderivation{\vlin {\refl}{}{\G, \Left, x \colon a \Rightarrow x \colon a}{\vlin {\ids}{}{\G,x \le x, \Left, x\colon a \Rightarrow x \colon a}{\vlhy {}}}}$
	\end{center}
		 
		
		\vspace{5mm}
		\item Regla $\sorones$ y $\sotwos$: 
		
		\begin{center}
		$\vlderivation {\vlin {\sorones}{}{\G, \Left \Rightarrow x \colon \vls[A.B]}{\vlpd {\Done}{}{\G, \Left \Rightarrow x \colon A}}}$\hspace{2mm} or \hspace{2mm}$\vlderivation {\vlin {\sotwos}{}{\G, \Left \Rightarrow x \colon \vls[A.B]}{\vlpd {\Done}{}{\G, \Left \Rightarrow x \colon B}}}$ \hspace{4mm} \begin{huge}$\rightarrow$\end{huge} \hspace{4mm} $\vlderivation {\vlin {\sorig}{}{\G, \Left \Rightarrow x \colon \vls[A.B]}{\vlpd{\Dwone}{}{\G, \Left \Rightarrow x \colon A, x \colon B}}}$
	\end{center}
	
		\newpage
		
		\item Regla $\sils$:
		
		\begin{center}
		$\vlderivation { \vliin {\sils}{}{\G, \Left, x \colon A \vljm B \Rightarrow z \colon C}{\vlpd {\Done}{}{\G, \Left \Rightarrow x \colon A}}{\vlpd {\Dtwo}{}{\G, \Left, x \colon B \Rightarrow z \colon C}}}$ \bigskip 
		
			\begin{huge}$\downarrow$\end{huge}\\
		
		 \bigskip
		 
		  $\vlderivation {\vlin {\refl}{}{\G, \Left, x \colon A \vljm B \Rightarrow z \colon C}{\vliin {\sil}{}{\G, \Left, x\le x, x \colon A \vljm B \Rightarrow z \colon C}{\vlpd {\Dwone}{}{\G, \Left, x \le x, x \colon A \vljm B \Rightarrow x\colon A}}{\vlpd {\Dwtwo}{}{\G, \Left, x \le x, x \colon B \Rightarrow z \colon C}}}}$
	\end{center}
		%\vspace{5mm}
		%\item Regla $\sirs$:
		
		%\begin{center}
		%$\vlderivation {\vlin {\sirs}{}{\G, \Left \Rightarrow x \colon A \vljm B}{\vlpd {\Done}{}{\G, \Left, x \colon A \Rightarrow x \colon B}}}$ \hspace{4mm} \begin{huge}$\rightarrow$\end{huge} \hspace{4mm} $\vlderivation {\vlin {\sir}{}{\G, \Left \Rightarrow x \colon A \vljm B}{\vlpd {\Dwone}{}{\G, \Left, x \le x, x \colon A \Rightarrow x \colon B}}}$
		%\end{center}
		
		\vspace{5mm}
		\item Regla $\sbls$:
		
		\begin{center}
		$\vlderivation {\vlin {\sbls}{}{\G, \Left, xRy, x \colon \square A \Rightarrow z \colon B}{\vlpd {\Done}{}{\G, \Left, xRy, x \colon \square A, y \colon A \Rightarrow z \colon B}}}$ \bigskip
		
			\begin{huge}$\downarrow$\end{huge}\\
		
		\bigskip
		
		 $\vlderivation {\vlin {\refl}{}{\G, \Left, xRy, x \colon \square A \Rightarrow z \colon B}{\vlin {\sbl}{}{\G, \Left, x \le x, xRy, x \colon \square A \Rightarrow z \colon B}{\vlpd {\Dwone}{}{\G, \Left, x \le x, xRy, x \colon \square A, y \colon A \Rightarrow z \colon B}}}}$
	\end{center}
	
		%\vspace{5mm}
		%\item Regla $\sbrs$:
		
		%\begin{center}
		%$\vlderivation {\vlin {\sbrs}{}{\G, \Left \Rightarrow x \colon \square A}{\vlpd {\Done}{}{\G, \Left, xRy \Rightarrow y \colon A}}}$ \hspace{4mm} \begin{huge}$\rightarrow$ \end{huge} \hspace{4mm} $\vlderivation {\vlin {\sbr}{}{\G, \Left \Rightarrow x \colon \square A}{\vlpd {\Dwone}{}{\G, \Left, x \le x, xRy \Rightarrow y \colon A}}}$
	%\end{center}
	
		\vspace{5mm}
		\item Regla $\sdrs$:
		
		\begin{center}
		$\vlderivation {\vlin {\sdrs}{}{\G, xRy, \Left \Rightarrow x \colon \Diamond A}{\vlpd {\Done}{}{\G, xRy, \Left \Rightarrow y \colon A}}}$ \hspace{4mm} \begin{huge}$\rightarrow$ \end{huge} \hspace{4mm} $\vlderivation {\vlin {\sdr}{}{\G, xRy, \Left \Rightarrow x \colon \Diamond A}{\vlpd {\Dwone}{}{\G, xRy, \Left \Rightarrow x \colon \Diamond A, y \colon A}}}$
	\end{center}
	
		\end{itemize}
	\end{center}
\end{proof}

Nuestra conjetura es que es posible demostrar completitud del sistema $\labIKh$ haciendo un análisis de cada una de las reglas de secuentes presentes en el sistema de Simpson (a partir del uso de las reglas de nuestro sistema).



%\chapter{Sistema $\labIKh$ + $\agklmn$}
\section{Extensiones del sistema $\labIKh$}

%Esta sección hace referencia a trabajo futuro por terminar. 
La idea parte de generar lógicas más fuertes extendiendo nuestro sistema con otros axiomas. Decimos \emph{una lógica más fuerte} para hacer referencia a que estamos restringiendo la clase de frames que queremos considerar, imponiendo algunas restricciones en la relación de accesibilidad. En particular, nuestro objetivo es realizar esto utilizando el \emph{axioma de Scott-Lemmon} o \emph{axioma $\agklmn$}:
\begin{center}
	 $\lozenge^{k} \square^{l} A \vljm \square^{m}\lozenge^{n} A$
\end{center}

Escribiendo nuestro axioma tanto en la lógica clásica como en la intuicionista en lenguaje de primer orden obtenemos lo siguiente:
 
$\star$ \emph{Caso clásico} \hspace{14mm} $\rightsquigarrow$ \hspace{3.7mm}$\forall x,y,z ( xR^{k}y \vlan xR^{m}z \rightarrow \exists u yR^{l}u \vlan zR^{n}u)$ 

$\star$ \emph{Caso intuicionista} \hspace{3mm} $\rightsquigarrow$ \hspace{3mm} $\forall x,y,z((xR^{k}y \vlan xR^{m}z) \vljm \exists y' (y \le y' \vlan \exists u (y'R^{l}u \vlan zR^{n} u)))$\\

Para entenderlo en mayor detalle podemos observar la Figura \ref{fig:gklmn} que representa al axioma de Scott-Lemmon para el caso intuicionista la cual nos dice que:

\begin{quote}
	Sean $w$, $u$ y $v$ mundos (o etiquetas). Si $wR^k u$ y $wR^m v$ entonces existe un mundo (o etiqueta) $u'$ tal que $u \le u'$ y existe un mundo (o etiqueta) $x$ tal que $u'R^l x$ y $vR^n x$.
\end{quote}

\begin{figure}[h]
	\begin{center}
		$
		\xymatrix{
			& u' \ar@{.>}[ddr]^{R^l} \\
			& u \ar@{.>}[u]^{\le} \\
			w \ar@{->}[ur]^{R^k}\ar@{->}[dr]_{R^m} && x \\
			& v \ar@{.>}[ur]_{R^n}
		}
		$
	\end{center}
	\label{fig:gklmn}
	\caption{Axioma $\agklmn$ para el caso intuicionista}
\end{figure}




Ahora añadimos el axioma $\agklmn$ ($\agklmn$ para el caso intuicionista) y creamos una nueva regla de sequente para nuestro sistema $\labIKh$ con el objetivo de capturar el axioma en cuestión:

\begin{center}
	$\vlderivation { \vlin {\gklmn}{y', u$ fresh$}{\G, xR^{k}y, xR^{m}z, \Left \Rightarrow \Right}{\vlhy {\G, y \le y', xR^{k}y, xR^{m}z, y'R^{l}u, zR^{n}u, \Left\Rightarrow \Right}}}$
\end{center}
%\section{Completitud}

\begin{teo}
El sistema $\labIKh$ $\mathsf{+}$ $\mathsf{ axioma}$ $\agklmn$	es completo para k = l = m = n = $\mathsf{1}$.
\end{teo}

\begin{proof}
	Como resultado de la sección \emph{Completitud Sintáctica} sabemos que $\labIKh$+$\mathsf{cut}$ es completo. Por lo tanto, sólo necesitamos mostrar que el axioma $\agklmn$ se puede probar a partir de las reglas que componen a nuestro sistema.
	Primero demostraremos completitud para \emph{k = l = m = n = $\mathsf{1}$} como se ve a continuación:
	\begin{center}
		\scalebox{0.93}{
		$\vlderivation {\vlin {\sir}
			{y$ fresh$}
			{\Rightarrow x \colon \lozenge \square A \vljm \square \lozenge A}
			{\vlin {\sbr}
				{z, w $ fresh $}
				{x \le y, y \colon \lozenge \square A \Rightarrow y \colon \square \lozenge A}
				{\vlin {\sdl}
					{u$ fresh$}
					{x \le y, y \le z, zRw, y \colon \lozenge \square A \Rightarrow w \colon \lozenge A}
					{\vlin {\ftwo}
						{t$ fresh$}
						{x \le y, y \le z, zRw, yRu, u \colon \square A \Rightarrow w \colon \lozenge A}
						{\vlin {\gklmn}
							{t', j$ fresh$}
							{x \le y, y \le z, u \le t, zRw, yRu, zRt, u \colon \square A \Rightarrow w \colon \lozenge A}
							{\vlin {\sdr}
								{}
								{x \le y, y \le z, u \le t, t \le t', zRw, yRu, zRt, t'Rj, wRj, u \colon \square A \Rightarrow w \colon \lozenge A}
								{\vlin {\trans}
									{}
									{x \le y, y \le z, u \le t, t \le t', zRw, yRu, zRt, t'Rj, wRj, u \colon \square A \Rightarrow w \colon \lozenge A, j \colon A}
									{\vlin {\sbl}
										{}
										{x \le y, y \le z, u \le t, t \le t', u \le t', zRw, yRu, zRt, t'Rj, wRj, u \colon \square A \Rightarrow w \colon \lozenge A, j \colon A}
										{\vlin {\refl}
											{}
											{x \le y, y \le z, u \le t, t \le t', u \le t', zRw, yRu, zRt, t'Rj, wRj, u \colon \square A, j \colon A \Rightarrow w \colon \lozenge A, j \colon A}
											{\vlin {\ids}
												{}
												{x \le y, y \le z, u \le t, t \le t', u \le t',j \le j, zRw, yRu, zRt, t'Rj, wRj, u \colon \square A, j \colon A \Rightarrow w \colon \lozenge A, j \colon A}
												{\vlhy {}}}}}}}}}}}}$}
	\end{center}
\end{proof}

Para la demostración de completitud del caso general, necesitamos introducir las reglas que se presentan en el Lema~\ref{lemma:admis}. 

\begin{lemma}\label{lemma:admis} Las siguientes reglas son admisibles en $\labIKh$:
	\begin{enumerate}
		\item{$\vlderivation {\vlin {\boxlk}{}{\G, \Left, x(\le \circ $R$)^{k}y, x \colon \square^{k} A\Rightarrow \Right}{\vlhy {\G, \Left, x(\le \circ $R$)^{k}y, x \colon \square^{k} A, z \colon A \Rightarrow \Right}}}$}
		\item{$\vlderivation{\vlin {\boxk}{}{\G, \Left \Rightarrow \Right, x \colon \square^{k} A}{\vlhy {\G, x(\le \circ $R$)^{k}y,\Left \Rightarrow \Right, y \colon A}}}$}
		\item{$\vlderivation { \vlin {\diamk}{}{\G, \Left, x \colon \lozenge^{k} A \Rightarrow \Right}{\vlhy {\G, xR^{k}y, \Left, y \colon A \Rightarrow \Right}}}$ }
		\item{$\vlderivation { \vlin {\diamrk}{}{\G, \Left, xR^{k}y \Rightarrow \Right, x \colon \lozenge^{k}A}{\vlhy {\G, \Left, xR^{k}y \Rightarrow \Right, x \colon \lozenge^{k}A, y \colon A}}}$}
	\end{enumerate}
	%\textbf{Probably F2g???}
\end{lemma}

Nuestra conjetura es que es posible demostrar completitud del sistema para el caso general, haciendo una demostración por inducción en el parámetro $k$. Esta demostración se deja como trabajo futuro.

% \begin{proof}
% 	Por inducción en k.
% 	\emph{\textbf{To be continued}}
% \end{proof}


%%\chapter{Completitud utilizando el sistema de Simpson}
\section{Completitud utilizando el sistema de Simpson}
El objetivo de esta sección, es discutir una demostración alternativa del resultado de completitud. Para ello utilizaremos el sistema de Simpson \cite{simpson1994} que se puede observar en la Figura \ref{fig:simpson}. En la sección anterior se realizó una prueba de completitud sintáctica para el nuevo sistema utilizando la regla de $\mathsf{cut}$ para simular modus ponens. Esto quiere decir que pudimos demostrar \emph{completitud con $\mathsf{cut}$} para nuestro sistema. Lo que deseamos lograr ahora, es tener un sistema completo sin $\mathsf{cut}$. Si bien se puede realizar una prueba conocida como \emph{cut elimination}, decidimos resolver este problema a partir de saber que el sistema propuesto por Simpson ya es un sistema libre de $\mathsf{cut}$. Como podemos observar en la Figura \ref{fig:simpson}, la mayoría de las reglas de secuentes que se encuentran en el sistema de Simpson (lo llamamos $\labIKs$) también están presentes en el sistema $\labIKh$, por lo que nos resta ver que las reglas que son distintas o bien no se encuentran en nuestro sistema, pueden ser deducidas a partir de la aplicación de reglas de secuentes del sistema $\labIKh$. Realizando este análisis a cada una de las reglas con éxito, podemos concluir que nuestro sistema $\labIKh$ es completo sin $\mathsf{cut}$.

\begin{teo}
	El sistema $\labIKs$ es completo sin la regla de $\mathsf{cut}$.
\end{teo}

\begin{figure}[h]
	\begin{center}
		$\vlderivation { \vlin {\idsim}{}{\G, \Left, x \colon a \Rightarrow x\colon a}{\vlhy {}}}$ \hspace{9mm} $\vlderivation { \vlin {\sbot}{}{\G, \Left, x \colon \bot \Rightarrow z\colon A}{\vlhy {}}}$
		
		\vspace{4mm}
		$\vlderivation {\vlin {\svlefs}{}{\G, \Left, x \colon \vls(A.B) \Rightarrow z \colon C}{\vlhy {\G, \Left, x \colon A, x \colon B \Rightarrow z \colon C}}}$
		\hspace{9mm}$\vlderivation { \vliin {\svrigs}{}{\G, \Left, \Rightarrow x \colon \vls(A.B)}{\vlhy {\G, \Left \Rightarrow x \colon A }}{\vlhy {\G, \Left \Rightarrow x \colon B}}}$
		
		\vspace{4mm}
		$\vlderivation {\vliin {\solefs}{}{\G, \Left, x \colon \vls[A.B] \Rightarrow z \colon C}{\vlhy {\G, \Left, x \colon A \Rightarrow z \colon C}}{\vlhy {\G, \Left, x \colon B \Rightarrow z \colon C}}}$
		\hspace{5mm}$\vlderivation { \vlin{\sorones}{}{\G, \Left \Rightarrow x \colon \vls[A.B]}{\vlhy {\G, \Left \Rightarrow x \colon A}}}$
		\hspace{5mm}$\vlderivation { \vlin {\sotwos}{}{\G, \Left \Rightarrow x \colon \vls[A.B]}{\vlhy {\G, \Left \Rightarrow x \colon B}}}$
		
		\vspace{4mm}
		$\vlderivation {\vliin{\sils}{}{\G, \Left, x \colon A \vljm B \Rightarrow z \colon C}{\vlhy {\G, \Left \Rightarrow x \colon A}}{\vlhy {\G, \Left, x \colon B \Rightarrow z \colon C}}}$
		\hspace{7mm}$\vlderivation {\vlin{\sirs}{}{\G,  \Left, x \colon A \Rightarrow x \colon B}{\vlhy {\G, \Left, x \colon A \Rightarrow x \colon B}}}$
		
		\vspace{4mm}
		$\vlderivation { \vlin {\sbls}{}{\G, xRy, \Left, x \colon \square A \Rightarrow z\colon B}{\vlhy {\G, xRy, \Left, x \colon \square A, y \colon A \Rightarrow z\colon B}}}$
		\hspace{23mm}$\vlderivation { \vlin {\sbrs}{y$ fresh$}{\G, \Left \Rightarrow x \colon \square A}{\vlhy {\G, xRy, \Left \Rightarrow y \colon A}}}$
		
		\vspace{4mm}
		$\vlderivation { \vlin{\sdls}{y$ fresh$}{\G, \Left, x \colon \Diamond A \Rightarrow z \colon B}{\vlhy {\G, xRy, \Left, y \colon A \Rightarrow z \colon B}}}$
		\hspace{7mm}$\vlderivation {\vlin {\sdrs}{}{\G,xRy, \Left \Rightarrow x \colon \Diamond A}{\vlhy {\G, xRy, \Left \Rightarrow y \colon A }}}$
	\end{center}
	\caption{System $\labIKs$}
	\label{fig:simpson}
\end{figure}

%\raul{Aca mencionar que es solo un sketch, y decir en una frase qué falta}
Lo siguiente representa un boceto de la prueba que se continuará como trabajo a futuro. Faltan incluir las demostraciones para la regla de $\sirs$ y para $\sbrs$.

\newpage
\begin{proof}
	La demostración de completitud utilizando el sistema propuesto por Simpson, como se detalló rápidamente al comienzo de esta sección, será a través de
	 análisis de casos. La mayoría de las reglas en $\labIKs$ son las mismas reglas que en el sistema $\labIKh$ excepto por las siguientes:
	\begin{center}
		\begin{itemize}
		
		\item Regla $\idsim$:
		
		\begin{center}
		$\vlderivation { \vlin {\idsim}{}{\G, \Left, x \colon A \Rightarrow x \colon a}{\vlhy {}}}$ \hspace{4mm}  \begin{huge}$  \rightarrow$ \end{huge} \hspace{4mm} $\vlderivation{\vlin {\refl}{}{\G, \Left, x \colon a \Rightarrow x \colon a}{\vlin {\ids}{}{\G,x \le x, \Left, x\colon a \Rightarrow x \colon a}{\vlhy {}}}}$
	\end{center}
		 
		
		\vspace{5mm}
		\item Regla $\sorones$ y $\sotwos$: 
		
		\begin{center}
		$\vlderivation {\vlin {\sorones}{}{\G, \Left \Rightarrow x \colon \vls[A.B]}{\vlpd {\Done}{}{\G, \Left \Rightarrow x \colon A}}}$\hspace{2mm} or \hspace{2mm}$\vlderivation {\vlin {\sotwos}{}{\G, \Left \Rightarrow x \colon \vls[A.B]}{\vlpd {\Done}{}{\G, \Left \Rightarrow x \colon B}}}$ \hspace{4mm} \begin{huge}$\rightarrow$\end{huge} \hspace{4mm} $\vlderivation {\vlin {\sorig}{}{\G, \Left \Rightarrow x \colon \vls[A.B]}{\vlpd{\Dwone}{}{\G, \Left \Rightarrow x \colon A, x \colon B}}}$
	\end{center}
	
		\newpage
		
		\item Regla $\sils$:
		
		\begin{center}
		$\vlderivation { \vliin {\sils}{}{\G, \Left, x \colon A \vljm B \Rightarrow z \colon C}{\vlpd {\Done}{}{\G, \Left \Rightarrow x \colon A}}{\vlpd {\Dtwo}{}{\G, \Left, x \colon B \Rightarrow z \colon C}}}$ \bigskip 
		
			\begin{huge}$\downarrow$\end{huge}\\
		
		 \bigskip
		 
		  $\vlderivation {\vlin {\refl}{}{\G, \Left, x \colon A \vljm B \Rightarrow z \colon C}{\vliin {\sil}{}{\G, \Left, x\le x, x \colon A \vljm B \Rightarrow z \colon C}{\vlpd {\Dwone}{}{\G, \Left, x \le x, x \colon A \vljm B \Rightarrow x\colon A}}{\vlpd {\Dwtwo}{}{\G, \Left, x \le x, x \colon B \Rightarrow z \colon C}}}}$
	\end{center}
		%\vspace{5mm}
		%\item Regla $\sirs$:
		
		%\begin{center}
		%$\vlderivation {\vlin {\sirs}{}{\G, \Left \Rightarrow x \colon A \vljm B}{\vlpd {\Done}{}{\G, \Left, x \colon A \Rightarrow x \colon B}}}$ \hspace{4mm} \begin{huge}$\rightarrow$\end{huge} \hspace{4mm} $\vlderivation {\vlin {\sir}{}{\G, \Left \Rightarrow x \colon A \vljm B}{\vlpd {\Dwone}{}{\G, \Left, x \le x, x \colon A \Rightarrow x \colon B}}}$
		%\end{center}
		
		\vspace{5mm}
		\item Regla $\sbls$:
		
		\begin{center}
		$\vlderivation {\vlin {\sbls}{}{\G, \Left, xRy, x \colon \square A \Rightarrow z \colon B}{\vlpd {\Done}{}{\G, \Left, xRy, x \colon \square A, y \colon A \Rightarrow z \colon B}}}$ \bigskip
		
			\begin{huge}$\downarrow$\end{huge}\\
		
		\bigskip
		
		 $\vlderivation {\vlin {\refl}{}{\G, \Left, xRy, x \colon \square A \Rightarrow z \colon B}{\vlin {\sbl}{}{\G, \Left, x \le x, xRy, x \colon \square A \Rightarrow z \colon B}{\vlpd {\Dwone}{}{\G, \Left, x \le x, xRy, x \colon \square A, y \colon A \Rightarrow z \colon B}}}}$
	\end{center}
	
		%\vspace{5mm}
		%\item Regla $\sbrs$:
		
		%\begin{center}
		%$\vlderivation {\vlin {\sbrs}{}{\G, \Left \Rightarrow x \colon \square A}{\vlpd {\Done}{}{\G, \Left, xRy \Rightarrow y \colon A}}}$ \hspace{4mm} \begin{huge}$\rightarrow$ \end{huge} \hspace{4mm} $\vlderivation {\vlin {\sbr}{}{\G, \Left \Rightarrow x \colon \square A}{\vlpd {\Dwone}{}{\G, \Left, x \le x, xRy \Rightarrow y \colon A}}}$
	%\end{center}
	
		\vspace{5mm}
		\item Regla $\sdrs$:
		
		\begin{center}
		$\vlderivation {\vlin {\sdrs}{}{\G, xRy, \Left \Rightarrow x \colon \Diamond A}{\vlpd {\Done}{}{\G, xRy, \Left \Rightarrow y \colon A}}}$ \hspace{4mm} \begin{huge}$\rightarrow$ \end{huge} \hspace{4mm} $\vlderivation {\vlin {\sdr}{}{\G, xRy, \Left \Rightarrow x \colon \Diamond A}{\vlpd {\Dwone}{}{\G, xRy, \Left \Rightarrow x \colon \Diamond A, y \colon A}}}$
	\end{center}
	
		\end{itemize}
	\end{center}
\end{proof}

Nuestra conjetura es que es posible demostrar completitud del sistema $\labIKh$ haciendo un análisis de cada una de las reglas de secuentes presentes en el sistema de Simpson (a partir del uso de las reglas de nuestro sistema).



%%\chapter{Sistema $\labIKh$ + $\agklmn$}
\section{Extensiones del sistema $\labIKh$}

%Esta sección hace referencia a trabajo futuro por terminar. 
La idea parte de generar lógicas más fuertes extendiendo nuestro sistema con otros axiomas. Decimos \emph{una lógica más fuerte} para hacer referencia a que estamos restringiendo la clase de frames que queremos considerar, imponiendo algunas restricciones en la relación de accesibilidad. En particular, nuestro objetivo es realizar esto utilizando el \emph{axioma de Scott-Lemmon} o \emph{axioma $\agklmn$}:
\begin{center}
	 $\lozenge^{k} \square^{l} A \vljm \square^{m}\lozenge^{n} A$
\end{center}

Escribiendo nuestro axioma tanto en la lógica clásica como en la intuicionista en lenguaje de primer orden obtenemos lo siguiente:
 
$\star$ \emph{Caso clásico} \hspace{14mm} $\rightsquigarrow$ \hspace{3.7mm}$\forall x,y,z ( xR^{k}y \vlan xR^{m}z \rightarrow \exists u yR^{l}u \vlan zR^{n}u)$ 

$\star$ \emph{Caso intuicionista} \hspace{3mm} $\rightsquigarrow$ \hspace{3mm} $\forall x,y,z((xR^{k}y \vlan xR^{m}z) \vljm \exists y' (y \le y' \vlan \exists u (y'R^{l}u \vlan zR^{n} u)))$\\

Para entenderlo en mayor detalle podemos observar la Figura \ref{fig:gklmn} que representa al axioma de Scott-Lemmon para el caso intuicionista la cual nos dice que:

\begin{quote}
	Sean $w$, $u$ y $v$ mundos (o etiquetas). Si $wR^k u$ y $wR^m v$ entonces existe un mundo (o etiqueta) $u'$ tal que $u \le u'$ y existe un mundo (o etiqueta) $x$ tal que $u'R^l x$ y $vR^n x$.
\end{quote}

\begin{figure}[h]
	\begin{center}
		$
		\xymatrix{
			& u' \ar@{.>}[ddr]^{R^l} \\
			& u \ar@{.>}[u]^{\le} \\
			w \ar@{->}[ur]^{R^k}\ar@{->}[dr]_{R^m} && x \\
			& v \ar@{.>}[ur]_{R^n}
		}
		$
	\end{center}
	\label{fig:gklmn}
	\caption{Axioma $\agklmn$ para el caso intuicionista}
\end{figure}




Ahora añadimos el axioma $\agklmn$ ($\agklmn$ para el caso intuicionista) y creamos una nueva regla de sequente para nuestro sistema $\labIKh$ con el objetivo de capturar el axioma en cuestión:

\begin{center}
	$\vlderivation { \vlin {\gklmn}{y', u$ fresh$}{\G, xR^{k}y, xR^{m}z, \Left \Rightarrow \Right}{\vlhy {\G, y \le y', xR^{k}y, xR^{m}z, y'R^{l}u, zR^{n}u, \Left\Rightarrow \Right}}}$
\end{center}
%\section{Completitud}

\begin{teo}
El sistema $\labIKh$ $\mathsf{+}$ $\mathsf{ axioma}$ $\agklmn$	es completo para k = l = m = n = $\mathsf{1}$.
\end{teo}

\begin{proof}
	Como resultado de la sección \emph{Completitud Sintáctica} sabemos que $\labIKh$+$\mathsf{cut}$ es completo. Por lo tanto, sólo necesitamos mostrar que el axioma $\agklmn$ se puede probar a partir de las reglas que componen a nuestro sistema.
	Primero demostraremos completitud para \emph{k = l = m = n = $\mathsf{1}$} como se ve a continuación:
	\begin{center}
		\scalebox{0.93}{
		$\vlderivation {\vlin {\sir}
			{y$ fresh$}
			{\Rightarrow x \colon \lozenge \square A \vljm \square \lozenge A}
			{\vlin {\sbr}
				{z, w $ fresh $}
				{x \le y, y \colon \lozenge \square A \Rightarrow y \colon \square \lozenge A}
				{\vlin {\sdl}
					{u$ fresh$}
					{x \le y, y \le z, zRw, y \colon \lozenge \square A \Rightarrow w \colon \lozenge A}
					{\vlin {\ftwo}
						{t$ fresh$}
						{x \le y, y \le z, zRw, yRu, u \colon \square A \Rightarrow w \colon \lozenge A}
						{\vlin {\gklmn}
							{t', j$ fresh$}
							{x \le y, y \le z, u \le t, zRw, yRu, zRt, u \colon \square A \Rightarrow w \colon \lozenge A}
							{\vlin {\sdr}
								{}
								{x \le y, y \le z, u \le t, t \le t', zRw, yRu, zRt, t'Rj, wRj, u \colon \square A \Rightarrow w \colon \lozenge A}
								{\vlin {\trans}
									{}
									{x \le y, y \le z, u \le t, t \le t', zRw, yRu, zRt, t'Rj, wRj, u \colon \square A \Rightarrow w \colon \lozenge A, j \colon A}
									{\vlin {\sbl}
										{}
										{x \le y, y \le z, u \le t, t \le t', u \le t', zRw, yRu, zRt, t'Rj, wRj, u \colon \square A \Rightarrow w \colon \lozenge A, j \colon A}
										{\vlin {\refl}
											{}
											{x \le y, y \le z, u \le t, t \le t', u \le t', zRw, yRu, zRt, t'Rj, wRj, u \colon \square A, j \colon A \Rightarrow w \colon \lozenge A, j \colon A}
											{\vlin {\ids}
												{}
												{x \le y, y \le z, u \le t, t \le t', u \le t',j \le j, zRw, yRu, zRt, t'Rj, wRj, u \colon \square A, j \colon A \Rightarrow w \colon \lozenge A, j \colon A}
												{\vlhy {}}}}}}}}}}}}$}
	\end{center}
\end{proof}

Para la demostración de completitud del caso general, necesitamos introducir las reglas que se presentan en el Lema~\ref{lemma:admis}. 

\begin{lemma}\label{lemma:admis} Las siguientes reglas son admisibles en $\labIKh$:
	\begin{enumerate}
		\item{$\vlderivation {\vlin {\boxlk}{}{\G, \Left, x(\le \circ $R$)^{k}y, x \colon \square^{k} A\Rightarrow \Right}{\vlhy {\G, \Left, x(\le \circ $R$)^{k}y, x \colon \square^{k} A, z \colon A \Rightarrow \Right}}}$}
		\item{$\vlderivation{\vlin {\boxk}{}{\G, \Left \Rightarrow \Right, x \colon \square^{k} A}{\vlhy {\G, x(\le \circ $R$)^{k}y,\Left \Rightarrow \Right, y \colon A}}}$}
		\item{$\vlderivation { \vlin {\diamk}{}{\G, \Left, x \colon \lozenge^{k} A \Rightarrow \Right}{\vlhy {\G, xR^{k}y, \Left, y \colon A \Rightarrow \Right}}}$ }
		\item{$\vlderivation { \vlin {\diamrk}{}{\G, \Left, xR^{k}y \Rightarrow \Right, x \colon \lozenge^{k}A}{\vlhy {\G, \Left, xR^{k}y \Rightarrow \Right, x \colon \lozenge^{k}A, y \colon A}}}$}
	\end{enumerate}
	%\textbf{Probably F2g???}
\end{lemma}

Nuestra conjetura es que es posible demostrar completitud del sistema para el caso general, haciendo una demostración por inducción en el parámetro $k$. Esta demostración se deja como trabajo futuro.

% \begin{proof}
% 	Por inducción en k.
% 	\emph{\textbf{To be continued}}
% \end{proof}


\section{Conclusion}

In this paper we embrace the fully labelled approach to intuitionistic modal logic as pioneered by~\cite{maffezioli:etal:synthese13} and generalise it to the class of logics defined by (one-sided intuitionistic) Scott-Lemmon axioms.
%
We establish that it is a valid approach to intuitionistic modal logic by proving soundness and completeness of our system, via a reductive cut-elimination argument.

For a restricted class of logics defined by so-called \emph{path axioms}: $(\DIA^k\BOX A \IMP\BOX^m A) \AND (\DIA^k A \IMP \BOX^m\DIA A)$ the standard labelled framework with one relation $R$ was enough for Simpson to get a strong connection between the sequent system, the axiomatisation and the birelational semantics~\ref{simpson:phd}.
%
%\todo{For path axioms we must have that adding the rule for the two sides of the axiom is equivalent to adding the rule that doesn't make use of the pre-order. Is it possible to see that directly on the proof theoretical side?}
%\sonia{probably not because the preorder is introduced/used in other rules for $\BOX$ and $\IMP$}
%
We believe that the framework presented here might be the more appropriate way to treat logics outside of the path axioms definable fragment.

However, we have not showed that our system satisfies Simpson's 6th requirement, that is, "there is an intuitionistically comprehensible explanation of the meaning of the modalities relative to which [our system] is sound and complete".
%
To make sure that his system satisfies this requirement, Simpson chose to depart from the direct correspondence with modal axioms and their corresponding class of Kripke frames, and to study intuitionistic purely as a fragment of intuitionistic first-order logic.
%
We instead took the way of a direct correspondence of our system with the class of frames defined by Scott-Lemmon axioms as uncovered by~\cite{plotkin:stirling:86}, but as this class of logics seems to be rather well-behaved, we believe it shoud be possible to prove the satisfaction of Simpson's 6th requirement too.

As for more general future work, there is a real necessity of a global view on intutionistic modal logics.
%
The work of~\cite{dalmonte:grellois:olivetti:arxiv19} is a great first step in understanding them in the context of non-normal modalities and neighbourhood semantics.
%
It would be interesting to know how and where the class of logics we considered can be included in their framework.
%\addtocontents{toc}{\vspace{2em}} 
%\appendix 


\bibliographystyle{named}
\addcontentsline{toc}{chapter}{ Bibliografía}
\bibliography{bibliography}


\end{document}
