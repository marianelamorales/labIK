\newcommand{\G}{\mathcal{G}}
\newcommand{\F}{\mathcal{F}}
\newcommand{\Right}{\mathcal{R}}
\newcommand{\M}{\mathcal{M}}
\newcommand{\Left}{\mathcal{L}}
\newcommand{\f}{f^{\mathcal{M}}}
\newcommand{\taxiom}{\mathsf{t}}
\newcommand{\daxiom}{\mathsf{d}}
\newcommand{\baxiom}{\mathsf{b}}
\newcommand{\fouraxiom}{\mathsf{4}}
\newcommand{\fiveaxiom}{\mathsf{5}}
\newcommand{\twoaxiom}{\mathsf{2}}
\newcommand{\sfive}{\mathsf{S5}}
\newcommand{\labK}{\mathsf{labK}}
\newcommand{\K}{\mathsf{K}}
\newcommand{\IK}{\mathsf{IK}}
\newcommand{\labIKh}{\mathsf{labIK\heartsuit}}
\newcommand{\labIKs}{\mathsf{labIK_{s}}}
\newcommand{\system}{\mathsf{S}}
\newcommand{\agklmn}{\mathsf{g_{klmn}}}
\newcommand{\kaxiom}{\mathsf{k}}

% For Simpson
\newcommand{\idsim}{id_{s}}
\newcommand{\svlefs}{\wedge_{Ls}}
\newcommand{\svrigs}{\wedge_{Rs}}
\newcommand{\solefs}{\vlor_{Ls}}
\newcommand{\sorones}{\vlor_{R1s}}
\newcommand{\sotwos}{\vlor_{R2s}}
\newcommand{\sirs}{\vljm_{Rs}}
\newcommand{\sils}{\vljm_{Ls}}
\newcommand{\sdls}{\Diamond_{Ls}}
\newcommand{\sdrs}{\Diamond_{Rs}}
\newcommand{\sbls}{\square_{Ls}}
\newcommand{\sbrs}{\square_{Rs}}

%Symbols for System labK
\newcommand{\id}{id^{lab}}
\newcommand{\tolab}{\top^{lab}}
\newcommand{\vlab}{\wedge^{lab}}
\newcommand{\olab}{\vlor^{lab}}
\newcommand{\blab}{\square^{lab}}
\newcommand{\dlab}{\Diamond^{lab}}

%Symbols for cut rule
\newcommand{\Gone}{\mathcal{G}_{1}}
\newcommand{\Gtwo}{\mathcal{G}_{2}}
\newcommand{\Dw}{\mathcal{D}^{w}}
\newcommand{\Dwone}{\mathcal{D}_{1}^{w}}
\newcommand{\Dwtwo}{\mathcal{D}_{2}^{w}}
\newcommand{\D}{\mathcal{D}}
\newcommand{\Done}{\mathcal{D}_{1}}
\newcommand{\Dtwo}{\mathcal{D}_{2}}

%Labelled proof system

\newcommand{\toprule}{\G \Rightarrow \Right, x  \colon   \top}
%\newcommand{\vlabr}{G \Rightarrow Right, x   \colon  \vls(A.B)}
\newcommand{\vlabr}{\G \Rightarrow \Right, x  \colon   A}
\newcommand{\vlabu}{\G \Rightarrow \Right, x  \colon   B}
\newcommand{\olabr}{\G \Rightarrow \Right, x  \colon   A, x  \colon   B}
\newcommand{\blabr}{\G \Rightarrow \Right, x  \colon   \square A}
\newcommand{\blabu}{\G, x$R$y \Rightarrow \Right, y  \colon   A}
\newcommand{\dlabr}{\G, x$R$y \Rightarrow \Right, x  \colon   \Diamond A}
\newcommand{\dlabu}{\G, x$R$y \Rightarrow \Right, x  \colon   \Diamond A, y  \colon   A}

%System labIK+gklmn
\newcommand{\gklmn}{\boxtimes_{gklmn}}
\newcommand{\boxk}{\square_{R}^{k}}
\newcommand{\boxlk}{\square_{L}^{k}}
\newcommand{\diamk}{\lozenge_{L}^{k}}
\newcommand{\diamrk}{\lozenge_{R}^{k}}


%New system
\newcommand{\conjrig}{\G, \Left \Rightarrow \Right, x \colon A}
\newcommand{\conjrigh}{\G, \Left \Rightarrow \Right, x  \colon B}
\newcommand{\conjlef}{\G, \Left, x  \colon  A, x \colon B \Rightarrow \Right}

\newcommand{\ids}{id}
\newcommand{\idg}{id_{g}}
\newcommand{\refl}{refl}
\newcommand{\trans}{trans}
\newcommand{\cut}{cut}
\newcommand{\fone}{F1}
\newcommand{\ftwo}{F2}
\newcommand{\sbot}{\bot_{L}}
\newcommand{\Stop}{\top_{R}}
\newcommand{\svlef}{\wedge_{L}}
\newcommand{\svrig}{\wedge_{R}}
\newcommand{\solef}{\vlor_{L}}
\newcommand{\sorig}{\vlor_{R}}
\newcommand{\sorone}{\vlor_{R1}}
\newcommand{\sotwo}{\vlor_{R2}}
\newcommand{\sir}{\vljm_{R}}
\newcommand{\sil}{\vljm_{L}}
\newcommand{\sdl}{\Diamond_{L}}
\newcommand{\sdr}{\Diamond_{R}}
\newcommand{\sbl}{\square_{L}}
\newcommand{\sbr}{\square_{R}}
\newcommand{\smon}{mon_{L}}


%%%%%%%%%%%%%%%%%%%%%%%%%%%%%%%%%%%%%%
%------ ---Tuples and sets-----------%
%%%%%%%%%%%%%%%%%%%%%%%%%%%%%%%%%%%%%%
% Tuples \tup{x,y} = <x,y>
\newcommand{\tup}[1]{\langle #1 \rangle}
% Tuples \ttup{x,y} = <<x,y>>
\newcommand{\ttup}[1]{\langle\!\langle \mathrm{#1} \rangle\!\rangle}
% Sets \cset{x,y} = {x,y}
\newcommand{\cset}[1]{\{ #1 \}}


%\newcommand{\raul}[1]{\todo[color=yellow!55]{{\bf Raul:} #1}\xspace}
\newcommand{\raul}[1]{\todo[inline,color=yellow!55]{{\bf Raul:} #1}}

%\newcommand{\male}[1]{\todo[color=blue!20]{{\bf Male:} #1}\xspace}
\newcommand{\male}[1]{\todo[inline,color=green!20]{{\bf Male:} #1}}

%%%%%%%%%%%%%%%%%%%%%%%%%%%%%%%%%%%%%%
%-----------for tikz-----------------%
%%%%%%%%%%%%%%%%%%%%%%%%%%%%%%%%%%%%%%
	\tikzset{
		>=stealth',
		ar/.style={
			shorten <=2pt,
			shorten >=2pt,}
	}
	
	\tikzstyle{blankback}=[fill=white]
	
	% Double arrows 
	\tikzstyle{vecArrow} = [thick, decoration={markings,mark=at position
		1 with {\arrow[semithick]{open triangle 60}}},
	double distance=1.4pt, shorten >= 5.5pt,
	preaction = {decorate},
	postaction = {draw,line width=1.4pt, white,shorten >= 4.5pt}]
	
%k1 Macro
\newcommand{\kone}{
	\begin{center}
			$\vlderivation{
				\vlin{\sir}
				{y$ fresh$}
				{\Rightarrow x \colon \square (A \vljm B) \vljm (\square A \vljm \square B)}
				{\vlin {\sir}
					{z$ fresh$}
					{x \le y, y \colon \square(A \vljm B) \Rightarrow y \colon \square A \vljm \square B}
					{\vlin {\sbr}
						{w, u$ fresh$}
						{x \le y,y \le z, y \colon \square(A \vljm B), z \colon \square A \Rightarrow z \colon \square B}
						{\vlin {\sbl}
							{}
							{x \le y,y \le z, z \le w, wRu, y \colon \square(A \vljm B), z \colon \square A \Rightarrow u \colon B}
							{\vlin {\trans}
								{}
								{x \le y,y \le z, z \le w, wRu, y \colon \square(A \vljm B), z \colon \square A, u \colon A \Rightarrow u \colon B}
								{\vlin {\sbl}
									{}
									{x \le y,y \le z, z \le w, y \le w, wRu, y \colon \square(A \vljm B), z \colon \square A, u \colon A \Rightarrow u \colon B}
									{\vlin {\refl}
										{}
										{x \le y,y \le z, z \le w, y \le w, wRu, y \colon \square(A \vljm B), z \colon \square A, u \colon A, u \colon A \vljm B \Rightarrow u \colon B}
										{\vliin{\sil}
											{}
											{x \le y,y \le z, z \le w, y \le w, u \le u, wRu, y \colon \square(A \vljm B), z \colon \square A, u \colon A, u \colon A \vljm B \Rightarrow u \colon B}
											{\vlhy{\circledast \hspace{75mm}}}
											{\vlhy{\star}}}}}}}}}
			}$
	\end{center}
	
	
	La derivación continúa con dos premisas luego de aplicar la regla de implicación izquierda $\sil$. Por un lado la premisa obtenida en $\circledast$ resulta en la siguiente:
	
	\scalebox{0.95}{
	$\vlderivation{
		\vlin{\sil}
		{}
		{\circledast}
		{\vlin {\ids}
			{}
			{x \le y,y \le z, z \le w, y \le w, u \le u, wRu, y \colon \square(A \vljm B), z \colon \square A, u \colon A, u \colon A \vljm B \Rightarrow u \colon B, u \colon A}
			{\vlhy {}}}
	}$}
	
	La otra premisa obtenida en $\star$ está dado por:
	
	\scalebox{0.95}{
	$\vlderivation{
		\vlin{\sil}
		{}
		{\star}
		{\vlin {\ids}
			{}
			{x \le y,y \le z, z \le w, y \le w, u \le u, wRu, y \colon \square(A \vljm B), z \colon \square A, u \colon A, u \colon A \vljm B, u \colon B \Rightarrow u \colon B}
			{\vlhy {}}}
	}$}}

%k2 Macro
\newcommand{\ktwo}{\begin{center}
		$\vlderivation {
			\vlin{\sir}
			{y$ fresh$}
			{ \Rightarrow x\colon \square (A \vljm B) \vljm (\Diamond A \vljm \Diamond B)}
			{\vlin {\sir}
				{z$ fresh$}
				{ x \le y, y \colon \square (A \vljm B) \Rightarrow y \colon (\Diamond A \vljm \Diamond B)}
				{\vlin {\sdl}
					{u$ fresh$}
					{x \le y, y \le z, y \colon \square (A \vljm B), z \colon \Diamond A \Rightarrow z \colon \Diamond B}
					{\vlin{\sdr}
						{}
						{x \le y, y \le z, zRu, y \colon \square (A \vljm B), u \colon A \Rightarrow z \colon \Diamond B}
						{\vlin {\sbl}
							{}
							{x \le y, y \le z, zRu, y \colon \square (A \vljm B), u \colon A \Rightarrow z \colon \Diamond B, u \colon B}
							{\vlin {\refl}
								{}
								{x \le y, y \le z, zRu, y \colon \square (A \vljm B), u \colon A, u \colon A \vljm B \Rightarrow z \colon \Diamond B, u \colon B}
								{\vliin{\sil}
									{}
									{x \le y, y \le z, zRu, u\le u, y \colon \square (A \vljm B), u \colon A, u \colon A \vljm B \Rightarrow z \colon \Diamond B, u \colon B}
									{\vlhy{\circledast \hspace{75mm}}}
									{\vlhy{\star}}								}}}}}}
		}$
	\end{center}
	
	Podemos ver que nuestra derivación continua con dos premisas como resultado de aplicar la regla de implicación izquierda que resultan de la siguiente manera:
	
	En $\circledast$ la derivación se sigue de:
	
	\begin{center}
	$\vlderivation{
		\vlin{\sil}
		{}
		{\circledast}
		{\vlin {\ids}
			{}
			{x \le y, y \le z, z$R$u, u \le u, y \colon \square (A \vljm B), u \colon A, u \colon A \vljm B \Rightarrow z \colon \Diamond B, u \colon B, u \colon A}
			{\vlhy {}}}
	}$
\end{center}

	Y en $\star$ la premisa obtenida está dada por:
	
	\begin{center}
			$\vlderivation{
				\vlin{\sil}
				{}
				{\star}
				{\vlin {\ids}
					{}
					{x \le y, y \le z, zRu, u \le u, y \colon \square (A \vljm B), u \colon A, u \colon A \vljm B, u \colon B \Rightarrow z \colon \Diamond B, u \colon B}
					{\vlhy {}}}
			}$
	\end{center}
}


\newcommand{\orthree}{\begin{center}
		
		$\vlderivation {
			\vlin{\sir}
			{}
			{\Rightarrow x \colon (A \vljm C)\vljm ((B \vljm C) \vljm ( A \vlor B \vljm C))}
			{\vlin {\sir}
				{}
				{x \le y, y \colon A \vljm C \Rightarrow y \colon (B \vljm C) \vljm (A \vlor B \vljm C)}
				{\vlin {\sir}
					{}
					{x \le y, y \le z, y \colon A \vljm C, z \colon B \colon C \Rightarrow z \colon A \vlor B  \vljm C}
					{\vlin {\solef}
						{}
						{x \le y, y \le z, z \le w, y \colon A \vljm C, z \colon B \vljm C, A \vlor B \Rightarrow w \colon C}
						{\vliin {\sil}
							{}
							{x \le y, y \le z, z \le w, y \colon A \vljm C, z \colon B \vljm C, w \colon A, w \colon B \Rightarrow w \colon C }
							{\vlhy{\circledast \hspace{75mm}}}
							{\vlhy{\star}}}}}}
		}$
	\end{center}
	
	\bigskip
	
	La derivación del axioma \textbf{OR-3} continua por las ramas representadas por $\circledast$ y por $\star$ como resultado de haber aplicado la regla izquierda de implicación $\sil$.
	
	La derivación proveniente de $\circledast$ se sigue de:
	
	\begin{center}
		$\vlderivation{
			\vlin{\sil}{}{\circledast}{\vlin {\refl}
				{}
				{x \le y, y \le z, z \le w, y \colon A \vljm C, z \colon B \vljm C, w \colon A, w \colon B \Rightarrow w \colon C, w \colon A}
				{\vlin {\ids}
					{}
					{x \le y, y \le z, z \le w, w \le w, y \colon A \vljm C, z \colon B \vljm C, w \colon A, w \colon B \Rightarrow w \colon C, w \colon A}
					{\vlhy {}}}}
		}$
	\end{center}	
	
	La derivación que se sigue de $\star$ es la siguiente:
	
	\begin{center}
		$\vlderivation{
			\vlin{\sil}{}{\star}{\vlin {\refl}
				{}
				{x \le y, y \le z, z \le w, y \colon A \vljm C, z \colon B \vljm C, w \colon A, w \colon B, w \colon C \Rightarrow w \colon C }
				{\vlin {\ids}
					{}
					{x \le y, y \le z, z \le w, w \le w, y \colon A \vljm C, z \colon B \vljm C, w \colon A, w \colon B, w \colon C \Rightarrow w \colon C}
					{\vlhy {}}}}
		}$
	\end{center}}