%%%%%%%%%%%%%%%%%%%%%%%%%%%%%%%%%%%%%%%%%%%%%%%%%%%%%%%%%
%%%%%%%%%%%%%%%%%%%%%%%%%%%%%%%%%%%%%%%%%%%%%%%%%%%%%%%%%
%%%%%%%%%%%%%%%%%%%%%%%%%%%%%%%%%%%%%%%%%%%%%%%%%%%%%%%%%
%
%
\section{The system}\label{sec:system}

In this section we present our fully labelled sequent proof system
for intuitionistic modal logics. The starting point is the notion of a
\emph{labelled formula} which is a pair $\labels xA$ of a label $\lb
x$ and a formula $\fm A$. The intended meaning is that $\fm A$ holds
at world $\lb x$. A \emph{relation atom} is either an expression
$\accs xy$ or $\futs xy$ where $\lb x$ and $\lb y$ are labels and the
intended meaning is that the corresponing relation holds between the
worlds $\lb x$ and $\lb y$. 
\sonia{Not sure we want to make it sound so semantical here, with not much context.}
%
This is enough to define the notion of
\emph{(labelled) sequent}, which is a triple $\lseq\B\Left\Right$,
where $\B$ is a set of relational and pre-order atoms, 
\sonia{inconsistent naming with intro}
and $\Left$ and $\Right$ are
multi-sets of labelled formulas, all written as list, separated by commas.

\begin{figure}[!t]
  \begin{center}
	\fbox{
%	\begin{minipage}{.95\textwidth}
	\begin{tabular}{c@{\quad}c}
		$\vlinf{\rn{id}}{}{\B, \futs xy, \Left, \labels{x}{a} \SEQ \Right, \labels{y}{a} }{}$
		&
		$\vlinf{\llabrn\bot}{}{\B, \Left, \labels{x}{\BOT} \SEQ \Right}{}$
		%\quad
		%$\vlinf{\rlabrn\top}{}{\B, \Left \SEQ \Right, \labels{x}{\TOP}}{}$
		\\\\
		$\vlinf{\llabrn\AND}{}{\B,\Left, \labels{x}{A \AND B} \SEQ \Right}{\B, \Left, \labels{x}{A}, \labels{x}{B} \SEQ \Right}$
		&
		$\vliinf{\rlabrn\AND}{}{\B,\Left \SEQ \Right, \labels{x}{A \AND B}}{\B, \Left \SEQ \Right, \labels{x}{A}}{\B, \Left \SEQ \Right, \labels{x}{B}}$
		\\\\
		$\vliinf{\llabrn\OR}{}{\B, \Left, \labels{x}{A \OR B} \SEQ \Right}{\B, \Left, \labels{x}{A} \SEQ \Right}{\B, \Left, \labels{x}{B} \SEQ \Right}$
		&
		$\vlinf{\rlabrn\OR}{}{\B, \Left \SEQ \Right, \labels{x}{A \OR B}}{\B, \Left \SEQ \Right, \labels{x}{A}, \labels{x}{B}}$
		\\\\
		\multicolumn{2}{c}{
		$\vlinf{\rlabrn\IMP}{\lb y \mbox{ fresh}}{\B, \Left \SEQ \Right, \labels{x}{A \IMP B}}{\B, \futs xy, \Left, \labels{y}{A} \SEQ \Right, \labels{y}{B}}$
		}
		\\\\
		\multicolumn{2}{c}{
 		$\vliinf{\llabrn\IMP}{}{\B, \futs xy, \Left, \labels{x}{A \IMP B} \SEQ \Right}{\B, \futs xy, \labels{x}{A \IMP B}, \Left \SEQ \Right, \labels{y}{A}}{\B, \futs xy, \Left, \labels{y}{B} \SEQ \Right}$ 
% 		&$\vliinf{\rlabrn\IMP}{\text{\scriptsize $x \le y \in \B$}}{\B, \Left, \labels{x}{A \IMP B} \SEQ \Right}{\B, \Left \SEQ \Right, \labels{y}{A}}{\B, \Left, \labels{y}{B} \SEQ \Right}$
 		}
		\\\\
		$\vlinf{\llabrn\BOX}{}{\B, \futs xy, \accs yz, \Left, \labels{x}{\BOX A} \SEQ \Right}{\B, x \le y, \accs yz, \Left, \labels{x}{\BOX A}, \labels{z}{A} \SEQ \Right}$
		&
		$\vlinf{\rlabrn\BOX}{\lb y, \lb z \mbox{ fresh}}{\B, \Left \SEQ \Right, \labels{x}{\BOX A}}{\B, \futs xy, \accs yz, \Left \SEQ \Right, \labels{z}{A}}$
		\\\\
		$\vlinf{\llabrn\DIA}{\lb y \mbox{ fresh}}{\B, \Left, \labels{x}{\DIA A} \SEQ \Right}{\B, \accs xy, \Left, \labels{y}{A} \SEQ \Right}$
		&
		$\vlinf{\rlabrn\DIA}{}{\B, \accs xy, \Left \SEQ \Right, \labels{x}{\DIA A}}{\B, \accs xy, \Left \SEQ \Right, \labels{x}{\DIA A}, \labels{y}{A}}$
		\\
		\multicolumn{2}{c}{
		$\mbox{\hbox to .9\linewidth{\dotfill}}$
		}
		\\
		$\vlinf{\rn{refl}}{}{\B, \Left \SEQ \Right}{\B, \futs xx, \Left \SEQ \Right}$
		&
		$\vlinf{\rn{trans}}{}{\B, \futs xy, \futs yz, \Left \SEQ \Right}{\B, \futs xy, \futs yz, \futs xz, \Left \SEQ \Right}$
		\\\\
		\multicolumn{2}{c}{
		$\vlinf{\rn{F_1}}{\lb u \mbox{ fresh}}{\B, \accs xy, \futs yz, \Left \SEQ \Right}{\B, \accs xy, \futs yz, \futs xu, \accs uz, \Left \SEQ \Right}$
		}
		\\\\
		\multicolumn{2}{c}{
		$\vlinf{\rn{F_2}}{\lb u \mbox{ fresh}}{\B, \accs xy, \futs xz, \Left \SEQ \Right}{\B, \accs xy, \futs xz, \futs yu, \accs zu, \Left \SEQ \Right }$		
		}
	\end{tabular}		
%\end{minipage}
}		
  \end{center}
  \caption{System $\labIKp$}
	\label{fig:labIKp}
\end{figure}

Now we can present the inference rules for \emph{system $\labIKp$} for the logic $\IK$.
We obtained this system, shown in Figure~\ref{fig:labIKp}, as follows.
\sonia{This is an interesting alternative view but for me the starting point really is the Maffezioli, Naibo and Negri paper. Maybe rephrase slightly to acknowledge their work too.}
Our starting point was the multiple-conlusion nested sequent system \emph{\`a la} Maehara (as presented in~\cite{str:2017maehara}), which can straightforwardly translated into the labelled setting, and which yields the rules $\llabrn\bot$, $\llabrn\AND$, $\rlabrn\AND$, $\llabrn\OR$, $\rlabrn\OR$, $\llabrn\DIA$, and $\rlabrn\DIA$ as we show them in Figure~\ref{fig:labIKp}. However, this naive translation wpould also yield the rules $\rn{id'}$, $\llabrn\IMP'$, and~$\llabrn\BOX'$:
\begin{equation}
  \label{eq:nomon}
  \vlinf{\rn{id}'}{}{\B, \Left, \labels{x}{a} \SEQ \Right, \labels{x}{a} }{}
  \quad
  \vliinf{\llabrn\IMP'}{}{\B,\Left, \labels{x}{A \IMP B} \SEQ \Right}{\B,\labels{x}{A \IMP B}, \Left \SEQ \Right, \labels{x}{A}}{\B, \Left, \labels{x}{B} \SEQ \Right}
  \quad
  \vlinf{\llabrn\BOX}{}{\B, \accs xz, \Left, \labels{x}{\BOX A} \SEQ \Right}{\B, \accs xz, \Left, \labels{x}{\BOX A}, \labels{z}{A} \SEQ \Right}
\end{equation}
that are not sufficient for a complete system, and we will see below
why. Before, let us first look at the rules $\rlabrn\IMP$ and
$\rlabrn\BOX$. In the multiple-conlusion nested sequent system
of~\cite{str:2017maehara}, these are the two rules that force
single-conclusion. This is the reason why in our system we need the
$\le$ in the relational atoms; we need to go to the future of the
current world.  In the Kripke-semantics in~\eqref{eq:kripke} the two
connectives $\IMP$ and $\BOX$ are the ones that need the
future-relation $\le$. This relation is reflexive and transitive. In
order to capture that in the proof system, we need to add the rules
$\rn{refl}$ and $\rn{trans}$.

Finally, in the semantics, the two relations $\rel$ and $\le$ are
strongly connected through the two conditions $\rn{F_1}$ and $\rn{F_2}$. These need to be reflected in the proof
system, which is done by the two rules $\rn{F_1}$ and $\rn{F_2}$.
%
However, these two rule create new labels, and in order to be
complete, the system needs the \emph{monotonicity} rule $\llabrn{mon}$, shown on the left below. 
\begin{equation}
  \label{eq:mono}
  \vlinf{\llabrn{mon}}{}{\lseq{\B, \futs{x}{x'}}{\Left, \labels{x}{A}}\Right}{
    \lseq{\B, \futs{x}{x'}}{\Left, \labels{x}{A}, \labels{x'}{A}}\Right}
  \hskip6em
  \vlinf{\rlabrn{mon}}{}{\lseq{\B, \futs{x}{x'}}{\Left}{\Right, \labels{x'}{A}}}{
    \lseq{\B, \futs{x}{x'}}{\Left}{\Right, \labels{x}{A}, \labels{x'}{A}}}
\end{equation}
Since this rule is a form of contraction, it would cause the same
problems as contraction in a cut elimination proof. Hence, it is
preferable to have a system in which this rule is admissible. This is
the reason why we have monotonicity incorporated in the rules
$\rn{id}$, $\llabrn\IMP$ and $\llabrn\BOX$ in Figure~\ref{fig:labIKp},
instead of using the rules in~\eqref{eq:nomon}. Then, not only
$\llabrn{mon}$ but also its right-hand side version $\rlabrn{mon}$,
shown on the right in~\eqref{eq:mono} above is admissible.

\begin{proposition}
  \label{prop:mon-adm}
  The rules $\llabrn{mon}$ and $\rlabrn{mon}$ are admissible for $\labIKp$.
\end{proposition}

One can prove this proposition in the same way as one usually proves
admissibity of contraction in a sequent calculus, by induction on the
height of the derivation, which in fact would yield a stronger result,
namely that $\llabrn{mon}$ and $\rlabrn{mon}$ are \emph{height
  preserving} admissible for $\labIKp$. However, we do not need this
result in this paper, and therefore we leave it to the interested
reader. Nonetheless, we will give a short proof of
Proposition~\ref{prop:mon-adm} at the end of this section.

Before, let us give another indication of the fact that $\labIKp$ is well-designed, namely that the general identity axiom is admissible.

\begin{proposition}
  \label{prop:id}
	The following general idenity axiom~
	{$\vlinf{\labrn{id_g}}{}{\B, \futs xy, \Left, \labels{x}{A} \SEQ \Right, \labels{y}{A}}{}$}
        ~is admissible for $\labIKp$. 
\end{proposition}

\begin{proof}
	As standard, we proceed by structural induction on $\fm A$. The two base cases $\fm A=\fm a$ and $\fm A=\fm\bot$ are trivial. The inductive cases are shown below.
	\begin{itemize}
		%% \item Atomic case
		%% \begin{smallequation*}
		%% 	%\vlinf{\labrn{id_g}}{}{\B, x \le y; \Left, \labels{x}{a} \SEQ \Right, \labels{y}{a}}{}
		%% 	%\reducesto
		%% 	\vlinf{\labrn{id}}{}{\B, \futs xy, \Left, \labels{x}{a} \SEQ \Right, \labels{y}{a}}{}
		%% \end{smallequation*}
		\item $\fm{A \AND B}$
		%\begin{smallequation*}
		%	\vlinf{\labrn{id_g}}{}{\B, x \le y; \Left, \labels{x}{A \AND B} \SEQ \Right, \labels{y}{A \AND B}}{}
		%	\reducesto
		%\end{smallequation*}
		\begin{smallequation*}
			\vlderivation{
				\vlin{\llabrn\AND}{}{\B, \futs xy, \Left, \labels{x}{A \AND B} \SEQ \Right, \labels{y}{A \AND B}}{
					\vliin{\rlabrn\AND}{}{\B, \futs xy, \Left, \labels{x}{A}, \labels{x}{B} \SEQ \Right, \labels{y}{A \AND B}}{
						\vlin{\labrn{id_g}}{}{\B, \futs xy, \Left, \labels{x}{A}, \labels{x}{B} \SEQ \Right, \labels{y}{A}}{
							\vlhy{}
						}
					}{
					\vlin{\labrn{id_g}}{}{\B, \futs xy, \Left, \labels{x}{A}, \labels{x}{B} \SEQ \Right, \labels{y}{B}}{
						\vlhy{}
					}
				}
			}
		}
	\end{smallequation*}
	\item $\fm{A \OR B}$
	%\begin{smallequation*}
	%	\vlinf{\labrn{id_g}}{}{\B, x \le y; \Left, \labels{x}{A \OR B} \SEQ \Right, \labels{y}{A \OR B}}{}
	%	\reducesto
	%\end{smallequation*}
	\begin{smallequation*}
		\vlderivation{
			\vliin{\llabrn\OR}{}{\B, \futs xy, \Left, \labels{x}{A \OR B} \SEQ \Right, \labels{y}{A \OR B}}{
				\vlin{\rlabrn\OR}{}{\B, \futs xy, \Left, \labels{x}{A} \SEQ \Right, \labels{y}{A \OR B}}{
					\vlin{\labrn{id_g}}{}{\B, \futs xy, \Left, \labels{x}{A} \SEQ \Right, \labels{y}{A}}{
						\vlhy{}
					}
				}
			}{
			\vlin{\rlabrn\OR}{}{\B, \futs xy, \Left, \labels{x}{B} \SEQ \Right, \labels{y}{A \OR B}}{
				\vlin{\labrn{id_g}}{}{\B, \futs xy, \Left, \labels{x}{B} \SEQ \Right, \labels{y}{B}}{
					\vlhy{}
				}
			}
		}
	}
	\end{smallequation*}
	
	\item $\fm{A \IMP B}$
%	\begin{smallequation*}
%		\vlinf{\labrn{id_g}}{}{\B, x \le y; \Left, \labels{x}{A \IMP B} \SEQ \Right, \labels{y}{A \IMP B}}{}
%		\reducesto
%	\end{smallequation*}
	\begin{smallequation*}
		\vlderivation{
			\vlin{\rlabrn\IMP}{\lb z \mbox{ fresh}}{\B, \futs xy, \Left, \labels{x}{A \IMP B} \SEQ \Right, \labels{y}{A \IMP B}}{
				\vlin{\rn{trans}}{}{\B, \futs xy, \futs yz, \Left, \labels{x}{A \IMP B}, \labels{z}{A} \SEQ \Right, \labels{z}{B}}{
					\vliin{\llabrn\IMP}{}{\B, \futs xy, \futs yz, \futs xz, \Left, \labels{x}{A \IMP B}, \labels{z}{A} \SEQ \Right, \labels{z}{B}}{
						\vlin{\rn{refl}}{}{\B, \futs xy, \futs yz, \futs xz, \Left, \labels{x}{A \IMP B}, \labels{z}{A} \SEQ \Right, \labels{z}{B}, \labels{z}{A}}{
							\vlin{\labrn{id_g}}{}{\B, \futs xy, \futs yz, \futs xz, \futs zz, \Left, \labels{x}{A \IMP B}, \labels{z}{A} \SEQ \Right, \labels{z}{B}, \labels{z}{A}}{
								\vlhy{}
							}
						}
					}{
					\vlin{\rn{refl}}{}{\B, \futs xy, \futs yz, \futs xz, \Left, \labels{z}{B}, \labels{z}{A} \SEQ \Right, \labels{z}{B}}{
						\vlin{\labrn{id_g}}{}{\B, \futs xy, \futs yz, \futs xz, \futs zz, \Left, \labels{z}{B}, \labels{z}{A} \SEQ \Right, \labels{z}{B}}{
							\vlhy{}
						}
					}
				}
			}
		}
	}
	\end{smallequation*}

	\item $\fm{\BOX A}$
%	\begin{smallequation*}
%		\vlinf{\labrn{id_g}}{}{\B, x \le y; \Left, \labels{x}{\BOX A} \SEQ \Right, \labels{y}{\BOX A}}{}
%		\reducesto
%	\end{smallequation*}
	\begin{smallequation*}
		\vlderivation{
			\vlin{\rlabrn\BOX}{\lb z, \lb w \mbox{ fresh}}{\B, \futs xy, \Left, \labels{x}{\BOX A} \SEQ \Right, \labels{y}{\BOX A}}{
				\vlin{\rn{trans}}{}{\B, \futs xy, \futs yz, \accs zw, \Left, \labels{x}{\BOX A} \SEQ \Right, \labels{w}{A}}{
					\vlin{\llabrn\BOX}{}{\B, \futs xy, \futs yz, \futs xz, \accs zw, \Left, \labels{x}{\BOX A} \SEQ \Right, \labels{w}{A}}{
						\vlin{\rn{refl}}{}{\B, \futs xy, \futs yz, \futs xz, \accs zw; \Left, \labels{z}{\BOX A}, \labels{w}{A} \SEQ \Right, \labels{w}{A}}{
							\vlin{\labrn{id_g}}{}{\B, \futs xy, \futs yz, \futs xz, \accs zw, \futs ww, \Left, \labels{z}{\BOX A}, \labels{w}{A} \SEQ \Right, \labels{w}{A}}{
								\vlhy{}
							}
						}
					}
				}
			}
		}
	\end{smallequation*}
	
	\item $\fm{\DIA A}$
%	\begin{smallequation*}
%		\vlinf{\labrn{id_g}}{}{\B, x \le y; \Left, \labels{x}{\DIA A} \SEQ \Right, \labels{y}{\DIA A}}{}
%		\reducesto
%	\end{smallequation*}
	\begin{smallequation*}
		\vlderivation{
			\vlin{\llabrn\DIA}{}{\B, \futs xy; \Left, \labels{x}{\DIA A} \SEQ \Right, \labels{y}{\DIA A}}{
				\vlin{\color{red}\rn{F_2}}{}{\B, \futs xy, \accs xz, \Left, \labels{z}{A} \SEQ \Right, \labels{y}{\DIA A}}{
					\vlin{\rlabrn\DIA}{}{\B, \futs xy, \accs xz, \futs zu, \accs yu, \Left, \labels{z}{A} \SEQ \Right, \labels{y}{\DIA A}}{
						\vlin{\labrn{id_g}}{}{\B, \futs xy, \accs xz, \futs zu, \accs yu, \Left, \labels{z}{A} \SEQ \Right, \labels{y}{\DIA A}, \labels{u}{A}}{
							\vlhy{}
						}
					}
				}
			}
		}
	\end{smallequation*}
        \qedhere
	\end{itemize}
\end{proof}

In the following sections, we will show that the system $\labIKp$ is sound and complete. For the completeness proof we proceed via cut elimination. The cut rule has the following shape:
\begin{equation}
  \label{eq:cut}
  \vliiinf{\labrn{cut}}{}{
    \lseq\B\Left\Right}{
    \lseq\B\Left{\Right,\labels{z}{C}}}{}{
    \lseq\B{\Left,\labels{z}{C}}{\Right}}
\end{equation}
Then we can summarize soundness and completeness, and cut
admissibil\texttt{}ity of $\labIKp$ in the following Theorem:

\begin{theorem}\label{thm:cutfree-compl}
	%	Let $\CC$ be a set of geometric frame properties as in~\eqref{eq:cla-geometric} and $\labbrn{\CC}$ be the corresponding set of rules following schema~\eqref{eq:modal-grs}.
	%
	For any formula $A$, the following are equivalent.
	%
	\begin{enumerate}
		\item\label{i} $A$ is a theorem of $\IK$. 
		%
		\item\label{ii} $A$ is provable in $\labIKp +\labrn{cut}$.
		\item\label{iii} $A$ is provable in $\labIKp$.
		%
		\item\label{iv} $A$ is valid in every birelational frame. %satisfying the properties in $\CC$.
	\end{enumerate}
\end{theorem}

The proof of this theorem is the topic of the following sections. The
equivalence of \ref{i} and \ref{iv} has already been stated in
Theorem~\ref{thm:plotkin}~\cite{fischer-servi:84, plotkin:stirling:86}. The implication
\ref{i}$\implies$\ref{ii} is shown in Section~\ref{sec:completeness},
the implication~\ref{ii}$\implies$\ref{iii} is shown in
Section~\ref{sec:cut-elim}, and finally, the implication \ref{iii}$\implies$\ref{iv} is shown in Section~\ref{sec:soundness}.

Once we have shown cut elimination (the implcation \ref{ii}$\implies$\ref{iii} of Theorem~\ref{thm:cutfree-compl}), the proof of Proposition~\ref{prop:mon-adm} becomes trivial.

\begin{proof}[Proof of Proposition~\ref{prop:mon-adm}]
  The rule $\llabrn{mon}$ can be derived using the general identity and cut:
  \begin{smallequation*}
    \vlderivation{
      \vliiin{\labrn{cut}}{}{\lseq{\B, \futs{x}{x'}}{\Left, \labels{x}{A}}\Right}{
        \vlin{\labrn{id_g}}{}{\lseq{\B, \futs{x}{x'}}{\Left, \labels{x}{A}}{\Right, \labels{x'}{A}}}{
          \vlhy{}}}{
        \vlhy{\quad}}{
        \vlhy{\lseq{\B, \futs{x}{x'}}{\Left, \labels{x}{A}, \labels{x'}{A}}\Right}}
    }
  \end{smallequation*}
  And both these rules are admissible by Proposition~\ref{prop:id} and
  Theorem~\ref{thm:cutfree-compl}. The case for $\rlabrn{mon}$ is
  similar.
\end{proof}
