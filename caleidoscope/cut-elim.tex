\section{Cut Admissibility}\label{sec:cut-elim}
In this section we are going to prove the admissibility of cut for $\labIKp$.

\begin{theorem}
	\label{thm:cut-adm}
	The rule $\labrn{cut}$ is admissible for $\labIKp$.
\end{theorem}

This theorem directly entails the implication \ref{ii}$\implies$\ref{iii} of Theorem~\ref{thm:cutfree-compl}. But before we can prove it, we need a series of auxiliary  lemmas.

The \emph{height} of a derivation $\DD$, denoted by $\height\DD$, is the height of D when seen as a tree, i.e., the length of
the longest path in the tree from its root to one of its leaves.

We say that a rule is \emph{height-preserving admissible} if for every derivation $\DD$ of its premise there is a derivation $\DD'$ of its conclusion such that $\height{\DD'}\le\height\DD$. A rule is \emph{height-preserving invertible} if for every
derivation of the conclusion of the rule there are derivations for each of its premisses with at most the same height.

The following two lemmas are standard.

%\begin{lemma}
%	\label{lem:weak-adm}
%	The weakening rule
%	~$\vlinf{\rn{weak}}{}{\lseq{\B,\B'}{\Left,\Left'}{\Right,\Right'}}{\lseq{\B}{\Left}{\Right}}$~
%	is height-preserving admissible for $\labIKp$.
%\end{lemma}


	\begin{lemma}
		\label{lem:weak-adm}
		If there exists a proof  $\vlderivation {\vlpd{\D}{}{\B, \Left \Rightarrow \Right}}$ then there exists a proof $\vlderivation {\vlpd {\Dw}{}{\B, \accs xy, \futs uv, \Left, \labels{z}{A }\SEQ \Right, \labels{w}{B}}}$
		
	\end{lemma}
	
	\vspace{3mm}
	
	\begin{proof}
		By induction on the height of $\D$.
		
		For a proof of height 1:
		
		if $\D$ = $\vlderivation {\vlin{\rn{id}}{}{\B, \futs xy, \Left, \labels{x}{a} \SEQ \Right, \labels{y}{a}}{\vlhy {}}}$ then we take $\Dw$ to be $\vlderivation {\vlin{\rn{id}}{}{\B, \futs xy, \accs xy, \futs uv, \Left, \labels{x}{a}, \labels{z}{A} \SEQ \Right, \labels{w}{B}, \labels{y}{a}}{\vlhy {}}}$.
		\vspace{2mm}
		
		For a proof $\D$ of height greater than 1:
		
		\begin{center}
			
			$\vlderivation{\vlin {$r$}{}{\B, \Left \Rightarrow \Right}{\vlpd {\Done}{}{\B', \Left' \Rightarrow \Right'}}}$
			
		\end{center}
		
		Then by induction hypothesis there exists a proof 
		
		\begin{center}
			
			$\vlderivation {\vlpd {\Dwone}{}{\B', \accs xy, \futs uv, \Left', \labels{z}{A} \SEQ \Right', \labels{w}{B}}}$
			
		\end{center}
		
		Therefore, we have the proof 
		
		\begin{center}
			
			$\Dw = \vlderivation {\vlin{}{}{\B, \accs xy, \futs uv, \Left, \labels{z}{A} \SEQ \Right, \labels{w}{B}}{\vlpd {\Dwone}{}{\B', \accs xy, \futs uv, \Left', \labels{z}{A} \SEQ \Right', \labels{w}{B}}}}$
			
		\end{center}
		
	\end{proof}


\begin{lemma}
	\label{lem:inv}
	The rules ..... are height-preserving invertible in
	$\labIKp$. Furthermore, the rule $\llabrn{\IMP}$ is
	height-preserving invertible for the right premise.
\end{lemma}

\begin{proof}
	\todo{}
\end{proof}

The next lemma looks like a special case of
Proposition~\ref{prop:mon-adm}, but it is not. First, we need to
preserve the height, and second, we cannot prove it using cut as we
need it to eliminate the cut.

\begin{lemma}
	The atomic version of $\llabrn{mon}$
	$$
	\vlinf{\labrn{mon_a}}{}{\lseq{\B, \futs{x}{x'}}{\Left, \labels{x}{a}}\Right}{
		\lseq{\B, \futs{x}{x'}}{\Left, \labels{x}{a}, \labels{x'}{a}}\Right}
	$$
	is height-preserving admissible for $\labIKp$.
\end{lemma}

\begin{proof}
	\begin{smallequation*}
		\vlderivation{
			\vlin{\labrn{mon_a}}{}{\B, \futs{x}{x'}, \futs{x'}{x''}, \Left, \labels{x}{a} \SEQ \Right, \labels{x''}{a}}{
				\vlin{\labrn{id}}{}{\B, \futs{x}{x'}, \futs{x'}{x''}, \Left, \labels{x}{a}, \labels{x'}{a} \SEQ \Right, \labels{x''}{a}}{
					\vlhy{}
				}
			}
		}
		\reducesto
		\vlderivation{
			\vlin{\color{red}\labrn{trans}}{}{\B, \futs{x}{x'}, \futs{x'}{x''}, \Left, \labels{x}{a} \SEQ \Right, \labels{x''}{a}}{\vlin{\labrn{id}}{}{\B, \futs{x}{x'}, \futs{x'}{x''}, \futs{x}{x''}, \Left, \labels{x}{a} \SEQ \Right, \labels{x''}{a}}{
					\vlhy{}
				}}
			}
		\end{smallequation*}

	  	\begin{smallequation*}
	  		\vlderivation{
	  			\vlin{\labrn{mon_a}}{}{\B, \futs{x}{x'}, \Left, \labels{x}{a} \SEQ \Right, \labels{x}{a}}{
	  				\vlin{\labrn{id}}{}{\B, \futs{x}{x'}, \labels{x}{a}, \labels{x'}{a} \SEQ \Right, \labels{x}{a}}{
	  					\vlhy{}
	  				}
	  			}
	  		}
	  		\reducesto
	  		\vlderivation{
	  		\vlin{\color{red}\labrn{refl}}{}{\B, \futs{x}{x'}, \Left, \labels{x}{a} \SEQ \Right, \labels{x}{a}}{\vlin{\labrn{id}}{}{\B, \futs{x}{x'}, \futs{x}{x}, \labels{x}{a} \SEQ \Right, \labels{x}{a}}{
	  				\vlhy{}
	  			}}}
	  		\qedhere
	  	\end{smallequation*}
			  	
	  	\begin{smallequation*}
	  		\vlderivation{
	  			\vlin{\labrn{mon_a}}{}{\B, \futs{x}{x'}, \Left, \labels{x}{a} \SEQ \Right, \labels{x'}{a}}{
	  				\vlin{\labrn{id}}{}{\B, \futs{x}{x'}, \labels{x}{a}, \labels{x'}{a} \SEQ \Right, \labels{x'}{a}}{
	  					\vlhy{}
	  				}
	  			}
	  		}
	  		\reducesto
	  		\vlinf{\labrn{id}}{}{\B, \futs{x}{x'}, \Left, \labels{x}{a} \SEQ \Right, \labels{x'}{a}}{}
	  		\qedhere
	  	\end{smallequation*}
	  	
	\end{proof}
	
	The next lemma shows that all the rules in our system are invertible, as already mentioned in the introduction. 
	
	\begin{lemma}\label{lem:inv}\hbox{}\quad\todo{change this $\bullet$ notation!!}
		\begin{enumerate}
			%\item
			%If there exists a proof $\vlderivation{\vlhtr{\DD}{\B, \Left, \labels{x}{A \AND B} \SEQ \Right}}$ then there exists a proof $\vlderivation{\vlhtr{\DD^{\invr{\llabrn\AND}}}{\B, \Left, \labels{x}{A}, \labels{x}{B} \SEQ \Right}}$ of the same (or smaller) height, for any label $x$ that does not appear in $\DD$.
			
			\item
			If there exists a proof 
			$\vlderivation{\vlhtr{\DD}{\B, \Left, \labels{x}{A \IMP B} \SEQ \Right}}$ 
			then there exists a proof 
			$\vlderivation{\vlhtr{\DD^{\invr{\llabrn\IMP}}}{\B, \Left, \labels{y}{B} \SEQ \Right}}$
			of the same (or smaller) height, for any label $y$ that does not appear in $\DD$.
			\sonia{this is wrong! $x\le y$ should appear in the conclusion of $\DD$. }
			
			\item
			If there exists a proof 
			$\vlderivation{\vlhtr{\DD}{\B, \Left \SEQ \Right, \labels{x}{A \IMP B}}}$ 
			then there exists a proof 
			$\vlderivation{\vlhtr{\DD^{\invr{\rlabrn\IMP}}}{\B, x \le y, \Left, \labels{y}{A} \SEQ \Right, \labels{y}{B}}}$
			of the same (or smaller) height, for any label $y$ that does not appear in $\DD$.
			
			\item 
			If there exists a proof 
			$\vlderivation{\vlhtr{\DD}{\B, \Left \SEQ \Right, \labels{x}{\BOX A}}}$ 
			then there exists a proof 
			$\vlderivation{\vlhtr{\DD^{\invr{\rlabrn\BOX}}}{\B, x \le u, u \rel v, \Left \SEQ \Right, \labels{v}{A}}}$
			of the same (or smaller) height, for any label $u$ and $v$ that do not appear in $\DD$.
			
			\item 
			If there exists a proof 
			$\vlderivation{\vlhtr{\DD}{\B, \Left, \labels{x}{\DIA A} \SEQ \Right}}$ 
			then there exists a proof 
			$\vlderivation{\vlhtr{\DD^{\invr{\llabrn\DIA}}}{\B, x \rel y, \Left, \labels{y}{A} \SEQ \Right}}$
			of the same (or smaller) height, for any label $y$ that does not appear in $\DD$.
			
		\end{enumerate}
	\end{lemma}
	
	\todo{add missing cases}\sonia{they are just weakenings.}
	
	\begin{proof}
		In each case, we reason by induction on the height of $\DD$.
		
		\begin{itemize}
			\item $\llabrn\IMP$:
			For a proof of height $1$, it is straightforward.
			%
			For example, if $\DD = \vlinf{\labrn{id}}{}{\B, u \le v, \Left, \labels{u}{a}, \labels{x}{A \IMP B} \SEQ \Right, \labels{v}{a}}{}$, 
			%
			then we take $\DD^{\invr{\llabrn\IMP}}$ to be $\vlinf{\labrn{id}}{}{\B, u \le v, \Left, \labels{u}{a},\labels{y}{B} \SEQ \Right, \labels{v}{a}}{}$.
			
			For a proof $\DD$ of height greater than $1$ we have two cases, depending on whether the last rule of $\DD$ acts on $\labels{x}{A \IMP B}$ or only on some part of the context.
			
			First let us fix a given index $y$ that does not appear in $\DD$.
			
			If we start with a proof
			\begin{smallequation*}
				\DD = 
				\vlderivation{
					\vlin{\rn{r}}{}{\B, \Left, \labels{x}{A \IMP B} \SEQ \Right}{
						\vlhtr{\DD'}{\B', \Left', \labels{x}{A \IMP B} \SEQ \Right'}
					}	
				}
			\end{smallequation*}
			
			Then by induction hypothesis there exists a proof 
			\begin{smallequation*}
				\vlderivation{\vlhtr{{\DD'}^{\invr{\llabrn\IMP}}}{\B', \Left', \labels{y}{B} \SEQ \Right'}}
			\end{smallequation*}
			of the same (or smaller) height as $\DD'$ (as $y$ also does not appear in $\DD'$).
			
			Therefore, we have the proof
			\begin{smallequation*}
				\DD^{\invr{\llabrn\IMP}} = 
				\vlderivation{
					\vlin{\rn{r}}{}{\B, \Left, \labels{y}{B} \SEQ \Right}{
						\vlhtr{{\DD'}^{\invr{\llabrn\IMP}}}{\B', \Left', \labels{y}{B} \SEQ \Right'}
					}
				}
			\end{smallequation*}
			of the same (or smaller) height as $\DD$.
			
			If we start with a proof 
			\begin{smallequation*}
				\vlderivation{
					\vliin{\llabrn\IMP}{\text{\footnotesize $x \le z$ appears in $\B$}}{\B, \Left, \labels{x}{A \IMP B} \SEQ \Right}{
						\vlhtr{\DD_1}{\B, \Left, \labels{x}{A \IMP B} \SEQ \Right, \labels{z}{A}}
					}{
					\vlhtr{\DD_2}{\B, \Left, \labels{z}{B} \SEQ \Right}
				}
			}
		\end{smallequation*}
		then we take $\DD^{\invr{\llabrn\IMP}}$ to be $\vlderivation{\vlhtr{\DD_2[y/z]}{\B, \Left, \labels{y}{B} \SEQ \Right}}$ (as $y$ also does not appear in $\DD_2$) and its height is smaller than the one of $\DD$.
		
		%
		\item $\llabrn\DIA$:
		
		If we start with a proof
		\begin{smallequation*}
			\DD = 
			\vlderivation{
				\vlin{\rn{r}}{}{\B, \Left, \labels{x}{\DIA A} \SEQ \Right}{
					\vlhtr{\DD'}{\B', \Left', \labels{x}{\DIA A} \SEQ \Right'}
				}	
			}
		\end{smallequation*}
		
		Then by induction hypothesis there exists a proof 
		\begin{smallequation*}
			\vlderivation{\vlhtr{{\DD'}^{\invr{\llabrn\DIA}}}{\B', x \rel y, \Left', \labels{y}{A} \SEQ \Right'}}
		\end{smallequation*}
		of the same (or smaller) height as $\DD'$.
		
		Therefore, we have the proof
		\begin{smallequation*}
			\DD^{\invr{\llabrn\DIA}} = 
			\vlderivation{
				\vlin{\rn{r}}{}{\B, x \rel y, \Left, \labels{y}{A} \SEQ \Right}{
					\vlhtr{{\DD'}^{\invr{\llabrn\DIA}}}{\B', x \rel y, \Left', \labels{y}{A} \SEQ \Right'}
				}
			}
		\end{smallequation*}
		of the same (or smaller) height as $\DD$, for any label $y$ that does not appear in $\DD$.
		
		If we start with a proof 
		\begin{smallequation*}
			\vlderivation{
				\vlin{\llabrn\DIA}{\text{\footnotesize $y'$ fresh}}{\B, \Left, \labels{x}{\DIA A} \SEQ \Right}{
					\vlhtr{\DD'}{\B, x \rel y', \Left, \labels{y'}{A} \SEQ \Right}
				}	
			}
		\end{smallequation*}
		then we take $\DD^{\invr{\llabrn\DIA}}$ to be $\vlderivation{\vlhtr{\DD'[y/y']}{\B, x \rel y, \Left, \labels{y}{A} \SEQ \Right}}$ for any label $y$ that does not appear in $\DD$.
		
		\item $\rlabrn\IMP$ and $\rlabrn\BOX$ : Similar.
	\end{itemize}
\end{proof}


	
	The next lemma is the core of our cut elimination proof.
	
	\begin{lemma}
		\label{lem:reduction}
		Given a derivation of shape
		$$
		\vlderivation{
			\vliiin{\labrn{cut}}{}{\lseq\B\Left\Right}{
				\vlhtr{\DD_1}{\lseq\B\Left{\Right,\labels{z}{C}}}}{
				\vlhy{}}{
				\vlhtr{\DD_2}{\lseq\B{\Left,\labels{z}{C}}{\Right}}}}
		$$
		where $\DD_1$ and $\DD_2$ are both cut-free, there is a cut-free
		derivation of ${\lseq\B\Left\Right}$
	\end{lemma}
	
	%$\vlderivation{\vlhtr{\DD}{\lseq\B\Left\Right}}$
	
	
	Following the proof of the Theorem \ref{thm:cutfree-compl}, in this section, we prove the implication \ref{ii}$\implies$\ref{iii}, i.e, we want to prove that our system $\labIKp$ is complete without the \rn{cut} rule.
	
	For proving this theorem, we need a series of auxiliary lemmas (see Lemma \ref{lem:inv}).
	
	\begin{proof}
		
		By induction on number of cuts + (rank,height) of the leftmost topmost cut.
		
		\textbf{Commutative cases:}\label{commutative-cases}
		\todo{we probably don't need all the cases here?}
		
		\begin{smallequation*}
			\vlderiibase{\labrn{cut}}{}{\B, x \le y, \Left, \labels{x}{A \IMP B} \SEQ \Right}{
				\vliin{\llabrn\IMP}{}{\B, x \le y, \Left, \labels{x}{A \IMP B} \SEQ \Right, \labels{z}{C}}{
					\vlhtr{\DD_1}{\B, x \le y, \Left, \labels{x}{A \IMP B} \SEQ \Right, \labels{z}{C}, \labels{y}{A}}
				}{
				\vlhtr{\DD_2}{\B, x \le y, \Left, \labels{y}{B} \SEQ \Right, \labels{z}{C}}
			}
		}{
		\vlhtr{\DD_3}{\B, x \le y, \Left, \labels{x}{A \IMP B}, \labels{z}{C} \SEQ \Right}
	}
	\reducesto
\end{smallequation*}

\begin{smallequation*}\hspace*{-3.5em}
	\vlderiibase{\llabrn\IMP}{}{\B, x \le y, \Left, \labels{x}{A \IMP B} \SEQ \Right}{
		\vliin{\labrn{cut}}{}{\B, x \le y, \Left, \labels{x}{A \IMP B} \SEQ \Right, \labels{y}{A}}{
			\vlhtr{\DD_1}{\B, x \le y, \Left, \labels{x}{A \IMP B} \SEQ \Right, \labels{z}{C}, \labels{y}{A}}
		}{
		\vlhtr{\DD_3^{\rn w}}{\B, x \le y, \Left, \labels{x}{A \IMP B}, \labels{z}{C} \SEQ \Right, \labels{y}{A}}
	}
}{
\vliin{\labrn{cut}}{}{\B, x \le y, \Left, \labels{y}{B} \SEQ \Right}{
	\vlhtr{\DD_2}{\B, x \le y, \Left, \labels{y}{B} \SEQ \Right, \labels{z}{C}}
}{
\vlhtr{\DD_3[v/y]^{\invr{\llabrn\IMP}}}{\B, x \le y, \Left, \labels{y}{B}, \labels{z}{C} \SEQ \Right}
}
}
\end{smallequation*}

We need to make sure that $y$ does not appear in $\DD_3$, before applying Lemma~\ref{lem:inv}.
%
If it does we rewrite it with a fresh variable $v$ first.
\sonia{check what is actually required since that Lemma is wrong!}

%\begin{smallequation*}
%	\reducesto
%	\vlderibase{\llabrn{cont}}{}{\B_1, \B_2, x \le y, \Left, \labels{x}{A \IMP B} \SEQ \Right}{
%		\vliin{\llabrn\IMP}{}{\B_1, \B_2, x \le y, \Left, \labels{x}{A \IMP B}, \labels{x}{A \IMP B} \SEQ \Right}{
%%			\vliin{\labrn{cut}}{}{\B_1, \B_2, x \le y, \Left, \labels{x}{A \IMP B}, \labels{x}{A \IMP B} \SEQ \Right, \labels{y}{A}}{
%%				\vlhtr{\DD_1^{\rn w}}{\B_1, x \le y, \Left, \labels{x}{A \IMP B}, \labels{x}{A \IMP B} \SEQ \Right, \labels{z}{C}, \labels{y}{A}}
%%				}{
%%				\vlhtr{\DD_3^{\rn w}}{\B_2, \Left, \labels{x}{A \IMP B}, \labels{x}{A \IMP B}, \labels{z}{C} \SEQ \Right}
%%				}
%				\vlhtr{\DD_4}{\B_1, \B_2, x \le y, \Left, \labels{x}{A \IMP B}, \labels{x}{A \IMP B} \SEQ \Right, \labels{y}{A}}
%			}{
%%			\vliin{\labrn{cut}}{}{\B_1, \B_2, x \le y, \Left, \labels{x}{A \IMP B}, \labels{y}{B} \SEQ \Right}{
%%				\vlhtr{\DD_2^{\rn w}}{\B_1, x \le y, \Left, \labels{x}{A \IMP B}, \labels{y}{B} \SEQ \Right}
%%				}{
%%				\vlhtr{\DD_3^{\rn w}}{\B_2, \Left, \labels{x}{A \IMP B}, \labels{y}{B} \SEQ \Right}
%%				}
%				\vlhtr{\DD_5}{\B_1, \B_2, x \le y, \Left, \labels{x}{A \IMP B}, \labels{y}{B} \SEQ \Right}
%			}
%		}
%\end{smallequation*}
%
%\begin{smallalign*}&
%	\text{with } \DD_4 = \quad
%	\vlderivation{
%		\vliin{\labrn{cut}}{}{\B_1, \B_2, x \le y, \Left, \labels{x}{A \IMP B}, \labels{x}{A \IMP B} \SEQ \Right, \labels{y}{A}}{
%			\vlhtr{\DD_1^{\rn w}}{\B_1, x \le y, \Left, \labels{x}{A \IMP B}, \labels{x}{A \IMP B} \SEQ \Right, \labels{z}{C}, \labels{y}{A}}
%		}{
%			\vlhtr{\DD_3^{\rn w}}{\B_2, \Left, \labels{x}{A \IMP B}, \labels{x}{A \IMP B}, \labels{z}{C} \SEQ \Right, \labels{y}{A}}
%		}
%	}
%	\\&
%	\text{and } \DD_5 = \quad
%	\vlderivation{
%		\vliin{\labrn{cut}}{}{\B_1, \B_2, x \le y, \Left, \labels{x}{A \IMP B}, \labels{y}{B} \SEQ \Right}{
%			\vlhtr{\DD_2^{\rn w}}{\B_1, x \le y, \Left, \labels{x}{A \IMP B}, \labels{y}{B} \SEQ \Right}
%		}{
%			\vlhtr{\DD_3^{\rn w}}{\B_2, \Left, \labels{x}{A \IMP B}, \labels{y}{B} \SEQ \Right}
%		}
%	}
%\end{smallalign*}

\begin{smallequation*}
	\vlderiibase{\labrn{cut}}{}{\B, \Left \SEQ \Right, \labels{x}{A \IMP B}}{
		\vlin{\rlabrn\IMP}{\text{\footnotesize $x'$ fresh}}{\B, \Left \SEQ \Right, \labels{x}{A \IMP B}, \labels{z}{C}}{
			\vlhtr{\DD_1}{\B, x \le x', \Left, \labels{x'}{A} \SEQ \Right, \labels{x'}{B}, \labels{z}{C}}
		}
	}{
	\vlhtr{\DD_2}{\B, \Left, \labels{z}{C} \SEQ \Right, \labels{x}{A \IMP B}}
}
\end{smallequation*}

\begin{smallequation*}
	\reducesto
	\vlderibase{\rlabrn\IMP}{\text{\footnotesize $x''$ fresh (also in $\DD_2$)}}{\B, \Left \SEQ \Right, \labels{x}{A \IMP B}}{
		\vliin{\labrn{cut}}{}{\B, x \le x'' \Left, \labels{x''}{A} \SEQ \Right, \labels{x''}{B}}{
			\vlhtr{\DD_1[x''/x']}{\B, x \le x'', \Left, \labels{x''}{A} \SEQ \Right, \labels{x''}{B}, \labels{z}{C}}
		}{
		\vlhtr{\DD_2^{\invr{\rlabrn\IMP}}}{\B, x \le x'', \Left, \labels{z}{C}, \labels{x''}{A} \SEQ \Right, \labels{x''}{B}}
	}
}
\end{smallequation*}

\begin{smallequation*}
	\vlderiibase{\labrn{cut}}{}{\B, x \le u, u \rel v, \Left, \labels{x}{\BOX A} \SEQ \Right}{
		\vlin{\llabrn\BOX}{}{\B, x \le u, u \rel v, \Left, \labels{x}{\BOX} \SEQ \Right, \labels{z}{C}}{
			\vlhtr{\DD_1}{\B, x \le u, u \rel v, \Left, \labels{x}{\BOX A}, \labels{v}{A} \SEQ \Right, \labels{z}{C}}
		}
	}{
	\vlhtr{\DD_2}{\B, x \le u, u \rel v, \Left, \labels{x}{\BOX A}, \labels{z}{C} \SEQ \Right}
}
\end{smallequation*}

\begin{smallequation*}
	\reducesto
	\vlderibase{\llabrn\BOX}{}{\B, x \le u, u \rel v, \Left, \labels{x}{\BOX A} \SEQ \Right}{
		\vliin{\labrn{cut}}{}{\B, x \le u, u \rel v, \Left, \labels{x}{\BOX A}, \labels{v}{A} \SEQ \Right}{
			\vlhtr{\DD_1}{\B, x \le u, u \rel v, \Left, \labels{x}{\BOX A}, \labels{v}{A} \SEQ \Right, \labels{z}{C}}
		}{
		\vlhtr{\DD_2^{\rn w}}{\B, x \le u, u \rel v, \Left, \labels{x}{\BOX A}, \labels{v}{A}, \labels{z}{C} \SEQ \Right}
	}
}
\end{smallequation*}

\begin{smallequation*}
	\vlderiibase{\labrn{cut}}{}{\B, \Left \SEQ \Right, \labels{x}{\BOX A}}{
		\vlin{\rlabrn\BOX}{\text{\footnotesize $x',y'$ fresh}}{\B, \Left \SEQ \Right, \labels{x}{\BOX A}, \labels{z}{C}}{
			\vlhtr{\DD_1}{\B, x \le x', x' \rel y' \Left \SEQ \Right, \labels{y'}{A}, \labels{z}{C}}
		}
	}{
	\vlhtr{\DD_2}{\B, \Left, \labels{z}{C} \SEQ \Right, \labels{x}{\BOX A}}
}
\end{smallequation*}

\begin{smallequation*}
	\reducesto
	\vlderibase{\rlabrn\BOX}{\text{\footnotesize $u,v$ fresh (also in $\DD_2$)}}{\B, \Left \SEQ \Right, \labels{x}{\BOX A}}{
		\vliin{\labrn{cut}}{}{\B, x \le u, u \rel v, \Left \SEQ \Right, \labels{v}{A}}{
			\vlhtr{\DD_1}{\B, x \le u, u \rel v, \Left \SEQ \Right, \labels{v}{A}, \labels{z}{C}}
		}{
		\vlhtr{\DD_2^{\invr{\rlabrn\BOX}}}{\B, x \le u, u \rel v, \Left, \labels{z}{C} \SEQ \Right, \labels{v}{A}}
	}
}
\end{smallequation*}

\begin{smallequation*}\hspace*{-2em}
	\vlderiibase{\labrn{cut}}{}{\B, \Left, \labels{x}{\DIA A} \SEQ \Right}{
		\vlin{\llabrn\DIA}{\text{\footnotesize $y'$ fresh}}{\B, \Left, \labels{x}{\DIA A} \SEQ \Right, \labels{z}{C}}{
			\vlhtr{\DD_1}{\B, x \rel y', \Left, \labels{y'}{A} \SEQ \Right, \labels{z}{C}}
		}
	}{
	\vlhtr{\DD_2}{\B, \Left, \labels{x}{\DIA A}, \labels{z}{C} \SEQ \Right}
}
\reducesto
\vlderibase{\llabrn\DIA}{\text{\footnotesize $y''$ fresh (also in $\DD_2$)}}{\B, \Left, \labels{x}{\DIA A} \SEQ \Right}{
	\vliin{\labrn{cut}}{}{\B, x \rel y'', \Left, \labels{y''}{A} \SEQ \Right}{
		\vlhtr{\DD_1[y''/y']}{\B, x \rel y'', \Left, \labels{y''}{A} \SEQ \Right, \labels{z}{C}}
	}{
	\vlhtr{\DD_2^{\invr{\llabrn\DIA}}}{\B, x \rel y'', \Left, \labels{y''}{A}, \labels{z}{C} \SEQ \Right}
}
}
\end{smallequation*}
\todo{is this correct wrt to invertibility Lemma~\ref{lem:inv}??}

\begin{smallequation*}\hspace*{-2em}
	\vlderiibase{\labrn{cut}}{}{\B, x \rel y, \Left \SEQ \Right, \labels{x}{\DIA A}}{
		\vlin{\rlabrn\DIA}{}{\B, x \rel y, \Left \SEQ \Right, \labels{x}{\DIA A}, \labels{z}{C}}{
			\vlhtr{\DD_1}{\B, x \rel y, \Left \SEQ \Right, \labels{x}{\DIA A}, \labels{y}{A}, \labels{z}{C}}
		}
	}{
	\vlhtr{\DD_2}{\B, \Left, \labels{z}{C} \SEQ \Right, \labels{x}{\DIA A}}
}
\reducesto
%	\end{smallequation*}
%	
%	\begin{smallequation*}
\vlderibase{\rlabrn\DIA}{}{\B, x \rel y, \Left \SEQ \Right, \labels{x}{\DIA A}}{
	\vliin{\labrn{cut}}{}{\B, x \rel y, \Left \SEQ \Right, \labels{x}{\DIA A}, \labels{y}{A}}{
		\vlhtr{\DD_1}{\B, x \rel y, \Left \SEQ \Right, \labels{x}{\DIA A}, \labels{y}{A}, \labels{z}{C}}
	}{
	\vlhtr{\DD_2^{\rn w}}{\B, \Left, \labels{z}{C} \SEQ \Right, \labels{x}{\DIA A}, \labels{y}{A}}
}
}
\end{smallequation*}

\textbf{Key cases:}\label{key-cases}

\begin{smallequation*}
	\vlderiibase{\labrn{cut}}{}{\B, x \le y, \Left, \labels{x}{a} \SEQ \Right}{
		\vlin{\labrn{id}}{}{\B, x \le y, \Left, \labels{x}{a} \SEQ \Right, \labels{y}{a}}{
			\vlhy{}	
		}
	}{
	\vlhtr{\DD_2}{\B, \Left, \labels{x}{a}, \labels{y}{a} \SEQ \Right}		
}
\reducesto
\vlderibase{\llabrn{mon}}{}{\B, x \le y, \Left, \labels{x}{a} \SEQ \Right}{
	\vlhtr{\DD_2^{\rn w}}{\B, x \le y, \Left, \labels{x}{a}, \labels{y}{a} \SEQ \Right}
}
\end{smallequation*}

\begin{smallequation*}
	\vlderiibase{\labrn{cut}}{}{\B, x \le y, \Left \SEQ \Right, \labels{y}{a}}{
		\vlhtr{\DD_1}{\B, \Left \SEQ \Right, \labels{x}{a}, \labels{y}{a}}		
	}{
	\vlin{\labrn{id}}{}{\B, x \le y, \Left, \labels{x}{a} \SEQ \Right, \labels{y}{a}}{
		\vlhy{}	
	}
}
\reducesto
\vlderibase{\rlabrn{mon}}{}{\B, x \le y, \Left \SEQ \Right, \labels{y}{a}}{
	\vlhtr{\DD_1^{\rn w}}{\B, x \le y, \Left \SEQ \Right, \labels{x}{a}, \labels{y}{a}}
}
\end{smallequation*}

\begin{smallequation*}
	\vlderiibase{\labrn{cut}}{}{\B, x \le y, \Left \SEQ \Right}{
		\vlin{\rlabrn\IMP}{}{\B, x \le y, \Left \SEQ \Right, \labels{x}{A \IMP B}}{
			\vlhtr{\DD_1}{\B, x \le y, x \le x', \Left, \labels{x'}{A} \SEQ \Right, \labels{x'}{B}}
		}
	}{
	\vliin{\llabrn\IMP}{}{\B, x \le y, \Left, \labels{x}{A \IMP B} \SEQ \Right}{
		\vlhtr{\DD_2}{\B, x \le y, \Left, \labels{x}{A \IMP B} \SEQ \Right, \labels{y}{A}}
	}{
	\vlhtr{\DD_3}{\B, x \le y, \Left, \labels{y}{B} \SEQ \Right}
}
}
\reducesto
\end{smallequation*}

\begin{smallequation*}
	\hspace*{-2em}
	\vlderiibase{\labrn{cut}}{}{\B, x \le y, \Left \SEQ \Right}{
		\vliin{\labrn{cut}}{}{\B, x \le y, \Left \SEQ \Right, \labels{y}{A}}{
			\vlin{\rlabrn\IMP}{}{\B, x \le y, \Left \SEQ \Right, \labels{x}{A \IMP B}, \labels{y}{A}}{
				\vlhtr{\DD_1^{\rn w}}{\B, x \le y, x \le x', \Left, \labels{x'}{A} \SEQ \Right, \labels{x'}{B}, \labels{y}{A}}
			}
			%			\vlhtr{\DD_{1\rn w}'}{\B_1, \Left \SEQ \Right, \labels{x}{A \IMP B}, \labels{y}{A}}
		}{
		\vlhtr{\DD_{2}}{\B, x \le y, \Left, \labels{x}{A \IMP B} \SEQ \Right, \labels{y}{A}}
	}
}{
\vliin{\labrn{cut}}{}{\B, x \le y, \Left, \labels{y}{A} \SEQ \Right}{
	\vlhtr{\DD_1[y/x']}{\B, x \le y, \Left, \labels{y}{A} \SEQ \Right, \labels{y}{B}}
}{
\vlhtr{\DD_3}{\B, x \le y, \Left, \labels{y}{B} \SEQ \Right}
}
}
\end{smallequation*}
\todo{check that substitution!}

\begin{smallequation*}
	\vlderiibase{\labrn{cut}}{}{\B, x \le u, u \rel v, \Left \SEQ \Right}{
		\vlin{\rlabrn\BOX}{}{\B, x \le u, u \rel v, \Left \SEQ \Right, \labels{x}{\BOX A}}{
			\vlhtr{\DD_1}{\B, x \le u, u \rel v, x \le x', x' \rel y', \Left \SEQ \Right, \labels{y'}{A}}	
		}
	}{
	\vlin{\llabrn\BOX}{}{\B, x \le u, u \rel v, \Left, \labels{x}{\BOX A} \SEQ \Right}{
		\vlhtr{\DD_2}{\B, x \le u, u \rel v, \Left, \labels{x}{\BOX A}, \labels{v}{A} \SEQ \Right}
	}
}
\reducesto
\end{smallequation*}

\begin{smallequation*}\hspace*{-7em}
	\vlderiibase{\labrn{cut}}{}{\B, x \le u, u \rel v, \Left \SEQ \Right}{
		%		\vliin{\labrn{cut}}{}{\B_1, \B_2, x \le u, u \rel v, \Left \SEQ \Right, \labels{v}{A}}{
		\vlhtr{\DD_1[u/x',v/y']}{\B, x \le u, u \rel v, \Left \SEQ \Right, \labels{v}{A}}	
		%			}{
		%			\vlin{\llabrn\BOX}{}{\B_2, x \le u, u \rel v, \Left, \labels{x}{\BOX A} \SEQ \Right, \labels{v}{A}}{
		%				\vlin{\labrn{id}}{}{\B_2, x \le u, u \rel v, \Left, \labels{x}{\BOX A}, \labels{v}{A} \SEQ \Right, \labels{v}{A}}{
		%					\vlhy{}
		%					}
		%				}
		%			}
	}{
	\vliin{\labrn{cut}}{}{\B, x \le u, u \rel v, \Left, \labels{v}{A} \SEQ \Right}{
		\vlin{\rlabrn\BOX}{}{\B, x \le u, u \rel v, \Left, \labels{v}{A} \SEQ \Right, \labels{x}{\BOX A}}{
			\vlhtr{\DD_1^{\rn w}}{\B, x \le u, u \rel v, x \le x', x' \rel y', \Left, \labels{v}{A} \SEQ \Right, \labels{x}{\BOX A}, \labels{y'}{A}}	
		}
	}{
	\vlhtr{\DD_2}{\B, x \le u, u \rel v, \Left, \labels{x}{\BOX A}, \labels{v}{A} \SEQ \Right}
}
}
\end{smallequation*}

\begin{smallequation*}
	\vlderiibase{\labrn{cut}}{}{\B, x \rel y, \Left \SEQ \Right}{
		\vlin{\rlabrn\DIA}{}{\B, x \rel y, \Left \SEQ \Right, \labels{x}{\DIA A}}{
			\vlhtr{\DD_1}{\B, x \rel y, \Left \SEQ \Right, \labels{x}{\DIA A}, \labels{y}{A}}
		}
	}{
	\vlin{\llabrn\DIA}{\text{\footnotesize $y'$ is fresh}}{\B, x \rel y, \Left, \labels{x}{\DIA A} \SEQ \Right}{
		\vlhtr{\DD_2}{\B, x \rel y, x \rel y',\Left, \labels{y'}{A} \SEQ \Right}
	}	
}
\end{smallequation*}

\begin{smallequation*}
	\vlderiibase{\labrn{cut}}{}{\B, x \rel y, \Left \SEQ \Right}{
		\vliin{\labrn{cut}}{}{\B, x \rel y, \Left \SEQ \Right, \labels{y}{A}}{
			\vlhtr{\DD_1}{\B, x \rel y, \Left \SEQ \Right, \labels{y}{A}, \labels{x}{\DIA A}}
		}{
		\vlin{\llabrn\DIA}{\text{\footnotesize $y'$ is fresh}}{\B, x \rel y, \Left, \labels{x}{\DIA A} \SEQ \Right, \labels{y}{A}}{
			\vlhtr{\DD_2^{\rn w}}{\B, x \rel y, x \rel y',\Left, \labels{y'}{A} \SEQ \Right, \labels{y}{A}}
		}
	}
}{
\vlhtr{\DD_2[y/y']}{\B, x \rel y,\Left, \labels{y}{A} \SEQ \Right}
}
\end{smallequation*}
\todo{check!}
\end{proof}


\begin{theorem}
	\label{thm:mon-adm}
	The rule $\rn{mon}$ is admissible for system $\labIKp$.
\end{theorem}


%\begin{proof}[Proof of Theorem~\ref{thm:mon-adm}]
%	
%	\todo{say that we can obtain the proof of monotonicity using the proof of cut.}
%	
%\end{proof}
