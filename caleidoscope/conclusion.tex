\section{Conclusion}

In this paper we embrace the fully labelled approach to intuitionistic modal logic as pioneered by~\cite{maffezioli:naibo:negri} and generalise it to the class of logics defined by (one-sided intuitionistic) Scott-Lemmon axioms.
%
We establish that it is a valid approach to intuitionistic modal logic by proving soundness and completeness of our system, via a reductive cut-elimination argument.

We believe that it is furthermore the more appropriate way to treat logics outside of the path axioms definable fragment.
%
However, we have not showed that our system satisfies Simpson's 6th requirement, that is, "there is an intuitionistically comprehensible explanation of the meaning of the modalities relative to which [our system] is sound and complete".
%
To make sure that his system satisfies this requirement, Simpson chose to depart from the direct correspondence with modal axioms and their corresponding class of Kripke frames.
%
We instead took the way of a direct correspondence of our system with the class of frames defined by Scott-Lemmon axioms as uncovered by~\cite{plotkin:sterling}, but as this class of logics seems to be rather well-behaved, we believe it shoud be possible to prove the satisfaction of Simpson's 6th requirement too.

As for more general future work, there is a real necessity of a global view on intutionistic modal logics.
%
The work of~\cite{dalmonte:grellois:olivetti:arxiv19} is a great first step in understanding them in the context of non-normal modalities and neighbourhood semantics.
%
It would be interesting to know how and where the class of logics we considered can be included in their framework.