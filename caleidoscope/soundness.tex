%%%%%%%%%%%%%%%%%%%%%%%%%%%%%%%%%%%%%%%%%%%%%%%%%%%%%%%%%
%%%%%%%%%%%%%%%%%%%%%%%%%%%%%%%%%%%%%%%%%%%%%%%%%%%%%%%%%
%%%%%%%%%%%%%%%%%%%%%%%%%%%%%%%%%%%%%%%%%%%%%%%%%%%%%%%%%

\section{Soundness}\label{sec:soundness}

In order to prove the implication \ref{iii}$\implies$\ref{iv} from
Theorem~\ref{thm:cutfree-compl} we need to show that each sequent rule
of our system $\labIKp$ is sound. To make precise what that actually
means, we have to extend the relation $\Vdash$, defined in
Section~\ref{sec:intmod} from formulas to sequents. This is the
purpose of the following definitions.

\begin{definition}\label{def:force-seq}
	Let $\M = \langle W, \rel_\M, \le_\M, V \rangle$ be a model, and let
	$\SG$ be the sequent $\lseq\B\Left\Right$. A \emph{$\SG$-interpretation
		in $\M$} is a mapping $\inter{\cdot}$ from the labels in $\SG$ to the set $W$ of worlds in $\M$, such that whenever $\accs xy$ in $\B$, then $\inter x\rel_\M\inter y$, and whenever $\futs xy$ in $\B$, then $\inter x\le_\M\inter y$. Now we can define 
	\begin{equation}
	\M,\inter\cdot\Vdash\SG\qquad\mbox{iff}\qquad
	\parbox{20em}{if for all $\labels{x}{A} \in \Gamma$, we have $\M, \inter x \Vdash \fm A$, then there exists $\labels{z}{B} \in \Delta$ such that $\M, \inter z \Vdash \fm B$.}
	\end{equation}
\end{definition}

\begin{definition}\label{def:valid-seq}
	A sequent $\SG$ is \emph{satisfied} in $\M = \langle W, \rel, \le, V
	\rangle$ iff for all $\SG$-interpretations $\inter\cdot$ we have
	$\M,\inter\cdot\Vdash\SG$.
	%%
	A sequent $\SG$ is \emph{valid} in a frame $\F = \langle W, \rel,
	\le \rangle$, if for all valuations $V$, the sequent $\SG$ is
	satisfied in $\langle W, R, \le, V \rangle$.
\end{definition}

We are now ready to state the main theorem of this section, of which
the implication \ref{iii}$\implies$\ref{iv} in
Theorem~\ref{thm:cutfree-compl} is an immediate consequence.

\begin{theorem}\label{thm:soundness}
	If a sequent $\SG$ is provable in $\labIKp$, then it is valid in every birelational frame.
\end{theorem}

\begin{proof}
	We proceed by induction on the height of the derivation of $\SG$, and we show for all rules in $\labIKp$
	$$
	\vliiinf{\labrn{r}}{}{\SG}{\SGi1}{\cdots}{\SGi n}
	$$ for $n\in\{0,1,2\}$, that whenever $\SGi1,\ldots,\SGi n$ are
	valid in all birelational frame, then so is $\SG$. It follows a case analysis on $\rn r$:
	\begin{itemize}
		\item $\rlabrn\bot$: This is trivial because $\bot$ is never forced.
		\item $\rn{id}$: This follows immediately from Proposition~\ref{prop:mon}.
		\item $\llabrn\BOX$: By way of contradiction, assume $\B, \Left,
		\futs xy, \accs yz, \labels{x}{\BOX A}, \labels{z}{A} \SEQ \Right$
		is valid in all birelational frames, but $\B,\Left, \futs xy,
		\accs yz, \labels{x}{\BOX A} \SEQ \Right$ is not. This means that
		we have a model $\M$ and an interpretation $\inter\cdot$, such
		that $\M,\inter\cdot\not\Vdash\B,\Left, \futs xy, \accs yz,
		\labels{x}{\BOX A} \SEQ \Right$, i.e., $\inter x\le_\M\inter y$
		and $\inter y\rel_\M\inter z$ and $\cforce \M x{\BOX A}$ and
		$\cnforce \M wB$ for all $\labels wB\in\Right$. However, by the
		definition of forcing in~\eqref{eq:kripke} we also have $\cforce
		\M zA$, and consequently $\nforce \M{\inter\cdot}{\B, \Left, \futs
			xy, \accs yz, \labels{x}{\BOX A}, \labels{z}{A} \SEQ
			\Right}$. Contradiction.
		\item $\rlabrn\BOX$: By way of contradiction, assume that $\futs xy,
		\accs yz, \Left \SEQ \Right, \labels{z}{A}$ is valid in all
		birelational frames, but $\B, \Left \SEQ \Right, \labels{x}{\BOX
			A}$ is not, where $\lb y$ and $\lb z$ do not occur in $\B$ or
		$\Left$ or $\Right$.This means that we have a model $\M$ and an
		interpretation $\inter\cdot$, such that
		$\nforce\M{\inter\cdot}{\B, \Left \SEQ \Right, \labels{x}{\BOX
				A}}$. This means that there are worlds $\lb{y'}$ and $\lb{z'}$
		in $\M$ such that $\inter x\le_\M \lb{y'}$ and
		$\lb{y'}\rel_\M\lb{z'}$ and $\cforce\M{z'}{A}$. Now we let
		$\inter\cdot'$ be the extension of $\inter\cdot$ such that $\inter
		y'=\lb{y'}$ and $\inter z'=\lb{z'}$ and $\inter\cdot'=\inter\cdot$
		on all other labels. Then $\nforce\M{\inter\cdot'}{\futs xy,
			\accs yz, \Left \SEQ \Right, \labels{z}{A}}$. Contradiction.
		\item \lutz{if you want, you can add some more cases here, following the pattern above.}
	\end{itemize}
	The other cases are similar (and simpler), and we leave them to the
	reader. In particular, note that the cases for the rules $\rn{refl}$,
	$\rn{trans}$, $\rn{F_1}$ and $\rn{F_2}$ are trivial, as all
	birelations frames have to obey the corresponding conditions.
\end{proof}

%% We want to prove that for all model $\M$ that satisfies the premise, then $\M$ satifies the conclusion.

%% In order to obtain the proof, let us introduce the next definitions:

%% \begin{definition}
%% 	Let $\M = \langle W, \rel, \le, V \rangle$ be a model. We say that $\f$ is an \emph{assignment function} such that $\f : Labels \rightarrow W$.
%% \end{definition} 

%% \begin{definition}
%% 	Let $\M = \langle W, \rel, \le, V \rangle$ be a model and $\B, \Gamma \SEQ \Delta$ a sequent. We say that $\M \Vdash \B, \Gamma \SEQ \Delta$ if for all $\f$ such that:
%% 	\begin{enumerate}
%% 		\item For all $\labels{x}{A} \in \Gamma$, we have $\M, f(\lb x) \Vdash \fm A$. \marianela{should we say that this is equal to $\M \Vdash \labels{x}{A}$?}
%% 		\item For all $\lb x, \lb y \in \B$, we have $f(\lb x) \rel f(\lb y)$.
%% 		\item For all $\futs xy \in \B$, we have $f(\lb x) \le f(\lb y)$.
%% 	\end{enumerate}
%% 	then there exists $\labels{z}{B} \in \Delta$ such that $\M, f(\lb z) \Vdash \fm B$. We say that $\M \not \Vdash \B, \Gamma \SEQ \Delta$ if it is not the case that $\M \Vdash \B, \Gamma \SEQ \Delta$.
%% \end{definition}


%% Following the definitions, we display the proof of soundness for some rules since it is similar for each rule. 

%% \begin{itemize}
%% 	\item Rule $\llabrn\BOX$: 

%% 	We want to prove that the rule for $\llabrn\BOX$ introduced in the Figure \ref{fig:labIKp} is sound. What we want to prove is:

%% 	\begin{center}
%% 		\emph{For all model $\M$, if $\M \Vdash \B,\Left, \futs xy, \accs yz, \labels{x}{\BOX A}, \labels{z}{A} \SEQ \Right$, then $\M \Vdash \B, \Left, \futs xy, \accs yz, \labels{x}{\BOX A} \SEQ \Right$.}
%% 	\end{center}

%% 	By way of contradiction, we assume that there exists a model $\M_{0}$ such that $\M_{0} \not \Vdash \B, \futs xy, \accs yz,\Left,\break \labels{x}{\BOX A} \SEQ \Right$, i.e $\M_{0} \Vdash \B$, $\M_{0} \Vdash \Left$, $\M_{0} \Vdash \futs xy$, $ \M_{0} \Vdash \accs yz$, $\M_{0} \Vdash \labels{x}{\BOX A}$ and  $\M_{0} \not \Vdash \Right$. Therefore, in particular from $\M_{0} \Vdash \labels{x}{\BOX A}$ we have that for all $\lb y, \lb z \in W$ such that $\futs xy$ and $\accs yz$, $\M_{0}, f(\lb z) \Vdash \fm A$, i.e $\M_{0} \Vdash \labels{z}{\fm A}$. From this result, we can say that $\M_{0} \not \Vdash \B, \Left, \futs xy, \accs yz, \labels{x}{\BOX A}, \labels{z}{A} \SEQ \Right$. Thus, we have already proved that if there exists a model $\M_{0}$ such that $\M_{0} \not \Vdash \B, \futs xy, \accs yz,\Left,\break \labels{x}{\BOX A} \SEQ \Right$, then $\M_{0} \not \Vdash \B, \Left, \futs xy, \accs yz, \labels{x}{\BOX A}, \labels{z}{A} \SEQ \Right$. According to the definition, we proved that the rule $\llabrn\BOX$ is sound.




%% 	\item Rule $\rlabrn\BOX$: 

%% 	In order to prove that the rule $\rlabrn\BOX$ of the system $\labIKp$ is sound, we need to show that:

%% 	\begin{center}
%% 		\emph{For all model $\M$, if $\M \Vdash \B, \futs xy, \accs yz, \Left \SEQ \Right, \labels{z}{A}$, then $\M \Vdash \B, \Left \SEQ \Right, \labels{x}{\BOX A}$.}
%% 	\end{center}

%% 	By way of contradiction, we want to prove that if there exists a model that does not satisfy the conclusion, then the same model does not satisfy the premise. Let us assume that there exists a model $\M_{0}$ such that $\M_{0} \not \Vdash \B, \Left \SEQ \Right, \labels{x}{\BOX A}$. In particular, $\M_{0} \not \Vdash \labels{x}{\BOX A}$, which means that there exist worlds $\lb y, \lb z$, such that $\futs xy$ and $\accs yz$ where $\M_{0} \not \Vdash \labels{z}{A}$. Finally, we have that $\M_{0} \Vdash \B, \futs xy, \accs yz, \Left$ and $\M_{0} \not \Vdash \labels{x}{\BOX A}, \labels{z}{A}$,  which is the same as $\M_{0} \not \Vdash \B, \futs xy, \accs yz, \Left \SEQ \Right, \labels{z}{A}$. Then, by way of contradiction, we proved that the rule $\rlabrn\BOX$ is sound.

%% 	\item Rule $\rlabrn\IMP$: 

%% 	We display the proof of soundness for the rule $\rlabrn\IMP$ showing that:
%% 	\begin{center}
%% 		\emph{For all model $\M$, if $\M \Vdash \B, \futs xy, \Left, \labels{y}{A} \SEQ \Right \labels{y}{B}$, then $\M \Vdash \B, \Left \SEQ \Right, \labels{x}{A \IMP B}$.}
%% 	\end{center}

%% 	We prove this statement by way of contradiction. Assume there exists a model $\M_{0}$ such that $\M_{0}$ does not satisfy the conclusion, i.e. $\M_{0} \not \Vdash \B, \Left \SEQ \Right, \labels{x}{A \IMP B}$. In particular, $\M_{0} \not \Vdash \labels{x}{A \IMP B}$. Therefore, there exists $\lb y$ such that $\futs xy$ and $\M_{0}, f(\lb y) \Vdash A$ (i.e. $\M_{0} \Vdash \labels{y}{A}$) but $\M_{0}, f(\lb y) \Vdash B$, i.e. $\M_{0} \not \Vdash \labels{y}{B}$. Then, we have $\M_{0} \not \Vdash \B, \futs xy, \Left, \labels{y}{A} \SEQ \labels{y}{B}$. This means that the model $\M_{0}$ does not satisfy the premise and that is what we wanted to prove. Finally, we showed that the rule $\rlabrn\IMP$ is sound.

%% 	\item Rule $\llabrn\DIA$:

%% 	Following the same reasoning that we have been tabling, we define what means soundness of the rule $\llabrn\DIA$:

%% 	\begin{center}
%% 		\emph{For all model $\M$, if $\M \Vdash \B, \accs xy,  \Left, \labels{y}{A} \SEQ \Right$, then $\M \Vdash \B, \Left, \labels{x}{\DIA A} \SEQ \Right$.}
%% 	\end{center}

%% 	In order to prove this statement, we assume there exists a model $\M_{0}$ such that $\M_{0} \not \Vdash \B, \Left, \labels{x}{\DIA A} \SEQ \Right$ and we want to see that $\M_{0} \not \Vdash \B, \accs xy, \Left, \labels{y}{A} \SEQ \Right$.
%% 	From our assumption, we have that from $\M_{0} \not \Vdash \B, \Left, \labels{x}{\DIA A} \SEQ \Right$ we obtain $\M_{0} \Vdash \B$, $\M_{0} \Vdash \Left$, $\M_{0} \Vdash \labels{x}{\DIA A}$ and $\M_{0} \not \Vdash \Right$. From $\M_{0} \Vdash \labels{x}{\DIA A}$, according to the definition of the $\DIA$ operator, we have that there exists a world $\lb y$ in $\M_{0}$ such that $\accs xy$ and $\M_{0} \Vdash \labels{y}{A}$. Therefore, we have that $\M_{0} \not \Vdash \B, \accs xy, \Left, \labels{y}{A} \SEQ \Right$ and we prove that the rule $\llabrn\DIA$ is sound.


%% 	\item Rule $\rlabrn\DIA$:

%% 	We want to prove that the rule $\rlabrn\DIA$ is sound, which means that:

%% 	\begin{center}
%% 		For all model $\M$, if $\M \Vdash \B, \accs xy, \Left \SEQ \Right, \labels{x}{\DIA A}, \labels{y}{A}$ then $\M \Vdash \B, \accs xy, \Left \SEQ \Right, \labels{x}{\DIA A}$
%% 	\end{center}

%% 	We prove this statement by way of contradiction. Let us assume there exists a model $\M_{0}$ such that $\M_{0} \not \Vdash \B, \accs xy, \Left \SEQ \Right, \labels{x}{\DIA A}$, we want to show that $\M_{0} \not \Vdash \B, \accs xy, \Left \SEQ \Right, \labels{x}{\DIA A}, \labels{y}{A}$.
%% 	From our hypothesis we have that $\M_{0} \not \Vdash \labels{x}{\DIA A}$, i.e. for all world $\lb y \in \M_{0}$ where $\accs xy$ we have $\M_{0} \not \Vdash \labels{y}{A}$. Therefore, we have that $\M_{0} \not \Vdash \B, \accs xy, \Left \SEQ \Right, \labels{x}{\DIA A}, \labels{y}{A}$. Finally, assuming that $\M_{0}$ does not satisfy the conclusion, we proved that $\M_{0}$ does not satisfy the premise. Then we conclude that the rule $\rlabrn\DIA$ of the system $\labIKp$ is sound.
%% \end{itemize}



%% The other cases are similar and we leave them to the reader.