%%%%%%%%%%%%%%%%%%%%%%%%%%%%%%%%%%%%%%%%%%%%%%%%%%%%%%%%%
%%%%%%%%%%%%%%%%%%%%%%%%%%%%%%%%%%%%%%%%%%%%%%%%%%%%%%%%%
%%%%%%%%%%%%%%%%%%%%%%%%%%%%%%%%%%%%%%%%%%%%%%%%%%%%%%%%%

\section{Completeness}\label{sec:completeness}

In this section we show our system at work, as most of the section
consists of derivations of axioms of $\IK$ in $\labIKp$. More precisely, we prove completeness of $\labIKp$, i.e., the implication \ref{i}$\implies$\ref{ii} of Theorem~\ref{thm:cutfree-compl}, which is stated again below:

\begin{theorem}\label{thm:completeness}
	For any formula $\fm A$. If $\fm A$ is a theorem of $\IK$ then $\fm A$ is provable in $\labIKp +\labrn{cut}$.
\end{theorem}

\begin{remark}
  We have seem already in the proof of Proposition~\ref{prop:id} the
  need of the rule $\rn{F_2}$. In the following proof of
  Theorem~\ref{thm:completeness} we also see the need of the rules
  $\rn{F_1}$, $\rn{refl}$, and $\rn{trans}$.
\end{remark}

\begin{proof}[Proof of Theorem~\ref{thm:completeness}]
  We begin by showning how the axioms $\kax[1]$--$\kax[5]$ are proved in system $\labIKp$.
  \begin{itemize}
  \item $\kax[1]$:
    $$
    \vlderivation{
				\vlin{\rlabrn{\IMP}}
				{\lb y \mbox{ fresh}}
				{\SEQ \labels{x}{\BOX (A \IMP B) \IMP (\BOX A \IMP \BOX B)}}
				{\vlin {\rlabrn{\IMP}}
					{\lb z \mbox{ fresh}}
					{\lseq{\futs xy}{\labels{y}{\BOX(A \IMP B)}}{\labels{y}{\BOX A \IMP \BOX B}}}
					{\vlin {\rlabrn{\BOX}}
						{\lb w, \lb u \mbox{ fresh}}
						{\lseq{\futs xy, \futs yz}{\labels{y}{\BOX(A \IMP B)}, \labels{z}{\BOX A}}{\labels{z}{\BOX B}}}
						{\vlin {\llabrn{\BOX}}
							{}
							{\lseq{\futs xy, \futs yz, \futs zw, \accs wu}{\labels{y}{\BOX(A \IMP B)}, \labels{z}{\BOX A}}{\labels{u}{B}}}
							{\vlin {\color{red}\rn{trans}}
								{}
								{\lseq{\futs xy, \futs yz, \futs zw, \accs wu}{\labels{y}{\BOX(A \IMP B)}, \labels{z}{\BOX A}, \labels{u}{A}}{\labels{u}{B}}}
								{\vlin {\llabrn{\BOX}}
									{}
									{\lseq{\futs xy, \futs yz, \futs zw, \futs yw, \accs wu}{\labels{y}{\BOX(A \IMP B)}, \labels{z}{\BOX A}, \labels{u}{A}}{\labels{u}{B}}}
									{\vlin {\color{red}\rn{refl}}
										{}
										{\lseq{\futs xy, \futs yz, \futs zw, \futs yw, \accs wu}{\labels{y}{\BOX(A \IMP B)}, \labels{z}{\BOX A}, \labels{u}{A}, \labels{u}{A \IMP B}}{\labels{u}{B}}}
										{\vliin{\llabrn{\IMP}}
											{}
											{\lseq{\futs xy, \futs yz, \futs zw, \futs yw, \futs uu, \accs wu}{\labels{y}{\BOX(A \IMP B)}, \labels{z}{\BOX A}, \labels{u}{A}, \labels{u}{A \IMP B}}{\labels{u}{B}}}
											{\vlin {\rn{id}}
												{}
												{\lseq{\B}{\labels{y}{\BOX(A \IMP B)}, \labels{z}{\BOX A}, \labels{u}{A}, \labels{u}{A \IMP B}}{\labels{u}{B}, \labels{u}{A}}}
												{\vlhy {}}}
											{\vlin {\rn{id}}
												{}
												{\lseq{\B}{\labels{y}{\BOX(A \IMP B)}, \labels{z}{\BOX A}, \labels{u}{A}, \labels{u}{A \IMP B}, \labels{u}{B}}{\labels{u}{B}}}
												{\vlhy {}}}}}}}}}}
    }
    $$
    where $\B$ is equal to: $\futs xy, \futs yz, \futs zw, \futs yw, \futs uu, \accs wu$.
  \item $\kax[2]$:
    $$
    \vlderivation {
		\vlin{\rlabrn{\IMP}}
		{\lb y \mbox{ fresh}}
		{ \SEQ \labels{x}{\BOX (A \IMP B) \IMP (\DIA A \IMP \DIA B)}}
		{\vlin {\rlabrn{\IMP}}
			{\lb z \mbox{ fresh}}
			{\lseq{\futs xy}{\labels{y}{\BOX (A \IMP B)}}{\labels{y}{(\DIA A \IMP \DIA B)}}}
			{\vlin {\llabrn{\DIA}}
				{\lb u \mbox{ fresh}}
				{\lseq{\futs xy, \futs yz}{\labels{y}{\BOX (A \IMP B)}, \labels{z}{\DIA A}}{\labels{z}{\DIA B}}}
				{\vlin{\rlabrn{\DIA}}
					{}
					{\lseq{\futs xy, \futs yz, \accs zu}{\labels{y}{\BOX (A \IMP B)}, \labels{u}{A}}{\labels{z}{\DIA B}}}
					{\vlin {\llabrn{\BOX}}
						{}
						{\lseq{\futs xy, \futs yz, \accs zu}{\labels{y}{\BOX (A \IMP B)}, \labels{u}{A}}{\labels{z}{\DIA B}, \labels{u}{B}}}
						{\vlin {\color{red}\rn{refl}}
							{}
							{\lseq{\futs xy, \futs yz, \accs zu}{\labels{y}{\BOX (A \IMP B)}, \labels{u}{A}, \labels{u}{A \IMP B}}{\labels{z}{\DIA B}, \labels{u}{B}}}
							{\vliin{\llabrn{\IMP}}
								{}
								{\lseq{\futs xy, \futs yz, \accs zu, \futs uu}{\labels{y}{\BOX (A \IMP B)}, \labels{u}{A}, \labels{u}{A \IMP B}}{\labels{z}{\DIA B}, \labels{u}{B}}}
								{\vlin {\rn{id}}
									{}
									{\lseq{\B}{\labels{y}{\BOX (A \IMP B)}, \labels{u}{A}, \labels{u}{A \IMP B}}{\labels{z}{\DIA B}, \labels{u}{B}, \labels{u}{A}}}
									{\vlhy {}}}
								{\vlin {\rn{id}}
									{}
									{\lseq{\B}{\labels{y}{\BOX (A \IMP B)}, \labels{u}{A}, \labels{u}{A \IMP B}, \labels{u}{B}}{\labels{z}{\DIA B}, \labels{u}{B}}}
									{\vlhy {}}}								}}}}}}
    }
    $$
    where $\B$ is equal to $\futs xy, \futs yz, \accs zu, \futs uu$.
  \item $\kax[3]$:
    $$
    \vlderivation {
			\vlin{\rlabrn{\IMP}}
			{}
			{\SEQ \labels{x}{\DIA (A \OR B) \IMP (\DIA A \OR \DIA B)}}
			{\vlin {\llabrn{\DIA}}
				{}
				{\lseq{\futs xy}{\labels{y}{\DIA (A \OR B)}}{\labels{y}{\DIA A \OR \DIA B}}}
				{\vliin{\llabrn{\OR}}{}{\lseq{\futs xy, \accs yz}{\labels{z}{A \OR B}}{\labels{y}{\DIA A \OR \DIA B}}}{\vlin {\rlabrn{\OR}}
						{}
						{\lseq{\futs xy, \accs yz}{\labels{z}{A}}{\labels{y}{\DIA A \OR \DIA B}}}
						{\vlin {\rlabrn{\DIA}}
							{}
							{\lseq{\futs xy, \accs yz}{\labels{z}{A}}{\labels{y}{\DIA A}, \labels{y}{\DIA B}}}
							{\vlin {\color{red}\rn{refl}}
								{}
								{\lseq{\futs xy, \accs yz}{\labels{z}{A}}{\labels{y}{\DIA A}, \labels{z}{A}, \labels{y}{\DIA B}}}
								{\vlin {\rn{id}}
									{}
									{\lseq{\futs xy, \futs zz, \accs yz}{\labels{z}{A}}{\labels{y}{\DIA A}, \labels{z}{A}, \labels{y}{\DIA B}}}
									{\vlhy {}}}}}}{\vlin {\rlabrn{\OR}}
						{}
						{\lseq{\futs xy, \accs yz}{\labels{z}{B}}{\labels{y}{\DIA A \OR \DIA B}}}
						{\vlin {\rlabrn{\DIA}}
							{}
							{\lseq{\futs xy, \accs yz}{\labels{z}{B}}{\labels{y}{\DIA A}, \labels{y}{\DIA B}}}
							{\vlin {\color{red}\rn{refl}}
								{}
								{\lseq{\futs xy, \accs yz}{\labels{z}{B}}{\labels{y}{\DIA A}, \labels{y}{\DIA B}, \labels{z}{B}}}
								{\vlin {\rn{id}}
									{}
									{\lseq{\futs xy, \futs zz, \accs yz}{\labels{z}{B}}{\labels{y}{\DIA A}, \labels{y}{\DIA B}, \labels{z}{B}}}
									{\vlhy {}}}}}}}}
		}
    $$
  \item $\kax[4]$:
    $$
\vlderivation {
		\vlin{\rlabrn{\IMP}}
		{\lb y \mbox{ fresh}}
		{\SEQ \labels{x}{(\DIA A \IMP \BOX B) \IMP \BOX (A \IMP B)}}
		{\vlin {\rlabrn{\BOX}}
			{\lb z, \lb w \mbox{ fresh}}
			{\lseq{\futs xy}{\labels{y}{\DIA A \IMP \BOX B}}{\labels{y}{\BOX (A \IMP B)}}}
			{\vlin {\rlabrn{\IMP}}
				{\lb u \mbox{ fresh}}
				{\lseq{\futs xy, \futs yz, \accs zw}{\labels{y}{\DIA A \IMP \BOX B}}{\labels{w}{A \IMP B}}}
				{\vlin {\color{red}{\rn{F_1}}}
					{}
					{\lseq{\futs xy, \futs yz, \futs wu, \accs zw}{\labels{y}{\DIA A \IMP \BOX B}, \labels{u}{A}}{\labels{u}{B}}}
					{\vlin {\color{red}\rn{trans}}
						{}
						{\lseq{\futs xy, \futs yz, \futs wu, \futs zt, \accs zw, \accs tu}{\labels{y}{\DIA A \IMP \BOX B}, \labels{u}{A}}{\labels{u}{B}}}
						{\vliin {\llabrn{\IMP}}
							{}
							{\lseq{\futs xy, \futs yz, \futs wu, \futs zt, \futs yt, \accs zw, \accs tu}{\labels{y}{\DIA A \IMP \BOX B}, \labels{u}{A}}{\labels{u}{B}}}
							{\vlin {\rlabrn{\DIA}}
									{}
									{\lseq{\B}{\labels{y}{\DIA A \IMP \BOX B}, \labels{u}{A}}{\labels{u}{B}, \labels{t}{\DIA A}}}
									{\vlin {\color{red}\rn{refl}}
										{}
										{\lseq{\B}{\labels{y}{\DIA A \IMP \BOX B}, \labels{u}{A}}{\labels{u}{B}, \labels{t}{\DIA A}, \labels{u}{A}}}
										{\vlin {\rn{id}}
											{}
											{\lseq{\B, \futs uu}{\labels{y}{\DIA A \IMP \BOX B}, \labels{u}{A}}{\labels{u}{B}, \labels{t}{\DIA A}, \labels{u}{A}}}
											{\vlhy {}}}}}
							{\vlin {\color{red}\rn{refl}}
								{}
								{\lseq{\B}{\labels{y}{\DIA A \IMP \BOX B}, \labels{u}{A}, \labels{t}{\BOX B}}{\labels{u}{B}}}
								{\vlin {\llabrn{\BOX}}
									{}
									{\lseq{\B, \futs tt}{\labels{y}{\DIA A \IMP \BOX B}, \labels{u}{A}, \labels{t}{\BOX B}}{\labels{u}{B}}}
									{\vlin {\color{red}\rn{refl}}
										{}
										{\lseq{\B, \futs tt}{\labels{y}{\DIA A \IMP \BOX B}, \labels{u}{A}, \labels{t}{\BOX B}, \labels{u}{B}}{\labels{u}{B}}}
										{\vlin {\rn{id}}
											{}
											{\lseq{\B, \futs tt, \futs uu}{\labels{y}{\DIA A \IMP \BOX B}, \labels{u}{A}, \labels{t}{\BOX B}, \labels{u}{B}}{\labels{u}{B}}}
											{\vlhy {}}}}}}}}}}}
}
$$
where $\B$ is equal to $\futs xy, \futs yz, \futs wu, \futs zt, \futs yt, \accs zw, \accs tu$.
\item $\kax[5]$:
  $$
  \vlderivation {
		\vlin{\rlabrn{\IMP}}
		{}
		{\SEQ \labels{x}{\DIA \BOT \IMP \BOT}}
		{\vlin {\llabrn{\DIA}}
			{}
			{\lseq{\futs xy}{\labels{y}{\DIA \BOT}}{\labels{y}{\BOT}}}
			{\vlin {\llabrn{\BOT}}
				{}
				{\lseq{\futs xy, \accs yz}{\labels{z}{\BOT}}{\labels{y}{\BOT}}}
				{\vlhy {}}}}
  }
  $$
  \end{itemize}
  Next, we have to prove all axioms of intuitionistic propositional logic can be shown in $\labIKp$. We do this only for $\fm{A \AND B \IMP B}$ and leave the rest to the reader:
  $$
  \vlderivation {
		\vlin{\rlabrn{\IMP}}
		{}
		{\SEQ \labels{x}{A \AND B \IMP B}}
		{\vlin {\llabrn{\AND}}
			{}
			{\lseq{\futs xy}{\labels{y}{A \AND B}}{\labels{y}{B}}}
			{\vlin {\color{red}\rn{refl}}
				{}
				{\lseq{\futs xy}{\labels{y}{A}, \labels{y}{B}}{\labels{y}{B}}}
				{\vlin {\rn{id}}
					{}
					{\lseq{\futs xy, \futs yy}{\labels{y}{A}, \labels{y}{B}}{\labels{y}{B}}}
					{\vlhy {}}}}}
  }
  $$
  
  
  Finally, we have to show how the rules of modus ponens and
  necessitation can be simulated in our system. For modus ponens, this
  is standard using the cut rule and for necessitation, we can
  transform a proof of $\fm A$ into a proof of $\fm{\BOX A}$ as
  follows:
  
\begin{lemma}
	If there exists a proof of $\vlderivation {\vlpd{\Done}{}{\SEQ \labels{z}{A}}}$ then there exists a proof of $\vlderivation { \vlpd{\Dtwo}{}{\SEQ \labels{x}{\BOX A}}}$
\end{lemma}

\begin{proof}
	
	We assume that there exists a proof of $\vlderivation {\vlpd {\Done}{}{\SEQ \labels{z}{A}}}$ and we want to obtain a proof of $\vlderivation { \vlpd{\Dtwo}{}{\SEQ \labels{x}{\BOX A}}}$.
	
	Using the rule $\rlabrn{\BOX}$ introduced in the system $\labIKp$ and the proof of $\labels{z}{A}$ from our hypothesis, we can build the following proof:
	$\vlderivation{\vlin{\rlabrn{\BOX}}{}{\SEQ \labels{x}{\BOX A}}{\vlin{\rn{w}}{}{\futs xy, \accs yz \SEQ \labels{z}{A}}{\vlhy{ \SEQ \labels{z}{A}}}}}$ \hspace{3mm} or what it is the same $\vlderivation{\vlin{\rlabrn{\BOX}}{}{\SEQ \labels{x}{\BOX A}}{\vlpd {\Dwone}{}{\futs xy, \accs yz \SEQ \labels{z}{A}}}}$.
	
	Let $\Dtwo$ be equal to: $\Dtwo = \vlderivation {\vlpd {\Dwone}{}{\futs xy, \accs yz \SEQ \labels{z}{A}}}$. Therefore, we have the proof $\vlderivation { \vlpd{\Dtwo}{}{\SEQ \labels{x}{\BOX A}}}$.
	
%	If we have a proof of $\vlderivation {\vlpd {\Done}{}{\SEQ \labels{z}{A}}}$ then we can obtain a proof of $\labels{x}{\BOX A}$ using the rule $\rlabrn{\BOX}$ introduced in the system $\labIKp$ and the proof of $\labels{z}{A}$ that we assumed  $\vlderivation{\vlin{\rlabrn{\BOX}}{}{\SEQ \labels{x}{\BOX A}}{\vlpd {\Dwone}{}{\futs xy, \accs yz \SEQ \labels{z}{A}}}}$
%	
%	Let $\Dtwo$ be equal to: $\Dtwo = \vlderivation {\vlpd {\Dwone}{}{\futs xy, \accs yz \SEQ \labels{z}{A}}}$.
%	Then using the \emph{weakening} rule, we have the following proof:
%	
%	\bigskip
%	\begin{center}
%		$\vlderivation{\vlin{\rlabrn{\BOX}}{}{\SEQ \labels{x}{\BOX A}}{\vlin{\rn{w}}{}{\futs xy, \accs yz \SEQ \labels{z}{A}}{\vlhy{ \SEQ \labels{z}{A}}}}}$
%	\end{center}
%	
	
	
\end{proof}

  
  This completes the proof of Proof of Theorem~\ref{thm:completeness}.
\end{proof}