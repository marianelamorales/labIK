% easychair.tex,v 3.5 2017/03/15

\documentclass{easychair}
%\documentclass[EPiC]{easychair}
%\documentclass[EPiCempty]{easychair}
%\documentclass[debug]{easychair}
%\documentclass[verbose]{easychair}
%\documentclass[notimes]{easychair}
%\documentclass[withtimes]{easychair}
%\documentclass[a4paper]{easychair}
%\documentclass[letterpaper]{easychair}

\usepackage{doc}
\newcommand{\G}{\mathcal{G}}
\newcommand{\Left}{\Gamma}
\newcommand{\Right}{\Delta}
%\newcommand{\Left}{\mathcal{L}}
%\newcommand{\Right}{\mathcal{R}}
\newcommand{\agklmn}{\mathsf{g_{klmn}}}

%Symbols for System labK
\newcommand{\id}{id^{lab}}
\newcommand{\tolab}{\top^{lab}}
\newcommand{\vlab}{\wedge^{lab}}
\newcommand{\olab}{\vlor^{lab}}
\newcommand{\blab}{\square^{lab}}
\newcommand{\dlab}{\lozenge^{lab}}

%Labelled proof system
\newcommand{\toprule}{\B \Rightarrow \Right, x  \colon   \top}
\newcommand{\vlabr}{\B \Rightarrow \Right, x  \colon   A}
\newcommand{\vlabu}{\B \Rightarrow \Right, x  \colon   B}
\newcommand{\olabr}{\B \Rightarrow \Right, x  \colon   A, x  \colon   B}
\newcommand{\blabr}{\B \Rightarrow \Right, x  \colon   \square A}
\newcommand{\blabu}{\B, x$R$y \Rightarrow \Right, y  \colon   A}
\newcommand{\dlabr}{\B, x$R$y \Rightarrow \Right, x  \colon   \lozenge A}
\newcommand{\dlabu}{\B, x$R$y \Rightarrow \Right, x  \colon   \lozenge A, y  \colon} 


%Symbols for system labIK
\newcommand{\botlab}{\bot_{L}^{lab}}
\newcommand{\toplab}{\top_{R}^{lab}}
\newcommand{\andleflab}{\wedge_{L}^{lab}}
\newcommand{\andriglab}{\wedge_{R}^{lab}}
\newcommand{\orleflab}{\vlor_{L}^{lab}}
\newcommand{\orriglabo}{\vlor_{R1}^{lab}}
\newcommand{\orriglabt}{\vlor_{R2}^{lab}}
\newcommand{\irlab}{\vljm_{R}^{lab}}
\newcommand{\illab}{\vljm_{L}^{lab}}
\newcommand{\dllab}{\lozenge_{L}^{lab}}
\newcommand{\drlab}{\lozenge_{R}^{lab}}
\newcommand{\bllab}{\square_{L}^{lab}}
\newcommand{\brlab}{\square_{R}^{lab}}

%System labIK+gklmn
\newcommand{\gklmn}{\boxtimes_{\mathsf{gklmn}}}
\newcommand{\boxk}{\square_{R}^{k}}
\newcommand{\boxlk}{\square_{L}^{k}}
\newcommand{\diamk}{\lozenge_{L}^{k}}
\newcommand{\diamrk}{\lozenge_{R}^{k}}

%Symbols for System labheartIK
\newcommand{\ids}{id}
\newcommand{\idg}{id_{g}}
\newcommand{\refl}{refl}
\newcommand{\trans}{trans}
\newcommand{\cut}{cut}
\newcommand{\fone}{F1}
\newcommand{\ftwo}{F2}
\newcommand{\sbot}{\bot_{L}}
\newcommand{\Stop}{\top_{R}}
\newcommand{\svlef}{\wedge_{L}}
\newcommand{\svrig}{\wedge_{R}}
\newcommand{\solef}{\vlor_{L}}
\newcommand{\sorig}{\vlor_{R}}
\newcommand{\sorone}{\vlor_{R1}}
\newcommand{\sotwo}{\vlor_{R2}}
\newcommand{\sir}{\vljm_{R}}
\newcommand{\sil}{\vljm_{L}}
\newcommand{\sdl}{\lozenge_{L}}
\newcommand{\sdr}{\lozenge_{R}}
\newcommand{\sbl}{\square_{L}}
\newcommand{\sbr}{\square_{R}}
\newcommand{\smon}{mon_{L}}
\newcommand{\M}{\mathcal{M}}
\newcommand{\F}{\mathcal{F}}
\newcommand{\Gone}{\mathcal{G}_{1}}
\newcommand{\Gtwo}{\mathcal{G}_{2}}
\newcommand{\Dw}{\mathcal{D}^{w}}
\newcommand{\Dwone}{\mathcal{D}_{1}^{w}}
\newcommand{\Dwtwo}{\mathcal{D}_{2}^{w}}
\newcommand{\D}{\mathcal{D}}
\newcommand{\Done}{\mathcal{D}_{1}}
\newcommand{\Dtwo}{\mathcal{D}_{2}}


%System LABIK
\newcommand{\conjrig}{\G, \Left \Rightarrow \Right, x \colon A}
\newcommand{\conjrigh}{\G, \Left \Rightarrow \Right, x  \colon B}
\newcommand{\conjlef}{\G, \Left, x  \colon  A, x \colon B \Rightarrow \Right}


% use this if you have a long article and want to create an index
% \usepackage{makeidx}

% In order to save space or manage large tables or figures in a
% landcape-like text, you can use the rotating and pdflscape
% packages. Uncomment the desired from the below.
%
% \usepackage{rotating}
% \usepackage{pdflscape}

% Some of our commands for this guide.
%
\newcommand{\easychair}{\textsf{easychair}}
\newcommand{\miktex}{MiK{\TeX}}
\newcommand{\texniccenter}{{\TeX}nicCenter}
\newcommand{\makefile}{\texttt{Makefile}}
\newcommand{\latexeditor}{LEd}

%\makeindex

%% Front Matter
%%
% Regular title as in the article class.
%
\title{A bicolor nested sequents for intuitionistic modal logics}

% Authors are joined by \and. Their affiliations are given by \inst, which indexes
% into the list defined using \institute
%
\author{
	Sonia Marin\inst{1}%\thanks{Designed and implemented the class style}
\and
    Marianela Morales\inst{2}%\thanks{Did numerous tests and provided a lot of suggestions}
}

% Institutes for affiliations are also joined by \and,
\institute{
	University College London  \\
  \email{s.marin@ucl.ac.uk}
  \and
  LIX, \'Ecole Polytechnique  \&  Inria Saclay\\
	\email{marianela.morales@polytechnique.edu}
 }

%  \authorrunning{} has to be set for the shorter version of the authors' names;
% otherwise a warning will be rendered in the running heads. When processed by
% EasyChair, this command is mandatory: a document without \authorrunning
% will be rejected by EasyChair

\authorrunning{Marin and Morales}

% \titlerunning{} has to be set to either the main title or its shorter
% version for the running heads. When processed by
% EasyChair, this command is mandatory: a document without \titlerunning
% will be rejected by EasyChair
\titlerunning{bicolor NS}

\begin{document}

\maketitle

\begin{abstract}
We present a nested sequent system for intuitionistic modal logic in which we capture explicitly two relations: one for the accessibility relation associated with the Kripke semantics for normal modal logics (represented with one bracket in a formula or context) and one for the preorder relation associated with the Kripke semantics for intuitionistic logic (represented with two brackets in a formula or context).
%

\emph{Keywords:} Nested sequents, Intuitionistic modal logic, Proof theory. 
\end{abstract}

\section{Introduction}
\todo{Say: working on multi-conclusion nested sequents for intuitionistic modal logic. A bit of story? Why is useful?}

Structural proof theoretic accounts of intuitionistic modal logic can adopt the paradigm of \emph{labelled deduction} in the form of labelled natural deduction and sequent systems \cite{simpson1994proof},
or the one of \emph{unlabelled deduction} in the form of sequent~\cite{bierman2000} or nested sequent systems \cite{strassburger2013cut}.
Nested sequents were introduced by Kashima \cite{kashima1994cut} and then independently rediscovered by Br\"{u}nnler \cite{brunnler2009deep} and Poggiolesi \cite{poggiolesi2009method}.

\section{Preliminaries}

Nested sequents are a generalisation of sequents from a multiset
of formulas to a tree of multisets of formulas.
In nested sequent notation, brackets are used to indicate the parent-child relation in the tree, and can be interpreted as the modal $\BOX$ (similarly to how the comma is interpreted as $\OR$). for intuitionistic logic we need two-sided sequents, which are formally generated by:
\begin{center}
	$\Cx \coloncolonequals \lf{A_{1}}, ..., \lf{A_{n}}, \rt{B_{1}},..., \rt{B_{k}}, \BR{\Cx[1]},..., \BR{\Cx[m]}$
\end{center}

where $A_{1}, . . . , A_{n}$ are the formulas that would occur on the left of the turnstile if there was a turnstile, $B_{1}, . . . , B_{k}$ are the formulas that would occur on the right of the turnstile if there was a turnstile, and $\Cx[1], . . . , \Cx[m]$ are nested sequents. We use the $\bullet$ and $\circ$ as polarities indication for formulas.

In this short article, we present a nested sequent system for intuiniotistic modal logic. Intuitionistic modal logics are intuitionistic propositional logic extended with the modalities $\BOX$ and $\DIA$, and the necessitation rule, obeying some variants of the $\kax$-axiom since De Morgan duality between the modalities is not present in the intuitionistic case:

\begin{equation*}
\label{eq:ik}%\hskip-2em
\begin{array}[t]{r@{\;}l@{\qquad}r@{\;}l@{\qquad}r@{\;}l}
\kax[1]\colon&\BOX(A\IMP B)\IMP(\BOX A\IMP\BOX B)
&
\kax[3]\colon&\DIA(A\OR B)\IMP(\DIA A\OR\DIA B)
&
\kax[5]\colon&\DIA\BOT\IMP\BOT
\\
\kax[2]\colon&\BOX(A\IMP B)\IMP(\DIA A\IMP\DIA B)
&
\kax[4]\colon&(\DIA A\IMP \BOX B)\IMP\BOX(A\IMP B)\\%x[1ex]
\end{array}
\end{equation*}
%%
We recall the birelational models \cite{plotkin1986, ewald1986} for intuitionistic modal logics, which are a combination of the Kripke semantics for propositional intuitionistic logic and the one for classical modal logic. A \emph{birelational frame} $\F = \langle W, R, \le \rangle$ is a non-empty set $W$ of worlds together with two binary relations $\le, \rel \subseteq W \times W$, where $\le$ is a preorder(i.e., reflexive and transitive), such that the following two conditions hold:

\begin{enumerate}
	\item[($\rn{F_1}$)] For $u, v, v' \in W$, if $u \rel v$ and $v \le v'$, there exists $u'$ s.t.~$u \le u'$ and $u' \rel v'$.
	
	\item[($\rn{F_2}$)] For $u', u, v \in W$, if $u \le v$, there exists $v'$ s.t.~$u' \rel v'$ and $v\le v'$.
\end{enumerate}

\section{Bicolor nested sequent system}

In order to capture intuitionistic modal logics, we build a nested sequent system with the two relations explicitly: one of for the accessibility relation $\rel$ associated with the Kripke semantics for normal modal logics (represented with one bracket in the sequent) and one for the preorder relation $\le$ associated with the Kripje semantics for intuitionistic logic (represented with two brackets in the system).

In the intuitionistic setting, the validity of a modal formula has to be defined using both the $\rel$ and the $\le$ relation as:
$\force{x}{\BOX A}$ iff for all $y$ and $z$ s.t. $\futs xy$ and $\accs yz$, $\force{z}{A}$. Our system reflects exactly this definition in the rules introducing the $\BOX$-operator:

\begin{center}
	$\vlinf{\lrn\BOX}{}{\Cx[1]{\lf{\BOX A}, \BR{\Cx[2]}}}{\Cx[1]{\lf{\BOX A}, \BR{\lf A, \Cx[2]}}}
	\quad
	\vlinf{\rrn\BOX}{}{\Cx{\rt{\BOX A}}}{\Cx{\bBR{\BR{\rt A}}}}$
\end{center}

\marianela{add F1 rule?}

\bibliographystyle{plain}
%\bibliographystyle{alpha}
%\bibliographystyle{unsrt}
%\bibliographystyle{abbrv}
\bibliography{references}


\end{document}