\chapter{Conclusiones}
\label{cap:conclusion}

Este último capítulo abarca, por un lado, un pequeño resumen de lo realizado a lo largo de esta tesis, como así también algunos detalles sobre las líneas de trabajo futuro. Por otro lado, se incluye una breve sección sobre la experiencia que tuve a la hora de realizar este trabajo para finalizar mi carrera como Licenciada en Ciencias de la Computación.

\section{Sobre la tesis}

En este trabajo estudiamos un sistema de prueba etiquetado para la lógica modal intuicionista. Para lograr este resultado se requirió el estudio de conceptos fundamentales para el desarrollo de esta contribución. Los mismos fueron introducidos a lo largo de los primeros capítulos de la tesis. Comenzamos repasando la sintaxis y la semántica de la lógica modal básica, para luego introducir una axiomatización a la Hilbert y otra a la Gentzen, en particular, nos enfocamos en el sistema de secuentes etiquetados propuesto en~\cite{negri2005}. Luego nos centramos en la lógica modal intuicionista, donde nuevamente estudiamos su sintaxis y semántica, para comprender las herramientas necesarias para introducir un nuevo sistema de secuentes etiquetados que nos permitiera capturar este tipo de lógica. 

El enfoque adoptado en este trabajo final consiste  en extender el sistema para la lógica modal básica propuesto por Negri para capturar la lógica modal intuicionista, que denominamos $\labIKh$. Para ello agregamos un nuevo símbolo de relación: la relación futura o mejor conocida como relación de pre-orden $\le$. A diferencia de la lógica modal básica, los operadores modales $\Diamond$ y $\square$ no son duales, debido a que la sintaxis de la lógica modal intuicionista no presenta el operador de negación $\neg$. Por tal motivo, resulta necesario contar con dos reglas de secuentes para cada operador para obtener un sistema completo. Otro aspecto a tener en cuenta fue la necesidad de reglas que caractericen las condiciones de la relación de pre-orden (es decir, reflexividad y transitividad). Este enfoque resulta novedoso ya que nos permite poner el sistema de prueba en estrecha relación con la semántica birelacional de Kripke.


En este trabajo final demostramos que el cálculo de secuentes para la lógica modal intuicionista que definimos es correcto y completo. Esto quiere decir, por un lado, que cada una de las reglas introducidas son semánticamente correctas, y por el otro, que son suficientes para caracterizar todos los teoremas de la lógica. En particular, demostramos completitud sintácticamente por medio del sistema a la Hilbert, lo cual nos permite concluir que el sistema $\labIKh$ es correcto y completo con la regla de $\mathsf{cut}$. Sin embargo, continuamos trabajando para obtener un sistema que sea completo libre de $\mathsf{cut}$. Una alternativa para lograr este tipo de sistemas, consiste en demostrar la propiedad de \emph{cut elimination}, aunque decidimos continuar nuestro trabajo con el estudio de sistemas completos libres de $\mathsf{cut}$, por el enorme desafío que representa dicha prueba. El sistema propuesto por Simpson~\cite{simpson1994} se presenta como una buena alternativa, debido a que resulta simple establecer una correspondencia entre las reglas de tal sistema y las reglas de nuestro sistema $\labIKh$. Para completar esta demostración, es necesario demostrar que el resto de las reglas de Simpson pueden ser deducidas en nuestro sistema. Esta conjetura terminará de ser formalizada y estudiada en nuestro trabajo futuro. Vale la pena destacar que, a pesar de poseer reglas similares a nuestro sistema, el sistema de Simpson no es suficiente para capturar la semántica de la lógica modal intuicionista. Como se puede observar en la Figura \ref{fig:simpson}, las reglas presentadas para el operador $\square$ no capturan la relación de pre-orden así como tampoco el sistema nos permite hablar de la implicación intuicionista. Dado que el cálculo de secuentes etiquetados propuesto en este trabajo nos permite hablar tanto de la relación de pre-orden como de la implicación intuicionista, es posible continuar con el estudio de esta demostración para lograr que el sistema $\labIKh$ sea un sistema completo sin la regla de $\mathsf{cut}$.

Por otra parte, nuestro trabajo futuro también contempla el estudio de extensiones del sistema para caracterizar lógicas más fuertes, en particular, para restringir la clase de frames que queremos considerar. En el último capítulo discutimos extensiones con el axioma $\agklmn$, presentando una nueva regla de secuente para el sistema $\labIKh$, y demostrando completitud para este nuevo sistema. Por el momento, sólo se discutió completitud para el caso $k = l = m = n = \mathsf{1}$ pero se continuará con el desarrollo del caso general. Esta demostración y el estudio de otras extensiones es parte de lo que seguiremos trabajando en un futuro próximo.

Vale la pena agregar que una de las principales ventajas del sistema de secuentes etiquetado $\labIKh$, es que nos permite extender de una manera sencilla a otros sistemas. Por otro lado, gracias al uso de etiquetas, es posible obtener una demostración simple de completitud.

Finalmente, consideramos que el sistema propuesto en este trabajo final, constituye una herramienta importante para el estudio de la complejidad computacional de las tareas de inferencia de la lógica modal intuicionista. En particular, del problema de decidir si una fórmula de esta lógica es válida. Esto también forma parte de las posibles líneas de investigación futuras.

 
 \section{Sobre mi experiencia de trabajo final de Licenciatura}
 
 Los comienzos de mi tesis surgieron a partir de una pasantía realizada en Francia desde diciembre de 2017 a marzo de 2018. El objetivo propuesto fue obtener mi primer experiencia en el mundo de la investigación y fue allí donde descubrí con certeza que esto era lo quería para mi futuro. Mi pasantía se realizó en INRIA bajo la supervisión del Dr. Lutz Strassburger en el equipo PARSIFAL, dirigido por el Dr. Dale Miller. 
 
 Mis estudios que constituyen la base de mi trabajo final de Licenciatura comenzaron allí en Palaiseau - Saclay. Luego de tener la oportunidad de indagar en otros temas como \emph{nested sequents} y \emph{combinatorial proofs}, decidí inclinarme por el estudio de la lógica modal intuicionista y su teoría de prueba. A pesar de que antes de esta pasantía realicé un curso de lógicas modales en FaMAF con el Dr. Carlos Areces y el Dr. Raul Fervari, este fue mi primer acercamiento a la teoría de prueba. Comenzar con el estudio de dicha área, constituyó uno de mis mayores desafíos en conjunto con un nuevo idioma, una nueva cultura, un nuevo grupo humano y mi primera vez en investigación. Durante mi pasantía tuve la oportunidad de visitar y colaborar con Sonia Marin (post-doc en IT-University de Copenhague y PhD obtenido con Dr. Lutz Strassburger) quien fue uno de mis mayores soportes a lo largo de mi trabajo en INRIA. 
 
Una vez concluida mi pasantía continué con mi trabajo de Licenciatura acompañada del Dr. Raul Fervari de la FaMAF quien actualmente es uno de mis directores de tesis junto con el Dr. Lutz Strassburger. En cada uno de mis avances fui también recibiendo sugerencias, consejos y correcciones del Dr. Carlos Areces.

Considero que dicha pasantía fue una buena oportunidad para afianzar los conocimientos obtenidos en diferentes etapas de la carrera. Para concluir, es importante destacar que los aprendizajes adquiridos resultan una base fundamental para continuar con mis estudios de doctorado.
 