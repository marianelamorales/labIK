\section{El problema principal}

Como se dijo anteriormente, nuestro trabajo consiste en la definición de un sistema de prueba etiquetado para lógicas modales intuicionistas. La extensión directa de la deducción etiquetada al entorno intuicionista consiste en la utilización de dos tipos de átomos relaciones: uno para la relación de accesibilidad modal representada con $R$, y otro para la relación intuicionista o mejor conocida como relación de preorden $\le$. Esto resulta interesante ya que nos permite poner el sistema de prueba en estrecha relación con la semántica birelacional de Kripke que se detalló en el Capítulo~\ref{cap:logintuicionista}.

En \cite{negri2005} se introdujo un cálculo de secuentes etiquetado para la lógica modal básica. En este trabajo extenderemos dicho sistema con el objetivo de capturar lógicas modales intuicionistas. Además de las fórmulas etiquetadas de la forma $x \colon A$, y de las expresiones $xRy$, capturando la relación de accesibilidad (donde $x, y$ pertenecen a un conjunto de etiquetas, y $A$ es un fórmula), utilizaremos expresiones del tipo $x \le y$ para capturar la relación de preorden intuicionista. De una manera más formal introducimos la siguiente definición:

%La idea es extender un cálculo de secuentes etiquetado con un símbolo de relación de preorden con el objetivo de capturar lógicas modales intuicionistas; esto quiere decir, buscamos definir secuentes etiquetados intuicionistas a partir de fórmulas etiquetadas $x \colon A$, de la relación de accesibilidad modal $x$R$y$, y la nueva relación de preorden de la forma $x \le y$, donde $x, y$ pertenecen a un conjunto de etiquetas y $A$ es una fórmula modal intuicionista. El cálculo de secuentes etiquetado en el cual nos basamos fue el propuesto por Negri para la lógica modal básica \cite{negri2005}.



\dfn{Un \emph{secuente etiquetado intuicionista} es de la forma $\G$, $\Left \Rightarrow \Right$ donde $\G$ denota un conjunto de átomos relacionales y de preorden, y  $\Left$ y $\Right$ son conjuntos de fórmulas etiquetadas. }

\vspace{2mm}

De esta manera obtenemos el sistema de prueba $\labIKh$ para lógicas modales intuicionistas en este formalismo, y podemos demostrar el siguiente teorema:

\begin{quote}
 \emph{Una fórmula $A$ es demostrable en el cálculo $\labIKh$ si y sólo si $A$ es válida en cada frame bi-relacional.}
\end{quote}

La dirección de izquierda a derecha de la sentencia anterior indica que el cálculo es \emph{correcto}: las reglas me permiten inferir solo fórmulas válidas. Por otro lado, la implicación de derecha a izquierda se conoce como \emph{completitud}: todos los teoremas son demostrables en el cálculo.

En la siguiente sección, se presenta el nuevo sistema de cálculo de secuentes etiquetados para las lógicas modales intuicionistas. 

