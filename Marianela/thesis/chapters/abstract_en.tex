\paragraph{Abstract}


The study of modal logic \cite{blackburn01}, going back to Aristotle, comes from the desire to analyse certain philosophical arguments, and thus qualify finely the truth of a proposition: for example a proposition may be false now but true later, or on the contrary true and necessarily so, and so on. What is now called modal logic describes the behaviour of the abstract modalities $\square$ and $\Diamond$, but covers a wide range of ‘real’ modalities in linguistic expressions: time, necessity, possibility, obligation, knowledge, belief, etc. Semantically, the modal operators and more precisely the $\Diamond$, allows to describe properties of the different states that you can reach from the evaluation point. 

Since the modal model theory began to develop, there is a tendency to use methods based on model theory instead of proof theory, because the classical proof systems were generally insufficient. However, in the last years, new techniques were developed, and proof systems began to be more familiar, that’s why they have certain advantages when it comes to analysis or standardization of proofs. Look at \cite{negri2005} to have more details of this dichotomy. Labelled deduction has been more generally proposed by Gabbay \cite{gabbay1996} in the 80’s as a unifying framework throughout proof theory in order to provide proof systems for a wide range of logics. For modal logics it can also take the form of labelled natural deduction and labelled sequent systems as used, for example, by Simpson \cite{simpson1994}, Vigano \cite{vigano2013}, and Negri \cite{negri2005}. These formalisms make explicit use not only of labels, but also of relational atoms that reference the accessibility relation with a Kripke model \cite{kripke1959}. 

We will continue with a choice of a sequent presentation. More precisely, in this work, the main goal consists of developing a labelled sequent system for intuitionistic modal logic, that comes with two relation symbols: one for the accessible world relation associated with the Kripke semantics for classical modal logics, and one for the preorder relation associated with the Kripke semantics for intuitionistic logic. To obtain this result, we used a labelled sequent calculus proposed by Negri \cite{negri2005} for classical modal logics and we extended it with a preorder relation. That allows to have a labelled system in close correspondence to the birelational Kripke models. 

\bigskip
\bigskip
\bigskip


\textbf{Palabras clave:} lógica modal, lógica intuicionista, cálculo de secuentes etiquetados, teoría de prueba, Gentzen, modelos bi-relacionales.


\textbf{Key words:} modal logic, intuitionistic logic, labelled sequent calculus, proof theory, Gentzen, bi-relational models.