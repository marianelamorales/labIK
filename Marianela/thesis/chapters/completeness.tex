\chapter{Completitud}
\label{cap:completeness}
%En esta sección demostramos completitud con respecto al sistema de Hilbert utilizando nuestro sistema de prueba etiquetado \textbf{labIK$\heartsuit$}.
Como se mencionó en el capítulo previo, necesitamos demostrar una propiedad importante de nuestro sistema que llamamos completitud. Intuitivamente, esta propiedad indica que nuestro sistema posee todo aquello que es necesario para demostrar las verdades de la lógica que caracteriza.

En esta tesis, para la demostración de completitud de $\labIKh$ seguiremos dos estrategias. La primera consiste en demostrar que con las reglas de $\labIKh$ podemos obtener todos los axiomas y reglas del sistema $\IK$. Dado que $\IK$ ya es completo, obtenemos la completitud de nuestro sistema. Por otro lado, demostraremos que nuestro sistema permite simular todas las reglas del sistema propuesto por Simpson \cite{simpson1994} que también es completo. Esta última estrategia además nos permite eliminar la regla de $\mathsf{cut}$. Veremos esto en mayor detalle en el Capítulo \ref{cap:future}.

En este capítulo seguiremos la primer estrategia, es decir, realizamos una demostración sintáctica en la cual las reglas de nuestro sistema son suficientes para la prueba de cada uno de los axiomas que necesitamos probar.

\begin{teo} Cada teorema de la lógica $\IK$ es demostrable en $\labIKh$.
\end{teo}
\emph{Demostración}. Como se dijo anteriormente, nuestra demostración demuestra completitud con respecto del sistema a la Hilbert. Para esta prueba necesitamos:
\begin{enumerate}
	\item Probar las variantes $\kaxiom_{1}$, ..., $\kaxiom_{5}$ del axioma de distributividad $\kaxiom$ presentados en el Capítulo \ref{cap:logintuicionista}.
	\item Demostrar todos los axiomas de la lógica modal proposicional.
	\item Simular modus ponens.
	\item Simular necesitación.
\end{enumerate}

Para llevar a cabo todo lo anterior, realizaremos nuestras demostraciones a partir del uso de las reglas de nuestro sistema $\labIKh$. De esta manera, como el sistema \emph{a la Hilbert} es completo, obtenemos completitud de nuestro sistema.

En primer lugar, veremos que los axiomas $\kaxiom_{1}$, ..., $\kaxiom_{5}$ se pueden demostrar en el sistema $\labIKh$ como se muestra a continuación:

\section{Axiomas modales}

\begin{lemma}
	Los axiomas $\kaxiom_{1}$, ..., $\kaxiom_{5}$ son demostrables en $\labIKh$
\end{lemma}

\paragraph{Axioma $\kaxiom_{1}$}
\kone


\paragraph{Axioma $\kaxiom_{2}$}
	\ktwo


\paragraph{Axioma $\kaxiom_{3}$}

\begin{center}

	$\vlderivation {
		\vlin{\sir}
		{}
		{\Rightarrow x \colon \Diamond (\vls[A.B]) \vljm (\vls[\Diamond A. \Diamond B ])}
		{\vlin {\sdl}
			{}
			{x \le y, y \colon \Diamond (\vls[A.B]) \Rightarrow y \colon \vls[\Diamond A. \Diamond B] }
			{\vliin{\solef}{}{x \le y, yRz, z \colon \vls[A.B] \Rightarrow y \colon \vls[\Diamond A. \Diamond B]}{\vlhy{\circledast \hspace{40mm}}}{\vlhy{\star}}}}
	}$
	
	\bigskip
	
\end{center}

	Luego de aplicar la regla de disjunción para el lado izquierdo, nuestra derivación continúa con dos premisas. Por un lado tenemos la premisa que se sigue de $\circledast$:
	
	\bigskip
	\begin{center}
	$\vlderivation{\vlin{\solef}{}{\circledast}{\vlin {\sorig}
			{}
			{x \le y, yRz, z \colon A \Rightarrow y \colon \vls[\Diamond A. \Diamond B]}
			{\vlin {\sdr}
				{}
				{x \le y, yRz, z \colon A \Rightarrow y \colon \Diamond A, y \colon \Diamond B}
				{\vlin {\refl}
					{}
					{x \le y, yRz, z \colon A \Rightarrow y \colon \Diamond A, z \colon A, y \colon \Diamond B}
					{\vlin {\ids}
						{}
						{x \le y, z \le z, yRz, z \colon A \Rightarrow y \colon \Diamond A, z \colon A, y \colon \Diamond B}
						{\vlhy {}}}}}}}$
	\end{center}	
		Y por otra parte, tenemos la derivación que continua a partir de $\star$:
		
		\begin{center}
	$\vlderivation{\vlin {\solef}
	{}
	{\star}
	{\vlin {\sorig}
		{}
		{x \le y, yRz, z \colon B \Rightarrow y \colon \vls[\Diamond A. \Diamond B]}
		{\vlin {\sdr}
			{}
			{x \le y, yRz, z \colon B \Rightarrow y \colon \Diamond A, y \colon \Diamond B}
			{\vlin {\refl}
				{}
				{x \le y, yRz, z \colon B \Rightarrow y \colon \Diamond A, y \colon \Diamond B, z \colon B}
				{\vlin {\ids}
					{}
					{x \le y, z \le z, yRz, z \colon B \Rightarrow y \colon \Diamond A, y \colon \Diamond B, z \colon B}
					{\vlhy {}}}}}}}$
\end{center}

\paragraph{Axioma $\kaxiom_{4}$}
	\kfour


\paragraph{Axioma $\kaxiom_{5}$}

\begin{center}
	
	$\vlderivation {
		\vlin{\sir}
		{}
		{\Rightarrow x \colon \Diamond \bot \vljm \bot}
		{\vlin {\sdl}
			{}
			{x \le y, y \colon \Diamond \bot \Rightarrow y \colon \bot}
			{\vlin {\sbot}
				{}
				{x \le y, yRz, z \colon \bot \Rightarrow y \colon \bot}
				{\vlhy {}}}}
	}$
	
\end{center}

\vspace{4mm}

\section{Axiomas proposicionales intuicionistas}
Para continuar con esta demostración, también debemos ver que todos los axiomas de la lógica proposicional intuicionista son demostrables en $\labIKh$. A continuación se puede observar una demostración para cada uno de los axiomas que fueron introducidos en el Capítulo 3. Continuamos con un método sintáctico de prueba en el que utilizamos las reglas de nuestro sistema.

\begin{lemma}
	Los axiomas la lógica proposicional intuicionista son demostrables en $\labIKh$.
\end{lemma}


\begin{center}
	\textbf{THEN-1}
	
	$\vlderivation{
		\vlin{\sir}
		{}
		{\Rightarrow x \colon A \vljm (B \vljm A)}
		{\vlin {\sir}
			{}
			{x \le y, y \colon A \Rightarrow y\colon B \vljm A}
			{\vlin {\ids}
				{}
				{x \le y, y \le z, y \colon A, z \colon B \Rightarrow z \colon A}
				{\vlhy {}}}}
	}$
	
\end{center}


\bigskip

\begin{center}
	\textbf{AND-1}
	
	$\vlderivation {
		\vlin{\sir}
		{}
		{\Rightarrow x \colon \vls(A.B) \vljm A}
		{\vlin {\svlef}
			{}
			{x \le y, y \colon \vls(A.B) \Rightarrow y \colon A}
			{\vlin {\refl}
				{}
				{x \le y, y \colon A, y \colon B \Rightarrow y \colon A}
				{\vlin {\ids}
					{}
					{x \le y, y \le y, y \colon A, y \colon B \Rightarrow y \colon A}
					{\vlhy {}}}}}
	}$
\end{center}

\bigskip

\begin{center}
	\textbf{AND-2}
	
	$\vlderivation {
		\vlin{\sir}
		{}
		{\Rightarrow x \colon \vls(A.B) \vljm B}
		{\vlin {\svlef}
			{}
			{x \le y, y \colon \vls(A.B) \Rightarrow y \colon B}
			{\vlin {\refl}
				{}
				{x \le y, y \colon A, y \colon B \Rightarrow y \colon B}
				{\vlin {\ids}
					{}
					{x \le y, y \le y, y \colon A, y \colon B \Rightarrow y \colon B}
					{\vlhy {}}}}}
	}$
\end{center}

\bigskip

\begin{center}
	\textbf{AND-3}
	
	$\vlderivation{
		\vlin{\sir}
		{}
		{\Rightarrow x \colon A \vljm (B \vljm (\vls(A.B)))}
		{\vlin {\sir}
			{}
			{x \le y, y \colon A \Rightarrow y \colon B \vljm (\vls(A.B))}
			{\vliin {\svrig}
				{}
				{x \le y, y \le z, y \colon A, z \colon B \Rightarrow z \colon \vls(A.B)}
				{\vlin {\ids}
					{}
					{x \le y, y \le z, y \colon A, z \colon B \Rightarrow z \colon A}
					{\vlhy {}}}
				{\vlin {\refl}
					{}
					{x \le y, y \le z, y \colon A, z \colon B \Rightarrow z \colon B}
					{\vlin {\ids}
						{}
						{x \le y, y \le z, z \le z, y \colon A, z \colon B \Rightarrow z \colon B}
						{\vlhy {}}}}}}
	}$
	
\end{center}

\bigskip

\begin{center}
	\textbf{OR-1}
	
	$\vlderivation {
		\vlin{\sir}
		{}
		{\Rightarrow x \colon A \vljm \vls[A.B]}
		{\vlin {\sorig}
			{}
			{x \le y, y \colon A \Rightarrow y \colon \vls[A.B]}
			{\vlin {\refl}
				{}
				{x \le y, y \colon A \Rightarrow y \colon A, y \colon B}
				{\vlin {\ids}
					{}
					{x \le y, y \le y, y \colon A \Rightarrow y \colon A, y \colon B}
					{\vlhy {}}}}}
	}$
	
\end{center}

\bigskip

\begin{center}
	\textbf{OR-2}
	
	$\vlderivation {
		\vlin{\sir}
		{}
		{\Rightarrow x \colon B \vljm \vls[A.B]}
		{\vlin {\sorig}
			{}
			{x \le y, y \colon B \Rightarrow y \colon \vls[A.B]}
			{\vlin {\refl}
				{}
				{x \le y, y \colon B \Rightarrow y \colon A, y \colon B}
				{\vlin {\ids}
					{}
					{x \le y, y \le y, y \colon B \Rightarrow y \colon A, y \colon B}
					{\vlhy {}}}}}
	}$
	
\end{center}


\bigskip

\begin{center}
	\textbf{OR-3}
\end{center}
\orthree

\bigskip

\begin{center}
	\textbf{FALSE}
	
	$\vlderivation {
		\vlin{\sir}
		{}
		{\Rightarrow x \colon \bot \vljm A}
		{\vlin {\sbot}
			{}
			{x \le y, y \colon \bot \Rightarrow y \colon A}
			{\vlhy {}}}
	}$
	
\end{center}

\vspace{3mm}

\begin{center}
	\textbf{THEN-2}
\end{center}
\thentwo


\vspace{3mm}

\section{Simulando Modus Ponens}
Una vez concluida esta demostración para cada uno de los axiomas de la lógica proposicional intuicionista, seguimos adelante con el resto de nuestra demostración para poder probar la completitud de nuestro sistema $\labIKh$. Para ello, continuamos mostrando que podemos simular de modus ponens utilizando \emph{la regla de cut} donde $\mathsf{cut}$ es:

\begin{center}
	
	$\vliinf{\mathsf{cut}}{}{\Gone, \Gtwo, \Left \Rightarrow \Right}{\Gone, \Left \Rightarrow \Right, z \colon C}{\Gtwo, \Left, z \colon C \Rightarrow \Right}$
	
\end{center}

Es así que utilizando $\mathsf{cut}$ vemos que la regla de modus ponens $\vliinf{}{}{B}{A}{A \vljm B}$ es simulada de la siguiente manera:

\vspace{3mm}


\begin{center}
		
		$\vlderivation {
			\vliin{\cut}
			{}
			{\Rightarrow x \colon B}
			{\vlin {w}
				{}
				{\Rightarrow x \colon A, x \colon B}
				{\vlhy {\Rightarrow x \colon A}}}
			{\vliin {\mathsf{cut}}
				{}
				{x \colon A \Rightarrow x \colon B}
				{\vlhy{\circledast}}
				{\vlhy{\star}}}
		}$
\end{center}

\bigskip
	
	Hemos representado con $\circledast$ y $\star$ la continuación de esta derivación en donde de $\circledast$ se sigue:
	
	\bigskip
	
	\begin{center}
	$\vlderivation{\vlin{\mathsf{cut}}{}{\circledast}{\vlin {\mathsf{w}}
			{}
			{x \colon A \Rightarrow x \colon B, x \colon A \vljm B}
			{\vlhy {\Rightarrow x \colon A \vljm B}}}}$
	
\end{center}

Mientras que de $\star$ se obtiene:

\begin{center}
	$\vlderivation{\vlin{\mathsf{cut}}{}{\star}{\vlin {\refl}
			{}
			{x \colon A, x \colon A \vljm B \Rightarrow x \colon B}
			{\vliin {\sil}
				{}
				{x \le x, x \colon A, x \colon A \vljm B \Rightarrow x \colon B}
				{\vlin {\ids}
					{}
					{x \le x, x \colon A, x \colon A \vljm B \Rightarrow x \colon B, x \colon A}
					{\vlhy {}}}
				{\vlin {\ids}
					{}
					{x \le x, x \colon A, x \colon A \vljm B, x \colon B \Rightarrow x \colon B}
					{\vlhy {}}}}}}$
\end{center}

Finalmente, queda demostrado que pudimos simular la regla de modus ponens a partir del uso de las reglas de secuentes que conforman el sistema $\labIKh$ y de la regla de $\mathsf{cut}$.

\section{Simulando Necesitación}

Una vez demostrado que podemos simular modus ponens utilizando la regla cut y las reglas de nuestro sistema, nos resta ver que podemos simular la regla de necesitación donde \emph{necesitación} es: Si existe una prueba de $A$, entonces existe una prueba de $\square A$.

En otras palabras, queremos probar que:

\begin{lemma}
	Si existe una prueba de $\vlderivation {\vlpd{\Done}{}{\Rightarrow z \colon A}}$ entonces existe una prueba de $\vlderivation { \vlpd{\Dtwo}{}{\Rightarrow x \colon \square A}}$.
\end{lemma}

\begin{proof}
	
	Podemos mostrar que:

		
		Si $\vlderivation {\vlpd {\mathcal{D}_{1}}{}{\Rightarrow z \colon A}}$ \hspace{4mm} entonces tenemos que \hspace{4mm} $\vlderivation{\vlin{\sbr}{}{\Rightarrow x \colon \square A}{\vlpd {\Dwone}{}{x \le y, yRz \Rightarrow z \colon A}}}$
		

	\bigskip

		     donde llamaremos $\Dtwo$ a lo siguiente: $\Dtwo = \vlderivation {\vlpd {\Dwone}{}{x \le y, yRz \Rightarrow z \colon A}}$ 
	\bigskip
		
Luego, utilizando la regla de debilitamiento (o en inglés \emph{weakening}) tenemos que: 

\bigskip
\begin{center}
	$\vlderivation{\vlin{\sbr}{}{\Rightarrow x \colon \square A}{\vlin{\mathsf{w}}{}{x \le y, yRz \Rightarrow z \colon A}{\vlhy{ \Rightarrow z \colon A}}}}$
\end{center}


	
\end{proof}

A continuación veremos la demostración de otros lemas que nos permitieron realizar las distintas demostraciones de las reglas para probar completitud.

En el Lema \ref{lemaw} demostramos la admisibilidad de la regla de debilitamiento o \emph{weakening} que fue ya utilizada para simular la regla de necesitación.

\begin{lemma}
	\label{lemaw}
	Si existe una prueba $\vlderivation {\vlpd{\D}{}{\G, \Left \Rightarrow \Right}}$ entonces existe una prueba \break $\vlderivation {\vlpd {\Dw}{}{\G, xRy, u \le v, \Left, z \colon A \Rightarrow \Right, w \colon B}}$
	
\end{lemma}

\vspace{3mm}

\begin{proof}
	Por induccion en la altura de $\D$.
	
	Para una prueba de altura 1:
	
	Si $\D$ = $\vlderivation {\vlin{\ids}{}{\G, \Left, x \le y, x \colon a \Rightarrow \Right, y \colon a}{\vlhy {}}}$ entonces tenemos que:
	
		\hspace{40mm}$\Dw$ = $\vlderivation {\vlin{\ids}{}{\G, \Left, x \le y, x$R$y, u \le v, x \colon a, z \colon A \Rightarrow \Right, w \colon B, y \colon a}{\vlhy {}}}$.
		
	\vspace{3mm}
	
	Para una prueba en la altura de $\D$ mayor a 1:
	
	\begin{center}
		
		$\vlderivation{\vlin {$r$}{}{\G, \Left \Rightarrow \Right}{\vlpd {\Done}{}{\G', \Left' \Rightarrow \Right'}}}$
		
	\end{center}
	
	Luego, por hipótesis inductiva existe una prueba
	\begin{center}
		
		$\vlderivation {\vlpd {\Dwone}{}{\G', xRy, u \le v, \Left', z \colon A \Rightarrow \Right', w \colon B}}$
		
	\end{center}
	
	Por lo tanto, tenemos la prueba
	
	\begin{center}
		
		$\Dw = \vlderivation {\vlin{}{}{\G, xRy, u \le v, \Left, z \colon A \Rightarrow \Right, w \colon B}{\vlpd {\Dwone}{}{\G', xRy, u \le v, \Left', z \colon A \Rightarrow \Right', w \colon B}}}$
		
	\end{center}
	
\end{proof}

A lo largo de nuestras pruebas sintácticas, hemos utilizado la regla de identidad para las fórmulas. Ya que $\labIKh$ presenta únicamente la regla de identidad para átomos, en el siguiente lema vemos la admisibilidad de la regla para el caso general en nuestro sistema. 

\begin{lemma}
	La siguiente regla es admisible en $\labIKh$:
	
	\begin{center}
		
		$\vlderivation {\vlin{\idg}{}{\G, x \le y,  \Left, x \colon A \Rightarrow \Right, y \colon A }{\vlhy {}}}$
		
	\end{center}
	
\end{lemma}

\vspace{3mm}

\begin{proof}  Por inducción en el tamaño de $A$.
	
	\begin{itemize}
		\item{$A=a$}: 
		
		Tenemos que la regla $\vlderivation {\vlin{\idg}{}{\G, x \le y,  \Left, x \colon a \Rightarrow \Right, y \colon a }{\vlhy {}}}$ es la misma regla que forma parte del sistema $\labIKh$:  $\vlderivation {\vlin{\ids}{}{\G, x \le y,  \Left, x \colon a \Rightarrow \Right, y \colon a }{\vlhy {}}}$ 
		
		\item{$A= \vls(A.B)$:}
		
		\begin{center}
			$\vlderivation{\vliin{\svrig}{}{\G, x \le y, x\colon \vls(A.B), \Left \Rightarrow \Right, y \colon \vls(A.B)}{\vlin{\svlef}{}{\G, x \le y, x\colon \vls(A.B), \Left \Rightarrow \Right, y \colon A}{\vlin {}{}{\G, x \le y, x\colon A,x \colon B, \Left \Rightarrow \Right, y \colon A}{\vlhy {$Por hipótesis inductiva, tamaño($A$)$\le n$$}}}}{\vlin{\svlef}{}{\G, x \le y, x\colon \vls(A.B), \Left \Rightarrow \Right, y \colon B}{\vlin {}{}{\G, x \le y, x\colon A,x \colon B, \Left \Rightarrow \Right, y \colon B}{\vlhy {$Por hipótesis inductiva, tamaño($B$)$\le n}}}}}$
		\end{center}
		\item{$A = \vls[A.B]$:}
		\begin{center}
			$\vlderivation { \vlin{\sorig}{}{\G, x\le y, x \colon \vls[A.B], \Left \Rightarrow \Right, y \colon \vls[A.B]}{\vliin {\solef}{}{\G, x\le y, x \colon \vls[A.B], \Left \Rightarrow \Right, y \colon A, y \colon B}{\vlin {}{}{\G, x\le y, x \colon A, \Left \Rightarrow \Right, y \colon A, y \colon B}{\vlhy {$Por hipótesis inductiva, tamaño($A$)$ \le n}}}{\vlin {}{}{\G, x\le y, x \colon B, \Left \Rightarrow \Right, y \colon A, y \colon B}{\vlhy {$Por hipótesis inductiva, tamaño($B$)$\le n}}}}}$
		\end{center}
		\item{$A = \square A$}
		\begin{center}
			$\vlderivation {\vlin {\sbr}
				{}
				{\G, x \le y, \Left, x \colon \square A \Rightarrow \Right, y \colon \square A}{\vlin {\sbl}
					{}
					{\G,\Left, x \le y, x \le z, zRw, x \colon \square A \Rightarrow \Right, w \colon A }
					{\vlin {\refl}
						{}
						{\G,\Left, x \le y, x \le z, zRw, x \colon \square A, w \colon A \Rightarrow \Right, w \colon A }
						{\vlin {}
							{}
							{\G,\Left, x \le y, x \le z, w \le w, zRw, x \colon \square A, w \colon A \Rightarrow \Right, w \colon A }
							{\vlhy {$Por hipótesis inductiva, tamaño($A$)$\le n}}}}}}$
		\end{center}
		\item{$A= \Diamond A$}
		\begin{center}
			$\vlderivation{\vlin {\sdl}
				{}
				{\G, \Left, x \le y, x \colon \Diamond A \Rightarrow \Right, y \colon \Diamond A}
				{\vlin {\ftwo}
					{}
					{\G, \Left, x \le y, xRz, z \colon A \Rightarrow \Right, y \colon \Diamond A}
					{\vlin {\sdr}
						{}
						{\G, \Left, x \le y, z \le u, xRz, yRu, z \colon A \Rightarrow \Right, y \colon \Diamond A}
						{\vlin {}
							{}
							{\G, \Left, x \le y, z \le u, xRz, yRu, z \colon A \Rightarrow \Right, y \colon \Diamond A, u \colon A}
							{\vlhy {$Por hipótesis inductiva, tamaño($B$)$\le n}}}}}}$
		\end{center}
	\end{itemize}
\end{proof}

%%\chapter{Completitud utilizando el sistema de Simpson}
\section{Completitud utilizando el sistema de Simpson}
El objetivo de esta sección, es discutir una demostración alternativa del resultado de completitud. Para ello utilizaremos el sistema de Simpson \cite{simpson1994} que se puede observar en la Figura \ref{fig:simpson}. En la sección anterior se realizó una prueba de completitud sintáctica para el nuevo sistema utilizando la regla de $\mathsf{cut}$ para simular modus ponens. Esto quiere decir que pudimos demostrar \emph{completitud con $\mathsf{cut}$} para nuestro sistema. Lo que deseamos lograr ahora, es tener un sistema completo sin $\mathsf{cut}$. Si bien se puede realizar una prueba conocida como \emph{cut elimination}, decidimos resolver este problema a partir de saber que el sistema propuesto por Simpson ya es un sistema libre de $\mathsf{cut}$. Como podemos observar en la Figura \ref{fig:simpson}, la mayoría de las reglas de secuentes que se encuentran en el sistema de Simpson (lo llamamos $\labIKs$) también están presentes en el sistema $\labIKh$, por lo que nos resta ver que las reglas que son distintas o bien no se encuentran en nuestro sistema, pueden ser deducidas a partir de la aplicación de reglas de secuentes del sistema $\labIKh$. Realizando este análisis a cada una de las reglas con éxito, podemos concluir que nuestro sistema $\labIKh$ es completo sin $\mathsf{cut}$.

\begin{teo}
	El sistema $\labIKs$ es completo sin la regla de $\mathsf{cut}$.
\end{teo}

\begin{figure}[h]
	\begin{center}
		$\vlderivation { \vlin {\idsim}{}{\G, \Left, x \colon a \Rightarrow x\colon a}{\vlhy {}}}$ \hspace{9mm} $\vlderivation { \vlin {\sbot}{}{\G, \Left, x \colon \bot \Rightarrow z\colon A}{\vlhy {}}}$
		
		\vspace{4mm}
		$\vlderivation {\vlin {\svlefs}{}{\G, \Left, x \colon \vls(A.B) \Rightarrow z \colon C}{\vlhy {\G, \Left, x \colon A, x \colon B \Rightarrow z \colon C}}}$
		\hspace{9mm}$\vlderivation { \vliin {\svrigs}{}{\G, \Left, \Rightarrow x \colon \vls(A.B)}{\vlhy {\G, \Left \Rightarrow x \colon A }}{\vlhy {\G, \Left \Rightarrow x \colon B}}}$
		
		\vspace{4mm}
		$\vlderivation {\vliin {\solefs}{}{\G, \Left, x \colon \vls[A.B] \Rightarrow z \colon C}{\vlhy {\G, \Left, x \colon A \Rightarrow z \colon C}}{\vlhy {\G, \Left, x \colon B \Rightarrow z \colon C}}}$
		\hspace{5mm}$\vlderivation { \vlin{\sorones}{}{\G, \Left \Rightarrow x \colon \vls[A.B]}{\vlhy {\G, \Left \Rightarrow x \colon A}}}$
		\hspace{5mm}$\vlderivation { \vlin {\sotwos}{}{\G, \Left \Rightarrow x \colon \vls[A.B]}{\vlhy {\G, \Left \Rightarrow x \colon B}}}$
		
		\vspace{4mm}
		$\vlderivation {\vliin{\sils}{}{\G, \Left, x \colon A \vljm B \Rightarrow z \colon C}{\vlhy {\G, \Left \Rightarrow x \colon A}}{\vlhy {\G, \Left, x \colon B \Rightarrow z \colon C}}}$
		\hspace{7mm}$\vlderivation {\vlin{\sirs}{}{\G,  \Left, x \colon A \Rightarrow x \colon B}{\vlhy {\G, \Left, x \colon A \Rightarrow x \colon B}}}$
		
		\vspace{4mm}
		$\vlderivation { \vlin {\sbls}{}{\G, xRy, \Left, x \colon \square A \Rightarrow z\colon B}{\vlhy {\G, xRy, \Left, x \colon \square A, y \colon A \Rightarrow z\colon B}}}$
		\hspace{23mm}$\vlderivation { \vlin {\sbrs}{y$ fresh$}{\G, \Left \Rightarrow x \colon \square A}{\vlhy {\G, xRy, \Left \Rightarrow y \colon A}}}$
		
		\vspace{4mm}
		$\vlderivation { \vlin{\sdls}{y$ fresh$}{\G, \Left, x \colon \Diamond A \Rightarrow z \colon B}{\vlhy {\G, xRy, \Left, y \colon A \Rightarrow z \colon B}}}$
		\hspace{7mm}$\vlderivation {\vlin {\sdrs}{}{\G,xRy, \Left \Rightarrow x \colon \Diamond A}{\vlhy {\G, xRy, \Left \Rightarrow y \colon A }}}$
	\end{center}
	\caption{System $\labIKs$}
	\label{fig:simpson}
\end{figure}

%\raul{Aca mencionar que es solo un sketch, y decir en una frase qué falta}
Lo siguiente representa un boceto de la prueba que se continuará como trabajo a futuro. Faltan incluir las demostraciones para la regla de $\sirs$ y para $\sbrs$.

\newpage
\begin{proof}
	La demostración de completitud utilizando el sistema propuesto por Simpson, como se detalló rápidamente al comienzo de esta sección, será a través de
	 análisis de casos. La mayoría de las reglas en $\labIKs$ son las mismas reglas que en el sistema $\labIKh$ excepto por las siguientes:
	\begin{center}
		\begin{itemize}
		
		\item Regla $\idsim$:
		
		\begin{center}
		$\vlderivation { \vlin {\idsim}{}{\G, \Left, x \colon A \Rightarrow x \colon a}{\vlhy {}}}$ \hspace{4mm}  \begin{huge}$  \rightarrow$ \end{huge} \hspace{4mm} $\vlderivation{\vlin {\refl}{}{\G, \Left, x \colon a \Rightarrow x \colon a}{\vlin {\ids}{}{\G,x \le x, \Left, x\colon a \Rightarrow x \colon a}{\vlhy {}}}}$
	\end{center}
		 
		
		\vspace{5mm}
		\item Regla $\sorones$ y $\sotwos$: 
		
		\begin{center}
		$\vlderivation {\vlin {\sorones}{}{\G, \Left \Rightarrow x \colon \vls[A.B]}{\vlpd {\Done}{}{\G, \Left \Rightarrow x \colon A}}}$\hspace{2mm} or \hspace{2mm}$\vlderivation {\vlin {\sotwos}{}{\G, \Left \Rightarrow x \colon \vls[A.B]}{\vlpd {\Done}{}{\G, \Left \Rightarrow x \colon B}}}$ \hspace{4mm} \begin{huge}$\rightarrow$\end{huge} \hspace{4mm} $\vlderivation {\vlin {\sorig}{}{\G, \Left \Rightarrow x \colon \vls[A.B]}{\vlpd{\Dwone}{}{\G, \Left \Rightarrow x \colon A, x \colon B}}}$
	\end{center}
	
		\newpage
		
		\item Regla $\sils$:
		
		\begin{center}
		$\vlderivation { \vliin {\sils}{}{\G, \Left, x \colon A \vljm B \Rightarrow z \colon C}{\vlpd {\Done}{}{\G, \Left \Rightarrow x \colon A}}{\vlpd {\Dtwo}{}{\G, \Left, x \colon B \Rightarrow z \colon C}}}$ \bigskip 
		
			\begin{huge}$\downarrow$\end{huge}\\
		
		 \bigskip
		 
		  $\vlderivation {\vlin {\refl}{}{\G, \Left, x \colon A \vljm B \Rightarrow z \colon C}{\vliin {\sil}{}{\G, \Left, x\le x, x \colon A \vljm B \Rightarrow z \colon C}{\vlpd {\Dwone}{}{\G, \Left, x \le x, x \colon A \vljm B \Rightarrow x\colon A}}{\vlpd {\Dwtwo}{}{\G, \Left, x \le x, x \colon B \Rightarrow z \colon C}}}}$
	\end{center}
		%\vspace{5mm}
		%\item Regla $\sirs$:
		
		%\begin{center}
		%$\vlderivation {\vlin {\sirs}{}{\G, \Left \Rightarrow x \colon A \vljm B}{\vlpd {\Done}{}{\G, \Left, x \colon A \Rightarrow x \colon B}}}$ \hspace{4mm} \begin{huge}$\rightarrow$\end{huge} \hspace{4mm} $\vlderivation {\vlin {\sir}{}{\G, \Left \Rightarrow x \colon A \vljm B}{\vlpd {\Dwone}{}{\G, \Left, x \le x, x \colon A \Rightarrow x \colon B}}}$
		%\end{center}
		
		\vspace{5mm}
		\item Regla $\sbls$:
		
		\begin{center}
		$\vlderivation {\vlin {\sbls}{}{\G, \Left, xRy, x \colon \square A \Rightarrow z \colon B}{\vlpd {\Done}{}{\G, \Left, xRy, x \colon \square A, y \colon A \Rightarrow z \colon B}}}$ \bigskip
		
			\begin{huge}$\downarrow$\end{huge}\\
		
		\bigskip
		
		 $\vlderivation {\vlin {\refl}{}{\G, \Left, xRy, x \colon \square A \Rightarrow z \colon B}{\vlin {\sbl}{}{\G, \Left, x \le x, xRy, x \colon \square A \Rightarrow z \colon B}{\vlpd {\Dwone}{}{\G, \Left, x \le x, xRy, x \colon \square A, y \colon A \Rightarrow z \colon B}}}}$
	\end{center}
	
		%\vspace{5mm}
		%\item Regla $\sbrs$:
		
		%\begin{center}
		%$\vlderivation {\vlin {\sbrs}{}{\G, \Left \Rightarrow x \colon \square A}{\vlpd {\Done}{}{\G, \Left, xRy \Rightarrow y \colon A}}}$ \hspace{4mm} \begin{huge}$\rightarrow$ \end{huge} \hspace{4mm} $\vlderivation {\vlin {\sbr}{}{\G, \Left \Rightarrow x \colon \square A}{\vlpd {\Dwone}{}{\G, \Left, x \le x, xRy \Rightarrow y \colon A}}}$
	%\end{center}
	
		\vspace{5mm}
		\item Regla $\sdrs$:
		
		\begin{center}
		$\vlderivation {\vlin {\sdrs}{}{\G, xRy, \Left \Rightarrow x \colon \Diamond A}{\vlpd {\Done}{}{\G, xRy, \Left \Rightarrow y \colon A}}}$ \hspace{4mm} \begin{huge}$\rightarrow$ \end{huge} \hspace{4mm} $\vlderivation {\vlin {\sdr}{}{\G, xRy, \Left \Rightarrow x \colon \Diamond A}{\vlpd {\Dwone}{}{\G, xRy, \Left \Rightarrow x \colon \Diamond A, y \colon A}}}$
	\end{center}
	
		\end{itemize}
	\end{center}
\end{proof}

Nuestra conjetura es que es posible demostrar completitud del sistema $\labIKh$ haciendo un análisis de cada una de las reglas de secuentes presentes en el sistema de Simpson (a partir del uso de las reglas de nuestro sistema).


