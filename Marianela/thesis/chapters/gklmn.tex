%\chapter{Sistema $\labIKh$ + $\agklmn$}
\section{Extensiones del sistema $\labIKh$}

%Esta sección hace referencia a trabajo futuro por terminar. 
La idea parte de generar lógicas más fuertes extendiendo nuestro sistema con otros axiomas. Decimos \emph{una lógica más fuerte} para hacer referencia a que estamos restringiendo la clase de frames que queremos considerar, imponiendo algunas restricciones en la relación de accesibilidad. En particular, nuestro objetivo es realizar esto utilizando el \emph{axioma de Scott-Lemmon} o \emph{axioma $\agklmn$}:
\begin{center}
	 $\lozenge^{k} \square^{l} A \vljm \square^{m}\lozenge^{n} A$
\end{center}

Escribiendo nuestro axioma tanto en la lógica clásica como en la intuicionista en lenguaje de primer orden obtenemos lo siguiente:
 
$\star$ \emph{Caso clásico} \hspace{14mm} $\rightsquigarrow$ \hspace{3.7mm}$\forall x,y,z ( xR^{k}y \vlan xR^{m}z \rightarrow \exists u yR^{l}u \vlan zR^{n}u)$ 

$\star$ \emph{Caso intuicionista} \hspace{3mm} $\rightsquigarrow$ \hspace{3mm} $\forall x,y,z((xR^{k}y \vlan xR^{m}z) \vljm \exists y' (y \le y' \vlan \exists u (y'R^{l}u \vlan zR^{n} u)))$\\

Para entenderlo en mayor detalle podemos observar la Figura \ref{fig:gklmn} que representa al axioma de Scott-Lemmon para el caso intuicionista la cual nos dice que:

\begin{quote}
	Sean $w$, $u$ y $v$ mundos (o etiquetas). Si $wR^k u$ y $wR^m v$ entonces existe un mundo (o etiqueta) $u'$ tal que $u \le u'$ y existe un mundo (o etiqueta) $x$ tal que $u'R^l x$ y $vR^n x$.
\end{quote}

\begin{figure}[h]
	\begin{center}
		$
		\xymatrix{
			& u' \ar@{.>}[ddr]^{R^l} \\
			& u \ar@{.>}[u]^{\le} \\
			w \ar@{->}[ur]^{R^k}\ar@{->}[dr]_{R^m} && x \\
			& v \ar@{.>}[ur]_{R^n}
		}
		$
	\end{center}
	\label{fig:gklmn}
	\caption{Axioma $\agklmn$ para el caso intuicionista}
\end{figure}




Ahora añadimos el axioma $\agklmn$ ($\agklmn$ para el caso intuicionista) y creamos una nueva regla de sequente para nuestro sistema $\labIKh$ con el objetivo de capturar el axioma en cuestión:

\begin{center}
	$\vlderivation { \vlin {\gklmn}{y', u$ fresh$}{\G, xR^{k}y, xR^{m}z, \Left \Rightarrow \Right}{\vlhy {\G, y \le y', xR^{k}y, xR^{m}z, y'R^{l}u, zR^{n}u, \Left\Rightarrow \Right}}}$
\end{center}
%\section{Completitud}

\begin{teo}
El sistema $\labIKh$ $\mathsf{+}$ $\mathsf{ axioma}$ $\agklmn$	es completo para k = l = m = n = $\mathsf{1}$.
\end{teo}

\begin{proof}
	Como resultado de la sección \emph{Completitud Sintáctica} sabemos que $\labIKh$+$\mathsf{cut}$ es completo. Por lo tanto, sólo necesitamos mostrar que el axioma $\agklmn$ se puede probar a partir de las reglas que componen a nuestro sistema.
	Primero demostraremos completitud para \emph{k = l = m = n = $\mathsf{1}$} como se ve a continuación:
	\begin{center}
		\scalebox{0.93}{
		$\vlderivation {\vlin {\sir}
			{y$ fresh$}
			{\Rightarrow x \colon \lozenge \square A \vljm \square \lozenge A}
			{\vlin {\sbr}
				{z, w $ fresh $}
				{x \le y, y \colon \lozenge \square A \Rightarrow y \colon \square \lozenge A}
				{\vlin {\sdl}
					{u$ fresh$}
					{x \le y, y \le z, zRw, y \colon \lozenge \square A \Rightarrow w \colon \lozenge A}
					{\vlin {\ftwo}
						{t$ fresh$}
						{x \le y, y \le z, zRw, yRu, u \colon \square A \Rightarrow w \colon \lozenge A}
						{\vlin {\gklmn}
							{t', j$ fresh$}
							{x \le y, y \le z, u \le t, zRw, yRu, zRt, u \colon \square A \Rightarrow w \colon \lozenge A}
							{\vlin {\sdr}
								{}
								{x \le y, y \le z, u \le t, t \le t', zRw, yRu, zRt, t'Rj, wRj, u \colon \square A \Rightarrow w \colon \lozenge A}
								{\vlin {\trans}
									{}
									{x \le y, y \le z, u \le t, t \le t', zRw, yRu, zRt, t'Rj, wRj, u \colon \square A \Rightarrow w \colon \lozenge A, j \colon A}
									{\vlin {\sbl}
										{}
										{x \le y, y \le z, u \le t, t \le t', u \le t', zRw, yRu, zRt, t'Rj, wRj, u \colon \square A \Rightarrow w \colon \lozenge A, j \colon A}
										{\vlin {\refl}
											{}
											{x \le y, y \le z, u \le t, t \le t', u \le t', zRw, yRu, zRt, t'Rj, wRj, u \colon \square A, j \colon A \Rightarrow w \colon \lozenge A, j \colon A}
											{\vlin {\ids}
												{}
												{x \le y, y \le z, u \le t, t \le t', u \le t',j \le j, zRw, yRu, zRt, t'Rj, wRj, u \colon \square A, j \colon A \Rightarrow w \colon \lozenge A, j \colon A}
												{\vlhy {}}}}}}}}}}}}$}
	\end{center}
\end{proof}

Para la demostración de completitud del caso general, necesitamos introducir las reglas que se presentan en el Lema~\ref{lemma:admis}. 

\begin{lemma}\label{lemma:admis} Las siguientes reglas son admisibles en $\labIKh$:
	\begin{enumerate}
		\item{$\vlderivation {\vlin {\boxlk}{}{\G, \Left, x(\le \circ $R$)^{k}y, x \colon \square^{k} A\Rightarrow \Right}{\vlhy {\G, \Left, x(\le \circ $R$)^{k}y, x \colon \square^{k} A, z \colon A \Rightarrow \Right}}}$}
		\item{$\vlderivation{\vlin {\boxk}{}{\G, \Left \Rightarrow \Right, x \colon \square^{k} A}{\vlhy {\G, x(\le \circ $R$)^{k}y,\Left \Rightarrow \Right, y \colon A}}}$}
		\item{$\vlderivation { \vlin {\diamk}{}{\G, \Left, x \colon \lozenge^{k} A \Rightarrow \Right}{\vlhy {\G, xR^{k}y, \Left, y \colon A \Rightarrow \Right}}}$ }
		\item{$\vlderivation { \vlin {\diamrk}{}{\G, \Left, xR^{k}y \Rightarrow \Right, x \colon \lozenge^{k}A}{\vlhy {\G, \Left, xR^{k}y \Rightarrow \Right, x \colon \lozenge^{k}A, y \colon A}}}$}
	\end{enumerate}
	%\textbf{Probably F2g???}
\end{lemma}

Nuestra conjetura es que es posible demostrar completitud del sistema para el caso general, haciendo una demostración por inducción en el parámetro $k$. Esta demostración se deja como trabajo futuro.

% \begin{proof}
% 	Por inducción en k.
% 	\emph{\textbf{To be continued}}
% \end{proof}

