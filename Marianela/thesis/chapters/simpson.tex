%\chapter{Completitud utilizando el sistema de Simpson}
\section{Completitud utilizando el sistema de Simpson}
El objetivo de esta sección, es discutir una demostración alternativa del resultado de completitud. Para ello utilizaremos el sistema de Simpson \cite{simpson1994} que se puede observar en la Figura \ref{fig:simpson}. En la sección anterior se realizó una prueba de completitud sintáctica para el nuevo sistema utilizando la regla de $\mathsf{cut}$ para simular modus ponens. Esto quiere decir que pudimos demostrar \emph{completitud con $\mathsf{cut}$} para nuestro sistema. Lo que deseamos lograr ahora, es tener un sistema completo sin $\mathsf{cut}$. Si bien se puede realizar una prueba conocida como \emph{cut elimination}, decidimos resolver este problema a partir de saber que el sistema propuesto por Simpson ya es un sistema libre de $\mathsf{cut}$. Como podemos observar en la Figura \ref{fig:simpson}, la mayoría de las reglas de secuentes que se encuentran en el sistema de Simpson (lo llamamos $\labIKs$) también están presentes en el sistema $\labIKh$, por lo que nos resta ver que las reglas que son distintas o bien no se encuentran en nuestro sistema, pueden ser deducidas a partir de la aplicación de reglas de secuentes del sistema $\labIKh$. Realizando este análisis a cada una de las reglas con éxito, podemos concluir que nuestro sistema $\labIKh$ es completo sin $\mathsf{cut}$.

\begin{teo}
	El sistema $\labIKs$ es completo sin la regla de $\mathsf{cut}$.
\end{teo}

\begin{figure}[h]
	\begin{center}
		$\vlderivation { \vlin {\idsim}{}{\G, \Left, x \colon a \Rightarrow x\colon a}{\vlhy {}}}$ \hspace{9mm} $\vlderivation { \vlin {\sbot}{}{\G, \Left, x \colon \bot \Rightarrow z\colon A}{\vlhy {}}}$
		
		\vspace{4mm}
		$\vlderivation {\vlin {\svlefs}{}{\G, \Left, x \colon \vls(A.B) \Rightarrow z \colon C}{\vlhy {\G, \Left, x \colon A, x \colon B \Rightarrow z \colon C}}}$
		\hspace{9mm}$\vlderivation { \vliin {\svrigs}{}{\G, \Left, \Rightarrow x \colon \vls(A.B)}{\vlhy {\G, \Left \Rightarrow x \colon A }}{\vlhy {\G, \Left \Rightarrow x \colon B}}}$
		
		\vspace{4mm}
		$\vlderivation {\vliin {\solefs}{}{\G, \Left, x \colon \vls[A.B] \Rightarrow z \colon C}{\vlhy {\G, \Left, x \colon A \Rightarrow z \colon C}}{\vlhy {\G, \Left, x \colon B \Rightarrow z \colon C}}}$
		\hspace{5mm}$\vlderivation { \vlin{\sorones}{}{\G, \Left \Rightarrow x \colon \vls[A.B]}{\vlhy {\G, \Left \Rightarrow x \colon A}}}$
		\hspace{5mm}$\vlderivation { \vlin {\sotwos}{}{\G, \Left \Rightarrow x \colon \vls[A.B]}{\vlhy {\G, \Left \Rightarrow x \colon B}}}$
		
		\vspace{4mm}
		$\vlderivation {\vliin{\sils}{}{\G, \Left, x \colon A \vljm B \Rightarrow z \colon C}{\vlhy {\G, \Left \Rightarrow x \colon A}}{\vlhy {\G, \Left, x \colon B \Rightarrow z \colon C}}}$
		\hspace{7mm}$\vlderivation {\vlin{\sirs}{}{\G,  \Left, x \colon A \Rightarrow x \colon B}{\vlhy {\G, \Left, x \colon A \Rightarrow x \colon B}}}$
		
		\vspace{4mm}
		$\vlderivation { \vlin {\sbls}{}{\G, xRy, \Left, x \colon \square A \Rightarrow z\colon B}{\vlhy {\G, xRy, \Left, x \colon \square A, y \colon A \Rightarrow z\colon B}}}$
		\hspace{23mm}$\vlderivation { \vlin {\sbrs}{y$ fresh$}{\G, \Left \Rightarrow x \colon \square A}{\vlhy {\G, xRy, \Left \Rightarrow y \colon A}}}$
		
		\vspace{4mm}
		$\vlderivation { \vlin{\sdls}{y$ fresh$}{\G, \Left, x \colon \Diamond A \Rightarrow z \colon B}{\vlhy {\G, xRy, \Left, y \colon A \Rightarrow z \colon B}}}$
		\hspace{7mm}$\vlderivation {\vlin {\sdrs}{}{\G,xRy, \Left \Rightarrow x \colon \Diamond A}{\vlhy {\G, xRy, \Left \Rightarrow y \colon A }}}$
	\end{center}
	\caption{System $\labIKs$}
	\label{fig:simpson}
\end{figure}

%\raul{Aca mencionar que es solo un sketch, y decir en una frase qué falta}
Lo siguiente representa un boceto de la prueba que se continuará como trabajo a futuro. Faltan incluir las demostraciones para la regla de $\sirs$ y para $\sbrs$.

\newpage
\begin{proof}
	La demostración de completitud utilizando el sistema propuesto por Simpson, como se detalló rápidamente al comienzo de esta sección, será a través de
	 análisis de casos. La mayoría de las reglas en $\labIKs$ son las mismas reglas que en el sistema $\labIKh$ excepto por las siguientes:
	\begin{center}
		\begin{itemize}
		
		\item Regla $\idsim$:
		
		\begin{center}
		$\vlderivation { \vlin {\idsim}{}{\G, \Left, x \colon A \Rightarrow x \colon a}{\vlhy {}}}$ \hspace{4mm}  \begin{huge}$  \rightarrow$ \end{huge} \hspace{4mm} $\vlderivation{\vlin {\refl}{}{\G, \Left, x \colon a \Rightarrow x \colon a}{\vlin {\ids}{}{\G,x \le x, \Left, x\colon a \Rightarrow x \colon a}{\vlhy {}}}}$
	\end{center}
		 
		
		\vspace{5mm}
		\item Regla $\sorones$ y $\sotwos$: 
		
		\begin{center}
		$\vlderivation {\vlin {\sorones}{}{\G, \Left \Rightarrow x \colon \vls[A.B]}{\vlpd {\Done}{}{\G, \Left \Rightarrow x \colon A}}}$\hspace{2mm} or \hspace{2mm}$\vlderivation {\vlin {\sotwos}{}{\G, \Left \Rightarrow x \colon \vls[A.B]}{\vlpd {\Done}{}{\G, \Left \Rightarrow x \colon B}}}$ \hspace{4mm} \begin{huge}$\rightarrow$\end{huge} \hspace{4mm} $\vlderivation {\vlin {\sorig}{}{\G, \Left \Rightarrow x \colon \vls[A.B]}{\vlpd{\Dwone}{}{\G, \Left \Rightarrow x \colon A, x \colon B}}}$
	\end{center}
	
		\newpage
		
		\item Regla $\sils$:
		
		\begin{center}
		$\vlderivation { \vliin {\sils}{}{\G, \Left, x \colon A \vljm B \Rightarrow z \colon C}{\vlpd {\Done}{}{\G, \Left \Rightarrow x \colon A}}{\vlpd {\Dtwo}{}{\G, \Left, x \colon B \Rightarrow z \colon C}}}$ \bigskip 
		
			\begin{huge}$\downarrow$\end{huge}\\
		
		 \bigskip
		 
		  $\vlderivation {\vlin {\refl}{}{\G, \Left, x \colon A \vljm B \Rightarrow z \colon C}{\vliin {\sil}{}{\G, \Left, x\le x, x \colon A \vljm B \Rightarrow z \colon C}{\vlpd {\Dwone}{}{\G, \Left, x \le x, x \colon A \vljm B \Rightarrow x\colon A}}{\vlpd {\Dwtwo}{}{\G, \Left, x \le x, x \colon B \Rightarrow z \colon C}}}}$
	\end{center}
		%\vspace{5mm}
		%\item Regla $\sirs$:
		
		%\begin{center}
		%$\vlderivation {\vlin {\sirs}{}{\G, \Left \Rightarrow x \colon A \vljm B}{\vlpd {\Done}{}{\G, \Left, x \colon A \Rightarrow x \colon B}}}$ \hspace{4mm} \begin{huge}$\rightarrow$\end{huge} \hspace{4mm} $\vlderivation {\vlin {\sir}{}{\G, \Left \Rightarrow x \colon A \vljm B}{\vlpd {\Dwone}{}{\G, \Left, x \le x, x \colon A \Rightarrow x \colon B}}}$
		%\end{center}
		
		\vspace{5mm}
		\item Regla $\sbls$:
		
		\begin{center}
		$\vlderivation {\vlin {\sbls}{}{\G, \Left, xRy, x \colon \square A \Rightarrow z \colon B}{\vlpd {\Done}{}{\G, \Left, xRy, x \colon \square A, y \colon A \Rightarrow z \colon B}}}$ \bigskip
		
			\begin{huge}$\downarrow$\end{huge}\\
		
		\bigskip
		
		 $\vlderivation {\vlin {\refl}{}{\G, \Left, xRy, x \colon \square A \Rightarrow z \colon B}{\vlin {\sbl}{}{\G, \Left, x \le x, xRy, x \colon \square A \Rightarrow z \colon B}{\vlpd {\Dwone}{}{\G, \Left, x \le x, xRy, x \colon \square A, y \colon A \Rightarrow z \colon B}}}}$
	\end{center}
	
		%\vspace{5mm}
		%\item Regla $\sbrs$:
		
		%\begin{center}
		%$\vlderivation {\vlin {\sbrs}{}{\G, \Left \Rightarrow x \colon \square A}{\vlpd {\Done}{}{\G, \Left, xRy \Rightarrow y \colon A}}}$ \hspace{4mm} \begin{huge}$\rightarrow$ \end{huge} \hspace{4mm} $\vlderivation {\vlin {\sbr}{}{\G, \Left \Rightarrow x \colon \square A}{\vlpd {\Dwone}{}{\G, \Left, x \le x, xRy \Rightarrow y \colon A}}}$
	%\end{center}
	
		\vspace{5mm}
		\item Regla $\sdrs$:
		
		\begin{center}
		$\vlderivation {\vlin {\sdrs}{}{\G, xRy, \Left \Rightarrow x \colon \Diamond A}{\vlpd {\Done}{}{\G, xRy, \Left \Rightarrow y \colon A}}}$ \hspace{4mm} \begin{huge}$\rightarrow$ \end{huge} \hspace{4mm} $\vlderivation {\vlin {\sdr}{}{\G, xRy, \Left \Rightarrow x \colon \Diamond A}{\vlpd {\Dwone}{}{\G, xRy, \Left \Rightarrow x \colon \Diamond A, y \colon A}}}$
	\end{center}
	
		\end{itemize}
	\end{center}
\end{proof}

Nuestra conjetura es que es posible demostrar completitud del sistema $\labIKh$ haciendo un análisis de cada una de las reglas de secuentes presentes en el sistema de Simpson (a partir del uso de las reglas de nuestro sistema).


