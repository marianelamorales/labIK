\documentclass[11pt]{article}
\usepackage{etex}
\usepackage{virginialake}
\usepackage{multicol}
\usepackage{graphicx}
\usepackage{rotating}
\usepackage{ragged2e}
\usepackage{amsthm}
\usepackage[left=2cm,top=1.5cm,right=2cm,bottom=2cm]{geometry} 
\usepackage{float}
\floatstyle{boxed} 
\restylefloat{figure}

\renewcommand\qedsymbol{$\blacksquare$}

\setlength{\parindent}{0cm}
\newcommand{\G}{\mathcal{G}}
\newcommand{\Left}{\Gamma}
\newcommand{\Right}{\Delta}
%\newcommand{\Left}{\mathcal{L}}
%\newcommand{\Right}{\mathcal{R}}
\newcommand{\agklmn}{\mathsf{g_{klmn}}}

%Symbols for System labK
\newcommand{\id}{id^{lab}}
\newcommand{\tolab}{\top^{lab}}
\newcommand{\vlab}{\wedge^{lab}}
\newcommand{\olab}{\vlor^{lab}}
\newcommand{\blab}{\square^{lab}}
\newcommand{\dlab}{\lozenge^{lab}}

%Labelled proof system
\newcommand{\toprule}{\B \Rightarrow \Right, x  \colon   \top}
\newcommand{\vlabr}{\B \Rightarrow \Right, x  \colon   A}
\newcommand{\vlabu}{\B \Rightarrow \Right, x  \colon   B}
\newcommand{\olabr}{\B \Rightarrow \Right, x  \colon   A, x  \colon   B}
\newcommand{\blabr}{\B \Rightarrow \Right, x  \colon   \square A}
\newcommand{\blabu}{\B, x$R$y \Rightarrow \Right, y  \colon   A}
\newcommand{\dlabr}{\B, x$R$y \Rightarrow \Right, x  \colon   \lozenge A}
\newcommand{\dlabu}{\B, x$R$y \Rightarrow \Right, x  \colon   \lozenge A, y  \colon} 


%Symbols for system labIK
\newcommand{\botlab}{\bot_{L}^{lab}}
\newcommand{\toplab}{\top_{R}^{lab}}
\newcommand{\andleflab}{\wedge_{L}^{lab}}
\newcommand{\andriglab}{\wedge_{R}^{lab}}
\newcommand{\orleflab}{\vlor_{L}^{lab}}
\newcommand{\orriglabo}{\vlor_{R1}^{lab}}
\newcommand{\orriglabt}{\vlor_{R2}^{lab}}
\newcommand{\irlab}{\vljm_{R}^{lab}}
\newcommand{\illab}{\vljm_{L}^{lab}}
\newcommand{\dllab}{\lozenge_{L}^{lab}}
\newcommand{\drlab}{\lozenge_{R}^{lab}}
\newcommand{\bllab}{\square_{L}^{lab}}
\newcommand{\brlab}{\square_{R}^{lab}}

%System labIK+gklmn
\newcommand{\gklmn}{\boxtimes_{\mathsf{gklmn}}}
\newcommand{\boxk}{\square_{R}^{k}}
\newcommand{\boxlk}{\square_{L}^{k}}
\newcommand{\diamk}{\lozenge_{L}^{k}}
\newcommand{\diamrk}{\lozenge_{R}^{k}}

%Symbols for System labheartIK
\newcommand{\ids}{id}
\newcommand{\idg}{id_{g}}
\newcommand{\refl}{refl}
\newcommand{\trans}{trans}
\newcommand{\cut}{cut}
\newcommand{\fone}{F1}
\newcommand{\ftwo}{F2}
\newcommand{\sbot}{\bot_{L}}
\newcommand{\Stop}{\top_{R}}
\newcommand{\svlef}{\wedge_{L}}
\newcommand{\svrig}{\wedge_{R}}
\newcommand{\solef}{\vlor_{L}}
\newcommand{\sorig}{\vlor_{R}}
\newcommand{\sorone}{\vlor_{R1}}
\newcommand{\sotwo}{\vlor_{R2}}
\newcommand{\sir}{\vljm_{R}}
\newcommand{\sil}{\vljm_{L}}
\newcommand{\sdl}{\lozenge_{L}}
\newcommand{\sdr}{\lozenge_{R}}
\newcommand{\sbl}{\square_{L}}
\newcommand{\sbr}{\square_{R}}
\newcommand{\smon}{mon_{L}}
\newcommand{\M}{\mathcal{M}}
\newcommand{\F}{\mathcal{F}}
\newcommand{\Gone}{\mathcal{G}_{1}}
\newcommand{\Gtwo}{\mathcal{G}_{2}}
\newcommand{\Dw}{\mathcal{D}^{w}}
\newcommand{\Dwone}{\mathcal{D}_{1}^{w}}
\newcommand{\Dwtwo}{\mathcal{D}_{2}^{w}}
\newcommand{\D}{\mathcal{D}}
\newcommand{\Done}{\mathcal{D}_{1}}
\newcommand{\Dtwo}{\mathcal{D}_{2}}


%System LABIK
\newcommand{\conjrig}{\G, \Left \Rightarrow \Right, x \colon A}
\newcommand{\conjrigh}{\G, \Left \Rightarrow \Right, x  \colon B}
\newcommand{\conjlef}{\G, \Left, x  \colon  A, x \colon B \Rightarrow \Right}


\newcommand{\kfour}{\scalebox{0.6}{
$\vlderivation {
\vlin{\sir}
{y$ fresh$}
{\Rightarrow x \colon (\lozenge A \vljm \square B) \vljm \square (A \vljm B)}
{\vlin {\sbr}
{z, w$ fresh$}
{x \le y, y\colon \lozenge A \vljm \square B \Rightarrow y \colon \square (A \vljm B) }
{\vlin {\sir}
{u$ fresh$}
{x \le y, y\le z, z$R$w, y \colon \lozenge A \vljm \square B \Rightarrow w \colon A \vljm B }
{\vlin {\fone}
{}
{x \le y, y \le z, w \le u, z$R$w, y \colon \lozenge A \vljm \square B, u \colon A \Rightarrow u \colon B}
{\vlin {\trans}
{}
{x \le y, y \le z, w \le u, z \le t, z$R$w, t$R$u, y \colon \lozenge A \vljm \square B, u \colon A \Rightarrow u \colon B}
{\vliin {\sil}
{}
{x \le y, y \le z, w \le u, z \le t, y \le t, z$R$w, t$R$u, y \colon \lozenge A \vljm \square B, u \colon A \Rightarrow u \colon B}
{\vlin {\sdr}
{}
{x \le y, y \le z, w \le u, z \le t, y \le t, z$R$w, t$R$u, y \colon \lozenge A \vljm \square B, u \colon A \Rightarrow u \colon B, t \colon \lozenge A}
{\vlin {\refl}
{}
{x \le y, y \le z, w \le u, z \le t, y \le t, z$R$w, t$R$u, y \colon \lozenge A \vljm \square B, u \colon A \Rightarrow u \colon B, t \colon \lozenge A, u \colon A}
{\vlin {\ids}
{}
{x \le y, y \le z, w \le u, z \le t, y \le t, u \le u, z$R$w, t$R$u, y \colon \lozenge A \vljm \square B, u \colon A \Rightarrow u \colon B, t \colon \lozenge A, u \colon A}
{\vlhy {}}}}}
{\vlin {\refl}
{}
{x \le y, y \le z, w \le u, z \le t, y \le t, z$R$w, t$R$u, y \colon \lozenge A \vljm \square B, u \colon A, t \colon \square B \Rightarrow u \colon B}
{\vlin {\sbl}
{}
{x \le y, y \le z, w \le u, z \le t, y \le t, t \le t, z$R$w, t$R$u, y \colon \lozenge A \vljm \square B, u \colon A, t \colon \square B \Rightarrow u \colon B}
{\vlin {\refl}
{}
{x \le y, y \le z, w \le u, z \le t, y \le t, t \le t, z$R$w, t$R$u, y \colon \lozenge A \vljm \square B, u \colon A, t \colon \square B, u \colon B \Rightarrow u \colon B}
{\vlin {\ids}
{}
{x \le y, y \le z, w \le u, z \le t, y \le t, t \le t, u \le u, z$R$w, t$R$u, y \colon \lozenge A \vljm \square B, u \colon A, t \colon \square B, u \colon B \Rightarrow u \colon B}
{\vlhy {}}}}}}}}}}}
}$
}
}

\newtheorem{lemma}{Lemma}
\newtheorem{teo}{Theorem}

\title{\textbf{Internship Project}}
\date{March 20, 2018}

\begin{document}
\maketitle
\section{Classical modal logic}

\subsection{Syntax}

The language of classical modal logic is obtained from the one of classical propositional logic by adding the modal connectives $\square$ and $\lozenge$, standing for example for $necessity$ and $possibility$. Starting with a set $A$ of atomic propositions denoted $a$ and their duals $\overline a$, modal formulas are constructed from the following grammar:
\begin{center}
 $A ::=  a$ $| $ $\vls-a$ $ | $ $\vls(A.A)$ $|$ $\top$ $|$ $\vls[A.A]$ $|$ $\bot $ $|$ $\square A$ $|$ $\lozenge A$ 
\end{center}


In the classical setting, we always assume that formulas are in negation normal form, that is, negation is restricted to atoms. When we write $\neg A$ in this case, we mean the result of computing the de Morgan dual of $A$, i.e. $\neg \neg A $ $\equiv$ $A$, $\neg (\vls(A.B))\equiv \vls[\neg A. \neg B]$ and $\neg \square A \equiv \lozenge \neg A$, where $\equiv$ denotes syntactic equality.

The classical modal logic is obtained from classical propositional logic by adding:

\begin{itemize}
\item{necessitation rule: if $A$ is a theorem of K, then so is $\square A$}.
\item{axiom of distributivity}:
\begin{center}
$k$ $\colon$ $\square(A \vljm B) \vljm (\square A \vljm \square B)$ 
\end{center}

\end{itemize}


\subsection{Semantics}

A frame is a pair $\langle W, R\rangle$ of a non-empty set $W$ of possible worlds and a binary relation $R \subseteq W \times W$(accesibility relation).\\

A model $\M$ is a frame with a valuation $V: W$$\rightarrow$ $2^{A}$ which assigns to each world $w$ a subset of propositional variables that are true in $w$.\\

The truth of a modal formula at a world $w$ in a relational structure is the smallest relation $\Vdash$ satisfying:


$w \Vdash a$ iff $a \in V(w)$

$w \Vdash \vls-a$ iff $a \not \in V(w)$

$w \Vdash \vls(A.B)$ iff $w \Vdash A$ and $w \Vdash B$

$w \Vdash \vls[A.B]$ iff $w \Vdash A$ or $w \Vdash B$

$w\Vdash \square A$ iff for all $v \in W$ such that $(w, v) \in$ R we have $v \Vdash A$

$w \Vdash \lozenge A$ iff there exists $v \in W$ such that $(w, v) \in$ R and $v \Vdash A$\\


A formula $A$ is satisfied in a model $\M = \langle W, R, V \rangle$ denoted by $\M \vDash A$, if for every $w \in W$ we have $w \Vdash A$.

A formula $A$ is valid in a frame $\F = \langle W, R \rangle$ denoted by $\F \vDash A$, if for every valuation $V$ we have $\langle W, R, V \rangle \vDash A$.



\begin{teo}(Kripke \cite{Kipke}). 
A formula $A$ is a theorem of K if and only if $A$ is valid in every frame.
\end{teo}

\newpage

\section{Labelled proof systems}

Labelled sequents are formed from labelled formulas of the form $x \colon A$ and relational atoms of the form $x$R$y$, where $x, y$ range over a set of variables (called labels) and $A$ is a modal formula.

A labelled sequent is of the form $\G \Rightarrow \Right$ where $\G$ denotes a set of relational atoms and $\Right$ a multiset of labelled formulas.

A simple proof system for classical modal logic K can be obtained in this formalism:

\vspace{3mm}

\begin{figure}[h]
\begin{center}
\begin{multicols}{2}

$\vlinf{\id}{}{\G \Rightarrow \Right, x: a, x: \vls-a}{}$

$\vlinf{\tolab}{}{\toprule}{}$

\end{multicols}

\begin{multicols}{2}

$\vliinf{\vlab}{}{\G \Rightarrow \Right, x :\vls(A.B)}{\vlabr}{\vlabu}$

$\vlinf{\olab}{}{\G \Rightarrow \Right, x \colon \vls[A.B]}{\olabr}$

\end{multicols}

\begin{multicols}{2}

$\vlinf{\blab}{y$ fresh$}{\blabr}{\blabu}$

$\vlinf{\dlab}{}{\dlabr}{\dlabu}$

\end{multicols}
\end{center}
\caption{System labK}
\end{figure}

\vspace{5mm}

\begin{teo}(Negri \cite{Negri}) 
 A formula $A$ is provable in the calculus labK if and only if $A$ is valid in every frame.
\end{teo}

\newpage

\section{Intuitionistic modal logic}

\subsection{Syntax}

In the intuitionistic case, we work with a different set of connectives. Starting with a set of atomic propositions still denoted $a$, formulas are constructed from the following grammar:


\begin{center}
 $A ::=  a$ $| $ $\vls(A.A)$ $|$ $\top$ $|$ $\vls[A.A]$ $|$ $\bot $ $|$ $A \vljm A$ $|$ $\square A$ $|$ $\lozenge A$ 
 \end{center}
 
 When we write $\neg A$ in this case, we mean $A \vljm \bot$.
 
 There are several variants of K that are classically but not intuitionistically equivalent.
 
The intuitionistic modal logic IK obtained from ordinary intuitionistic propositional logic:
\begin{itemize}

\item{THEN-1}: $A  \vljm (B \vljm A)$

\item{THEN-2}: $(A \vljm (B \vljm C)) \vljm ((A \vljm B) \vljm (A \vljm C))$

\item{AND-1}: $\vls(A.B)\vljm A$

\item{AND-2}: $\vls(A.B) \vljm B$

\item{AND-3}: $A \vljm (B \vljm (\vls(A.B)))$

\item{OR-1}: $A \vljm \vls[A.B]$

\item{OR-2}: $B \vljm \vls[A.B]$

\item{OR-3}: $(A \vljm C) \vljm ((B \vljm C) \vljm (\vls[A.B] \vljm C))$

\item{FALSE}: $\bot \vljm A$
\end{itemize}

 By adding:

\begin{itemize}
\item{necessitation rule}: if $A$ is a theorem, then so is $\square A$.
\item{the following five axioms}:

$k_{1}$: $\square(A \vljm B) \vljm (\square A \vljm \square B)$

$k_{2}$: $\square (A \vljm B) \vljm (\lozenge A \vljm \lozenge B)$

$k_{3}$: $\lozenge (\vls[A.B]) \vljm (( \vls [\lozenge A. \lozenge B]))$

$k_{4}$: $(\lozenge A \vljm \square B) \vljm \square(A \vljm B)$

$k_{5}$: $\lozenge \bot \vljm \bot$ 

\end{itemize} 
 
\subsection{Semantics}

The Kripke semantics for intuitionistic modal logic combines the Kripke semantics for intuitionistic propositional logic and the one for classical modal logic, using two distinct relations on the set of worlds.\\

A bi-relational frame $\F$ is a triple $\langle W, R, \le \rangle$ of a non-empty set of worlds W with two binary relations: $R$ $\subseteq$ $W \times W$ and $\le$ a preorder relation on $W$ satysfing the conditions:\\

(F1) For all worlds $u, v, v'$, if $u$R$v$ and $v \le v'$, there exists a $u'$ such that $u \le u'$ and $u'$R$v'$.\\

(F2) For all worlds $u', u, v$, if $u \le v$, there exists a $v'$ such that $u'$R$v'$ and $v\le v'$.\\

\newpage
A bi-relational model $\M$ is a quadruple $\langle W, R,\le,V \rangle$ with $\langle W, R, \le \rangle$ a bi-relational frame and $V$ a monotone valuation function $V: W$$\rightarrow$ $2^{A}$ which is a function that maps each world $w$ to the subset of propositional atoms that are true in $w$, subject to:
\begin{center}
$w \le w'$ $\Rightarrow$ $V(w)$ $\subseteq$ $V(w')$
\end{center}

As in the classical case, we write $w \Vdash a$ iff $a \in V(w)$ and we extend this relation to all formulas by induction, following the rules for both intuitionistic and modal Kripke models:

$w \Vdash \vls(A.B)$ iff $w \Vdash A$ and $w \Vdash B$

$w \Vdash \vls[A.B]$ iff $w \Vdash A$ or $w \Vdash B$

$w \Vdash A \vljm B$ iff for all $w'$ with $w \le w'$, if $w' \Vdash A$ then $w' \Vdash B$

$w \Vdash \square A$ iff for all $w'$ and $u$ with $w \le w'$ and $w'$R$u$, $u \Vdash A$

$w \Vdash \lozenge A$ iff there exists a world $u \in W$ such that $w$R$u$ and $u \Vdash A$.

We write $w \not \Vdash A$  if it is not the case that $w\Vdash A$, in particular $w \Vdash \bot $.\\

A formula $A$ is satisfied in a model $\M = \langle W, R, \le, V \rangle$, if for all $w \in W$ we have $w \Vdash A$.\\

A formula $A$ is valid in a frame $\F = \langle W, R, \le \rangle$, if for all valuations $V$ we have that $A$ is satisfied in $\langle W, R, \le, V \rangle$.

\begin{teo} (Fischer-Servi \cite{Fischer}, Plotin and Stirling \cite{Plotin})
 A formula $A$ is a theorem of IK if and only if $A$ is valid in every bi-relational frame.
\end{teo}

\section{The main problem}

The straightforward extension of labelled deduction to the intuitionistic setting would be to use two sorts of relational atoms, one for the modal relation R and another one for the intuitionistic relation $\le$.\\

The idea is to extend labelled sequents with a preorder relation symbol in order to capture intuitionistic modal logics; that is, to define intuitionistic labelled sequents from labelled formulas $x \colon A$, relational atoms $x$R$y$, and preorder atoms of the form $x \le y$, where $x, y$ range over a set of labels and $A$ is an inutitionistic modal formula.\\

A two-sided intuitionistic labelled sequent would be of the form $\G$, $\Left \Rightarrow \Right$ where $\G$ denotes a set of relational and preorder atoms, and $\Left$ and $\Right$ are multiset of labelled formulas. We then would want to obtain a proof system labIK for intuitionistic modal logic in this formalism and prove the following as a new theorem:\\

\underline{To prove}: A formula $A$ is provable in the calculus labIK if and only if $A$ is valid in every bi-relational frame.

\newpage

\begin{center}

Solving the problem: The following proof system is proposed.

\end{center}

\section{New system: System labIK}

\vspace{3mm}

\begin{figure}[h]
\begin{center}
\begin{multicols}{3}

$\vlinf{\sbot}{}{\G,\Left, x \colon \bot \Rightarrow \Right}{}$

$\vlinf{\ids}{}{\G, \Left,x \le y, x \colon a \Rightarrow \Right, y \colon a}{}$

$\vlinf{\Stop}{}{\G, \Left \Rightarrow \Right, x \colon \top}{}$

\end{multicols}

\vspace{2mm}
\begin{multicols}{2}

$\vlinf{\svlef}{}{\G,\Left, x \colon \vls(A.B) \Rightarrow \Right}{\conjlef}$

$\vliinf{\svrig}{}{\G,\Left \Rightarrow \Right, x \colon \vls(A.B)}{\conjrig}{\conjrigh}$

\end{multicols}

\vspace{2mm}

\begin{multicols}{2}


$\vliinf{\solef}{}{\G, \Left, x \colon \vls[A.B] \Rightarrow \Right}{\G, \Left, x   \colon   A \Rightarrow \Right}{\G, \Left, x   \colon   B \Rightarrow \Right}$

$\vlinf{\sorig}{}{\G, \Left \Rightarrow \Right, x \colon \vls[A.B]}{\G, \Left \Rightarrow \Right, x   \colon   A, x   \colon   B}$

\end{multicols}


\vspace{2mm}


\begin{multicols}{1}

$\vlinf{\sir}{$ $y$ fresh$}{\G, \Left \Rightarrow \Right, x \colon A \vljm B}{\G, \Left, x \le y, y \colon A \Rightarrow \Right, y \colon B}$

\end{multicols}


\vspace{2mm}


$\vliinf{\sil}{}{\G, \Left, x \le y, x \colon A \vljm B \Rightarrow \Right}{\G, \Left, x \le y, x \colon A \vljm B \Rightarrow \Right, y \colon A}{\G, \Left, x \le y, x \colon A \vljm B, y \colon B \Rightarrow \Right}$

\vspace{3mm}

\begin{multicols}{2}

$\vlinf{\sbl}{}{\G, \Left, x \le y, y$R$z, x \colon \square A \Rightarrow \Right}{\G,\Left, x \le y, y$R$z, x \colon \square A, z \colon A \Rightarrow \Right}$

$\vlinf{\sbr}{$ $y, z$ fresh$}{\G, \Left \Rightarrow \Right, x \colon \square A}{\G, \Left, x \le y, y$R$z \Rightarrow \Right, z \colon A}$

\end{multicols}

\vspace{2mm}

\begin{multicols}{2}

$\vlinf{\sdl}{$ $y$ fresh $}{\G, \Left, x \colon \lozenge A \Rightarrow \Right}{\G, \Left, x$R$y, y \colon A \Rightarrow \Right}$

$\vlinf{\sdr}{}{\G, \Left, x$R$y \Rightarrow \Right, x \colon \lozenge A}{\G, \Left, x$R$y \Rightarrow \Right, x \colon \lozenge A, y \colon A}$

\end{multicols}

\begin{multicols}{2}
$\vlinf{\refl}{}{\G, \Left \Rightarrow \Right}{\G, x\le x, \Left, \Right}$

$\vlinf{\trans}{}{\G, x \le y, y \le z, \Left \Rightarrow \Right}{\G, x \le y, y \le z, x \le z, \Left \Rightarrow \Right}$

\end{multicols}

\vspace{1mm}

\begin{multicols}{2}

$\vlinf{\fone}{$ $u$ fresh$}{\G, \Left, x$R$y, y \le z \Rightarrow \Right}{\G, \Left, x$R$y, y \le z, x \le u, u$R$z \Rightarrow \Right}$

$\vlinf{\ftwo}{u$ fresh$}{\G, \Left, x$R$y,x \le z \Rightarrow \Right}{\G, \Left, x$R$y, x \le z, y \le u, z$R$u \Rightarrow \Right }$

\end{multicols}
\end{center}

\caption{System labIK}
\end{figure}
\newpage
\section{Soundness}

\vspace{5mm}

In this section we will show that all rules presented in \textbf{Figure 2} are indeed sound.

\vspace{1mm}

\underline{Note:} The proofs are similar for each rule. TO DO: define lemmas.

\vspace{4mm}

$\vlinf{\sbr}{}{\G, \Left \Rightarrow \Right, x \colon \square A}{\G, \Left, x \le y, y$R$z \Rightarrow \Right, x \colon \square A, z \colon A}$

\vspace{3mm}
\textbf{Is this sound?}
\vspace{2mm}

Soundness means: If $\forall$ model $\M$, $\M \Vdash premise $ then $\forall$ $\M$, $\M \Vdash conclusion$.

By way of contradiction: Assume $\exists$ $\M_{0}$ such that $\M_{0} \not \Vdash$ $conclusion$, i.e. $\M_{0} \Vdash \vlan (\G,\Left)$ and  $\M_{0} \not \Vdash \vlor (\Right, x: \square A)$, i.e. $\M_{0} \not \Vdash \Right$ and $\M_{0},x \not \Vdash \square A $.

$\Rightarrow$ exists worlds $y, z$ in $\M_{0}$, such that $x \le y, y$R$z$ where $\M_{0}, z \not \Vdash A$.

$\Rightarrow \M_{0} \not \Vdash premise$.


\vspace{5mm}

$\vlinf{\sdr}{}{\G, \Left, x$R$y \Rightarrow \Right, x \colon \lozenge A}{\G, \Left, x$R$y \Rightarrow \Right, x \colon \lozenge A, y \colon A}$

\vspace{3mm}

\textbf{Is this sound?}
\vspace{2mm}

Soundness means: If $\forall$ model $\M$, $M \Vdash premise $ then $\forall$ $\M$, $\M \Vdash conclusion$.



By way of contradiction: Assume $\exists$ $\M_{0}$ such that $\M_{0} \not \Vdash$ $conclusion$, i.e. $\M_{0} \Vdash \vlan (\G, \Left, x$R$y)$ and $\M_{0} \not \Vdash \vlor (\Right, x: \lozenge A)$, i.e. $\M_{0} \not \Vdash \Right$ and $\M_{0}, x \not \Vdash \lozenge A $, i.e. $\forall$ worlds $y$ in $\M_{0}$ where $x$R$y$, $\M_{0}, y \not \Vdash A$.

$\Rightarrow$ $\M_{0} \not \Vdash premise$.


\vspace{5mm}

$\vlinf{\sdl}{}{\G, \Left, x: \lozenge A \Rightarrow \Right}{\G, \Left, x$R$y,  y \colon A \Rightarrow \Right}$

\vspace{3mm}

\textbf{Is this sound?}

\vspace{2mm}

Soundness means: If $\forall$ model $\M$, $\M \Vdash premise $ then $\forall$ $\M$, $\M \Vdash conclusion$.



By way of contradiction: Assume $\exists$ $\M_{0}$ such that $\M_{0} \not \Vdash$ $conclusion$, i.e. $\M_{0} \Vdash \vlan (\G, \Left, x \colon \lozenge A)$ and $\M_{0} \not \Vdash \Right$, i.e. $\M_{0} \Vdash G$, $\M_{0} \Vdash L$ and $\M_{0}, x \Vdash \lozenge A$, i.e. exists worlds $y$ in $\M_{0}$ such that $x$R$y$ and $\M_{0}, y \Vdash A$.

$\Rightarrow \M_{0} \not \Vdash premise$.



\vspace{5mm}

\emph{\textbf{To be continued}} 

\newpage
\section{Syntactic Completeness Proof}

We show completeness with respect to the Hilbert system.\\

\begin{teo} Every theorem of the logic IK is provable in labIK.
\end{teo}
\emph{Proof} For this proof we need to simulate modus ponens, simulate necessitation, prove $k_{1}$, ..., $k_{5}$ axioms and prove all the axioms in propositional modal logic.

The \textbf{axioms $k_{1}$, ..., $k_{5}$ are provable} in labIK as shown below:

\subsection{$k_{1}$}

\begin{center}

\scalebox{0.6}{
$\vlderivation{
\vlin{\sir}
{y$ fresh$}
{\Rightarrow x \colon \square (A \vljm B) \vljm (\square A \vljm \square B)}
{\vlin {\sir}
{z$ fresh$}
{x \le y, y \colon \square(A \vljm B) \Rightarrow y \colon \square A \vljm \square B}
{\vlin {\sbr}
{w, u$ fresh$}
{x \le y,y \le z, y \colon \square(A \vljm B), z \colon \square A \Rightarrow z \colon \square B}
{\vlin {\sbl}
{}
{x \le y,y \le z, z \le w, w$R$u, y \colon \square(A \vljm B), z \colon \square A \Rightarrow u \colon B}
{\vlin {\trans}
{}
{x \le y,y \le z, z \le w, w$R$u, y \colon \square(A \vljm B), z \colon \square A, u \colon A \Rightarrow u \colon B}
{\vlin {\sbl}
{}
{x \le y,y \le z, z \le w, y \le w, w$R$u, y \colon \square(A \vljm B), z \colon \square A, u \colon A \Rightarrow u \colon B}
{\vlin {\refl}
{}
{x \le y,y \le z, z \le w, y \le w, w$R$u, y \colon \square(A \vljm B), z \colon \square A, u \colon A, u \colon A \vljm B \Rightarrow u \colon B}
{\vliin {\sil}
{}
{x \le y,y \le z, z \le w, y \le w, u \le u, w$R$u, y \colon \square(A \vljm B), z \colon \square A, u \colon A, u \colon A \vljm B \Rightarrow u \colon B}
{\vlin {\ids}
{}
{x \le y,y \le z, z \le w, y \le w, u \le u, w$R$u, y \colon \square(A \vljm B), z \colon \square A, u \colon A, u \colon A \vljm B \Rightarrow u \colon B, u \colon A}
{\vlhy {}}}
{\vlin {\ids}
{}
{x \le y,y \le z, z \le w, y \le w, u \le u, w$R$u, y \colon \square(A \vljm B), z \colon \square A, u \colon A, u \colon A \vljm B, u \colon B \Rightarrow u \colon B}
{\vlhy {}}}}}}}}}}
}$
}
\end{center}

\subsection{$k_{2}$}

\begin{center}

\ktwo

\end{center}

\subsection{$k_{3}$}

\begin{center}

$\vlderivation {
\vlin{\sir}
{}
{\Rightarrow x \colon \lozenge (\vls[A.B]) \vljm (\vls[\lozenge A. \lozenge B ])}
{\vlin {\sdl}
{}
{x \le y, y \colon \lozenge (\vls[A.B]) \Rightarrow y \colon \vls[\lozenge A. \lozenge B] }
{\vliin {\solef}
{}
{x \le y, y$R$z, z \colon \vls[A.B] \Rightarrow y \colon \vls[\lozenge A. \lozenge B]}
{\vlin {\sorig}
{}
{x \le y, y$R$z, z \colon A \Rightarrow y \colon \vls[\lozenge A. \lozenge B]}
{\vlin {\sdr}
{}
{x \le y, y$R$z, z \colon A \Rightarrow y \colon \lozenge A, y \colon \lozenge B}
{\vlin {\refl}
{}
{x \le y, y$R$z, z \colon A \Rightarrow y \colon \lozenge A, z \colon A, y \colon \lozenge B}
{\vlin {\ids}
{}
{x \le y, z \le z, y$R$z, z \colon A \Rightarrow y \colon \lozenge A, z \colon A, y \colon \lozenge B}
{\vlhy {}}}}}}
{\vlin {\sorig}
{}
{x \le y, y$R$z, z \colon B \Rightarrow y \colon \vls[\lozenge A. \lozenge B]}
{\vlin {\sdr}
{}
{x \le y, y$R$z, z \colon B \Rightarrow y \colon \lozenge A, y \colon \lozenge B}
{\vlin {\refl}
{}
{x \le y, y$R$z, z \colon B \Rightarrow y \colon \lozenge A, y \colon \lozenge B, z \colon B}
{\vlin {\ids}
{}
{x \le y, z \le z, y$R$z, z \colon B \Rightarrow y \colon \lozenge A, y \colon \lozenge B, z \colon B}
{\vlhy {}}}}}}}}
}$

\end{center}

\subsection{$k_{4}$}

\begin{center}

\kfour

\end{center}

\subsection{$k_{5}$}

\begin{center}

$\vlderivation {
\vlin{\sir}
{}
{\Rightarrow x \colon \lozenge \bot \vljm \bot}
{\vlin {\sdl}
{}
{x \le y, y \colon \lozenge \bot \Rightarrow y \colon \bot}
{\vlin {\sbot}
{}
{x \le y, y$R$z, z \colon \bot \Rightarrow y \colon \bot}
{\vlhy {}}}}
}$

\end{center}


We can show that \textbf{all axioms of propositional intuitionistic logic are provable} in labIK:

\begin{center}
\textbf{THEN-1}

$\vlderivation{
\vlin{\sir}
{}
{\Rightarrow x \colon A \vljm (B \vljm A)}
{\vlin {\sir}
{}
{x \le y, y \colon A \Rightarrow y\colon B \vljm A}
{\vlin {\ids}
{}
{x \le y, y \le z, y \colon A, z \colon B \Rightarrow z \colon A}
{\vlhy {}}}}
}$

\end{center}

\vspace{3mm}


\begin{center}
\textbf{AND-1}

$\vlderivation {
\vlin{\sir}
{}
{\Rightarrow x \colon \vls(A.B) \vljm A}
{\vlin {\svlef}
{}
{x \le y, y \colon \vls(A.B) \Rightarrow y \colon A}
{\vlin {\refl}
{}
{x \le y, y \colon A, y \colon B \Rightarrow y \colon A}
{\vlin {\ids}
{}
{x \le y, y \le y, y \colon A, y \colon B \Rightarrow y \colon A}
{\vlhy {}}}}}
}$
\end{center}

\vspace{3mm}

\begin{center}
\textbf{AND-2}

$\vlderivation {
\vlin{\sir}
{}
{\Rightarrow x \colon \vls(A.B) \vljm B}
{\vlin {\svlef}
{}
{x \le y, y \colon \vls(A.B) \Rightarrow y \colon B}
{\vlin {\refl}
{}
{x \le y, y \colon A, y \colon B \Rightarrow y \colon B}
{\vlin {\ids}
{}
{x \le y, y \le y, y \colon A, y \colon B \Rightarrow y \colon B}
{\vlhy {}}}}}
}$
\end{center}

\newpage

\begin{center}
\textbf{AND-3}

$\vlderivation{
\vlin{\sir}
{}
{\Rightarrow x \colon A \vljm (B \vljm (\vls(A.B)))}
{\vlin {\sir}
{}
{x \le y, y \colon A \Rightarrow y \colon B \vljm (\vls(A.B))}
{\vliin {\svrig}
{}
{x \le y, y \le z, y \colon A, z \colon B \Rightarrow z \colon \vls(A.B)}
{\vlin {\ids}
{}
{x \le y, y \le z, y \colon A, z \colon B \Rightarrow z \colon A}
{\vlhy {}}}
{\vlin {\refl}
{}
{x \le y, y \le z, y \colon A, z \colon B \Rightarrow z \colon B}
{\vlin {\ids}
{}
{x \le y, y \le z, z \le z, y \colon A, z \colon B \Rightarrow z \colon B}
{\vlhy {}}}}}}
}$

\end{center}

\vspace{3mm}

\begin{center}
\textbf{OR-1}

$\vlderivation {
\vlin{\sir}
{}
{\Rightarrow x \colon A \vljm \vls[A.B]}
{\vlin {\sorig}
{}
{x \le y, y \colon A \Rightarrow y \colon \vls[A.B]}
{\vlin {\refl}
{}
{x \le y, y \colon A \Rightarrow y \colon A, y \colon B}
{\vlin {\ids}
{}
{x \le y, y \le y, y \colon A \Rightarrow y \colon A, y \colon B}
{\vlhy {}}}}}
}$

\end{center}

\vspace{3mm}

\begin{center}
\textbf{OR-2}

$\vlderivation {
\vlin{\sir}
{}
{\Rightarrow x \colon B \vljm \vls[A.B]}
{\vlin {\sorig}
{}
{x \le y, y \colon B \Rightarrow y \colon \vls[A.B]}
{\vlin {\refl}
{}
{x \le y, y \colon B \Rightarrow y \colon A, y \colon B}
{\vlin {\ids}
{}
{x \le y, y \le y, y \colon B \Rightarrow y \colon A, y \colon B}
{\vlhy {}}}}}
}$

\end{center}

\vspace{3mm}

\begin{center}
\textbf{OR-3}

\scalebox{0.73} {

$\vlderivation {
\vlin{\sir}
{}
{\Rightarrow x \colon (A \vljm C)\vljm ((B \vljm C) \vljm ( A \vlor B \vljm C))}
{\vlin {\sir}
{}
{x \le y, y \colon A \vljm C \Rightarrow y \colon (B \vljm C) \vljm (A \vlor B \vljm C)}
{\vlin {\sir}
{}
{x \le y, y \le z, y \colon A \vljm C, z \colon B \colon C \Rightarrow z \colon A \vlor B  \vljm C}
{\vlin {\solef}
{}
{x \le y, y \le z, z \le w, y \colon A \vljm C, z \colon B \vljm C, A \vlor B \Rightarrow w \colon C}
{\vliin {\sil}
{}
{x \le y, y \le z, z \le w, y \colon A \vljm C, z \colon B \vljm C, w \colon A, w \colon B \Rightarrow w \colon C }
{\vlin {\refl}
{}
{x \le y, y \le z, z \le w, y \colon A \vljm C, z \colon B \vljm C, w \colon A, w \colon B \Rightarrow w \colon C, w \colon A}
{\vlin {\ids}
{}
{x \le y, y \le z, z \le w, w \le w, y \colon A \vljm C, z \colon B \vljm C, w \colon A, w \colon B \Rightarrow w \colon C, w \colon A}
{\vlhy {}}}}
{\vlin {\refl}
{}
{x \le y, y \le z, z \le w, y \colon A \vljm C, z \colon B \vljm C, w \colon A, w \colon B, w \colon C \Rightarrow w \colon C }
{\vlin {\ids}
{}
{x \le y, y \le z, z \le w, w \le w, y \colon A \vljm C, z \colon B \vljm C, w \colon A, w \colon B, w \colon C \Rightarrow w \colon C}
{\vlhy {}}}}}}}}
}$
}

\end{center}

\vspace{3mm}

\begin{center}
\textbf{FALSE}

$\vlderivation {
\vlin{\sir}
{}
{\Rightarrow x \colon \bot \vljm A}
{\vlin {\sbot}
{}
{x \le y, y \colon \bot \Rightarrow y \colon A}
{\vlhy {}}}
}$

\end{center}

\newpage

\begin{center}
\begin{turn}{90}



\scalebox{0.4}{

$\vlderivation {
\vlin{\sir}
{}
{\Rightarrow x \colon (A \vljm (B \vljm C)) \vljm ((A \vljm B)\vljm (A \vljm C))}
{\vlin {\sir}
{}
{x \le y, y \colon A \vljm (B \vljm C)\Rightarrow y \colon (A \vljm B) \vljm (A \vljm C)}
{\vlin {\sir}
{}
{x \le y, y \le z, y \colon A \vljm (B \vljm C), z \colon A \vljm B \Rightarrow z \colon A \vljm C}
{\vlin {\trans}
{}
{x \le y, y \le z, z \le w, y \colon A \vljm (B \vljm C), z \colon A \vljm B, w \colon A \Rightarrow w \colon C}
{\vliin {\sil}
{}
{x \le y, y \le z, z \le w, y \le w, y \colon A \vljm (B \vljm C), z \colon A \vljm B, w \colon A \Rightarrow w \colon C}
{\vlin {\refl}
{}
{x \le y, y \le z, z \le w, y \le w, y \colon A \vljm (B \vljm C), z \colon A \vljm B, w \colon A \Rightarrow w \colon C, w \colon A}
{\vlin {\ids}
{}
{x \le y, y \le z, z \le w, y \le w, w \le w, y \colon A \vljm (B \vljm C), z \colon A \vljm B, w \colon A \Rightarrow w \colon C, w \colon A}
{\vlhy {}}}}
{\vlin {\trans}
{}
{x \le y, y \le z, z \le w, y \le w, y \colon A \vljm (B \vljm C), z \colon A \vljm B, w \colon A, w \colon B \vljm C \Rightarrow w \colon C}
{\vliin {\sil}
{}
{x \le y, y \le z, z \le w, y \le w, w \le w, y \colon A \vljm (B \vljm C), z \colon A \vljm B, w \colon A, w \colon B \vljm C \Rightarrow w \colon C}
{\vliin {\sil}
{}
{x \le y, y \le z, z \le w, y \le w, w \le w, y \colon A \vljm (B \vljm C), z \colon A \vljm B, w \colon A, w \colon B \vljm C \Rightarrow w \colon C, w \colon B}
{\vlin {\ids}
{}
{x \le y, y \le z, z \le w, y \le w, w \le w, y \colon A \vljm (B \vljm C), z \colon A \vljm B, w \colon A, w \colon B \vljm C \Rightarrow w \colon C, w \colon B, w \colon A}
{\vlhy {}}}
{\vlin {\ids}
{}
{x \le y, y \le z, z \le w, y \le w, w \le w, y \colon A \vljm (B \vljm C), z \colon A \vljm B, w \colon A, w \colon B \vljm C, w \colon B \Rightarrow w \colon C, w \colon B}
{\vlhy {}}}}
{\vlin {\ids}
{}
{x \le y, y \le z, z \le w, y \le w, w \le w, y \colon A \vljm (B \vljm C), z \colon A \vljm B, w \colon A, w \colon B \vljm C, w \colon C \Rightarrow w \colon C}
{\vlhy {}}}}}}}}}
}$
}


\end{turn}

\end{center}

\vspace{3mm}

\newpage
We show that we can \textbf{simulate modus ponens} using the cut rule where cut is:

\begin{center}

$\vliinf{\cut}{}{\Gone, \Gtwo, \Left \Rightarrow \Right}{\Gone, \Left \Rightarrow \Right, z \colon C}{\Gtwo, \Left, z \colon C \Rightarrow \Right}$

\end{center}

The modus ponens rule $\vliinf{}{}{B}{A}{A \vljm B}$ is simulated by:

\vspace{3mm}


\begin{center}

\scalebox{0.9}{

$\vlderivation {
\vliin{\cut}
{}
{\Rightarrow x \colon B}
{\vlin {w}
{}
{\Rightarrow x \colon A, x \colon B}
{\vlhy {\Rightarrow x \colon A}}}
{\vliin {\cut}
{}
{x \colon A \Rightarrow x \colon B}
{\vlin {w}
{}
{x \colon A \Rightarrow x \colon B, x \colon A \vljm B}
{\vlhy {\Rightarrow x \colon A \vljm B}}}
{\vlin {\refl}
{}
{x \colon A, x \colon A \vljm B \Rightarrow x \colon B}
{\vliin {\sil}
{}
{x \le x, x \colon A, x \colon A \vljm B \Rightarrow x \colon B}
{\vlin {\ids}
{}
{x \le x, x \colon A, x \colon A \vljm B \Rightarrow x \colon B, x \colon A}
{\vlhy {}}}
{\vlin {\ids}
{}
{x \le x, x \colon A, x \colon A \vljm B, x \colon B \Rightarrow x \colon B}
{\vlhy {}}}}}}
}$
} 
\end{center}



We can show that we can \textbf{simulate necessitation} where necessitation rule is:
If there exists a proof of A, then there exists a proof of $\square A$.

In other words, we want to prove:

\begin{lemma}
If there exists a proof $\vlderivation {\vlpd{\Done}{}{\Rightarrow z \colon A}}$ then there exists a proof $\vlderivation { \vlpd{\Dtwo}{}{\Rightarrow x \colon \square A}}$
\end{lemma}

\begin{proof}

We can show that:

wrong... to be completed

\begin{center}

If $\vlderivation {\vlpd {}{}{\Rightarrow z \colon A}}$ \hspace{4mm} $\rightsquigarrow$

\end{center}

\begin{center}
$\vlderivation{\vlin{\sbr}{}{\Rightarrow x \colon \square A}{\vlpd {\Dwone}{}{x \le y, y$R$z \Rightarrow z \colon A}}}$   where $\Dtwo = \vlderivation {\vlpd {\Dwone}{}{x \le y, y$R$z \Rightarrow z \colon A}}$ 

\end{center}

\end{proof}

\newpage
\begin{lemma}
If there exists a proof  $\vlderivation {\vlpd{\D}{}{\G, \Left \Rightarrow \Right}}$ then there exists a proof $\vlderivation {\vlpd {\Dw}{}{\G, x$R$y, u \le v, \Left, z \colon A \Rightarrow \Right, w \colon B}}$

\end{lemma}

\vspace{3mm}

\begin{proof}
 By induction on the height of $\D$.

For a proof of height 1:

if $\D$ = $\vlderivation {\vlin{\ids}{}{\G, \Left, x \le y, x \colon a \Rightarrow \Right, y \colon a}{\vlhy {}}}$ then we take $\Dw$ to be $\vlderivation {\vlin{\ids}{}{\G, \Left, x \le y, x$R$y, u \le v, x \colon a, z \colon A \Rightarrow \Right, w \colon B, y \colon a}{\vlhy {}}}$.
\vspace{2mm}

For a proof $\D$ of height greater than 1:

\begin{center}

$\vlderivation{\vlin {$r$}{}{\G, \Left \Rightarrow \Right}{\vlpd {\Done}{}{\G', \Left' \Rightarrow \Right'}}}$

\end{center}

Then by induction hypothesis there exists a proof 

\begin{center}

$\vlderivation {\vlpd {\Dwone}{}{\G', x$R$y, u \le v, \Left', z \colon A \Rightarrow \Right', w \colon B}}$

\end{center}

Therefore, we have the proof 

\begin{center}

$\Dw = \vlderivation {\vlin{}{}{\G, x$R$y, u \le v, \Left, z \colon A \Rightarrow \Right, w \colon B}{\vlpd {\Dwone}{}{\G', x$R$y, u \le v, \Left', z \colon A \Rightarrow \Right', w \colon B}}}$

\end{center}

\end{proof}
\newpage
\begin{lemma}
The following rule is admissible in labIK:

\begin{center}

$\vlderivation {\vlin{\idg}{}{\G, x \le y,  \Left, x \colon A \Rightarrow \Right, y \colon A }{\vlhy {}}}$

\end{center}

\end{lemma}

\vspace{3mm}

\begin{proof}  By induction on the size of $A$.

\begin{itemize}
\item{$A=a$}: we have $\vlderivation {\vlin{\idg}{}{\G, x \le y,  \Left, x \colon a \Rightarrow \Right, y \colon a }{\vlhy {}}}$ which is the same rule $\vlderivation {\vlin{\ids}{}{\G, x \le y,  \Left, x \colon a \Rightarrow \Right, y \colon a }{\vlhy {}}}$ that we have in the system labIK.

\item{$A= \vls(A.B)$:}

\begin{center}
$\vlderivation{\vliin{\svrig}{}{\G, x \le y, x\colon \vls(A.B), \Left \Rightarrow \Right, y \colon \vls(A.B)}{\vlin{\svlef}{}{\G, x \le y, x\colon \vls(A.B), \Left \Rightarrow \Right, y \colon A}{\vlin {}{}{\G, x \le y, x\colon A,x \colon B, \Left \Rightarrow \Right, y \colon A}{\vlhy {$By induction hypothesis, size($A$)$\le n$$}}}}{\vlin{\svlef}{}{\G, x \le y, x\colon \vls(A.B), \Left \Rightarrow \Right, y \colon B}{\vlin {}{}{\G, x \le y, x\colon A,x \colon B, \Left \Rightarrow \Right, y \colon B}{\vlhy {$By induction hypothesis, size($B$)$\le n}}}}}$
\end{center}


\item{$A = \vls[A.B]$:}

\begin{center}
$\vlderivation { \vlin{\sorig}{}{\G, x\le y, x \colon \vls[A.B], \Left \Rightarrow \Right, y \colon \vls[A.B]}{\vliin {\solef}{}{\G, x\le y, x \colon \vls[A.B], \Left \Rightarrow \Right, y \colon A, y \colon B}{\vlin {}{}{\G, x\le y, x \colon A, \Left \Rightarrow \Right, y \colon A, y \colon B}{\vlhy {$By induction hypothesis, size($A$)$\le n}}}{\vlin {}{}{\G, x\le y, x \colon B, \Left \Rightarrow \Right, y \colon A, y \colon B}{\vlhy {$By induction hypothesis, size($B$)$\le n}}}}}$
\end{center}

\item{$A = \square A$}

\begin{center}
$\vlderivation {\vlin {\sbr}
{}
{\G, x \le y, \Left, x \colon \square A \Rightarrow \Right, y \colon \square A}{\vlin {\sbl}
{}
{\G,\Left, x \le y, x \le z, z$R$w, x \colon \square A \Rightarrow \Right, w \colon A }
{\vlin {\refl}
{}
{\G,\Left, x \le y, x \le z, z$R$w, x \colon \square A, w \colon A \Rightarrow \Right, w \colon A }
{\vlin {}
{}
{\G,\Left, x \le y, x \le z, w \le w, z$R$w, x \colon \square A, w \colon A \Rightarrow \Right, w \colon A }
{\vlhy {$By induction hypothesis, size($A$)$\le n}}}}}}$
\end{center}

\item{$A= \lozenge A$}

\begin{center}

$\vlderivation{\vlin {\sdl}
{}
{\G, \Left, x \le y, x \colon \lozenge A \Rightarrow \Right, y \colon \lozenge A}
{\vlin {\ftwo}
{}
{\G, \Left, x \le y, x$R$z, z \colon A \Rightarrow \Right, y \colon \lozenge A}
{\vlin {\sdr}
{}
{\G, \Left, x \le y, z \le u, x$R$z, y$R$u, z \colon A \Rightarrow \Right, y \colon \lozenge A}
{\vlin {}
{}
{\G, \Left, x \le y, z \le u, x$R$z, y$R$u, z \colon A \Rightarrow \Right, y \colon \lozenge A, u \colon A}
{\vlhy {$By induction hypothesis, size($B$)$\le n}}}}}}$
\end{center}

\end{itemize}
\end{proof}

\newpage

\section{Completeness using Simpson system}

The main goal of this section is to prove completeness for our system labIK using the Simpson system (\textbf{Figure 3}). The idea comes from knowing that the Simpson system is a \emph{Cut-free system}, so this proof lets us know that our system is complete without the cut rule.

\begin{figure}[h]

\begin{center}

$\vlderivation { \vlin {\ids}{}{\G, \Left, x \colon a \Rightarrow x\colon a}{\vlhy {}}}$ \hspace{7mm} $\vlderivation { \vlin {\sbot}{}{\G, \Left, x \colon \bot \Rightarrow z\colon A}{\vlhy {}}}$

\vspace{4mm}

$\vlderivation {\vlin {\svlef}{}{\G, \Left, x \colon \vls(A.B) \Rightarrow z \colon C}{\vlhy {\G, \Left, x \colon A, x \colon B \Rightarrow z \colon C}}}$
\hspace{7mm}$\vlderivation { \vliin {\svrig}{}{\G, \Left, \Rightarrow x \colon \vls(A.B)}{\vlhy {\G, \Left \Rightarrow x \colon A }}{\vlhy {\G, \Left \Rightarrow x \colon B}}}$

\vspace{4mm}


$\vlderivation {\vliin {\solef}{}{\G, \Left, x \colon \vls[A.B] \Rightarrow z \colon C}{\vlhy {\G, \Left, x \colon A \Rightarrow z \colon C}}{\vlhy {\G, \Left, x \colon B \Rightarrow z \colon C}}}$
\hspace{7mm}$\vlderivation { \vlin{\sorone}{}{\G, \Left \Rightarrow x \colon \vls[A.B]}{\vlhy {\G, \Left \Rightarrow x \colon A}}}$
\hspace{7mm}$\vlderivation { \vlin {\sotwo}{}{\G, \Left \Rightarrow x \colon \vls[A.B]}{\vlhy {\G, \Left \Rightarrow x \colon B}}}$

\vspace{4mm}

$\vlderivation {\vliin{\sil}{}{\G, \Left, x \colon A \vljm B \Rightarrow z \colon C}{\vlhy {\G, \Left \Rightarrow x \colon A}}{\vlhy {\G, \Left, x \colon B \Rightarrow z \colon C}}}$
\hspace{7mm}$\vlderivation {\vlin{\sir}{}{\G,  \Left, x \colon A \Rightarrow x \colon B}{\vlhy {\G, \Left, x \colon A \Rightarrow x \colon B}}}$

\vspace{4mm}

$\vlderivation { \vlin {\sbl}{}{\G, x$R$y, \Left, x \colon \square A \Rightarrow z\colon B}{\vlhy {\G, x$R$y, \Left, x \colon \square A, y \colon A \Rightarrow z\colon B}}}$
\hspace{7mm}$\vlderivation { \vlin {\sbr}{y$ is fresh$}{\G, \Left \Rightarrow x \colon \square A}{\vlhy {\G, x$R$y, \Left \Rightarrow y \colon A}}}$

\vspace{4mm}

$\vlderivation { \vlin{\sdl}{y$ is fresh$}{\G, \Left, x \colon \lozenge A \Rightarrow z \colon B}{\vlhy {\G, x$R$y, \Left, y \colon A \Rightarrow z \colon B}}}$
\hspace{7mm}$\vlderivation {\vlin {\sdr}{}{\G,x$R$y, \Left \Rightarrow x \colon \lozenge A}{\vlhy {\G, x$R$y, \Left \Rightarrow y \colon A }}}$



\end{center}
\caption{System labIKs}
\end{figure}

\begin{proof}
By case analysis. Most of the rules in labIKs are the same as rules in the system labIK except for the following:

\begin{center}

$\vlderivation { \vlin {\ids}{}{\G, \Left, x \colon A \Rightarrow x \colon a}{\vlhy {}}}$ \hspace{4mm}  $  \rightsquigarrow$  \hspace{4mm} $\vlderivation{\vlin {\refl}{}{\G, \Left, x \colon a \Rightarrow x \colon a}{\vlin {\ids}{}{\G,x \le x, \Left, x\colon a \Rightarrow x \colon a}{\vlhy {}}}}$

\vspace{5mm}

$\vlderivation {\vlin {\sorone}{}{\G, \Left \Rightarrow x \colon \vls[A.B]}{\vlpd {\Done}{}{\G, \Left \Rightarrow x \colon A}}}$ or $\vlderivation {\vlin {\sotwo}{}{\G, \Left \Rightarrow x \colon \vls[A.B]}{\vlpd {\Done}{}{\G, \Left \Rightarrow x \colon B}}}$ \hspace{4mm} $\rightsquigarrow$ \hspace{4mm} $\vlderivation {\vlin {\sorig}{}{\G, \Left \Rightarrow x \colon \vls[A.B]}{\vlpd{\Dwone}{}{\G, \Left \Rightarrow x \colon A, x \colon B}}}$

\vspace{5mm}

$\vlderivation { \vliin {\sil}{}{\G, \Left, x \colon A \vljm B \Rightarrow z \colon C}{\vlpd {\Done}{}{\G, \Left \Rightarrow x \colon A}}{\vlpd {\Dtwo}{}{\G, \Left, x \colon B \Rightarrow z \colon C}}}$ \hspace{4mm} $\rightsquigarrow$ \hspace{4mm} $\vlderivation {\vlin {\refl}{}{\G, \Left, x \colon A \vljm B \Rightarrow z \colon C}{\vliin {\sil}{}{\G, \Left, x\le x, x \colon A \vljm B \Rightarrow z \colon C}{\vlpd {\Dwone}{}{\G, \Left, x \le x, x \colon A \vljm B \Rightarrow x\colon A}}{\vlpd {\Dwtwo}{}{\G, \Left, x \le x, x \colon B \Rightarrow z \colon C}}}}$

\vspace{5mm}
$\vlderivation {\vlin {\sir}{}{\G, \Left \Rightarrow x \colon A \vljm B}{\vlpd {\Done}{}{\G, \Left, x \colon A \Rightarrow x \colon B}}}$ \hspace{4mm} $\rightsquigarrow$ \hspace{4mm} $\vlderivation {\vlin {\sir}{}{\G, \Left \Rightarrow x \colon A \vljm B}{\vlpd {\Dwone}{}{\G, \Left, x \le x, x \colon A \Rightarrow x \colon B}}}$

\vspace{5mm}

$\vlderivation {\vlin {\sbl}{}{\G, \Left, x$R$y, x \colon \square A \Rightarrow z \colon B}{\vlpd {\Done}{}{\G, \Left, x$R$y, x \colon \square A, y \colon A \Rightarrow z \colon B}}}$ \hspace{4mm} $\rightsquigarrow$ \hspace{4mm} $\vlderivation {\vlin {\refl}{}{\G, \Left, x$R$y, x \colon \square A \Rightarrow z \colon B}{\vlin {\sbl}{}{\G, \Left, x \le x, x$R$y, x \colon \square A \Rightarrow z \colon B}{\vlpd {\Dwone}{}{\G, \Left, x \le x, x$R$y, x \colon \square A, y \colon A \Rightarrow z \colon B}}}}$

\vspace{5mm}

$\vlderivation {\vlin {\sbr}{}{\G, \Left \Rightarrow x \colon \square A}{\vlpd {\Done}{}{\G, \Left, x$R$y \Rightarrow y \colon A}}}$ \hspace{4mm} $\rightsquigarrow$ \hspace{4mm} $\vlderivation {\vlin {\sir}{}{\G, \Left \Rightarrow x \colon \square A}{\vlpd {\Dwone}{}{\G, \Left, x \le x, x$R$y \Rightarrow y \colon A}}}$

\vspace{5mm}

$\vlderivation {\vlin {\sdr}{}{\G, x$R$y, \Left \Rightarrow x \colon \lozenge A}{\vlpd {\Done}{}{\G, x$R$y, \Left \Rightarrow y \colon A}}}$ \hspace{4mm} $\rightsquigarrow$ \hspace{4mm} $\vlderivation {\vlin {\sdr}{}{\G, x$R$y, \Left \Rightarrow x \colon \lozenge A}{\vlpd {\Dwone}{}{\G, x$R$y, \Left \Rightarrow x \colon \lozenge A, y \colon A}}}$

\end{center}

\end{proof}

\section{System labIK + gklmn}

\begin{center}
\textbf{Axiom gklmn: $\lozenge^{k} \square^{l} A \vljm \square^{m}\lozenge^{n} A$} 
\end{center}

$\star$ \emph{Classical case} \hspace{3mm} $\rightsquigarrow$ \hspace{3mm}$\forall x,y,z ( x$R$^{k}y \vlan x$R$^{m}z \rightarrow \exists u y$R$^{l}u \vlan z$R$^{n}u)$ 

$\star$ \emph{Intuitionistic case} \hspace{3mm} $\rightsquigarrow$ \hspace{3mm} $\forall x,y,z((x$R$^{k}y \vlan x$R$^{m}z) \vljm \exists y' (y \le y' \vlan \exists u (y'$R$^{l}u \vlan z$R$^{n} u)))$\\


We add the axiom gklmn (gklmn for intuitionistic case) and we create a new sequent rule for our system labIK in order to capture this axiom:

\begin{center}

$\vlderivation { \vlin {\gklmn}{y', u$ fresh$}{\G, x$R$^{k}$$y, x$R$^{m}z, \Left \Rightarrow \Right}{\vlhy {\G, y \le y', x$R$^{k}y, x$R$^{m}z, y'$R$^{l}u, z$R$^{n}u, \Left\Rightarrow \Right}}}$


\end{center}

\subsection{Completeness}

\begin{proof}

As a result of section \emph{7. Syntactic Completeness Proof} we know that labIK+cut is complete. Therefore, we just need to show that gklmn is provable.

First we want to show completeness for \emph{k = l = m = n = 1}.

\begin{center}
$\vlderivation {\vlin {\sir}
{y$ fresh$}
{\Rightarrow x \colon \lozenge \square A \vljm \square \lozenge A}
{\vlin {\sbr}
{z, w $ fresh $}
{x \le y, y \colon \lozenge \square A \Rightarrow y \colon \square \lozenge A}
{\vlin {\sdl}
{u$ fresh$}
{x \le y, y \le z, z$R$w, y \colon \lozenge \square A \Rightarrow w \colon \lozenge A}
{\vlin {\ftwo}
{t$ fresh$}
{x \le y, y \le z, z$R$w, y$R$u, u \colon \square A \Rightarrow w \colon \lozenge A}
{\vlin {\gklmn}
{t', j$ fresh$}
{x \le y, y \le z, u \le t, z$R$w, y$R$u, z$R$t, u \colon \square A \Rightarrow w \colon \lozenge A}
{\vlin {\sdr}
{}
{x \le y, y \le z, u \le t, t \le t', z$R$w, y$R$u, z$R$t, t'$R$j, w$R$j, u \colon \square A \Rightarrow w \colon \lozenge A}
{\vlin {\trans}
{}
{x \le y, y \le z, u \le t, t \le t', z$R$w, y$R$u, z$R$t, t'$R$j, w$R$j, u \colon \square A \Rightarrow w \colon \lozenge A, j \colon A}
{\vlin {\sbl}
{}
{x \le y, y \le z, u \le t, t \le t', u \le t', z$R$w, y$R$u, z$R$t, t'$R$j, w$R$j, u \colon \square A \Rightarrow w \colon \lozenge A, j \colon A}
{\vlin {\refl}
{}
{x \le y, y \le z, u \le t, t \le t', u \le t', z$R$w, y$R$u, z$R$t, t'$R$j, w$R$j, u \colon \square A, j \colon A \Rightarrow w \colon \lozenge A, j \colon A}
{\vlin {\ids}
{}
{x \le y, y \le z, u \le t, t \le t', u \le t',j \le j, z$R$w, y$R$u, z$R$t, t'$R$j, w$R$j, u \colon \square A, j \colon A \Rightarrow w \colon \lozenge A, j \colon A}
{\vlhy {}}}}}}}}}}}}$

\end{center}

In order to prove the general case we need to use the rules from \textbf{Lemma 4}.

\emph{\textbf{To be continued}}
\end{proof}


\begin{lemma} The following rules are admissible in labIK:
\begin{enumerate}

\item{$\vlderivation {\vlin {\boxlk}{}{\G, \Left, x(\le \circ $R$)^{k}y, x \colon \square^{k} A\Rightarrow \Right}{\vlhy {\G, \Left, x(\le \circ $R$)^{k}y, x \colon \square^{k} A, z \colon A \Rightarrow \Right}}}$}

\item{$\vlderivation{\vlin {\boxk}{}{\G, \Left \Rightarrow \Right, x \colon \square^{k} A}{\vlhy {\G, x(\le \circ $R$)^{k}y,\Left \Rightarrow \Right, y \colon A}}}$}

\item{$\vlderivation { \vlin {\diamk}{}{\G, \Left, x \colon \lozenge^{k} A \Rightarrow \Right}{\vlhy {\G, x$R$^{k}y, \Left, y \colon A \Rightarrow \Right}}}$ }


\item{$\vlderivation { \vlin {\diamrk}{}{\G, \Left, x$R$^{k}y \Rightarrow \Right, x \colon \lozenge^{k}A}{\vlhy {\G, \Left, x$R$^{k}y \Rightarrow \Right, x \colon \lozenge^{k}A, y \colon A}}}$}

\end{enumerate}

\textbf{Probably F2g???}
\end{lemma}

\begin{proof}

By induction in k.

\emph{\textbf{To be continued}}

\end{proof}

\textbf{TO DO:}

\textbf{- NECESSITATION: lemma... is it ok?}

\textbf{- SOUNDNESS}

\textbf{- gklmn}

\textbf{- Improve proofs}

\textbf{- Cut elimination}

\newpage
\begin{thebibliography}{4}
\bibitem{Kripke}
Saul A. Kripke. Semantical analysis of modal logic I: Normal modal propositional calculi. 
\textit{Zeitshrift f ̈ur mathematische Logik and Grundlagen der Mathematik,} 9(5-6):67–96, 1963.

\bibitem{Negri}
Sara Negri. \textit{Proof analysis in modal logics. Journal of Philosophical Logic,} 34:507–544, 2005.

\bibitem{Fischer}
Gisèle Fischer Servi. Axiomatizations for some intuitionistic modal logics. \textit{Rendiconti del Seminario Matematico dell’ Universit`a Politecnica di Torino,} 42(3):179–194, 1984.

\bibitem{Plotin}
Gordon D. Plotkin and Colin P. Stirling. A framework for intuitionistic modal logic. In
J. Y. Halpern, editor, \textit{1st Conference on Theoretical Aspects of Reasoning About Knowledge.}
Morgan Kaufmann, 1986.
\end{thebibliography}


\end{document}