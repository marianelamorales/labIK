%%%%%%%%%%%%%%%%%%%%%%%%%%%%%%%%%%%%%%%%%%%%%%%%%%%%%%%%%
%%%%%%%%%%%%%%%%%%%%%%%%%%%%%%%%%%%%%%%%%%%%%%%%%%%%%%%%%
%%%%%%%%%%%%%%%%%%%%%%%%%%%%%%%%%%%%%%%%%%%%%%%%%%%%%%%%%

\section{Extensions}\label{sec:ext}

The main goal of this section is to generate stronger logics adding new axioms to our system. We say \emph{stronger logic} to refer to the fact that we are restricting the class of frames we want to consider, imposing some restrictions on the accessibility relation. 

In the \emph{fully labelled} framework, we are able for the first time to consider the logics defined by \emph{one-sided intuitionistic Scott-Lemmon axioms}:
\begin{equation}
\agklmn\colon\fm{\DIA^{k} \BOX^{l} A \IMP \BOX^{m}\DIA^{n} A}
\end{equation}
for any natural numbers $k,l,m,n$.

\begin{figure}
	$$
	\xymatrix{
		& y' \ar@{.>}[ddr]^{R^l} \\
		& y \ar@{.>}[u]^{\le} \\
		x \ar@{->}[ur]^{R^k}\ar@{->}[dr]_{R^m} && u \\
		& z \ar@{.>}[ur]_{R^n}
	}
	$$
	\caption{The intuitionistic klmn-incestuality condition}
	\label{fig:gklmn}
\end{figure}

Indeed, they are known to obey a strong correspondence with the class of frames satisfying the condition illustrated on Figure~\ref{fig:gklmn}, which we call by analogy to the classical case, \emph{intuitionistic klmn-incestuality condition}.

\begin{theorem}[\cite{plotkin:stirling:86}]\label{thm:gklmn-correspondence}
	An intuitionistic modal frame $\langle W, \rel, \le \rangle$ validates $\fm{\DIA^k\BOX^l A \IMP \BOX^m\DIA^n A}$ if and only if the frame satisfies:
	
	if $\lb w R^{k} \lb u$ and $\lb w R^{m} \lb v$ then there exists $\lb{u'}$ such that $\futs{u}{u'}$ and there exists $\lb x$ such that $\lb{u'}R^{l}\lb x$ and $\lb v R^{n} \lb x$.
\end{theorem}

\todo{explain notation $\rel^n$ and what when $n=0$?}

%\sonia{I don't think we need the actual \emph{correspondence} result. Soundness and completeness is enough for us}.


%Writing this axiom in both classical and intuitionistic logic in first-order language, we obtain:
%
%$\star$ \emph{Classical case} \hspace{2.2mm} $\rightsquigarrow$ \hspace{3.7mm}$\forall x,y,z ( xR^{k}y \vlan xR^{m}z \rightarrow \exists u yR^{l}u \vlan zR^{n}u)$ 
%
%$\star$ \emph{Intuitionistic case} $\rightsquigarrow$  $\forall x,y,z((xR^{k}y \vlan xR^{m}z) \IMP \exists y' (y \le y' \vlan \exists u (y'R^{l}u \vlan zR^{n} u)))$\\
%
%\sonia{in the intuitionistic case, with the $\le$ proposition, the formula is geometric.}

Following again the axiom-as-rule idea, to have a sound and complete system adding the axiom $\agklmn$ to the system $\labIKp$, we introduce the $\gklmn$ rule, for any natural numbers $k,l,m,n$.

\begin{equation}
  \label{eq:gklmn}
  \vlinf{\gklmn}{\lb{y'}, \lb {u} \mbox{ fresh}}{\B, \lb x \rel^{k} \lb y, \lb x \rel^{m} \lb z, \Left \SEQ \Right}{\B, \lb x \rel^{k} \lb y, \lb x \rel^{m} \lb z, \futs{y}{y'}, \lb{y'}\rel^{l} \lb u, \lb z \rel^{n} \lb u, \Left\SEQ\Right}
\end{equation}

\todo{actually when one of the number is 0 the u might not be fresh after substitution!! see $g_{1110}$  below}

\begin{remark}
  Observe, that in the rule $\gklmn$ above, $\lb xR^k\lb y$ is an abbrevation for $\lb xR\lb{x_1},\lb{x_1}R\lb{x_2},\ldots,\lb{x_{k-1}}R\lb y$, and the labels $\lb x,\lb{x_1},\ldots,\lb{x_{k-1}},\lb y$ do \emph{not} have to be distinct, and similarly for $\lb x R^{m} \lb z$.
(In case some of them coincide, they are not reapeated since $\B$ is a \emph{set} of atoms.)
   However, in the premise, the  expression $\lb{y'}R^{l} \lb u$ stands for  $\lb{y'}R\lb{u_1},\ldots,\lb{u_{l-1}}R \lb u$ where all $\lb{u_1},\ldots,\lb{u_{l-1}},\lb u$ are fresh (and therefore pairwise distinct), and similarly for $\lb z R^{n} \lb u$.
  %
  For example, the derivation below requires the (valid) application of $\rn{g_{1110}}$ 
  $$\vlinf{\rn g_{1110}}{\lb{y'} \mbox{ fresh}}{\B, \lb x \rel \lb y, \lb x \rel \lb z, \Left \SEQ \Right}{\B, \lb x \rel \lb y, \lb x \rel \lb z, \futs{y}{y'}, \lb{y'}\rel \lb z, \Left\SEQ\Right}
  \quad\mbox{as}\quad
  \vlinf{\rn g_{1110}}{\lb{y'} \mbox{ fresh}}{\B, \accs xz, \Left \SEQ \Right}{\B, \lb x \rel \lb z, \futs{z}{y'}, \lb{y'}\rel \lb z, \Left\SEQ\Right}
  $$
  i.e., the case where $\lb y=\lb z$, in order to derive $ \fm{\BOX(\BOX A \IMP A)}$.
  
  \todo{this is the crucial point: the original rule has $y\le y'$ but the "contracted" instance has $z\le y'$}
  
  	$$
		\vlderivation{
			\vlin{\rlabrn\BOX}{}{\SEQ \labels{x}{\BOX(\BOX A \IMP A)}}{
				\vlin{\rlabrn\IMP}{\lb v\mbox{ fresh}}{\futs xy, \accs yz \SEQ \labels{z}{\BOX A \IMP A}}{
					\vlin{\rn F_1}{\lb u\mbox{ fresh}}{\futs xy, \accs yz, \futs zv, \labels{v}{\BOX A} \SEQ \labels{v}{A}}{
						\vlin{\rn g_{1110}}{ \lb{w}\mbox{ fresh}}{\futs xy, \accs yz, \futs zv, \futs yu, \accs uv, \labels{v}{\BOX A} \SEQ \labels{v}{A}}{
							\vlin{\llabrn\BOX}{}{\futs xy, \accs yz, \futs zv, \futs yu, \accs uv, \futs{v}{w}, \accs{w}{v}, \labels{v}{\BOX A} \SEQ \labels{v}{A}}{
								\vlin{\rn{id}}{}{\futs xy, \accs yz, \futs zv, \futs yu, \accs uv, \futs{v}{w}, \accs{w}{v}, \labels{v}{\BOX A}, \labels{v}{A} \SEQ \labels{v}{A}}{
									\vlhy{}
								}
							}
						}
					}
				}
			}
		}
		$$
\end{remark}


%\todo{it is maybe a bit fishy to say this ``freshness-thing'' only in a remark...}

%\todo{move this comment to the system (it also applies to boxL, F1,F2...), and then refer to it again here}\lutz{I tried to do that, but I could not find a good place, and it does not really make sense. So, I'd just leave it here.}

		\review{In Section 7, if the rules G klmn are added without their contracted instances, the calculus will not be complete. For example, in the classical case, Euclidean transitivity $\forall a,b,c (\accs{a}{b} \& \accs{a}{c} \IMP \accs{b}{c})$ has the contracted instance $\forall a,b (\accs{a}{b} \IMP \accs{b}{b})$ that allows for the derivation of $ \BOX(\BOX A \IMP A)$. This, however, can hardly be derived without the rule’s contracted instance.}
		\lutz{Done by the remark above?}

%We start proving that our new system is complete with the base case of $\mathsf{k = l = m = n = 1}$:
%
%\begin{theorem}
%	The system $\labIKp +\agklmn$ is complete for $\mathsf{k = l = m = n = 1}$.
%\end{theorem}
%
%\begin{proof}
%		\begin{smallequation*}
%			\vlderivation {\vlin {\rlabrn{\IMP}}{y \mbox{ fresh}}{\SEQ \labels{x}{\DIA \BOX A \IMP \BOX \DIA A}}{
%					\vlin {\rlabrn{\BOX}}{z, w \mbox{ fresh} }{\lseq{\futs xy}{\labels{y}{\DIA \BOX A}}{\labels{y}{\BOX \DIA A}}}{
%						\vlin {\llabrn{\DIA}}{u \mbox{ fresh}}{\lseq{\futs xy, \futs yz, \accs zw}{\labels{y}{\DIA \BOX A}}{\labels{w}{\DIA A}}}{
%							\vlin {\rn{F_2}}{t \mbox{ fresh}}{\lseq{\futs xy, \futs yz, \accs zw, \accs yu}{\labels{u}{\BOX A}}{\labels{w}{\DIA A}}}{
%								\vlin {\gklmn}{t', j \mbox{ fresh}}{\lseq{\futs xy, \futs yz, \futs ut, \accs zw, \accs yu, \accs zt}{\labels{u}{\BOX A}}{\labels{w}{\DIA A}}}{
%									\vlin {\rlabrn{\DIA}}{}{\lseq{\futs xy, \futs yz, \futs ut, \futs{t}{t'}, \accs zw, \accs yu, \accs zt, \accs{t'}{j}, \accs wj}{\labels{u}{\BOX A}}{\labels{w}{\DIA A}}}{
%										\vlin {\rn{trans}}{}{\lseq{\futs xy, \futs yz, \futs ut, \futs{t}{t'}, \accs zw, \accs yu, \accs zt, \accs{t'}{j}, \accs wj}{\labels{u}{\BOX A}}{\labels{w}{\DIA A}, \labels{j}{A}}}{
%											\vlin {\llabrn{\BOX}}{}{\lseq{\futs xy, \futs yz, \futs ut, \futs{t}{t'}, \futs{u}{t'}, \accs zw, \accs yu, \accs zt, \accs{t'}{j}, \accs wj}{\labels{u}{\BOX A}}{\labels{w}{\DIA A}, \labels{j}{A}}}{
%												\vlin {\rn{refl}}{}{\lseq{\futs xy, \futs yz, \futs ut, \futs{t}{t'}, \futs{u}{t'}, \accs zw, \accs yu, \accs zt, \accs{t'}{j}, \accs wj}{\labels{u}{\BOX A}, \labels{j}{A}}{\labels{w}{\DIA A}, \labels{j}{A}}}{
%													\vlin {\rn{id}}{}{\lseq{\futs xy, \futs yz, \futs ut, \futs{t}{t'}, \futs{u}{t'},\futs jj, \accs zw, \accs yu, \accs zt, \accs{t'}{j}, \accs wj}{\labels{u}{\BOX A}, \labels{j}{A}}{\labels{w}{\DIA A}, \labels{j}{A}}}
%													{\vlhy {}}}}}}}}}}}}
%		\end{smallequation*}
%\end{proof}
%
%\begin{smallequation*}
%	\vlderivation {
%		\vlin {\rlabrn{\IMP}}{\futs{x}{x'}}{\SEQ \labels{x}{\DIA \BOX A \IMP \BOX \DIA A}}{
%			\vlin {\rlabrn{\BOX}}{\futs{x'}{x''}, \accs{x''}{y''}}{\labels{x'}{\DIA \BOX A} \SEQ\labels{x'}{\BOX \DIA A}}{
%				\vlin {\llabrn{\DIA}}{\accs{x'}{z'}}{\labels{x'}{\DIA \BOX A}\SEQ\labels{y''}{\DIA A}}{
%					\vlin {\rn{F_2}}{\futs{z'}{z''}, \accs{x''}{z''}}{\labels{z'}{\BOX A}\SEQ\labels{y''}{\DIA A}}{
%						\vlin {\gklmn}{\futs{z''}{z'''}, \accs{z'''}{w''}, \accs{y''}{w''}}{\labels{z'}{\BOX A}\SEQ\labels{y''}{\DIA A}}{
%							\vlin {\rlabrn{\DIA}}{}{\labels{z'}{\BOX A}\SEQ\labels{y''}{\DIA A}}{
%								\vlin {\rn{trans}+\rn{refl}}{\futs{z'}{z'''},\futs{w''}{w''}}{\labels{z'}{\BOX A}\SEQ\labels{y''}{\DIA A}, \labels{w''}{A}}{
%									\vlin {\llabrn{\BOX}}{}{\labels{z'}{\BOX A}\SEQ\labels{y''}{\DIA A}, \labels{w''}{A}}{
%										\vlin {\rn{id}}{}{\labels{z'}{\BOX A}, \labels{w''}{A}\SEQ\labels{y''}{\DIA A}, \labels{w''}{A}}{
%											\vlhy {}}}}}}}}}}}
%\end{smallequation*}

We can then show that Theorem~\ref{thm:cutfree-compl} generalises nicely to $\labIKp$ with any $\gklmn$ rule to provide a sound and cut-free complete system for this family of logics.

\begin{theorem}\label{thm:cutfree-compl-gklmn}
	For any formula $A$, the following are equivalent.
	%
	\begin{enumerate}
		\item\label{1} $A$ is a theorem of $\IK + \fm{\DIA^k\BOX^l A \IMP \BOX^m\DIA^n A}$. 
		%
		\item\label{2} $A$ is provable in $\labIKp + \gklmn +\labrn{cut}$.
		\item\label{3} $A$ is provable in $\labIKp + \gklmn $.
		%
		\item\label{4} $A$ is valid in every birelational frame satisfying the klmn-incestuality property.
	\end{enumerate}
\end{theorem}

\begin{proof} 
	The proof is similar to the one of Theorem~\ref{thm:cutfree-compl}.
	\begin{itemize}
		\item $1 \implies 2$:
		%	For any formula $\fm A$. If $\fm A$ is a theorem of $\IK + \DIA^k\BOX^l A \IMP \BOX^m\DIA^n A$, then $A$ is a theorem of $\labIKp + \gklmn + \rn{cut}$.
		Same as Thm.~\ref{thm:completeness} with the additional derivation of $\fm{\DIA^k\BOX^l A \IMP \BOX^m\DIA^n A}$:
		\begin{smallequation*}
			\vlderivation{
				\vlin{\rlabrn\IMP}{}{\SEQ \labels{x}{\DIA^k\BOX^l A \IMP \BOX^m\DIA^n A}}{
					\vliq{\rlabrn\BOX}{}{\futs{x}{x_0}, \labels{x_0}{\DIA^k\BOX^lA} \SEQ \labels{x_0}{\BOX^m\DIA^n A}}{
						\vliq{\rn{F_1}}{}{\{\futs{x_i}{x_i'}, \accs{x_i'}{x_{i+1}}\}_{0\le i \le m-1}, \labels{x_0}{\DIA^k\BOX^l A} \SEQ \labels{x_m}{\DIA^n A}}{
							\vliq{\llabrn\DIA}{}{\futs{x_{m-2}'}{x_{m-2}''}, \accs{x_{m-2}''}{x_{m-1}'},\{\futs{x_i'}{x_i''},\accs{x_i''}{x_{i+1}''}\}_{0\le i \le m-2}, \labels{x_0}{\DIA^k\BOX^l A} \SEQ \labels{x_m}{\DIA^n A}}{
								\vliq{\rn{F_2}}{}{\accs{x_0}{y_1}, \{\accs{y_j}{y_{j+1}}\}_{1\le j \le k-1}, \labels{y_k}{\BOX^l A}\SEQ\labels{x_m}{\DIA^n A}}{
									\vlin {\gklmn}{}{\accs{x_0''}{y_1'},\{\accs{y_{j}'}{y_{j+1}'}\}_{1\le j \le k-1}, \labels{y_k}{\BOX^l A}\SEQ\labels{x_m}{\DIA^n A}}{
										\vliq {\rlabrn{\DIA}}{}{\futs{y_k'}{y_k''},\lb{y_k''}R^l\lb{w}, \lb{x_m}R^n\lb{w}, \labels{y_k}{\BOX^l A}\SEQ\labels{x_m}{\DIA^n A}}{
											\vlin {\rn{trans}+\rn{refl}}{}{\labels{y_k}{\BOX^l A}\SEQ\labels{x_m}{\DIA^n A},\labels{w}{A}}{
												\vliq {\llabrn{\BOX}}{}{\futs{y_k}{y_k''}, \ldots, \futs{w}{w}, \labels{y_k}{\BOX^l A}\SEQ\labels{x_m}{\DIA^n A},\labels{w}{A}}{
													\vlin {\rn{id}}{}{\labels{y_k}{\BOX^l A},\labels{w}{A}\SEQ\labels{x_m}{\DIA^n A},\labels{w}{A}}{
														\vlhy{}
													}
												}
											}
										}
									}
								}
							}
						}
					}
				}
			}
		\end{smallequation*}
		where we omit the accumulated relational context for space reason.
		
		\item $2 \implies 3$:
		To prove that the rule $\rn{cut}$ is admissible for $\labIKp + \gklmn$, it is enough to insert a case for the rule $\gklmn$ in the proof of Theorem~\ref{thm:cutfree-compl}, which
		is straightforward as the $\gklmn$ rule only manipulates the relational context.
		
		\item $3 \implies 4$:
		As we already proved the rules of $\labIKp$ sound in Theorem~\ref{thm:soundness}, we only need to prove that $\gklmn$ is sound.
		%
		By way of contradiction, assume that $\B, \futs{y}{y'}, \lb x R^{k} \lb y, \lb x R^{m} \lb z, \lb{y'}R^{l} \lb u, \lb z R^{n} \lb u, \Left\SEQ\Right$ is valid in any klmn-incestuous frame,
		but that there is such a model $\M$ and an interpretation $\inter\cdot$, such
		that $\M,\inter\cdot\not\Vdash\B, \lb x R^{k} \lb y, \lb x R^{m} \lb z, \Left \SEQ \Right$.
		%
		That means, $\inter x \rel_\M^k \inter y$, $\inter x \rel_\M^m\inter z$, for all $\labels{x}{A} \in \Left$, $\cforce \M xA$ and for all $\labels wB\in\Right$,
		$\cnforce \M wB$. 
		%
		Since $\M$ is klmn-incestuous, there exists $v,w\in W_\M$, such that $\inter y \le_\M \lb{v}$, $\lb{v} R_\M^l \lb{w}$, and $\inter z R_\M^n \lb{w}$.
		%
		Now let $\inter\cdot^*$ be the extension of $\inter\cdot$ such that $\inter{y'}^* = v$, $\inter{u}^* = w$, and $\inter\cdot^* = \inter\cdot$ otherwise.
		%
		Then, $ \nforce \M {\inter\cdot^*} {\B, \futs{y}{y'}, \lb x R^{k} \lb y, \lb x R^{m} \lb z, \lb{y'}R^{l} \lb u, \lb z R^{n} \lb u, \Left\SEQ\Right}$. Contradiction.
	\end{itemize}
	
	The proof is completed by appealing to Theorem~\ref{thm:gklmn-correspondence} used as $4 \implies 1$ to close the equivalence.
\end{proof}

%In order to have a completeness proof for the general case, we need to introduce the rules from the Lemma~\ref{lemma:admis}.
%
%\begin{lemma}\label{lemma:admis} The following rules are derivable in $\labIKp$:
%	\todo{notation $\le \circ R$ etc. needs to be defined}
%	\begin{enumerate}
%		\item{$\vlderivation {\vlin {\boxlk}{}{\B, \Left, x(\le \circ $R$)^{k}y, \labels{x}{\BOX^{k} A}\Rightarrow \Right}{\vlhy {\B, \Left, x(\le \circ $R$)^{k}y, \labels{x}{\BOX^{k} A}, \labels{z}{A} \Rightarrow \Right}}}$}
%		\item{$\vlderivation{\vlin {\boxk}{}{\B, \Left \Rightarrow \Right, \labels{x}{\BOX^{k} A}}{\vlhy {\B, x(\le \circ $R$)^{k}y,\Left \Rightarrow \Right, \labels{y}{A}}}}$}
%		\item{$\vlderivation { \vlin {\diamk}{}{\B, \Left, \labels{x}{\DIA^{k} A} \Rightarrow \Right}{\vlhy {\B, xR^{k}y, \Left, \labels{y}{A} \Rightarrow \Right}}}$ }
%		\item{$\vlderivation { \vlin {\diamrk}{}{\B, \Left, xR^{k}y \Rightarrow \Right, \labels{x}{\DIA^{k}A}}{\vlhy {\B, \Left, xR^{k}y \Rightarrow \Right, \labels{x}{\DIA^{k}A}, \labels{y}{A}}}}$}
%	\end{enumerate}
%\end{lemma}
%
%\def\lef#1{#1}
%\def\rig#1{#1^{\circ}}
%\newcommand{\wbri}[2]{[\strut^{#1} #2]}

%		\begin{smallequation*}\hspace*{-1.5cm}
%			\vlderivation{
%				\vlin{\rlabrn\IMP,\llabrn\AND}{}{\SEQ \labels{x}{(\DIA(\BOX(a \OR b) \AND \DIA a) \AND \DIA(\BOX(a \OR b) \AND \DIA b)) \IMP \DIA(\DIA a \AND \DIA b)}}{
%					\vlin{\llabrn\DIA, \llabrn\AND,\llabrn\DIA}{xRy,yRu,xRz,zRv}{\labels{x}{\DIA(\BOX(a \OR b) \AND \DIA a)}, \labels{x}{\DIA(\BOX(a \OR b) \AND \DIA b)} \SEQ \labels{x}{\DIA(\DIA a \AND \DIA b)}}{
%						\vlin{g 1111}{yRw,zRw}{\labels{y}{\BOX(a \OR b)}, \labels{u}{a}, \labels{z}{\BOX(a \OR b)}, \labels{v}{b} \SEQ \labels{x}{\DIA(\DIA a \AND \DIA b)}}{
%							\vliin{\llabrn\BOX, \llabrn\OR}{}{\labels{y}{\BOX(a \OR b)}, \labels{u}{a}, \labels{z}{\BOX(a \OR b)}, \labels{v}{b} \SEQ \labels{x}{\DIA(\DIA a \AND \DIA b)}}{
%%								\vlhtr{\pi_2}{\labels{y}{\BOX(a \OR b)}, \labels{u}{a}, \labels{z}{\BOX(a \OR b)}, \labels{v}{b}, \labels{w}{b} \SEQ \labels{x}{\DIA(\DIA a \AND \DIA b)}}									
%									\vliin{\rlabrn\DIA,\rlabrn\AND}{}{\labels{y}{\BOX(a \OR b)}, \labels{u}{a}, \labels{w}{b} \SEQ \labels{x}{\DIA(\DIA a \AND \DIA b)}}{
%										\vlin{\rlabrn\DIA}{}{\labels{y}{\BOX(a \OR b)}, \labels{u}{a}, \labels{w}{b} \SEQ \labels{y}{\DIA a}}{
%											\vlin{\rn{id}}{}{\labels{y}{\BOX(a \OR b)}, \labels{u}{a}, \labels{w}{b} \SEQ \labels{u}{a}}{
%												\vlhy{}
%											}
%										}
%									}{
%									\vlin{\rlabrn\DIA}{}{\labels{y}{\BOX(a \OR b)}, \labels{u}{a}, \labels{w}{b} \SEQ \labels{y}{\DIA b}}{
%										\vlin{\rn{id}}{}{\labels{y}{\BOX(a \OR b)}, \labels{u}{a}, \labels{w}{b} \SEQ \labels{w}{b}}{
%											\vlhy{}
%										}
%									}
%								}
%							}{								
%%								\vlhtr{\pi_1}{\labels{y}{\BOX(a \OR b)}, \labels{u}{a}, \labels{z}{\BOX(a \OR b)}, \labels{v}{b}, \labels{w}{a} \SEQ \labels{x}{\DIA(\DIA a \AND \DIA b)}}
%								\vliin{\rlabrn\DIA,\rlabrn\AND}{}{\labels{z}{\BOX(a \OR b)}, \labels{v}{b}, \labels{w}{a} \SEQ \labels{x}{\DIA(\DIA a \AND \DIA b)}}{
%									\vlin{\rlabrn\DIA}{}{\labels{z}{\BOX(a \OR b)}, \labels{v}{b}, \labels{w}{a} \SEQ \labels{z}{\DIA a}}{
%										\vlin{\rn{id}}{}{\labels{z}{\BOX(a \OR b)}, \labels{v}{b}, \labels{w}{a} \SEQ \labels{w}{a}}{
%											\vlhy{}
%											}
%										}
%									}{
%									\vlin{\rlabrn\DIA}{}{\labels{z}{\BOX(a \OR b)}, \labels{v}{b}, \labels{w}{a} \SEQ \labels{z}{\DIA b}}{
%										\vlin{\rn{id}}{}{\labels{z}{\BOX(a \OR b)}, \labels{v}{b}, \labels{w}{a} \SEQ \labels{v}{b}}{
%											\vlhy{}
%											}
%										}
%									}
%								}
%						}
%					}
%				}
%			}
%		\end{smallequation*}




%\begin{equation*}
%\xymatrix{
%	& \lb{z''} \ar@{->}[rrdd]^R \\
%	& \lb{z'} \ar@{.>}[u]^{\le} \ar@{->}[rd]^{R}\\
%	\lb{x'} \ar@{->}[rd]_R\ar@{->}[ru]^R  & \lb{y''} \ar@{->}[r]_R & \lb{u''} & \lb{v''}\\
%	& \lb{y'} \ar@{.>}[u]_{\le} \ar@{->}[rru]_R \\
%	\lb{x} \ar@{.>}[uu]^{\le}
%}
%\end{equation*}

As we realise this theorem might look rather abstract, we come back to the example of the transitivity frame condition evoked in the introduction.
%%
The frames that validate the axiom $\rn{4}_\rn\DIA \colon \fm{\DIA\DIA A \IMP \DIA A}$ are exactly the ones satisfying the condition:
($g_{2001}$) if $\accs wv$ and $\accs vu$, there exists $\lb{u'}$ s.t.~$\futs{u}{u'}$ and $\accs{w}{u'}$.
%%
The frames that validate the axiom $\rn{4}_\rn\BOX \colon \fm{\BOX A \IMP \BOX \BOX }$ are exactly the ones satisfying the condition:
($g_{0120}$) if $\accs wv$ and $\accs vu$, there exists $\lb{w'}$ s.t.~$\futs{w}{w'}$ and $\accs{w'}{u}$.


We can therefore obtain a sound and complete proof system for the intuitionistic modal logic $\IK$ extended with axiom $\rn{4}_\rn\DIA$ or $\rn{4}_\rn\BOX$, respectively, by specialising the rule scheme above as:
$$
\vlinf{\rn{g_{2001}}}{\text{\footnotesize $\lb{u'}$ fresh}}{\B, \accs wv, \accs vu, \Left \SEQ \Right}{\B, \accs wv, \accs vu, \futs{u}{u'}, \accs{w}{u'} , \Left \SEQ \Right}
\qquad\mbox{or}\qquad
\vlinf{\rn{g_{0120}}}{\text{\footnotesize $\lb{w'}$ fresh}}{\B, \accs wv, \accs vu, \Left \SEQ \Right}{\B, \accs wv, \accs vu, \futs{w}{w'}, \accs{w'}{u} , \Left \SEQ \Right}
$$

Indeed, they allow us to derive their correspond axioms as required above in the completeness proof:
$$\vlderivation{
	\vlin{\rlabrn\IMP}{}{\labels{x}{\DIA\DIA A \IMP \DIA A}}{
		\vliq{\llabrn\DIA}{}{\futs xw, \labels{w}{\DIA\DIA A} \SEQ \labels{w}{\DIA A}}{
			\vlin{\rn{g_{2001}}}{}{\futs xw, \accs wv, \accs vu, \labels{u}{A} \SEQ \labels{w}{\DIA A}}{
				\vlin{\rlabrn\DIA}{}{\futs xw, \accs wv, \accs vu, \futs{u}{u'}, \accs{w}{u'},  \labels{u}{A} \SEQ \labels{w}{\DIA A}}{
					\vlin{\labrn{id}}{}{\futs xw, \accs wv, \accs vu, \futs{u}{u'}, \accs{w}{u'},  \labels{u}{A} \SEQ \labels{w}{\DIA A}, \labels{u'}{A}}{
						\vlhy{}
					}
				}
			}
		}
	}
}
$$
as well as,
$$\vlderivation{
	\vlin{\rlabrn\IMP}{}{\labels{x}{\BOX A \IMP \BOX\BOX A}}{	
		\vliq{\rlabrn\BOX}{}{x \le w, \labels{w}{\BOX A} \SEQ \labels{w}{\BOX\BOX A}}{
			\vlin{\rn{F_1}}{}{x \le w, w \le w', {w' \rel v}, {v \le v'}, v' \rel u, \labels{w}{\BOX A} \SEQ \labels{u}{A}}{
				\vlin{\rn{g_{0120}}}{}{x \le w, w \le w', {w' \rel v}, {v \le v'}, v' \rel u, w' \le t, t \rel v', \labels{w}{\BOX A} \SEQ \labels{u}{A}}{
					\vlin{\rn{trans_\le}}{}{x \le w, w \le w', {w' \rel v}, {v \le v'}, v' \rel u, w' \le t, t \rel v', t \le t', t' \rel u, \labels{w}{\BOX A} \SEQ \labels{u}{A}}{
						\vlin{\llabrn\BOX}{}{x \le w, w \le w', w' \rel v, v \le v', v' \rel u, w' \le t, t \rel v', t \le t', t' \rel u, w \le t, \labels{w}{\BOX A} \SEQ \labels{u}{A}}{
							\vlin{\labrn{id}}{}{x \le w, w \le w', w' \rel v, v \le v', v' \rel u, w' \le t, t \rel v', w \le t, t \rel u, \labels{w}{\BOX A}, \labels{u}{\A} \SEQ \labels{u}{A}}{
								\vlhy{}
							}
						}
					}
				}
			}
		}
	}
}
$$
\todo{text colouring!}

	As another illustration of our system, we reconsider an example that was problematic in previous approaches to the logic $\IK + \rn{g_{1111}}$.
	%
	Indeed, the formula $\fm{\DIA(\BOX(a \OR b) \AND \DIA a) \AND \DIA(\BOX(a \OR b) \AND \DIA b)) \IMP \DIA(\DIA a \AND \DIA b)}$ is not a theorem of this logic, but would become provable if we were to directly add to our system the rule 
	$$\vlinf{}{}{\B, \accs wv, \accs vu, \Left \SEQ \Right}{\B, \accs wv, \accs vu, \accs{v}{x}, \accs{u}{x}, \Left \SEQ \Right}$$
	%
	corresponding to the directedness condition $\forall \lb{xyz}. ((\accs xy \AND \accs xz) \IMP\exists u. (\accs yu \AND \accs zu))$). (See details in Sec.6.3 of~\cite{simpson:phd}.)
	%
	By representing birelational semantics precisely with both $R$ and $\le$ relations and by adding the  rule $\rn{g_{1111}}$ defined above, we can no longer derive this formula, as illustrated by the representation of the failed proof search below:	
	\review{it does not help at all that the	applications of the rule $\rn{g_1111}$ are merged with $\llabrn\BOX$ and $\rlabrn\DIA$ in the example derivation.} 
	\marianela{I made this new version.
		Also I found a mistake in the other version: the first time g111 is used an $\rel$ is created as $\accs{y'}{y''}$ when it has to be $\futs{y'}{y''}$. Besides that, I added the fresh variables.}
	\begin{smallequation*}\hspace*{-1.2cm}
		\vlderivation{
			\vlin{\rlabrn\IMP,\llabrn\AND}{}{\SEQ\labels{x}{(\DIA(\BOX(a \OR b) \AND \DIA a) \AND \DIA(\BOX(a \OR b) \AND \DIA b)) \IMP \DIA(\DIA a \AND \DIA b)}}{
				\vlin{\llabrn\DIA, \llabrn\AND, \rlabrn\DIA}{\lb y' \mbox{ fresh}}{\futs{x}{x'}, \labels{x'}{\DIA(\BOX(a \OR b) \AND \DIA a)}, \labels{x'}{\DIA(\BOX(a \OR b) \AND \DIA b)} \SEQ \labels{x'}{\DIA(\DIA a \AND \DIA b)}}{
					\vlin{\rlabrn\AND, \rn{id}}{}{\futs{x}{x'}, \accs{x'}{y'}, \labels{y'}{\BOX(a \OR b)}, \labels{y'}{\DIA a}, \labels{x'}{\DIA(\BOX(a \OR b) \AND \DIA b)} \SEQ \labels{x'}{\DIA(\DIA a \AND \DIA b)}, \labels{y'}{\DIA a \AND \DIA b}}{
						%						\vlin{\rn{id}}{}{\labels{y'}{\DIA a} \SEQ \labels{y'}{\DIA a}}{
						%							\vlhy{}
						%						}
						%					}{
						\vlin{\llabrn\DIA, \llabrn\AND, \rlabrn\DIA}{\lb z' \mbox{ fresh}}{\futs{x}{x'}, \accs{x'}{y'}, \labels{y'}{\BOX(a \OR b)}, \labels{y'}{\DIA a}, \labels{x'}{\DIA(\BOX(a \OR b) \AND \DIA b)} \SEQ \labels{x'}{\DIA(\DIA a \AND \DIA b)}, \labels{y'}{\DIA b}}{
							\vlin{\rlabrn\AND, \rn{id}}{}{\futs{x}{x'}, \accs{x'}{y'}, \accs{x'}{z'}, \labels{y'}{\BOX(a \OR b)}, \labels{y'}{\DIA a}, \labels{z'}{\BOX(a \OR b)}, \labels{z'}{\DIA b} \SEQ \labels{x'}{\DIA(\DIA a \AND \DIA b)}, \labels{y'}{\DIA b}, \labels{z'}{\DIA a \AND \DIA b}}{
								\vlin{\rn{g_{1111}}}{\lb{y''}, \lb{u''} \mbox{ fresh}}{\futs{x}{x'}, \accs{x'}{y'}, \accs{x'}{z'}, \labels{y'}{\BOX(a \OR b)}, \labels{y'}{\DIA a}, \labels{z'}{\BOX(a \OR b)}, \labels{z'}{\DIA b} \SEQ \labels{x'}{\DIA(\DIA a \AND \DIA b)}, \labels{y'}{\DIA b}, \labels{z'}{\DIA a}}{
									\vlin{\llabrn\BOX, \rlabrn\DIA}{}{\futs{x}{x'}, \futs{y'}{y''}, \accs{x'}{y'}, \accs{x'}{z'},  \accs{y''}{u''}, \accs{z'}{u''}, \labels{y'}{\BOX(a \OR b)}, \labels{y'}{\DIA a}, \labels{z'}{\BOX(a \OR b)}, \labels{z'}{\DIA b} \SEQ \labels{x'}{\DIA(\DIA a \AND \DIA b)}, \labels{y'}{\DIA b}, \labels{z'}{\DIA a}}{
										%										\vlin{\rn{id}}{}{\labels{u''}{a} \SEQ \labels{u''}{a}}{
										%											\vlhy{}
										%											}
										%										}{
										\vlin{\llabrn\OR, \rn{id}}{}{\futs{x}{x'}, \futs{y'}{y''}, \accs{x'}{y'}, \accs{x'}{z'}, \accs{y''}{u''}, \accs{z'}{u''}, \labels{y'}{\BOX(a \OR b)}, \labels{u''}{a \OR b}, \labels{y'}{\DIA a}, \labels{z'}{\BOX(a \OR b)}, \labels{z'}{\DIA b} \SEQ \labels{x'}{\DIA(\DIA a \AND \DIA b)}, \labels{y'}{\DIA b}, \labels{z'}{\DIA a}, \labels{u''}{a}}{
											\vlin{\rn{g_{1111}}}{\lb{z''}, \lb{v''}\mbox{ fresh}}{\B, \labels{y'}{\BOX(a \OR b)}, \labels{u''}{b}, \labels{y'}{\DIA a}, \labels{z'}{\BOX(a \OR b)}, \labels{z'}{\DIA b} \SEQ \labels{x'}{\DIA(\DIA a \AND \DIA b)}, \labels{y'}{\DIA b}, \labels{z'}{\DIA a}, \labels{u''}{a}}{
												\vlin{\llabrn\BOX, \rlabrn\DIA}{}{\B, \futs{z'}{z''}, \accs{z''}{v''}, \accs{y'}{v''},\labels{y'}{\BOX(a \OR b)}, \labels{u''}{b}, \labels{y'}{\DIA a}, \labels{z'}{\BOX(a \OR b)}, \labels{z'}{\DIA b} \SEQ \labels{x'}{\DIA(\DIA a \AND \DIA b)}, \labels{y'}{\DIA b}, \labels{z'}{\DIA a}, \labels{u''}{a}}{
													\vlin{\llabrn\OR, \rn{id}}{}
													{\B, \futs{z'}{z''}, \accs{z''}{v''}, \accs{y'}{v''},\labels{y'}{\BOX(a \OR b)}, \labels{u''}{b}, \labels{y'}{\DIA a}, \labels{z'}{\BOX(a \OR b)}, \labels{v''}{a \OR b}, \labels{z'}{\DIA b} \SEQ \labels{x'}{\DIA(\DIA a \AND \DIA b)}, \labels{y'}{\DIA b}, \labels{v''}{b}, \labels{z'}{\DIA a}, \labels{u''}{a}}
													{\vlid{}{}
														{\B, \futs{z'}{z''}, \accs{z''}{v''}, \accs{y'}{v''},\labels{y'}{\BOX(a \OR b)}, \labels{u''}{b}, \labels{y'}{\DIA a}, \labels{z'}{\BOX(a \OR b)}, \labels{v''}{a}, \labels{z'}{\DIA b} \SEQ \labels{x'}{\DIA(\DIA a \AND \DIA b)}, \labels{y'}{\DIA b}, \labels{v''}{b}, \labels{z'}{\DIA a}, \labels{u''}{a}}
														{}}
												}
											}
										}
									}
								}
								%							}{
								%							\vlin{\rn{id}}{}{\labels{z'}{\DIA b} \SEQ \labels{z'}{\DIA b}}{
								%								\vlhy{}		
								%								}
							}
						}
					}
				}
			}
		}
	\end{smallequation*}
	where $\B = \futs{x}{x'}, \futs{y'}{y''}, \accs{x'}{y'}, \accs{x'}{z'}, \accs{y''}{u''}, \accs{z'}{u''}$.

%\sonia{	
%	In general, it is possible to design rules corresponding to axioms if given by \emph{bipolar formulas}. 
%	%
%	They are defined as the formulas in the level $N_2$ of the following hierarchy:
%	
%	$P_0 = N_0 \coloncolonequals xRy \mid x \le y$
%	
%	for $n \ge 1$, $P_n \coloncolonequals N_{n-1} \mid \exists x. P_n \mid P_n \AND P_n \mid P_n \OR P_n \mid \TOP \mid \BOT$
%	
%	$N_n \coloncolonequals P_{n-1} \mid \forall x. N_n \mid N_n \AND N_n \mid P_n \IMP N_n \mid \TOP$ 
%	
%	Indeed, this hierarchy exactly carves out the fragment of first-order logic that can be given equivalent rules manipulating only atoms~\cite{marin:etal:submitted}.
%	\lutz{I think we should say this in another place in the paper, maybe an extra ``related work section''. It feels out of place here.}
%}
%\todo{add example from\cite{maffezioli2014} }

