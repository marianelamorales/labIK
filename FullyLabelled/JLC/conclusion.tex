\section{Conclusion}


In this paper we embrace the fully labelled approach to intuitionistic modal logic as pioneered by~\cite{maffezioli:etal:synthese13} and generalise it to the class of logics defined by (one-sided) intuitionistic Scott-Lemmon axioms.
%
We establish that it is a valid approach to intuitionistic modal logic by proving soundness and completeness of our system, via a reductive cut elimination argument.

For a restricted class of these logics defined by so-called \emph{path axioms} $\fm{(\DIA^k\BOX A \IMP\BOX^m A) \AND (\DIA^k A \IMP \BOX^m\DIA A)}$ the standard labelled framework with one relation $R$ was enough for Simpson to get a strong connection between the sequent system, the axiomatisation and the birelational semantics~\cite{simpson:phd}.
%
%\todo{For path axioms we must have that adding the rule for the two sides of the axiom is equivalent to adding the rule that doesn't make use of the pre-order. Is it possible to see that directly on the proof theoretical side?}
%\sonia{probably not because the preorder is introduced/used in other rules for $\BOX$ and $\IMP$}
%
We believe that the framework presented here might be the more appropriate way to treat logics outside of the path axioms definable fragment.

However, we have not shown that our system satisfies Simpson's 6th requirement, that is, ``there is an intuitionistically comprehensible explanation of the meaning of the modalities relative to which [the system] is sound and complete".
%
To make sure that his system satisfies this requirement, Simpson chose to depart from the direct correspondence with modal axioms and their corresponding class of Kripke frames, and to study intuitionistic modal logics purely as a fragment of intuitionistic first-order logic.
%
We instead took the way of a direct correspondence of our system with the class of frames defined by one-sided Scott-Lemmon axioms as uncovered by~\cite{plotkin:stirling:86}, but as this class of logics seems to be rather well-behaved, we believe it should be possible to prove the satisfaction of Simpson's 6th requirement too.

%% \review{Also the reason for preferring one semantics for intuitionistic modal logic over another should be discussed in detail, as otherwise the authors’ work would rather look a routine exercise.}
%% \todo{add a paragraph in conclusion: saying that it is an interesting idea to apply method to CK, but extension is tricky \cite{ADS:lmcs}} \sonia{done?}
We have considered in this work the logic known as $\IK$ (and its extensions) with respect to the birelational semantics that is its most well-studied semantics.
%
As we have mentioned, however, different basis can be considered for intuitionistic modal logic, for example the \emph{constructive modal logic} $\mathsf{CK}$~\cite{bierman:depaiva:sl00,mendler:scheele:ic11} and any in between.
%
These can also be studied within the birelational framework with some additional conditions, and we are convinced that the fully labelled framework, once extended with these conditions, will be suitable to treat constructive fragments equally.
%
However, treating extensions of $\mathsf{CK}$ with axioms such as we did here for Scott-Lemmon logics is known to be complex~\cite{arisaka:etal:lmcs15}.

As for more general future work, there is a real necessity of a global view on intutionistic modal logics.
%
The work of~\cite{dalmonte:grellois:olivetti:arxiv19} is a great first step in understanding them in the context of non-normal modalities and neighbourhood semantics.
%
It would be interesting to know how and where the class of logics we considered can be included in their framework.






