%\documentclass[preprint,authoryear,times,10pt]{elsart}  
\documentclass[10pt]{article}  

%%%%%%%%%%%%%%%%%%%%%%%%%%%%%%%%%%%%%%%%%%%%%%%%%%%%%%%%%%%%%%%%%
%%%%%%%%%%%%%%%%%%%%%%%%%%%%%%%%%%%%%%%%%%%%%%%%%%%%%%%%%%%%%%%%%
%%% PACKAGES
%%%%%%%%%%%%%%%%%%%%%%%%%%%%%%%%%%%%%%%%%%%%%%%%%%%%%%%%%%%%%%%%%
%%%%%%%%%%%%%%%%%%%%%%%%%%%%%%%%%%%%%%%%%%%%%%%%%%%%%%%%%%%%%%%%%

\usepackage{amsmath} 
%\usepackage{amssymb} 
%\usepackage{amsthm}

\usepackage{latexsym} % For \Box and \Diamond
\usepackage{colonequals} % for ::=
%\usepackage{bm} % nice boldface for maths

\usepackage[matrix,arrow]{xy}
%\usepackage{xcolor}
\usepackage[noxy]{virginialake}

%\usepackage{array}
\usepackage{graphicx} % For command \includegraphics and \scalebox                        
%\usepackage{tikz}

%\usepackage{lmodern} % Latin modern fonts in outline formats
%\usepackage{mathtools} % a se­ries of pack­ages de­signed to en­hance the ap­pear­ance of doc­u­ments con­tain­ing a lot of math­e­mat­ics
%\usepackage{enumerate} % adds an optional [style] argument to the enumerate environment

%\usepackage{pgf} % macro package for creating graphics
%\usepackage{pgffor}

%\usepackage{hyperref} % handle cross-referencing commands with hypertext links

%\usepackage{float} % for defining floating objects such as figures and tables
%\floatstyle{boxed} 
%\restylefloat{figure}

%%%%%%%%%%%%%%%%%%%%%%%%%%%%%%%%%%%%%%%%%%%%%%%%%%%%%%%%%%%%%%
%%%%%%%%%%%%%%%%%%%%%%%%%%%%%%%%%%%%%%%%%%%%%%%%%%%%%%%%%%%%%%
%%% Small equation environments
\newdimen\mydisplayskip
\mydisplayskip=.4\abovedisplayskip
\newenvironment{smallequation}
{\par\nobreak\vskip\mydisplayskip\noindent\bgroup\small\csname equation\endcsname}{\csname endequation\endcsname\egroup}
%%
\newenvironment{smallequation*}
{\par\nobreak\vskip\mydisplayskip\noindent\bgroup\small\csname equation*\endcsname}{\csname endequation*\endcsname\egroup}
%%
\newenvironment{smallalign}
{\par\nobreak\noindent\bgroup\small\csname align\endcsname}{\csname endalign\endcsname\egroup}
%%
\newenvironment{smallalign*}
{\par\nobreak\noindent\bgroup\small\csname align*\endcsname}{\csname endalign*\endcsname\egroup}
\newcommand*{\reducesto}{\quad{\leadsto}\quad}
%%%%%%%%%%%%%%%%%%%%%%%%%%%%%%%%%%%%%%%%%%%%%%%%%%%%%%%%%%%%%%%%%
%%%%%%%%%%%%%%%%%%%%%%%%%%%%%%%%%%%%%%%%%%%%%%%%%%%%%%%%%%%%%%%%%
%%%% 
%\tikzset{
%	annotated cuboid/.pic={
%		\tikzset{%
%			every edge quotes/.append style={midway, auto},
%			/cuboid/.cd,
%			#1
%		}
%		\draw [every edge/.append style={pic actions, opacity=.5}, pic actions]
%		(0,0,0) coordinate (o) -- ++(-\cubescale*\cubex,0,0) coordinate (a) -- ++(0,-\cubescale*\cubey,0) coordinate (b) edge coordinate [pos=1] (g) ++(0,0,-\cubescale*\cubez)  -- ++(\cubescale*\cubex,0,0) coordinate (c) -- cycle
%		(o) -- ++(0,0,-\cubescale*\cubez) coordinate (d) -- ++(0,-\cubescale*\cubey,0) coordinate (e) edge (g) -- (c) -- cycle
%		(o) -- (a) -- ++(0,0,-\cubescale*\cubez) coordinate (f) edge (g) -- (d) -- cycle;
%		
%		;
%	},
%	/cuboid/.search also={/tikz},
%	/cuboid/.cd,
%	width/.store in=\cubex,
%	height/.store in=\cubey,
%	depth/.store in=\cubez,
%	units/.store in=\cubeunits,
%	scale/.store in=\cubescale,
%	width=10,
%	height=10,
%	depth=10,
%	units=cm,
%	scale=.1,
%}
%%tikz parameters
%
%\tikzstyle{point}=[circle,draw]
%\usetikzlibrary{arrows,automata,shapes,decorations.markings,
%	decorations.pathmorphing,backgrounds,fit,snakes,calc}
%

%%%%%%%%%%%%%%%%%%%%%%%%%%%%%%%%%%%%%%%%%%%%%%%%%%%%%%%%%%%%%%%%%
%%%%%%%%%%%%%%%%%%%%%%%%%%%%%%%%%%%%%%%%%%%%%%%%%%%%%%%%%%%%%%%%%
%%%% Extracting symbols from MnSymbol 
\DeclareFontFamily{U} {MnSymbolC}{}
%
\DeclareFontShape{U}{MnSymbolC}{m}{n}{
	<-6>  MnSymbolC5
	<6-7>  MnSymbolC6
	<7-8>  MnSymbolC7
	<8-9>  MnSymbolC8
	<9-10> MnSymbolC9
	<10-12> MnSymbolC10
	<12->   MnSymbolC12}{}
\DeclareFontShape{U}{MnSymbolC}{b}{n}{
	<-6>  MnSymbolC-Bold5
	<6-7>  MnSymbolC-Bold6
	<7-8>  MnSymbolC-Bold7
	<8-9>  MnSymbolC-Bold8
	<9-10> MnSymbolC-Bold9
	<10-12> MnSymbolC-Bold10
	<12->   MnSymbolC-Bold12}{}
%
\DeclareSymbolFont{MnSyC}         {U}  {MnSymbolC}{m}{n}
%
\DeclareMathSymbol{\diamondplus}{\mathbin}{MnSyC}{124}
\DeclareMathSymbol{\boxtimes}{\mathbin}{MnSyC}{117}


%%%%%%%%%%%%%%%%%%%%%%%%%%%%%%%%%%%%%%%%%%%%%%%%%%%%%%%%%%%%%%%%%
%%%%%%%%%%%%%%%%%%%%%%%%%%%%%%%%%%%%%%%%%%%%%%%%%%%%%%%%%%%%%%%%%
%%%% Virginialake add-ons
\newcommand{\vlderivationauxnc}[1]{#1{\box\derboxone}\vlderivationterm}
\newcommand{\vlderivationnc}{\vlderivationinit\vlderivationauxnc}
%
%
\makeatletter
\newbox\@conclbox
\newdimen\@conclheight
%
%
\newcommand{\vlhtr}[2]{\vlpd{#1}{}{#2}}
\newcommand\vlderiibase[5]{{%
		\setbox\@conclbox=\hbox{$#3$}\relax%
		\@conclheight=\ht\@conclbox%
		\setbox\@conclbox=\hbox{$%
			\vlderivationnc{%
				\vliin{#1}{#2}{\box\@conclbox}{#4}{#5}%
			}$}%
		\lower\@conclheight\box\@conclbox%
	}}
	%
	\newcommand\vlderibase[4]{{%
			\setbox\@conclbox=\hbox{$#3$}\relax%
			\@conclheight=\ht\@conclbox%
			\setbox\@conclbox=\hbox{$%
				\vlderivationnc{%
					\vlin{#1}{#2}{\box\@conclbox}{#4}%
				}$}%
			\lower\@conclheight\box\@conclbox%
		}}
		%
		\newcommand\vlderidbase[4]{{%
				\setbox\@conclbox=\hbox{$#3$}\relax%
				\@conclheight=\ht\@conclbox%
				\setbox\@conclbox=\hbox{$%
					\vlderivationnc{%
						\vlid{#1}{#2}{\box\@conclbox}{#4}%
					}$}%
				\lower\@conclheight\box\@conclbox%
			}}
			%
			\makeatother
			%


%%%%%%%%%%%%%%%%%%%%%%%%%%%%%%%%%%%%%%%%%%%%%%%%%%%%%%%%%%%%%%%%%
%%%%%%%%%%%%%%%%%%%%%%%%%%%%%%%%%%%%%%%%%%%%%%%%%%%%%%%%%%%%%%%%%
%%% MACROS
%%%%%%%%%%%%%%%%%%%%%%%%%%%%%%%%%%%%%%%%%%%%%%%%%%%%%%%%%%%%%%%%%
%%%%%%%%%%%%%%%%%%%%%%%%%%%%%%%%%%%%%%%%%%%%%%%%%%%%%%%%%%%%%%%%%

%%%% Comments  
\definecolor{notgreen}{rgb}{.1,.6,.1}
\newcommand{\marianela}[1]{{\color{purple}[Marianela: #1]}}
\newcommand{\sonia}[1]{{\color{blue}[Sonia: #1]}}
\newcommand{\lutz}[1]{{\color{notgreen}[Lutz: #1]}}
\newcommand{\todo}[1]{{\color{red}[TODO: #1]}}

%%%% General
\newcommand*{\A}{\mathcal{A}}
%\newcommand{\G}{\mathcal{G}}
\newcommand*{\SG}{\fm{\mathcal{G}}}
\newcommand{\SGi}[1]{\fm{\mathcal{G}_{#1}}}
%
\newcommand{\quand}{\quad\mbox{and}\quad}
\newcommand{\qquand}{\qquad\mbox{and}\qquad}
\newcommand{\quadcm}{\rlap{\quad,}}
%
\newcommand{\proviso}[1]{\mbox{\scriptsize #1}}

%%%% Systems
\newcommand*{\ax}[1]{\mathsf{#1}}
\newcommand*{\kax}[1][]{\ax{k_{#1}}}
\newcommand{\fourax}{\ax{4}}
\newcommand{\agklmn}{\mathsf{g_{klmn}}}
%
\newcommand*{\IK}{\mathsf{IK}}
\newcommand*{\K}{\mathsf{K}}
%
\newcommand*{\IKfour}{\mathsf{IK4}}
\newcommand*{\Kfour}{\mathsf{K4}}
%
%\newcommand*{\ISfour}{\mathsf{IS4}}
%\newcommand*{\Sfour}{\mathsf{S4}}
%
\newcommand*{\labIKp}{\lab\IK_{\le}}

%%%% Connectives
\newcommand*{\NOT}{\neg}
\newcommand*{\AND}{\mathbin{\wedge}}
\newcommand*{\TOP}{\mathord{\top}}
\newcommand*{\OR}{\mathbin{\vee}}
\newcommand*{\BOT}{\mathord{\bot}}
\newcommand*{\IMP}{\mathbin{\supset}}%%\scalebox{.9}{\raise.2ex\hbox{$\supset$}}}}

\newcommand*{\BOX}{\mathord{\Box}}
\newcommand*{\DIA}{\mathord{\Diamond}}
%
%%%% Labelled sequents
\newcommand{\lseq}[3]{#1 , #2 \SEQ #3}
%
\newcommand{\B}{\mathcal{R}}
\newcommand{\Left}{\Gamma} %{\mathcal{L}}
\newcommand{\Right}{\Delta} %{\mathcal{R}}
%
\newcommand*{\fm}[1]{{\color{notgreen}{#1}}}
\newcommand*{\lb}[1]{{\color{blue}{#1}}}
%
\newcommand*{\rel}{R}
\newcommand*{\labels}[2]{\lb{#1}\mathord{:}\fm{#2}}
\newcommand*{\accs}[2]{\lb{#1}R\lb{#2}}
\newcommand*{\rels}[2]{\lb{#1}S\lb{#2}}
\newcommand*{\futs}[2]{\lb{#1}\le{\lb{#2}}}
%
\newcommand{\SEQ}{\Longrightarrow}
%

%%%% Labelled rules
\newcommand*{\rn}[1]  {\ensuremath{\mathsf{#1}}}
\newcommand*{\lab}{\mathsf{lab}}
%
\newcommand*{\labrn}[2][]  {\rn{#2}_{#1}}%^{\lab}}}
\newcommand*{\rlabrn}[2][]  {\rn{#2}_\rn{R#1}}%^\lab}}
\newcommand*{\llabrn}[2][]  {\rn{#2}_\rn{L#1}}%^\lab}}
%%
%\newcommand*{\brsym}{\boxtimes}%\mathord{\scalebox{.8}{$\blacksquare$}}}
\newcommand*{\diasym}{\diamondplus}%\mathord{\blacklozenge}}
%%
\newcommand*{\boxbrn}[1]{\boxtimes_\rn{#1}}%^{\lab}}}
\newcommand*{\diabrn}[1]{\diamondplus_\rn{#1}}

%%%% System labIK+gklmn
\newcommand{\gklmn}{\boxtimes_{\mathsf{gklmn}}}
\newcommand{\boxk}{\square_{R}^{k}}
\newcommand{\boxlk}{\square_{L}^{k}}
\newcommand{\diamk}{\lozenge_{L}^{k}}
\newcommand{\diamrk}{\lozenge_{R}^{k}}

%%%% Semantics
\newcommand{\f}{f^{\mathcal{M}}}
\newcommand{\M}{\mathcal{M}}
\newcommand{\F}{\mathcal{F}}
%
\newcommand{\inter}[1]{\lb{\llbracket #1\rrbracket}}
%\newcommand{\force}[3]{#1,#2\Vdash#3}
\newcommand{\nforce}[3]{#1,#2\not\Vdash#3}
\newcommand{\cforce}[3]{#1,\lb{#2}\Vdash\fm{#3}}
\newcommand{\cnforce}[3]{#1,\lb{#2}\not\Vdash\fm{#3}}

%%%% Derivations
\newcommand{\Dw}{\mathcal{D}^{\rn w}}
\newcommand{\Dwone}{\mathcal{D}_{1}^{\rn w}}
\newcommand{\Dwtwo}{\mathcal{D}_{2}^{\rn w}}
%
\newcommand{\D}{\mathcal{D}}
\newcommand*{\DD}{\mathcal{D}}
\newcommand{\Done}{\mathcal{D}_{1}}
\newcommand{\Dtwo}{\mathcal{D}_{2}}
%
\newcommand{\height}[1]{|#1|}
%
\newcommand*{\invr}[1]{#1}%^\bullet}



%%Symbols for System labK
%\newcommand{\id}{id^{lab}}
%\newcommand{\tolab}{\top^{lab}}
%\newcommand{\vlab}{\wedge^{lab}}
%\newcommand{\olab}{\vlor^{lab}}
%\newcommand{\blab}{\square^{lab}}
%\newcommand{\dlab}{\lozenge^{lab}}
%
%%Labelled proof system
%\newcommand{\toprule}{\B \Rightarrow \Right, x  \colon   \top}
%\newcommand{\vlabr}{\B \Rightarrow \Right, x  \colon   A}
%\newcommand{\vlabu}{\B \Rightarrow \Right, x  \colon   B}
%\newcommand{\olabr}{\B \Rightarrow \Right, x  \colon   A, x  \colon   B}
%\newcommand{\blabr}{\B \Rightarrow \Right, x  \colon   \square A}
%\newcommand{\blabu}{\B, x$R$y \Rightarrow \Right, y  \colon   A}
%\newcommand{\dlabr}{\B, x$R$y \Rightarrow \Right, x  \colon   \lozenge A}
%\newcommand{\dlabu}{\B, x$R$y \Rightarrow \Right, x  \colon   \lozenge A, y  \colon} 
%
%
%%Symbols for system labIK
%\newcommand{\botlab}{\bot_{L}^{lab}}
%\newcommand{\toplab}{\top_{R}^{lab}}
%\newcommand{\andleflab}{\wedge_{L}^{lab}}
%\newcommand{\andriglab}{\wedge_{R}^{lab}}
%\newcommand{\orleflab}{\vlor_{L}^{lab}}
%\newcommand{\orriglabo}{\vlor_{R1}^{lab}}
%\newcommand{\orriglabt}{\vlor_{R2}^{lab}}
%\newcommand{\irlab}{\vljm_{R}^{lab}}
%\newcommand{\illab}{\vljm_{L}^{lab}}
%\newcommand{\dllab}{\lozenge_{L}^{lab}}
%\newcommand{\drlab}{\lozenge_{R}^{lab}}
%\newcommand{\bllab}{\square_{L}^{lab}}
%\newcommand{\brlab}{\square_{R}^{lab}}
%
%%Symbols for System labheartIK
%\newcommand{\ids}{id}
%\newcommand{\idg}{id_{g}}
%\newcommand{\refl}{refl}
%\newcommand{\trans}{trans}
%\newcommand{\cut}{cut}
%\newcommand{\fone}{F1}
%\newcommand{\ftwo}{F2}
%\newcommand{\sbot}{\bot_{L}}
%\newcommand{\Stop}{\top_{R}}
%\newcommand{\svlef}{\wedge_{L}}
%\newcommand{\svrig}{\wedge_{R}}
%\newcommand{\solef}{\vlor_{L}}
%\newcommand{\sorig}{\vlor_{R}}
%\newcommand{\sorone}{\vlor_{R1}}
%\newcommand{\sotwo}{\vlor_{R2}}
%\newcommand{\sir}{\vljm_{R}}
%\newcommand{\sil}{\vljm_{L}}
%\newcommand{\sdl}{\lozenge_{L}}
%\newcommand{\sdr}{\lozenge_{R}}
%\newcommand{\sbl}{\square_{L}}
%\newcommand{\sbr}{\square_{R}}
%\newcommand{\smon}{mon_{L}}
%%
%\newcommand{\Gone}{\mathcal{G}_{1}}
%\newcommand{\Gtwo}{\mathcal{G}_{2}}
%
%% System LABIK
%\newcommand{\conjrig}{\G, \Left \Rightarrow \Right, x \colon A}
%\newcommand{\conjrigh}{\G, \Left \Rightarrow \Right, x  \colon B}
%\newcommand{\conjlef}{\G, \Left, x  \colon  A, x \colon B \Rightarrow \Right}


%\textbf{Comment:}\\ #1\\}
\newcommand{\R}[1]{\textbf{R#1}:\,}
\newcommand{\reponse}[1]{\textbf{#1}}

\begin{document}

\hfill \today
\bigskip

\noindent\textbf{Letter to the editors of the EICNCL Special Issue of JLC.}
\bigskip

\noindent{Dear Agata, Didier, Nicola, and Revantha,}
\medskip

Please find enclosed the revision of our paper entitled {\it A fully labelled 
proof system for intuitionistic modal logics}, taking into accounts the reviews
that we received concerning our first submission.
%
We detail below the main changes compared to the previous version that we operated 
on the paper in order to address remarks and questions by the reviewers.
\smallskip

One of the main concerns of Reviewer 1 is that we had not cited all
the relevant literature. We thank the reviewer for pointing out the
missing references and we added them accordingly.

We now address their other comments.

\begin{quote}\it
	Perhaps even more importantly, the authors of the paper under 
	review neglect an important caveat in the design of rules of 
	contraction-free labelled sequent calculi: to guarantee that 
	contraction be admissible one needs to ensure that a certain 
	closure condition be satisfied
\end{quote}

We mentioned in the submitted version that contraction is derivable 
in our system from the monotonicity rules, which in turn
are admissible. We made this clearer in the new version in
Remark 3.4.

\begin{quote}\it
	In Section 7, if the rules $\gklmn$ are added without their 
	contracted instances, the calculus will not be complete. For 
	example, in the classical case, Euclidean transitivity $\forall a, b, 
	c(aRb \AND aRc \IMP bRc)$ has the contracted instance $?a, b(aRb \IMP bRb)$ that 
	allows for the derivation of $\BOX(\BOX A \IMP A)$. This, however, can 
	hardly be derived without the rule?s contracted instance.
\end{quote}

In fact, the contracted instances are added. In the same way as for
other rules, occurring labels can be equal and substituted as needed. 
We failed to make this explicit in the first submitted version and 
corrected that mistake in Remark 7.2.

\begin{quote}\it
	On top of this, the authors' claim that labelled calculi ``work best 
	for logics with standard Kripke semantic'' is utterly wrong: 
	labelled calculi have been developed for several variants of Kripke 
	semantics, such as Lewis's semantics for counterfactuals, 
	neighbourhood semantics, Kripke semantics of impossible worlds, and 
	so on.
\end{quote}

We meant to say that for logics with standard Kripke semantics 
there is this close relationship between the semantics and the labels in 
the proof system which makes this method both efficient and elegant.

\begin{quote}\it
In conclusion, the paper does not look highly original. The 
birelational semantics is not a novelty, nor a calculus 
internalising that semantics appears an original idea.
\end{quote}

We do not claim the birelational semantics to be a novelty of this work.
(We acknowledge that it dates back to the 1980s.) Our paper is using the 
semantical results of Plotkin and Sterling, and others, to design proof 
systems for extensions of intuitionistic modal logic that had not been 
studied proof-theoretically before. We try to make this clearer in the new 
version (see Section 7).

\begin{quote}\it
Also the reason for preferring one semantics for intuitionistic 
modal logic over another should be discussed in detail, as 
otherwise the authors? work would rather look a routine exercise
\end{quote}

We chose to work the standard birelational semantics for intuitionistic modal 
logic IK. There are indeed other possible semantics that have been studied 
for other constructive interpretations of modal logics such as birelational 
semantics with impossible words or neighbourhood semantics. These could 
potentially be studied in a similar framework to ours and we commented on 
this topic in the conclusion.

\medskip
Reviewer 2 finds this paper well written, but has some 
technical concerns. We address them below.

\begin{quote}\it
The main issue is indeed with the ``wide range of logics''. While the
base logic is discussed in detail, the range of logics is treated
in an ``Extensions'' section that I believe could actually be more
detailed.
\end{quote}

We put more emphasis on Section 7 in this new version. We hope we made 
it clearer how to generate ``stronger logics" adding new axioms to our system. 
We extended this section with more explanation and also added a few comments 
and remarks. For example, as a better way to understand the rules and the 
completeness proof, we present the specific case of the 4-axioms.

\begin{quote}\it
Most importantly, I am not convinced by the $\gklmn$ rule. The side
condition stipulates that $y?$ and $u$ are fresh, whereas the frame
property contains existential quantifications. I was thus expecting
either a Skolemization in the style of Vigan\`o in the ``Labelled
Non-Classical Logics'' book, or a geometric rule with existential
quantification in the style of Simpson [Sim94] or, perhaps even
better, a ``mixed'' rule where the labels are assumed more or less in
the style of an $\exists$ elimination to derive a labeled formula.  I
think that a thorough discussion of the alternatives, along with a
more thorough proof of the correctness of the chosen solution, is
needed.
\end{quote}

We hope the additional Remark 7.4 answers these concerns. 
We are indeed using the same method that Simpson (or Negri) applied to 
geometric formulas, but in our case also allowing the preorder atoms to occur. 
This method enforces that both relational and preorder atoms only appear on 
the left-hand-side of a sequent, never on the right-hand-side. 
As suggested by Reviewer 2, it should also be possible to treat the Scott-Lemmon axioms in 
the style of Vigan\`o where atoms would then appear on the right of sequents 
only and indeed Skolemisation of some of the quantifiers would be required.

\begin{quote}\it
By the way, it does not help at all that the applications of the
rule $\rn{g_{1111}}$ are merged with $\llabrn\BOX$ and $\rlabrn\DIA$ in the example
derivation.
\end{quote}

Following this comment, we fixed the derivation which should now 
show better the use of rule $\rn{g_{1111}}$.

\begin {quote}\it
Also Remark 4.2 could benefit from some additional explanation. You
write that the proof of Theorem 4.1 shows the ``need'' of the rules
$\rn{F_1}$, $\rn{refl}$ and $\rn{trans}$. However, strictly speaking, the proof only
shows how the rules are used and that they are helpful to prove the
axioms. To show the ``need'' you would likely have to show that no
other proofs of the axioms are possible. That is of course the
case, but unless you prove it formally, it is not fully correct to
claim that the ``need'' is shown. 
\end{quote}

In this case, our intention was indeed only to show how to use those rules and how 
they play an interesting role in these specific cases.

\begin{quote}\it
Finally, I suggest that the authors reconsider the abstract, which, 
in my opinion, does not do their paper justice. It reads very ?operational? 
and does not clearly state the contributions of the paper.
\end{quote}

We rewrote the abstract adding some more context and with the aim of better reflecting 
our contributions
\medskip

We finally want to thank the reviewers for their time and for helping us improve our paper. 
We hope to have addressed all their insightful comments both in this letter and in the new version of this paper. 
\bigskip

Kind regards,
\medskip

Sonia Marin, Marianela Morales and Lutz Stra\ss burger



\end{document}
