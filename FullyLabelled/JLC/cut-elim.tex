\section{Cut Admissibility}\label{sec:cut-elim}
In this section we are going to prove the admissibility of cut for
$\labIKp$. The presentations follows the standard textbook
exposition (see, e.g., \cite{troelstra:schwichtenberg:00,negri:vonplato:01})
adapted to the system at hand.\footnote{As pointed out
	in~\cite{girard:87:a}, any minor change in a proof system demands
	to do the whole cut elimination argument from scratch.}
%%
%A similar system as
%ours has been presented in~\cite{maffezioli:etal:synthese13}, but 

%% \todo{comment on standard method, cite textbooks (TS00 and/or Negri, MNN paper)}
%% \lutz{done?}

\begin{theorem}
	\label{thm:cut-adm}
	The $\labrn{cut}$ rule is admissible for $\labIKp$.
\end{theorem}

This theorem directly entails the implication \ref{ii}$\implies$\ref{iii} of Theorem~\ref{thm:cutfree-compl}. But before we can prove it, we need a series of auxiliary  lemmas.

The \emph{height} of a derivation $\DD$, denoted by $\height\DD$, is the height of $\DD$ when seen as a tree, i.e., the length of
the longest path in the tree from its root to one of its leaves.

We say that a rule is \emph{height-preserving admissible} if for every derivation $\DD$ of its premise(s) there is a derivation $\DD'$ of its conclusion such that $\height{\DD'}\le\height\DD$. A rule is \emph{height-preserving invertible} if for every
derivation of the conclusion of the rule there are derivations for each of its premises with at most the same height.

%The following two lemmas are standard.

%\begin{lemma}
%	\label{lem:weak-adm}
%	The weakening rule
%	~$\vlinf{\rn{weak}}{}{\lseq{\B,\B'}{\Left,\Left'}{\Right,\Right'}}{\lseq{\B}{\Left}{\Right}}$~
%	is height-preserving admissible for $\labIKp$.
%\end{lemma}

The first lemma is the height-preserving admissibility of weakening on both relational atoms and labelled formulas.
	
\begin{lemma}
	\label{lem:weak-adm}
	The weakening rule
	~$\vlinf{\rn{weak}}{}{\lseq{\B,\B'}{\Left,\Left'}{\Right,\Right'}}{\lseq{\B}{\Left}{\Right}}$~
	is height-preserving admissible for $\labIKp$.
\end{lemma}
	
\begin{proof}
  By a straightforward induction on the height of the derivation, we can transform any derivation
  \begin{equation*}
    \vlderivation {\vlpd {\D}{}{\lseq{\B}{\Left}{\Right}}}
    \qquad
    \mbox{into}
    \qquad
    \vlderivation {\vlpd{\Dw}{}{\lseq{\B,\B'}{\Left,\Left'}{\Right,\Right'}}}
  \end{equation*}
of the same (or smaller) height.
\end{proof}


The next lemma looks like a special case of
Proposition~\ref{prop:mon-adm}, but it is not. First, we need to
preserve the height, and second, we cannot prove it using $\rn{cut}$ rule as we
are trying to eliminate it from derivations.

\begin{lemma}\label{lem:adm-mon-at}
	The atomic version of $\llabrn{mon}$ and $\rlabrn{mon}$
	$$
	\vlinf{\llabrn{mon_a}}{}{\lseq{\B, \futs{x}{y}}{\Left, \labels{x}{a}}\Right}{
		\lseq{\B, \futs{x}{y}}{\Left, \labels{x}{a}, \labels{y}{a}}\Right}
	\qquad
	\vlinf{\rlabrn{mon_a}}{}{\B, \futs xy, \Left \SEQ \Right, \labels{y}{a}}{\B, \futs xy, \Left \SEQ \Right, \labels{x}{a}, \labels{y}{a}}
	$$
	are height-preserving admissible for $\labIKp$.
%	\lutz{in the second base case we need the atomic version of $\rlabrn{mon}$ as well, either we add it here, or we drop that case}
\end{lemma}

\begin{proof}
	We show the details for $\llabrn{mon_a}$, the argument is the same for $\rlabrn{mon_a}$.
	By induction on the height of $\D$, we prove that for any proof of $\lseq{\B, \futs{x}{y}}{\Left, \labels{x}{a}, \labels{y}{a}}{\Right}$, there exists a proof of $\lseq{\B, \futs{x}{y}}{\Left, \labels{x}{a}}{\Right}$
	of the same (or smaller) height.
	The inductive step is straightforward by permutation of rules.
	The base cases are obtained as follows:
	\begin{smallequation*}
		\vlderivation{
			\vlin{\llabrn{mon_a}}{}{\B, \futs{x}{y}, \futs{y}{z}, \Left, \labels{x}{a} \SEQ \Right, \labels{z}{a}}{
				\vlin{\labrn{id}}{}{\B, \futs{x}{y}, \futs{y}{z}, \Left, \labels{x}{a}, \labels{y}{a} \SEQ \Right, \labels{z}{a}}{
					\vlhy{}
				}
			}
		}
		\reducesto
		\vlderivation{
			\vlin{\color{red}\labrn{trans}}{}{\B, \futs{x}{y}, \futs{y}{z}, \Left, \labels{x}{a} \SEQ \Right, \labels{z}{a}}{\vlin{\labrn{id}}{}{\B, \futs{x}{y}, \futs{y}{z}, \futs{x}{z}, \Left, \labels{x}{a} \SEQ \Right, \labels{z}{a}}{
					\vlhy{}
				}}
			}
		\end{smallequation*}

	  	\begin{smallequation*}
	  		\vlderivation{
	  			\vlin{\llabrn{mon_a}}{}{\B, \futs{x}{y}, \Left, \labels{x}{a} \SEQ \Right, \labels{x}{a}}{
	  				\vlin{\labrn{id}}{}{\B, \futs{x}{y}, \labels{x}{a}, \labels{y}{a} \SEQ \Right, \labels{x}{a}}{
	  					\vlhy{}
	  				}
	  			}
	  		}
	  		\reducesto
	  		\vlderivation{
	  		\vlin{\color{red}\labrn{refl}}{}{\B, \futs{x}{y}, \Left, \labels{x}{a} \SEQ \Right, \labels{x}{a}}{\vlin{\labrn{id}}{}{\B, \futs{x}{y}, \futs{x}{x}, \labels{x}{a} \SEQ \Right, \labels{x}{a}}{
	  				\vlhy{}
	  			}}}
	  	\end{smallequation*}
			  	
	  	\begin{smallequation*}
	  		\vlderivation{
	  			\vlin{\llabrn{mon_a}}{}{\B, \futs{x}{y}, \Left, \labels{x}{a} \SEQ \Right, \labels{y}{a}}{
	  				\vlin{\labrn{id}}{}{\B, \futs{x}{y}, \labels{x}{a}, \labels{y}{a} \SEQ \Right, \labels{y}{a}}{
	  					\vlhy{}
	  				}
	  			}
	  		}
	  		\reducesto
	  		\vlinf{\labrn{id}}{}{\B, \futs{x}{y}, \Left, \labels{x}{a} \SEQ \Right, \labels{y}{a}}{}
	  		\qedhere
	  	\end{smallequation*}
	  	
	\end{proof}
	
	The next lemma shows that all the rules in our system are invertible, as already mentioned in the introduction. 
	
	\begin{lemma}
	  \label{lem:inv}
          All single-premise rules of $\labIKp$ are height-preserving
          invertible. Furthermore, the rules $\llabrn{\OR}$ and
          $\rlabrn{\AND}$ are height-preserving invertible on both premises, and the
          rule $\llabrn{\IMP}$ is height-preserving invertible on the right premise.
%		Furthermore, the rule $\llabrn{\IMP}$ is
          %		height-preserving invertible for the right premise.
%          \lutz{"height-preseving'' was commented out. For a reason or by mistake?} 
	\end{lemma}

	\begin{proof}
		For each rule $\rn{r}$, we need to show that if there exists a proof $\DD$ of the conclusion, there exists a proof $\DD^{\rn r_i}$ of the $i$-th premise, of the same (or smaller) height.
		For $\rlabrn\AND$, $\llabrn\AND$, $\rlabrn\OR$, $\llabrn\OR$, and the right premise of $\llabrn\IMP$, we use a standard induction on the height of $\DD$.
		%
		For $\rlabrn\IMP$, $\rlabrn\BOX$, $\llabrn\DIA$ as well, but
		we need to make sure that the obtained derivation uses a fresh label by using substitution inside $\DD^{\rn r}$ when necessary.
		%
		The other rules can be shown invertible using Lemma~\ref{lem:weak-adm}. 
\end{proof}


	
	The next lemma is the central ingredient of our cut elimination proof.
	
	\begin{lemma}
		\label{lem:reduction}
		Given a derivation of shape
		$$
		\vlderivation{
			\vliiin{\labrn{cut}}{}{\lseq\B\Left\Right}{
				\vlhtr{\DD_1}{\lseq\B\Left{\Right,\labels{z}{C}}}}{
				\vlhy{}}{
				\vlhtr{\DD_2}{\lseq\B{\Left,\labels{z}{C}}{\Right}}}}
		$$
		where $\DD_1$ and $\DD_2$ are both cut-free, there is a cut-free
		derivation of ${\lseq\B\Left\Right}$
	\end{lemma}
	
	%$\vlderivation{\vlhtr{\DD}{\lseq\B\Left\Right}}$
	
	
%	Following the proof of the Theorem \ref{thm:cutfree-compl}, in this section, we prove the implication \ref{ii}$\implies$\ref{iii}, i.e, we want to prove that our system $\labIKp$ is complete without the \rn{cut} rule.
%	
%	For proving this theorem, we need a series of auxiliary lemmas (see Lemma \ref{lem:inv}).
	
	\begin{proof}
		%
		The proof is by a lexicographic induction on the complexity of the cut-formula $C$ and the sum of the heights $\height{\DD_1}+\height{\DD_2}$.
		%
		We perform a case analysis on the last rule used in $\DD_1$ above the $\rn{cut}$ and whether it applies to the cut-formula or not.
		%
		In case it does not, we are in a \emph{commutative} case; in case it does, we have to perform a similar analysis on $\DD_2$ to end up in a \emph{key} case.

                \begin{description}
                \item[Base cases:]\label{base-cases}
                When the last rule in $\DD_1$ is an axiom, 
%                the complexity of the cut-formula is $0$, i.e.~$C$ is atomic, or in general when the height of $\DD_1$ or $\DD_2$ is $0$, 
                we can produce directly a cut-free derivation of the conclusion.
                %
                In the first case, we appeal to Lemma~\ref{lem:adm-mon-at}, to use the atomic monotonicity rule freely.
        \begin{itemize}
        	                \item 
        	                \begin{smallequation*}
        	                	\vlderiibase{\labrn{cut}}{}{\B, \futs xy, \Left, \labels{x}{a} \SEQ \Right}{
        	                		\vlin{\labrn{id}}{}{\B, \futs xy, \Left, \labels{x}{a} \SEQ \Right, \labels{y}{a}}{
        	                			\vlhy{}	
        	                		}
        	                	}{
        	                	\vlhtr{\DD_2}{\B, \futs xy, \Left, \labels{x}{a}, \labels{y}{a} \SEQ \Right}		
        	                }
        	                \reducesto
        	                \vlderibase{\llabrn{mon_a}}{}{\B, \futs xy, \Left, \labels{x}{a} \SEQ \Right}{
        	                	\vlhtr{\DD_2^{\rn w}}{\B, \futs xy, \Left, \labels{x}{a}, \labels{y}{a} \SEQ \Right}
        	                }
        	            \end{smallequation*}
        	            
%        	            \item
%        	            \begin{smallequation*}
%        	            	\vlderiibase{\labrn{cut}}{}{\B, \futs xy, \Left \SEQ \Right, \labels{y}{a}}{
%        	            		\vlhtr{\DD_1}{\B, \Left \SEQ \Right, \labels{x}{a}, \labels{y}{a}}		
%        	            	}{
%        	            	\vlin{\labrn{id}}{}{\B, \futs xy, \Left, \labels{x}{a} \SEQ \Right, \labels{y}{a}}{
%        	            		\vlhy{}	
%        	            	}
%        	            }
%        	            \reducesto
%        	            \vlderibase{\rlabrn{mon_a}}{}{\B, \futs xy, \Left \SEQ \Right, \labels{y}{a}}{
%        	            	\vlhtr{\DD_1^{\rn w}}{\B, \futs xy, \Left \SEQ \Right, \labels{x}{a}, \labels{y}{a}}
%        	            }
%        	        \end{smallequation*}
%                
%                \item
%                \begin{smallequation*}
%                	\vlderiibase{\labrn{cut}}{}{\B, \Left \SEQ \Right}{
%                		\vlhtr{\DD_1}{\B, \Left \SEQ \Right, \labels{x}{\BOT}}
%                	}{
%                		\vlin{\llabrn{\BOT}}{}{\B, \Left, \labels{x}{\BOT} \SEQ \Right}{
%                			\vlhy{}}
%                	}
%                	\reducesto
%                	\vlderivation{
%                		\vlhtr{\DD_1}{\B, \Left \SEQ \Right}
%                	}
%                \end{smallequation*}
%            	Indeed, $\labels{x}{\BOT}$ is passive in $\DD_1$ as there is no right rule for $\BOT$ in $\labIKp$.
            	
            
            	\item
            	\begin{smallequation*}
            		\vlderiibase{\labrn{cut}}{}{\B, \futs xy, \Left, \labels{x}{a} \SEQ \Right, \labels{y}{a}}{
            			\vlin{\labrn{id}}{}{\B, \futs xy, \Left, \labels{x}{a} \SEQ \Right, \labels{y}{a}, \labels{z}{C}}{
            				\vlhy{}	
            			}
            		}{
            			\vlhtr{\DD_2}{\B, \futs xy, \Left, \labels{x}{a}, \labels{z}{C} \SEQ \Right, \labels{y}{a}}		
            		}
            		\reducesto
            		\vlderibase{\rn{id}}{}{\B, \futs xy, \Left, \labels{x}{a} \SEQ \Right, \labels{y}{a}}{
            			\vlhy{}
            		}
            	\end{smallequation*}
            
            	\item
            	\begin{smallequation*}
            		\vlderiibase{\labrn{cut}}{}{\B, \Left, \labels{x}{\BOT} \SEQ \Right}{
            			\vlin{\llabrn{\BOT}}{}{\B, \Left, \labels{x}{\BOT} \SEQ \Right, \labels{z}{C}}{
            				\vlhy{}	
            			}
            		}{
            			\vlhtr{\DD_2}{\B, \Left, \labels{x}{\BOT}, \labels{z}{C} \SEQ \Right}		
            		}
            		\reducesto
            		\vlderibase{\llabrn{\BOT}}{}{\B, \Left, \labels{x}{\BOT} \SEQ \Right}{
            			\vlhy{}
            		}
            	\end{smallequation*}
        \end{itemize}
                \item[Commutative cases:]\label{commutative-cases}
		In such a case, the complexity of the cut-formula stays constant, but the height of the derivation above the $\rn{cut}$ decreases.
		\begin{itemize}
                \item $\llabrn\IMP$:
		\begin{smallequation*}
			\vlderiibase{\labrn{cut}}{}{\B, \futs xy, \Left, \labels{x}{A \IMP B} \SEQ \Right}{
				\vliin{\llabrn\IMP}{}{\B, \futs xy, \Left, \labels{x}{A \IMP B} \SEQ \Right, \labels{z}{C}}{
					\vlhtr{\DD_1}{\B, \futs xy, \Left, \labels{x}{A \IMP B} \SEQ \Right, \labels{z}{C}, \labels{y}{A}}
				}{
				\vlhtr{\DD_2}{\B, \futs xy, \Left, \labels{y}{B} \SEQ \Right, \labels{z}{C}}
			}
		}{
		\vlhtr{\DD_3}{\B, \futs xy, \Left, \labels{x}{A \IMP B}, \labels{z}{C} \SEQ \Right}
	}
	\reducesto
\end{smallequation*}

\begin{smallequation*}\hspace*{-8.5em}
	\vlderiibase{\llabrn\IMP}{}{\B, \futs xy, \Left, \labels{x}{A \IMP B} \SEQ \Right}{
		\vliin{\labrn{cut}}{}{\B, \futs xy, \Left, \labels{x}{A \IMP B} \SEQ \Right, \labels{y}{A}}{
			\vlhtr{\DD_1}{\B, \futs xy, \Left, \labels{x}{A \IMP B} \SEQ \Right, \labels{z}{C}, \labels{y}{A}}
		}{
		\vlhtr{\DD_3^{\rn w}}{\B, \futs xy, \Left, \labels{x}{A \IMP B}, \labels{z}{C} \SEQ \Right, \labels{y}{A}}
	}
}{
\vliin{\labrn{cut}}{}{\B, \futs xy, \Left, \labels{y}{B} \SEQ \Right}{
	\vlhtr{\DD_2}{\B, \futs xy, \Left, \labels{y}{B} \SEQ \Right, \labels{z}{C}}
}{
\vlhtr{\DD_3^{\invr{\llabrn\IMP}}}{\B, \futs xy, \Left, \labels{y}{B}, \labels{z}{C} \SEQ \Right}
}
}
\end{smallequation*}
where $\DD_3^{\rn w}$ is obtained using Lemma~\ref{lem:weak-adm} and $\DD_3^{\llabrn\IMP}$ is obtained using Lemma~\ref{lem:inv}. We use the same naming scheme in the following cases.
\item $\rlabrn\IMP$:
\begin{smallequation*}\hskip-8em
	\vlderiibase{\labrn{cut}}{}{\B, \Left \SEQ \Right, \labels{x}{A \IMP B}}{
		\vlin{\rlabrn\IMP}{\text{\footnotesize $\lb{x'}$ fresh}}{\B, \Left \SEQ \Right, \labels{x}{A \IMP B}, \labels{z}{C}}{
			\vlhtr{\DD_1}{\B, \futs{x}{x'}, \Left, \labels{x'}{A} \SEQ \Right, \labels{x'}{B}, \labels{z}{C}}
		}
	}{
	\vlhtr{\DD_2}{\B, \Left, \labels{z}{C} \SEQ \Right, \labels{x}{A \IMP B}}
}
\end{smallequation*}

\begin{smallequation*}\hskip8em
	\reducesto
	\vlderibase{\rlabrn\IMP}{\text{\footnotesize $\lb{x''}$ fresh (also in $\DD_2$)}}{\B, \Left \SEQ \Right, \labels{x}{A \IMP B}}{
		\vliin{\labrn{cut}}{}{\B, \futs{x}{x''}, \Left, \labels{x''}{A} \SEQ \Right, \labels{x''}{B}}{
			\vlhtr{\DD_1[x''/x']}{\B, \futs{x}{x''}, \Left, \labels{x''}{A} \SEQ \Right, \labels{x''}{B}, \labels{z}{C}}
		}{
		\vlhtr{\DD_2^{\invr{\rlabrn\IMP}}}{\B, \futs{x}{x''}, \Left, \labels{z}{C}, \labels{x''}{A} \SEQ \Right, \labels{x''}{B}}
	}
}
\end{smallequation*}
\item $\llabrn\BOX$:
\begin{smallequation*}\hskip-8em
	\vlderiibase{\labrn{cut}}{}{\B, \futs xu, \accs uv, \Left, \labels{x}{\BOX A} \SEQ \Right}{
		\vlin{\llabrn\BOX}{}{\B, \futs xu, \accs uv, \Left, \labels{x}{\BOX A} \SEQ \Right, \labels{z}{C}}{
			\vlhtr{\DD_1}{\B, \futs xu, \accs uv, \Left, \labels{x}{\BOX A}, \labels{v}{A} \SEQ \Right, \labels{z}{C}}
		}
	}{
	\vlhtr{\DD_2}{\B, \futs xu, \accs uv, \Left, \labels{x}{\BOX A}, \labels{z}{C} \SEQ \Right}
}
\end{smallequation*}

\begin{smallequation*}\hskip8em
	\reducesto
	\vlderibase{\llabrn\BOX}{}{\B, \futs xu, \accs uv, \Left, \labels{x}{\BOX A} \SEQ \Right}{
		\vliin{\labrn{cut}}{}{\B, \futs xu, \accs uv, \Left, \labels{x}{\BOX A}, \labels{v}{A} \SEQ \Right}{
			\vlhtr{\DD_1}{\B, \futs xu, \accs uv, \Left, \labels{x}{\BOX A}, \labels{v}{A} \SEQ \Right, \labels{z}{C}}
		}{
		\vlhtr{\DD_2^{\rn w}}{\B, \futs xu, \accs uv, \Left, \labels{x}{\BOX A}, \labels{v}{A}, \labels{z}{C} \SEQ \Right}
	}
}
\end{smallequation*}

\item $\rlabrn\BOX$:
\begin{smallequation*}\hskip-8em
	\vlderiibase{\labrn{cut}}{}{\B, \Left \SEQ \Right, \labels{x}{\BOX A}}{
		\vlin{\rlabrn\BOX}{\text{\footnotesize $\lb{x'},\lb{y'}$ fresh}}{\B, \Left \SEQ \Right, \labels{x}{\BOX A}, \labels{z}{C}}{
			\vlhtr{\DD_1}{\B, \futs{x}{x'}, \accs{x'}{y'}, \Left \SEQ \Right, \labels{y'}{A}, \labels{z}{C}}
		}
	}{
	\vlhtr{\DD_2}{\B, \Left, \labels{z}{C} \SEQ \Right, \labels{x}{\BOX A}}
}
\end{smallequation*}

\begin{smallequation*}\hskip8em
	\reducesto
	\vlderibase{\rlabrn\BOX}{\text{\footnotesize $\lb u,\lb v$ fresh (also in $\DD_2$)}}{\B, \Left \SEQ \Right, \labels{x}{\BOX A}}{
		\vliin{\labrn{cut}}{}{\B, \futs xu, \accs uv, \Left \SEQ \Right, \labels{v}{A}}{
			\vlhtr{\DD_1}{\B, \futs xu, \accs uv, \Left \SEQ \Right, \labels{v}{A}, \labels{z}{C}}
		}{
		\vlhtr{\DD_2^{\invr{\rlabrn\BOX}}}{\B, \futs xu, \accs uv, \Left, \labels{z}{C} \SEQ \Right, \labels{v}{A}}
	}
}
\end{smallequation*}
\item $\llabrn\DIA$:

\begin{smallequation*}\hspace*{-5.5em}
	\vlderiibase{\labrn{cut}}{}{\B, \Left, \labels{x}{\DIA A} \SEQ \Right}{
		\vlin{\llabrn\DIA}{\text{\footnotesize $\lb{y'}$ fresh}}{\B, \Left, \labels{x}{\DIA A} \SEQ \Right, \labels{z}{C}}{
			\vlhtr{\DD_1}{\B, \accs{x}{y'}, \Left, \labels{y'}{A} \SEQ \Right, \labels{z}{C}}
		}
	}{
	\vlhtr{\DD_2}{\B, \Left, \labels{x}{\DIA A}, \labels{z}{C} \SEQ \Right}
}
\reducesto
\vlderibase{\llabrn\DIA}{\text{\footnotesize $\lb{y''}$ fresh (also in $\DD_2$)}}{\B, \Left, \labels{x}{\DIA A} \SEQ \Right}{
	\vliin{\labrn{cut}}{}{\B, \accs{x}{y''}, \Left, \labels{y''}{A} \SEQ \Right}{
		\vlhtr{\DD_1[y''/y']}{\B, \accs{x}{y''}, \Left, \labels{y''}{A} \SEQ \Right, \labels{z}{C}}
	}{
	\vlhtr{\DD_2^{\invr{\llabrn\DIA}}}{\B, \accs{x}{y''}, \Left, \labels{y''}{A}, \labels{z}{C} \SEQ \Right}
}
}
\end{smallequation*}
%\todo{is this correct wrt to invertibility Lemma~\ref{lem:inv}??}
\item $\rlabrn\DIA$:

\begin{smallequation*}\hspace*{-5.5em}
	\vlderiibase{\labrn{cut}}{}{\B, \accs xy, \Left \SEQ \Right, \labels{x}{\DIA A}}{
		\vlin{\rlabrn\DIA}{}{\B, \accs xy, \Left \SEQ \Right, \labels{x}{\DIA A}, \labels{z}{C}}{
			\vlhtr{\DD_1}{\B, \accs xy, \Left \SEQ \Right, \labels{x}{\DIA A}, \labels{y}{A}, \labels{z}{C}}
		}
	}{
	\vlhtr{\DD_2}{\B, \Left, \labels{z}{C} \SEQ \Right, \labels{x}{\DIA A}}
}
\reducesto
%	\end{smallequation*}
%	
%	\begin{smallequation*}
\vlderibase{\rlabrn\DIA}{}{\B, \accs xy, \Left \SEQ \Right, \labels{x}{\DIA A}}{
	\vliin{\labrn{cut}}{}{\B, \accs xy, \Left \SEQ \Right, \labels{x}{\DIA A}, \labels{y}{A}}{
		\vlhtr{\DD_1}{\B, \accs xy, \Left \SEQ \Right, \labels{x}{\DIA A}, \labels{y}{A}, \labels{z}{C}}
	}{
	\vlhtr{\DD_2^{\rn w}}{\B, \Left, \labels{z}{C} \SEQ \Right, \labels{x}{\DIA A}, \labels{y}{A}}
}
}
\end{smallequation*}

\item$\rn{refl}$:
\begin{smallequation*}
	\vlderiibase{\rn{cut}}{}
	{\B, \Left \SEQ \Right}
	{\vlin{\rn{refl}}{}{\B, \Left \SEQ \Right, \labels{z}{C}}{\vlhtr{\DD_1}{\B, \futs xx, \Left \SEQ \Right, \labels{z}{C}}}}
	{\vlhtr{\DD_2}{\B, \Left, \labels{z}{C} \SEQ \Right}}
	\reducesto
	\vlderibase{\rn{refl}}{}{\B, \Left \SEQ \Right}{\vliin{\rn{cut}}{}
		{\B, \futs xx, \Left \SEQ \Right}
		{\vlhtr{\DD_1}{\B, \futs xx, \Left \SEQ \Right, \labels{z}{C}}}
		{\vlhtr{\DD_2^{\rn w}}{\B, \futs xx, \Left, \labels{z}{C} \SEQ \Right}}}
\end{smallequation*}

\item$\rn{trans}$:
\begin{smallequation*}
	\vlderiibase{\rn{cut}}{}
	{\B, \futs xy, \futs yz, \Left \SEQ \Right}
	{\vlin{\rn{trans}}{}
		{\B, \futs xy, \futs yz, \Left \SEQ \Right, \labels{z}{C}}
		{\vlhtr{\DD_1}{\B, \futs xy, \futs yz, \futs xz, \Left \SEQ \Right, \labels{z}{C}}}}
	{\vlhtr{\DD_2}{\B, \futs xy, \futs yz, \Left, \labels{z}{C} \SEQ \Right}}
\end{smallequation*}
\begin{smallequation*}
	\reducesto
	\vlderibase{\rn{trans}}{}
	{\B, \futs xy, \futs yz, \Left \SEQ \Right}
	{\vliin{\rn{cut}}{}
		{\B, \futs xy, \futs yz, \futs xz, \Left \SEQ \Right}
		{\vlhtr{\DD_1}{\B, \futs xy, \futs yz, \futs xz, \Left \SEQ \Right, \labels{z}{C}}}
		{\vlhtr{\DD_2^{\rn w}}{\B, \futs xy, \futs yz, \futs xz, \Left, \labels{z}{C} \SEQ \Right}}}
\end{smallequation*}

\item$\rn{F_1}$:
 % \todo{say that the fresh variable is renamed (if present) in the other branch (for F1 and F2)}
\begin{smallequation*}
	\vlderiibase{\rn{cut}}{}
	{\B, \accs xy, \futs yz, \Left \SEQ \Right}
	{\vlin{\rn{F_1}}{\lb u \mbox{ fresh}}
		{\B, \accs xy, \futs yz, \Left \SEQ \Right, \labels{z}{C}}
		{\vlhtr{\DD_1}{\B, \accs xy, \futs yz, \futs xu, \accs uz, \Left \SEQ \Right, \labels{z}{C}}}}
	{\vlhtr{\DD_2}{\B, \accs xy, \futs yz, \Left, \labels{z}{C} \SEQ \Right}}
\end{smallequation*}
\begin{smallequation*}\hskip8em
	\reducesto
	\vlderibase{\rn{F_1}}{\lb v \mbox{ fresh (also in $\DD_2$)}}
	{\B, \accs xy, \futs yz, \Left \SEQ \Right}
	{\vliin{\rn{cut}}{}
		{\B, \accs xy, \futs yz, \futs xv, \accs vz, \Left \SEQ \Right}
		{\vlhtr{\DD_1[v/u]}{\B, \accs xy, \futs yz, \futs xv, \accs vz, \Left \SEQ \Right, \labels{z}{C}}}
		{\vlhtr{\DD_2^{\rn w}}{\B, \accs xy, \futs yz, \futs xv, \accs vz, \Left, \labels{z}{C} \SEQ \Right}}}
\end{smallequation*}
%where if $u$ appeared in $\DD_2$ it is renamed to a fresh label $v$.

\item$\rn{F_2}$:
\begin{smallequation*}
	\vlderiibase{\labrn{cut}}{}{\B, \accs xy, \futs xz, \Left \SEQ \Right}
	{\vlin{\rn{F_2}}{\lb u \mbox{ fresh}}{\B, \accs xy, \futs xz, \Left \SEQ \Right, \labels{z}{C}}{\vlhtr{\DD_1}{\B, \accs xy, \futs xz, \futs yu, \accs zu, \Left \SEQ \Right, \labels{z}{C}}}}
	{\vlhtr{\DD_2}{\B, \accs xy, \futs xz, \Left, \labels{z}{C} \SEQ \Right}}
\end{smallequation*}
\begin{smallequation*}\hskip8em
	\reducesto
	\vlderibase{\rn{F_2}}{\lb v \mbox{ fresh (also in $\DD_2$)}}{\B, \accs xy, \futs xz, \Left \SEQ \Right}{\vliin{\labrn{cut}}{}
		{\B, \accs xy, \futs xz, \futs yv, \accs zv, \Left \SEQ \Right}
		{\vlhtr{\DD_1[v/u]}{\B, \accs xy, \futs xz, \futs yv, \accs zv, \Left \SEQ \Right, \labels{z}{C}}}
		{\vlhtr{\DD_2^{\rn w}}{\B, \accs xy, \futs xz, \futs yv, \accs zv, \Left, \labels{z}{C} \SEQ \Right}}}
\end{smallequation*}
%where, again, if $u$ appeared in $\DD_2$ it is renamed to a fresh label $v$.

\end{itemize}
              \item[Key cases:]\label{key-cases}
                If the last rule in $\DD_1$ and the last rule in $\DD_2$ both apply to the cut-formulas, then it is the complexity of the cut-formula that is the decreasing inductive measure, save for the modal cases, where it is important to note the combination of induction on both height and formula size . 
                \begin{itemize}
				\item $C=A \AND B$:
				
				
				\begin{smallequation*}\hspace*{-7.5em}
					\vlderiibase{\labrn{cut}}{}{\B, \Left \SEQ \Right}{
						\vliin{\rlabrn\AND}{}{\B, \Left \SEQ \Right, \labels{x}{A \AND B}}{
							\vlhtr{\DD_1}{\B, \Left \SEQ \Right, \labels{x}{A}}}{
							\vlhtr{\DD_2}{\B, \Left \SEQ \Right, \labels{x}{B}}}}		
					{
						\vlin{\llabrn\AND}{}{\B, \Left, \labels{x}{A \AND B} \SEQ \Right}{
							\vlhtr{\DD_3}{\B, \Left, \labels{x}{A}, \labels{x}{B} \SEQ \Right}}
					}
					\reducesto
%				\end{smallequation*}
%				
%				\begin{smallequation*}
					\vlderiibase{\labrn{cut}}{}{\B, \Left \SEQ \Right}{
						\vlhtr{\DD_1}{\B, \Left \SEQ \Right, \labels{x}{A}}
%						\vliin{\labrn{cut}}{}{\B, \Left \SEQ \Right, \labels{x}{A}}{
%							\vlhtr{\DD_1^{\rn w}}{\B, \Left \SEQ \Right, \labels{x}{A}, \labels{x}{B}}}{
%							\vlhtr{\DD_1^{\rn w}}{\B, \Left, \labels{x}{B} \SEQ \Right, \labels{x}{A}}}
						}		
					{
						\vliin{\labrn{cut}}{}{\B, \Left, \labels{x}{A} \SEQ \Right}{
							\vlhtr{\DD_2^{\rn w}}{\B, \Left, \labels{x}{A} \SEQ \Right, \labels{x}{B}}
							}{
							\vlhtr{\DD_3}{\B, \Left, \labels{x}{A}, \labels{x}{B} \SEQ \Right}
						}
					}
				\end{smallequation*}
				
				\item $C=A \OR B$:
				
				\begin{smallequation*}\hspace*{-7.5em}
					\vlderiibase{\labrn{cut}}{}{\B, \Left \SEQ \Right}{
						\vlin{\rlabrn\OR}{}{\B, \Left \SEQ \Right, \labels{x}{A \OR B}}{
							\vlhtr{\DD_1}{\B, \Left \SEQ \Right, \labels{x}{A}, \labels{x}{B}}}}		
					{
						\vliin{\llabrn\OR}{}{\B, \Left, \labels{x}{A \OR B} \SEQ \Right}{
							\vlhtr{\DD_2}{\B, \Left, \labels{x}{A} \SEQ \Right}}{
							\vlhtr{\DD_3}{\B, \Left, \labels{x}{B} \SEQ \Right}}
					}
					\reducesto
%				\end{smallequation*}
%				
%				\begin{smallequation*}
					\vlderiibase{\labrn{cut}}{}{\B, \Left \SEQ \Right}{
						\vliin{\labrn{cut}}{}{\B, \Left \SEQ \Right, \labels{x}{A}}{
							\vlhtr{\DD_1}{\B, \Left \SEQ \Right, \labels{x}{A}, \labels{x}{B}}}{
							\vlhtr{\DD_3^{\rn w}}{\B, \Left, \labels{x}{B} \SEQ \Right, \labels{x}{A}}}}		
					{
						\vlhtr{\DD_2}{\B, \Left, \labels{x}{A} \SEQ \Right}
%						\vliin{\labrn{cut}}{}{\B, \Left, \labels{x}{A} \SEQ \Right}{
%							\vlhtr{\DD_2^{\rn w}}{\B, \Left, \labels{x}{A}, \labels{x}{B} \SEQ \Right}}{
%							\vlhtr{\DD_2^{\rn w}}{\B, \Left, \labels{x}{A} \SEQ \Right, \labels{x}{B}}}
					}
				\end{smallequation*}
				
                \item $C=A\IMP B$:

\begin{smallequation*}
	\vlderiibase{\labrn{cut}}{}{\B, \futs xy, \Left \SEQ \Right}{
		\vlin{\rlabrn\IMP}{}{\B, \futs xy, \Left \SEQ \Right, \labels{x}{A \IMP B}}{
			\vlhtr{\DD_1}{\B, \futs xy, \futs{x}{x'}, \Left, \labels{x'}{A} \SEQ \Right, \labels{x'}{B}}
		}
	}{
	\vliin{\llabrn\IMP}{}{\B, \futs xy, \Left, \labels{x}{A \IMP B} \SEQ \Right}{
		\vlhtr{\DD_2}{\B, \futs xy, \Left, \labels{x}{A \IMP B} \SEQ \Right, \labels{y}{A}}
	}{
	\vlhtr{\DD_3}{\B, \futs xy, \Left, \labels{y}{B} \SEQ \Right}
}
}
\reducesto
\end{smallequation*}

\begin{smallequation*}
	\hspace*{-7em}
	\vlderiibase{\labrn{cut}}{}{\B, \futs xy, \Left \SEQ \Right}{
		\vliin{\labrn{cut}}{}{\B, \futs xy, \Left \SEQ \Right, \labels{y}{A}}{
			\vlin{\rlabrn\IMP}{}{\B, \futs xy, \Left \SEQ \Right, \labels{x}{A \IMP B}, \labels{y}{A}}{
				\vlhtr{\DD_1^{\rn w}}{\B, \futs xy, \futs{x}{x'}, \Left, \labels{x'}{A} \SEQ \Right, \labels{x'}{B}, \labels{y}{A}}
			}
			%			\vlhtr{\DD_{1\rn w}'}{\B_1, \Left \SEQ \Right, \labels{x}{A \IMP B}, \labels{y}{A}}
		}{
		\vlhtr{\DD_{2}}{\B, \futs xy, \Left, \labels{x}{A \IMP B} \SEQ \Right, \labels{y}{A}}
	}
}{
\vliin{\labrn{cut}}{}{\B, \futs xy, \Left, \labels{y}{A} \SEQ \Right}{
	\vlhtr{\DD_1[y/x']}{\B, \futs xy, \Left, \labels{y}{A} \SEQ \Right, \labels{y}{B}}
}{
\vlhtr{\DD_3}{\B, \futs xy, \Left, \labels{y}{B} \SEQ \Right}
}
}
\end{smallequation*}
%\todo{check that substitution!}
                \item $C=\BOX A$:

\begin{smallequation*}
	\vlderiibase{\labrn{cut}}{}{\B, \futs xu, \accs uv, \Left \SEQ \Right}{
		\vlin{\rlabrn\BOX}{}{\B, \futs xu, \accs uv, \Left \SEQ \Right, \labels{x}{\BOX A}}{
			\vlhtr{\DD_1}{\B, \futs xu, \accs{u}{v}, \futs{x}{x'}, \accs{x'}{y'}, \Left \SEQ \Right, \labels{y'}{A}}	
		}
	}{
	\vlin{\llabrn\BOX}{}{\B, \futs xu, \accs uv, \Left, \labels{x}{\BOX A} \SEQ \Right}{
		\vlhtr{\DD_2}{\B, \futs xu, \accs uv, \Left, \labels{x}{\BOX A}, \labels{v}{A} \SEQ \Right}
	}
}
\reducesto
\end{smallequation*}

\begin{smallequation*}\hspace*{-7em}
	\vlderiibase{\labrn{cut}}{}{\B, \futs xu, \accs uv, \Left \SEQ \Right}{
		%		\vliin{\labrn{cut}}{}{\B_1, \B_2, x \le u, u \rel v, \Left \SEQ \Right, \labels{v}{A}}{
		\vlhtr{\DD_1[u/x',v/y']}{\B, \futs xu, \accs uv, \Left \SEQ \Right, \labels{v}{A}}	
		%			}{
		%			\vlin{\llabrn\BOX}{}{\B_2, x \le u, u \rel v, \Left, \labels{x}{\BOX A} \SEQ \Right, \labels{v}{A}}{
		%				\vlin{\labrn{id}}{}{\B_2, x \le u, u \rel v, \Left, \labels{x}{\BOX A}, \labels{v}{A} \SEQ \Right, \labels{v}{A}}{
		%					\vlhy{}
		%					}
		%				}
		%			}
	}{
	\vliin{\labrn{cut}}{}{\B, \futs xu, \accs uv, \Left, \labels{v}{A} \SEQ \Right}{
		\vlin{\rlabrn\BOX}{}{\B, \futs xu, \accs uv, \Left, \labels{v}{A} \SEQ \Right, \labels{x}{\BOX A}}{
			\vlhtr{\DD_1^{\rn w}}{\B, \futs xu, \accs uv, \futs{x}{x'}, \accs{x'}{y'}, \Left, \labels{v}{A} \SEQ \Right, \labels{x}{\BOX A}, \labels{y'}{A}}	
		}
	}{
	\vlhtr{\DD_2}{\B, \futs xu, \accs uv, \Left, \labels{x}{\BOX A}, \labels{v}{A} \SEQ \Right}
}
}
\end{smallequation*}
%\lutz{we need to explain why we can apply the IH}
The top $\rn{cut}$ is admissible by induction on the height, as the size of the cut-formula is constant. This however may increase the height above the right premiss of the bottom $\rn{cut}$ arbitrarily. The bottom $\rn{cut}$ is still admissible as the size of the cut-formula decreases.

                \item $C=\DIA A$:

\begin{smallequation*}
	\vlderiibase{\labrn{cut}}{}{\B, \accs xy, \Left \SEQ \Right}{
		\vlin{\rlabrn\DIA}{}{\B, \accs xy, \Left \SEQ \Right, \labels{x}{\DIA A}}{
			\vlhtr{\DD_1}{\B, \accs xy, \Left \SEQ \Right, \labels{x}{\DIA A}, \labels{y}{A}}
		}
	}{
	\vlin{\llabrn\DIA}{\text{\footnotesize $\lb{y'}$ is fresh}}{\B, \accs xy, \Left, \labels{x}{\DIA A} \SEQ \Right}{
		\vlhtr{\DD_2}{\B, \accs xy, \accs{x}{y'},\Left, \labels{y'}{A} \SEQ \Right}
	}	
}
\end{smallequation*}

\begin{smallequation*}
	\vlderiibase{\labrn{cut}}{}{\B, \accs xy, \Left \SEQ \Right}{
		\vliin{\labrn{cut}}{}{\B, \accs xy, \Left \SEQ \Right, \labels{y}{A}}{
			\vlhtr{\DD_1}{\B, \accs xy, \Left \SEQ \Right, \labels{y}{A}, \labels{x}{\DIA A}}
		}{
		\vlin{\llabrn\DIA}{\text{\footnotesize $\lb{y'}$ is fresh}}{\B, \accs xy, \Left, \labels{x}{\DIA A} \SEQ \Right, \labels{y}{A}}{
			\vlhtr{\DD_2^{\rn w}}{\B, \accs xy, \accs{x}{y'},\Left, \labels{y'}{A} \SEQ \Right, \labels{y}{A}}
		}
	}
}{
\vlhtr{\DD_2[y/y']}{\B, \accs xy, \Left, \labels{y}{A} \SEQ \Right}
}
\end{smallequation*}
%\lutz{Same. we need to explain why we can apply the IH}
The induction hypothesis is applied here again twice as above, on the height for the top $\rn{cut}$ and on formula size for the bottom one.
                \end{itemize}
                \end{description}
\end{proof}


                We can now complete the proof of Theorem~\ref{thm:cut-adm}.
\begin{proof}[Proof of Theorem~\ref{thm:cut-adm}]
			By induction on number of $\rn{cut}$ rules, always applying Lemma~\ref{lem:reduction} to the leftmost topmost cut.
\end{proof}

%\begin{theorem}
%	\label{thm:mon-adm}
%	The rule $\rn{mon}$ is admissible for system $\labIKp$.
%\end{theorem}
%
%
%\begin{proof}[Proof of Theorem~\ref{thm:mon-adm}]
%	
%	\todo{say that we can obtain the proof of monotonicity using the proof of cut.}
%	
%\end{proof}
