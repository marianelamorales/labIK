%%%%%%%%%%%%%%%%%%%%%%%%%%%%%%%%%%%%%%%%%%%%%%%%%%%%%%%%%
%%%%%%%%%%%%%%%%%%%%%%%%%%%%%%%%%%%%%%%%%%%%%%%%%%%%%%%%%
%%%%%%%%%%%%%%%%%%%%%%%%%%%%%%%%%%%%%%%%%%%%%%%%%%%%%%%%%

\section{Completeness}\label{sec:completeness}

In this section we show our system at work, as most of the section
consists of derivations of axioms of $\IK$ in $\labIKp$. More precisely, we prove completeness of $\labIKp+\rn{cut}$, i.e., the implication \ref{i}$\implies$\ref{ii} of Theorem~\ref{thm:cutfree-compl}, which is stated again below:

\begin{theorem}\label{thm:completeness}
	For any formula $\fm A$. If $\fm A$ is a theorem of $\IK$ then $\fm A$ is provable in $\labIKp +\labrn{cut}$.
\end{theorem}

\begin{remark}
  We have seen already in the proof of Proposition~\ref{prop:id} the
  use of the rule $\rn{F_2}$. In the following proof of
  Theorem~\ref{thm:completeness} we also see the use of the rules
  $\rn{F_1}$, $\rn{refl}$, and $\rn{trans}$.
\end{remark}

%% \review{\emph{Remark 4.2. Show the ``need" you would likely have to show that no other proofs of the axioms are possible. That is of course the case, but unless you prove it formally, it is not fully correct to claim that the ``need'' is shown.}}
%% \lutz{Done.}

\begin{proof}[Proof of Theorem~\ref{thm:completeness}]
  We begin by showing how the axioms $\kax[1]$--$\kax[5]$ are proved in system $\labIKp$.
  \begin{itemize}
  \item $\kax[1]$:
    \begin{smallequation*}
    \vlderivation{
				\vlin{\rlabrn{\IMP}}
				{\lb y \mbox{ fresh}}
				{\SEQ \labels{x}{\BOX (A \IMP B) \IMP (\BOX A \IMP \BOX B)}}
				{\vlin {\rlabrn{\IMP}}
					{\lb z \mbox{ fresh}}
					{\lseq{\futs xy}{\labels{y}{\BOX(A \IMP B)}}{\labels{y}{\BOX A \IMP \BOX B}}}
					{\vlin {\rlabrn{\BOX}}
						{\lb w, \lb u \mbox{ fresh}}
						{\lseq{\futs xy, \futs yz}{\labels{y}{\BOX(A \IMP B)}, \labels{z}{\BOX A}}{\labels{z}{\BOX B}}}
						{\vlin {\llabrn{\BOX}}
							{}
							{\lseq{\futs xy, \futs yz, \futs zw, \accs wu}{\labels{y}{\BOX(A \IMP B)}, \labels{z}{\BOX A}}{\labels{u}{B}}}
							{\vlin {\color{red}\rn{trans}}
								{}
								{\lseq{\futs xy, \futs yz, \futs zw, \accs wu}{\labels{y}{\BOX(A \IMP B)}, \labels{z}{\BOX A}, \labels{u}{A}}{\labels{u}{B}}}
								{\vlin {\llabrn{\BOX}}
									{}
									{\lseq{\futs xy, \futs yz, \futs zw, \futs yw, \accs wu}{\labels{y}{\BOX(A \IMP B)}, \labels{z}{\BOX A}, \labels{u}{A}}{\labels{u}{B}}}
									{\vlin {\color{red}\rn{refl}}
										{}
										{\lseq{\futs xy, \futs yz, \futs zw, \futs yw, \accs wu}{\labels{y}{\BOX(A \IMP B)}, \labels{z}{\BOX A}, \labels{u}{A}, \labels{u}{A \IMP B}}{\labels{u}{B}}}
										{\vliin{\llabrn{\IMP}}
											{}
											{\lseq{\futs xy, \futs yz, \futs zw, \futs yw, \futs uu, \accs wu}{\labels{y}{\BOX(A \IMP B)}, \labels{z}{\BOX A}, \labels{u}{A}, \labels{u}{A \IMP B}}{\labels{u}{B}}}
											{\vlin {\rn{id_g}}
												{}
												{\lseq{\B}{\labels{y}{\BOX(A \IMP B)}, \labels{z}{\BOX A}, \labels{u}{A}, \labels{u}{A \IMP B}}{\labels{u}{B}, \labels{u}{A}}}
												{\vlhy {}}}
											{\vlin {\rn{id_g}}
												{}
												{\lseq{\B}{\labels{y}{\BOX(A \IMP B)}, \labels{z}{\BOX A}, \labels{u}{A}, \labels{u}{A \IMP B}, \labels{u}{B}}{\labels{u}{B}}}
												{\vlhy {}}}}}}}}}}
    }
\end{smallequation*}
    where $\B$ is equal to: $\futs xy, \futs yz, \futs zw, \futs yw, \futs uu, \accs wu$.
  \item $\kax[2]$:
    \begin{smallequation*}
    \vlderivation {
		\vlin{\rlabrn{\IMP}}
		{\lb y \mbox{ fresh}}
		{ \SEQ \labels{x}{\BOX (A \IMP B) \IMP (\DIA A \IMP \DIA B)}}
		{\vlin {\rlabrn{\IMP}}
			{\lb z \mbox{ fresh}}
			{\lseq{\futs xy}{\labels{y}{\BOX (A \IMP B)}}{\labels{y}{(\DIA A \IMP \DIA B)}}}
			{\vlin {\llabrn{\DIA}}
				{\lb u \mbox{ fresh}}
				{\lseq{\futs xy, \futs yz}{\labels{y}{\BOX (A \IMP B)}, \labels{z}{\DIA A}}{\labels{z}{\DIA B}}}
				{\vlin{\rlabrn{\DIA}}
					{}
					{\lseq{\futs xy, \futs yz, \accs zu}{\labels{y}{\BOX (A \IMP B)}, \labels{u}{A}}{\labels{z}{\DIA B}}}
					{\vlin {\llabrn{\BOX}}
						{}
						{\lseq{\futs xy, \futs yz, \accs zu}{\labels{y}{\BOX (A \IMP B)}, \labels{u}{A}}{\labels{z}{\DIA B}, \labels{u}{B}}}
						{\vlin {\color{red}\rn{refl}}
							{}
							{\lseq{\futs xy, \futs yz, \accs zu}{\labels{y}{\BOX (A \IMP B)}, \labels{u}{A}, \labels{u}{A \IMP B}}{\labels{z}{\DIA B}, \labels{u}{B}}}
							{\vliin{\llabrn{\IMP}}
								{}
								{\lseq{\futs xy, \futs yz, \accs zu, \futs uu}{\labels{y}{\BOX (A \IMP B)}, \labels{u}{A}, \labels{u}{A \IMP B}}{\labels{z}{\DIA B}, \labels{u}{B}}}
								{\vlin {\rn{id_g}}
									{}
									{\lseq{\B}{\labels{y}{\BOX (A \IMP B)}, \labels{u}{A}, \labels{u}{A \IMP B}}{\labels{z}{\DIA B}, \labels{u}{B}, \labels{u}{A}}}
									{\vlhy {}}}
								{\vlin {\rn{id_g}}
									{}
									{\lseq{\B}{\labels{y}{\BOX (A \IMP B)}, \labels{u}{A}, \labels{u}{A \IMP B}, \labels{u}{B}}{\labels{z}{\DIA B}, \labels{u}{B}}}
									{\vlhy {}}}								}}}}}}
    }
        \end{smallequation*}
        
    where $\B$ is equal to $\futs xy, \futs yz, \accs zu, \futs uu$.
  \item $\kax[3]$:
    \begin{smallequation*}
    \vlderivation {
			\vlin{\rlabrn{\IMP}}
			{\lb y \mbox{ fresh}}
			{\SEQ \labels{x}{\DIA (A \OR B) \IMP (\DIA A \OR \DIA B)}}
			{\vlin {\llabrn{\DIA}}
				{\lb z \mbox{ fresh}}
				{\lseq{\futs xy}{\labels{y}{\DIA (A \OR B)}}{\labels{y}{\DIA A \OR \DIA B}}}
				{\vliin{\llabrn{\OR}}{}{\lseq{\futs xy, \accs yz}{\labels{z}{A \OR B}}{\labels{y}{\DIA A \OR \DIA B}}}{\vlin {\rlabrn{\OR}}
						{}
						{\lseq{\futs xy, \accs yz}{\labels{z}{A}}{\labels{y}{\DIA A \OR \DIA B}}}
						{\vlin {\rlabrn{\DIA}}
							{}
							{\lseq{\futs xy, \accs yz}{\labels{z}{A}}{\labels{y}{\DIA A}, \labels{y}{\DIA B}}}
							{\vlin {\color{red}\rn{refl}}
								{}
								{\lseq{\futs xy, \accs yz}{\labels{z}{A}}{\labels{y}{\DIA A}, \labels{z}{A}, \labels{y}{\DIA B}}}
								{\vlin {\rn{id_g}}
									{}
									{\lseq{\futs xy, \futs zz, \accs yz}{\labels{z}{A}}{\labels{y}{\DIA A}, \labels{z}{A}, \labels{y}{\DIA B}}}
									{\vlhy {}}}}}}{\vlin {\rlabrn{\OR}}
						{}
						{\lseq{\futs xy, \accs yz}{\labels{z}{B}}{\labels{y}{\DIA A \OR \DIA B}}}
						{\vlin {\rlabrn{\DIA}}
							{}
							{\lseq{\futs xy, \accs yz}{\labels{z}{B}}{\labels{y}{\DIA A}, \labels{y}{\DIA B}}}
							{\vlin {\color{red}\rn{refl}}
								{}
								{\lseq{\futs xy, \accs yz}{\labels{z}{B}}{\labels{y}{\DIA A}, \labels{y}{\DIA B}, \labels{z}{B}}}
								{\vlin {\rn{id_g}}
									{}
									{\lseq{\futs xy, \futs zz, \accs yz}{\labels{z}{B}}{\labels{y}{\DIA A}, \labels{y}{\DIA B}, \labels{z}{B}}}
									{\vlhy {}}}}}}}}
		}
    \end{smallequation*}
  \item $\kax[4]$:
    \begin{smallequation*}
	\vlderivation {
		\vlin{\rlabrn{\IMP}}
		{\lb y \mbox{ fresh}}
		{\SEQ \labels{x}{(\DIA A \IMP \BOX B) \IMP \BOX (A \IMP B)}}
		{\vlin {\rlabrn{\BOX}}
			{\lb z, \lb w \mbox{ fresh}}
			{\lseq{\futs xy}{\labels{y}{\DIA A \IMP \BOX B}}{\labels{y}{\BOX (A \IMP B)}}}
			{\vlin {\rlabrn{\IMP}}
				{\lb u \mbox{ fresh}}
				{\lseq{\futs xy, \futs yz, \accs zw}{\labels{y}{\DIA A \IMP \BOX B}}{\labels{w}{A \IMP B}}}
				{\vlin {\color{red}{\rn{F_1}}}
					{\lb t \mbox{ fresh}}
					{\lseq{\futs xy, \futs yz, \futs wu, \accs zw}{\labels{y}{\DIA A \IMP \BOX B}, \labels{u}{A}}{\labels{u}{B}}}
					{\vlin {\color{red}\rn{trans}}
						{}
						{\lseq{\futs xy, \futs yz, \futs wu, \futs zt, \accs zw, \accs tu}{\labels{y}{\DIA A \IMP \BOX B}, \labels{u}{A}}{\labels{u}{B}}}
						{\vliin {\llabrn{\IMP}}
							{}
							{\lseq{\futs xy, \futs yz, \futs wu, \futs zt, \futs yt, \accs zw, \accs tu}{\labels{y}{\DIA A \IMP \BOX B}, \labels{u}{A}}{\labels{u}{B}}}
							{\vlin {\rlabrn{\DIA}}
									{}
									{\lseq{\B}{\labels{y}{\DIA A \IMP \BOX B}, \labels{u}{A}}{\labels{u}{B}, \labels{t}{\DIA A}}}
									{\vlin {\color{red}\rn{refl}}
										{}
										{\lseq{\B}{\labels{y}{\DIA A \IMP \BOX B}, \labels{u}{A}}{\labels{u}{B}, \labels{t}{\DIA A}, \labels{u}{A}}}
										{\vlin {\rn{id_g}}
											{}
											{\lseq{\B, \futs uu}{\labels{y}{\DIA A \IMP \BOX B}, \labels{u}{A}}{\labels{u}{B}, \labels{t}{\DIA A}, \labels{u}{A}}}
											{\vlhy {}}}}}
							{\vlin {\color{red}\rn{refl}}
								{}
								{\lseq{\B}{\labels{y}{\DIA A \IMP \BOX B}, \labels{u}{A}, \labels{t}{\BOX B}}{\labels{u}{B}}}
								{\vlin {\llabrn{\BOX}}
									{}
									{\lseq{\B, \futs tt}{\labels{y}{\DIA A \IMP \BOX B}, \labels{u}{A}, \labels{t}{\BOX B}}{\labels{u}{B}}}
									{\vlin {\color{red}\rn{refl}}
										{}
										{\lseq{\B, \futs tt}{\labels{y}{\DIA A \IMP \BOX B}, \labels{u}{A}, \labels{t}{\BOX B}, \labels{u}{B}}{\labels{u}{B}}}
										{\vlin {\rn{id_g}}
											{}
											{\lseq{\B, \futs tt, \futs uu}{\labels{y}{\DIA A \IMP \BOX B}, \labels{u}{A}, \labels{t}{\BOX B}, \labels{u}{B}}{\labels{u}{B}}}
											{\vlhy {}}}}}}}}}}}}
    \end{smallequation*}

where $\B$ is equal to $\futs xy, \futs yz, \futs wu, \futs zt, \futs yt, \accs zw, \accs tu$.
\item $\kax[5]$:
  \begin{smallequation*}
  \vlderivation {
		\vlin{\rlabrn{\IMP}}
		{\lb y \mbox{ fresh}}
		{\SEQ \labels{x}{\DIA \BOT \IMP \BOT}}
		{\vlin {\llabrn{\DIA}}
			{\lb z \mbox{ fresh}}
			{\lseq{\futs xy}{\labels{y}{\DIA \BOT}}{\labels{y}{\BOT}}}
			{\vlin {\llabrn{\BOT}}
				{}
				{\lseq{\futs xy, \accs yz}{\labels{z}{\BOT}}{\labels{y}{\BOT}}}
				{\vlhy {}}}}
  }
    \end{smallequation*}
    
  \end{itemize}
  Next, we have to prove all axioms of intuitionistic propositional logic can be shown in $\labIKp$. We do this only for $\fm{A \AND B \IMP B}$ and leave the rest to the reader:
  \begin{smallequation*}
  \vlderivation {
		\vlin{\rlabrn{\IMP}}
		{\lb y \mbox{ fresh}}
		{\SEQ \labels{x}{A \AND B \IMP B}}
		{\vlin {\llabrn{\AND}}
			{}
			{\lseq{\futs xy}{\labels{y}{A \AND B}}{\labels{y}{B}}}
			{\vlin {\color{red}\rn{refl}}
				{}
				{\lseq{\futs xy}{\labels{y}{A}, \labels{y}{B}}{\labels{y}{B}}}
				{\vlin {\rn{id_g}}
					{}
					{\lseq{\futs xy, \futs yy}{\labels{y}{A}, \labels{y}{B}}{\labels{y}{B}}}
					{\vlhy {}}}}}
  }
    \end{smallequation*}
    
  
  
  Finally, we have to show how the rules of modus ponens and
  necessitation can be simulated in our system. For modus ponens, this
  is standard using the cut rule and for necessitation, we can
  transform a proof of $\fm A$ into a proof of $\fm{\BOX A}$ as
  follows.
  A proof of $\fm A$ is in fact a proof $\DD$ of the sequent
  $\SEQ{\labels{z}{A}}$ for some label $\lb z$. If $\lb x$ and $\lb y$
  are fresh labels, we can transform $\DD$ into a proof $\DD'$ of the
  sequent $\futs xy, \accs yz \SEQ \labels{z}{A}$ by adding $\futs xy,
  \accs yz$ to every line. We can now apply the $\rlabrn\BOX$-rule to
  obtain a proof of $\SEQ \labels{x}{\BOX A}$.
  
%\begin{lemma}
%	If there exists a proof of $\vlderivation {\vlpd{\Done}{}{\SEQ \labels{z}{A}}}$ then there exists a proof of $\vlderivation { \vlpd{\Dtwo}{}{\SEQ \labels{x}{\BOX A}}}$
%\end{lemma}
%
%\begin{proof}
%	
%	We assume that there exists a proof of $\vlderivation {\vlpd {\Done}{}{\SEQ \labels{z}{A}}}$ and we want to obtain a proof of $\vlderivation { \vlpd{\Dtwo}{}{\SEQ \labels{x}{\BOX A}}}$.
%	
%	Using the rule $\rlabrn{\BOX}$ introduced in the system $\labIKp$ and the proof of $\labels{z}{A}$ from our hypothesis, we can build the following proof:
%	$\vlderivation{\vlin{\rlabrn{\BOX}}{}{\SEQ \labels{x}{\BOX A}}{\vlin{\rn{w}}{}{\futs xy, \accs yz \SEQ \labels{z}{A}}{\vlhy{ \SEQ \labels{z}{A}}}}}$ \hspace{3mm} or what it is the same $\vlderivation{\vlin{\rlabrn{\BOX}}{}{\SEQ \labels{x}{\BOX A}}{\vlpd {\Dwone}{}{\futs xy, \accs yz \SEQ \labels{z}{A}}}}$.
%	
%	Let $\Dtwo$ be equal to: $\Dtwo = \vlderivation {\vlpd {\Dwone}{}{\futs xy, \accs yz \SEQ \labels{z}{A}}}$. Therefore, we have the proof $\vlderivation { \vlpd{\Dtwo}{}{\SEQ \labels{x}{\BOX A}}}$.
	
%	If we have a proof of $\vlderivation {\vlpd {\Done}{}{\SEQ \labels{z}{A}}}$ then we can obtain a proof of $\labels{x}{\BOX A}$ using the rule $\rlabrn{\BOX}$ introduced in the system $\labIKp$ and the proof of $\labels{z}{A}$ that we assumed  $\vlderivation{\vlin{\rlabrn{\BOX}}{}{\SEQ \labels{x}{\BOX A}}{\vlpd {\Dwone}{}{\futs xy, \accs yz \SEQ \labels{z}{A}}}}$
%	
%	Let $\Dtwo$ be equal to: $\Dtwo = \vlderivation {\vlpd {\Dwone}{}{\futs xy, \accs yz \SEQ \labels{z}{A}}}$.
%	Then using the \emph{weakening} rule, we have the following proof:
%	
%	\bigskip
%	\begin{center}
%		$\vlderivation{\vlin{\rlabrn{\BOX}}{}{\SEQ \labels{x}{\BOX A}}{\vlin{\rn{w}}{}{\futs xy, \accs yz \SEQ \labels{z}{A}}{\vlhy{ \SEQ \labels{z}{A}}}}}$
%	\end{center}
%	
%	
%	
%\end{proof}

  
  This completes the proof of Theorem~\ref{thm:completeness}.
\end{proof}

\sonia{We could have proved the same result via proof search and counter-model construction as rules are terminationg in $\labIKp$ but this technique would not be generalisable to the case of Sctt-Lemmon logics. }
\todo{This would allow us to show completeness directly for any labelled sequents. Here, we are interested only in formulas that can be expressed in the logic $\IK$ and extensions. }


%\sonia{I think the following needs to appear somewhere as it is rather illustrative of the key points. }
%
%This a a reflection on the proof-theoretic side of the model theoretic reasons to enforce $(F1)$ and $(F2)$ conditions on birelational models.
%
%In particular, in order to prove (i) $\rightarrow$ (ii), we present a syntactic completeness proof with respect the Hilbert system which means that we prove all Hilbert axioms using the rules from our system $\labIKp$: we give a proof for all the axioms of propositional intuionistic logic, for the five variants of $\mathsf{k}$ axiom from the intuitionistic syntax and finally, we simulate the necesssitation rule and modus ponens. As we mentioned, in the course of the proof of (i) $\rightarrow$ (ii), we have to derive the five $\kax$ axioms. 
%%
%As an example, we display the derivation of $\kax[4]$ which also illustrates the need of having the rule corresponding to $\rn{F_1}$ in the system.
%\bigskip
%
%\vspace*{-.9cm}
%$$
%\hspace*{-.5cm}
%\scalebox{.9}{
%	$
%	\vlderivation{
%		\vlin{\rlabrn\IMP}{\text{\scriptsize $y$ fresh}} {\SEQ \labels{x}{(\DIA A \IMP \BOX B) \SEQ \BOX (A \IMP B)}}{
%			\vlin{\rlabrn\BOX}{\text{\scriptsize $z, w$ fresh}}{x \le y, \labels{y}{\DIA A \IMP \BOX B} \SEQ \labels{y}{\BOX (A \IMP B)}}{
%				\vlin {\rlabrn\IMP}{\text{\scriptsize $u$ fresh}}{x \le y, y\le z, z \rel w, \labels{y}{\DIA A \IMP \BOX B} \SEQ \labels{w}{A \IMP B}}{
%					\vlin {\color{red}{\rn{F_1}}}{}{x \le y, y \le z, w \le u, z \rel w, \labels{y}{\DIA A \SEQ \BOX B}, \labels{u}{A} \SEQ \labels{u}{B}}{
%						\vlin {\rn{trans}}{}{x \le y, y \le z, w \le u, z \le t, z \rel w, t \rel u, \labels{y}{\DIA A \IMP \BOX B}, \labels{u}{A} \SEQ \labels{u}{B}}{
%							\vliin {\llabrn\IMP}{}{x \le y, y \le z, w \le u, z \le t, y \le t, z \rel w, t \rel u, \labels{y}{\DIA A \IMP \BOX B}, \labels{u}{A} \SEQ \labels{u}{B}}{
%								\vlin {\rlabrn\DIA}{}{x \le y, y \le z, w \le u, z \le t, y \le t, z \rel w, t \rel u, \labels{u}{A} \SEQ \labels{u}{B}, \labels{t}{\DIA A}}{
%									\vlin {\rn{refl}}{}{x \le y, y \le z, w \le u, z \le t, y \le t, z \rel w, t \rel u, \labels{u}{A} \SEQ \labels{u}{B}, \labels{t}{\DIA A}, \labels{u}{A}}{
%										\vlin {\labrn{id_g}}{}{x \le y, y \le z, w \le u, z \le t, y \le t, u \le u, z \rel w, t \rel u, \labels{u}{A} \SEQ \labels{u}{B}, \labels{t}{\DIA A}, \labels{u}{A}}{
%											\vlhy {}
%											}
%											}
%											}
%											}{
%											%						\vlin {\rn{refl}}{}{x \le y, y \le z, w \le u, z \le t, y \le t, z \rel w, t \rel u, \labels{y}{\DIA A \IMP \BOX B}, \labels{u}{A}, \labels{t}{\BOX B} \SEQ \labels{u}{B}}{
%											%							\vlin {\llabrn\BOX}{}{x \le y, y \le z, w \le u, z \le t, y \le t, t \le t, z \rel w, t \rel u, \labels{y}{\DIA A \IMP \BOX B}, \labels{u}{A}, \labels{t}{\BOX B} \SEQ \labels{u}{B}}{
%											%								\vlin {\rn{refl}}{}{x \le y, y \le z, w \le u, z \le t, y \le t, t \le t, z \rel w, t \rel u, \labels{y}{\DIA A \IMP \BOX B}, \labels{u}{A}, \labels{t}{\BOX B}, \labels{u}{B} \SEQ \labels{u}{B}}{
%											%									\vlin {\labrn{id_g}}{}{x \le y, y \le z, w \le u, z \le t, y \le t, t \le t, u \le u, z \rel w, t \rel u, \labels{y}{\DIA A \IMP \BOX B}, \labels{u}{A}, \labels{t}{\BOX B}, \labels{u}{B} \SEQ \labels{u}{B}}{
%											\vlhy {\qquad\vdots\qquad}
%											%										}
%											%									}
%											%								}
%											%							}
%											}
%											}
%											}
%											}
%											}
%											}
%											}$
%											}$$
%											%\end{example}
%											
%											%\vspace*{-.5cm}
%											
%Note that our system offers only an atomic version of the identity rule, though the above derivation uses a general version of the identity rule $\rn{id_g}$ that applies to generic formulas. 
%%
%We therefore have to show that such a rule is admissible in our system.
%%
%As an example, we display one step of this admissibility proof that also illustrates the need for the rule $\rn{F_2}$. The other cases are standard.
%
%\vspace*{-.5cm}
%%\begin{example}
%$$
%\scalebox{.9}{
%	$
%	\vlderivation{
%		\vlin{\llabrn\DIA}{}{\B, x \le y, \Left, \labels{x}{\DIA A} \SEQ \Right, \labels{y}{\DIA A}}{
%			\vlin{\color{red}{\rn{F_2}}}{}{\B, x \le y, x \rel z, \Left, \labels{z}{A} \SEQ \Right, \labels{y}{\DIA A}}{
%				\vlin{\rlabrn\DIA}{}{\B, x \le y, x \rel z, z \le u, y \rel u, \Left, \labels{z}{A} \SEQ \Right, \labels{y}{\DIA A}}{
%					\vlin{\labrn{id_g}}{}{\B, x \le y, x \rel z, z \le u, y \rel u, \Left, \labels{z}{A} \SEQ \Right, \labels{y}{\DIA A}, \labels{u}{A}}{
%						\vlhy{}
%						}
%						}
%						}
%						}
%						}
%						$
%						}
%						$$
%						%\end{example}
